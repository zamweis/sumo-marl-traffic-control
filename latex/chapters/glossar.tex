% glossar.tex

% --- Allgemeine Begriffe ---
\newglossaryentry{reinforcementlearning}{%
    name={Reinforcement Learning (RL)},%
    description={Ein Teilgebiet des maschinellen Lernens, bei dem ein Agent durch Interaktion mit einer Umgebung lernt, Entscheidungen zu treffen, um eine kumulative Belohnung zu maximieren.}%
}

\newglossaryentry{agent}{%
    name={Agent},%
    description={Eine Entscheidungsinstanz im Reinforcement Learning, die in der Umgebung handelt, Beobachtungen verarbeitet und eine Policy befolgt.}%
}

\newglossaryentry{policy}{%
    name={Policy},%
    description={Eine Strategie oder Abbildung, die festlegt, welche Aktion ein Agent in einem gegebenen Zustand ausführt.}%
}

\newglossaryentry{rewardfunction}{%
    name={Reward-Funktion},%
    description={Eine Funktion, die dem Agenten Rückmeldung über die Qualität einer Aktion gibt und das Lernen steuert. In dieser Arbeit wurden verschiedene Varianten untersucht (z.\,B. Diff-Waiting-Time, Queue, Real-World, Emissionen).}%
}

\newglossaryentry{episode}{%
    name={Episode},%
    description={Eine vollständige Simulationseinheit, bestehend aus einer Sequenz von Zeitschritten vom Start bis zur Terminierung.}%
}

\newglossaryentry{seed}{%
    name={Seed},%
    description={Startwert für Zufallszahlengeneratoren, der die Reproduzierbarkeit von Experimenten sicherstellt.}%
}

\newglossaryentry{hyperparameter}{%
    name={Hyperparameter},%
    description={Parameter, die das Lernverhalten eines Modells steuern, z.\,B. Lernrate, Discount-Faktor oder Explorationsparameter.}%
}

% --- Methoden und Algorithmen ---
\newacronym{ppo}{PPO}{Proximal Policy Optimization}
\newacronym{rl}{RL}{Reinforcement Learning}
\newacronym{cpu}{CPU}{Central Processing Unit}
\newacronym{gpu}{GPU}{Graphics Processing Unit}

\newglossaryentry{populationbased}{%
    name={Population-Based Training},%
    description={Ein Verfahren zur automatisierten Anpassung von Hyperparametern, bei dem mehrere Modelle parallel trainiert und Parameter zwischen erfolgreichen Modellen ausgetauscht werden.}%
}

\newglossaryentry{bayesianopt}{%
    name={Bayesian Optimization},%
    description={Ein Verfahren zur Hyperparameter-Optimierung, das auf probabilistischen Modellen basiert und gezielt vielversprechende Konfigurationen auswählt.}%
}

% --- Simulation und Daten ---
\newglossaryentry{SUMO}{%
    name={SUMO},%
    description={Simulation of Urban MObility, eine quelloffene mikroskopische Verkehrssimulationssoftware, die das Verhalten einzelner Fahrzeuge in Straßennetzen abbildet.}%
}

\newglossaryentry{OSM}{%
    name={OpenStreetMap (OSM)},%
    description={Ein kollaboratives Projekt, das Geodaten frei zur Verfügung stellt und als Grundlage für die Straßennetze in dieser Arbeit diente.}%
}

\newglossaryentry{Karlsruhe}{%
    name={Karlsruhe},%
    description={Die in dieser Arbeit verwendete Fallstudien-Stadt. Das Straßennetz basierte auf OpenStreetMap-Daten.}%
}

% --- Verkehrssteuerung ---
\newglossaryentry{trafficlight}{%
    name={Lichtsignalanlage},%
    description={Eine Ampelanlage, die den Verkehr an Knotenpunkten regelt. In dieser Arbeit Zielsystem für die Optimierung.}%
}

\newglossaryentry{fixedtime}{%
    name={Fixed-Time},%
    description={Eine klassische Steuerungsstrategie, bei der feste Signalzeiten für Grün- und Rotphasen verwendet werden. Diente als Baseline in den Experimenten.}%
}

\newglossaryentry{actuated}{%
    name={Actuated Control},%
    description={Eine dynamische Steuerung von Lichtsignalanlagen basierend auf Sensordaten. Diente als zweite Baseline, erwies sich jedoch in den Experimenten als ineffizient.}%
}

\newglossaryentry{queue}{%
    name={Queue},%
    description={Stausituation im Verkehr, gemessen als Anzahl der stoppenden Fahrzeuge. Gleichzeitig eine Reward-Funktion in den Experimenten.}%
}

\newglossaryentry{waitingtime}{%
    name={Wartezeit},%
    description={Die Zeit, die Fahrzeuge im Netz stillstehen oder stark verlangsamt sind.}%
}

\newglossaryentry{teleportation}{%
    name={Teleportation},%
    description={Eine in SUMO vorkommende Notfallmaßnahme, bei der Fahrzeuge versetzt werden, wenn sie in Deadlocks oder unrealistischen Situationen feststecken.}%
}

\newglossaryentry{deadlock}{%
    name={Deadlock},%
    description={Eine Verkehrssituation, in der sich Fahrzeuge gegenseitig blockieren und kein Fortschritt mehr möglich ist.}%
}

% --- Zukunftsperspektiven ---
\newglossaryentry{connectedvehicles}{%
    name={Connected Vehicles},%
    description={Fahrzeuge, die über drahtlose Kommunikation Daten mit Infrastruktur oder anderen Fahrzeugen austauschen können.}%
}

\newglossaryentry{autonomousvehicles}{%
    name={Autonome Fahrzeuge},%
    description={Fahrzeuge, die ohne menschliche Steuerung im Verkehr agieren und Entscheidungen selbstständig treffen.}%
}

\newglossaryentry{smartcity}{%
    name={Smart City},%
    description={Ein Konzept für Städte, die Informations- und Kommunikationstechnologien nutzen, um Effizienz, Nachhaltigkeit und Lebensqualität zu verbessern.}%
}
