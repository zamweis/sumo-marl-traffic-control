% glossar.tex

% --- Allgemeine Begriffe ---
\newglossaryentry{reinforcementlearning}{%
    name={Reinforcement Learning (RL)},%
    description={Ein Teilgebiet des maschinellen Lernens, bei dem ein Agent durch Interaktion mit einer Umgebung lernt, Entscheidungen zu treffen, um eine kumulative Belohnung zu maximieren.}%
}

\newglossaryentry{agent}{%
    name={Agent},%
    description={Eine Entscheidungsinstanz im Reinforcement Learning, die in der Umgebung handelt, Beobachtungen verarbeitet und eine Policy befolgt.}%
}

\newglossaryentry{policy}{%
    name={Policy},%
    description={Eine Strategie oder Abbildung, die festlegt, welche Aktion ein Agent in einem gegebenen Zustand ausführt.}%
}

\newglossaryentry{rewardfunction}{%
    name={Reward-Funktion},%
    description={Eine Funktion, die dem Agenten Rückmeldung über die Qualität einer Aktion gibt und das Lernen steuert. In dieser Arbeit wurden verschiedene Varianten untersucht (z.\,B. Diff-Waiting-Time, Queue, Real-World, Emissionen).}%
}

\newglossaryentry{episode}{%
    name={Episode},%
    description={Eine vollständige Simulationseinheit, bestehend aus einer Sequenz von Zeitschritten vom Start bis zur Terminierung.}%
}

\newglossaryentry{seed}{%
    name={Seed},%
    description={Startwert für Zufallszahlengeneratoren, der die Reproduzierbarkeit von Experimenten sicherstellt.}%
}

\newglossaryentry{hyperparameter}{%
    name={Hyperparameter},%
    description={Parameter, die das Lernverhalten eines Modells steuern, z.\,B. Lernrate, Discount-Faktor oder Explorationsparameter.}%
}

\newglossaryentry{populationbased}{%
    name={Population-Based Training},%
    description={Ein Verfahren zur automatisierten Anpassung von Hyperparametern, bei dem mehrere Modelle parallel trainiert und Parameter zwischen erfolgreichen Modellen ausgetauscht werden.}%
}

\newglossaryentry{induktionsschleifen}{%
    name={Induktionsschleifen},%
    description={Sensoren im Straßenbelag, die Fahrzeuge durch magnetische Feldänderungen detektieren. Häufig in verkehrsabhängigen Steuerungen genutzt.}%
}

\newglossaryentry{kamera}{%
    name={Kameras},%
    description={Verkehrssensorik, die Kamerabilder auswertet, um Fahrzeuge, Fußgänger oder Radfahrer zu detektieren.}%
}

\newglossaryentry{bayesianopt}{%
    name={Bayesian Optimization},%
    description={Ein Verfahren zur Hyperparameter-Optimierung, das auf probabilistischen Modellen basiert und gezielt vielversprechende Konfigurationen auswählt.}%
}

\newglossaryentry{osm}{%
    name={OpenStreetMap (OSM)},%
    description={Ein kollaboratives Projekt, das Geodaten frei zur Verfügung stellt und als Grundlage für die Straßennetze in dieser Arbeit diente.}%
}

\newglossaryentry{Karlsruhe}{%
    name={Karlsruhe},%
    description={Die in dieser Arbeit verwendete Fallstudien-Stadt. Das Straßennetz basierte auf OpenStreetMap-Daten.}%
}

% --- Verkehrssteuerung ---
\newglossaryentry{trafficlight}{%
    name={Lichtsignalanlagen},%
    description={Eine Ampelanlage, die den Verkehr an Knotenpunkten regelt. In dieser Arbeit Zielsystem für die Optimierung.}%
}

\newglossaryentry{fixedtime}{%
    name={Fixed-Time},%
    description={Eine klassische Steuerungsstrategie, bei der feste Signalzeiten für Grün- und Rotphasen verwendet werden. Diente als Baseline in den Experimenten.}%
}

\newglossaryentry{actuated}{%
    name={Actuated Control},%
    description={Eine dynamische Steuerung von Lichtsignalanlagen basierend auf Sensordaten. Diente als zweite Baseline, erwies sich jedoch in den Experimenten als ineffizient.}%
}

\newglossaryentry{queue}{%
    name={Queue},%
    description={Stausituation im Verkehr, gemessen als Anzahl der stoppenden Fahrzeuge. Gleichzeitig eine Reward-Funktion in den Experimenten.}%
}

\newglossaryentry{waitingtime}{%
    name={Wartezeit},%
    description={Die Zeit, die Fahrzeuge im Netz stillstehen oder stark verlangsamt sind.}%
}

\newglossaryentry{teleportation}{%
    name={Teleportation},%
    description={Eine in SUMO vorkommende Notfallmaßnahme, bei der Fahrzeuge versetzt werden, wenn sie in Deadlocks oder unrealistischen Situationen feststecken.}%
}

\newglossaryentry{deadlock}{%
    name={Deadlock},%
    description={Eine Verkehrssituation, in der sich Fahrzeuge gegenseitig blockieren und kein Fortschritt mehr möglich ist.}%
}

% --- Zukunftsperspektiven ---
\newglossaryentry{connectedvehicles}{%
    name={Connected Vehicles},%
    description={Fahrzeuge, die über drahtlose Kommunikation Daten mit Infrastruktur oder anderen Fahrzeugen austauschen können.}%
}

\newglossaryentry{autonomousvehicles}{%
    name={Autonome Fahrzeuge},%
    description={Fahrzeuge, die ohne menschliche Steuerung im Verkehr agieren und Entscheidungen selbstständig treffen.}%
}

\newglossaryentry{smartcity}{%
    name={Smart City},%
    description={Ein Konzept für Städte, die Informations- und Kommunikationstechnologien nutzen, um Effizienz, Nachhaltigkeit und Lebensqualität zu verbessern.}%
}

% --- Weitere Begriffe aus Einleitung & Methodik ---


\newglossaryentry{sumo-rl}{%
    name={SUMO-rl},%
    description={Ein Python-Framework, das SUMO mit Reinforcement-Learning-Bibliotheken verbindet (z.\,B. Stable-Baselines3, PyTorch) und Single- sowie Multi-Agent-Szenarien unterstützt.}%
}

\newglossaryentry{pettingzoo}{%
    name={PettingZoo},%
    description={Eine Python-Bibliothek für Multi-Agent-Reinforcement-Learning, die standardisierte Schnittstellen für Agentenumgebungen bietet und auch in sumo-rl genutzt wird.}%
}

\newglossaryentry{gymnasium}{%
    name={Gymnasium},%
    description={Eine Weiterentwicklung von OpenAI Gym, die standardisierte Schnittstellen für Reinforcement-Learning-Umgebungen bereitstellt.}%
}

\newglossaryentry{mdp}{%
    name={Markov Decision Process (MDP)},%
    description={Formale Beschreibung von Reinforcement-Learning-Problemen, bestehend aus Zuständen, Aktionen, Übergängen und Belohnungen.}%
}

\newglossaryentry{policygradient}{%
    name={Policy-Gradient},%
    description={Eine Klasse von Reinforcement-Learning-Algorithmen, die die Policy direkt optimieren, indem sie Gradienten der erwarteten Belohnung berechnen.}%
}

\newglossaryentry{overfitting}{%
    name={Überanpassung (Overfitting)},%
    description={Ein Phänomen beim maschinellen Lernen, bei dem ein Modell zu stark an die Trainingsdaten angepasst wird und dadurch auf neuen Daten schlecht generalisiert.}%
}

% --- Simulation & Netzaufbau ---

\newglossaryentry{netconvert}{%
    name={netconvert},%
    description={Ein SUMO-Tool zur Konvertierung von Straßendaten (z.\,B. aus OpenStreetMap) in SUMO-kompatible Netzwerke.}%
}

\newglossaryentry{netedit}{%
    name={netedit},%
    description={Ein grafisches Editor-Tool von SUMO, mit dem Straßennetze, Kreuzungen und Ampelsteuerungen manuell bearbeitet und validiert werden können.}%
}

\newglossaryentry{randomtrips}{%
    name={randomTrips},%
    description={Ein SUMO-Skript zur zufälligen Generierung von Fahrzeugfahrten im Simulationsnetz, nützlich für Testläufe und Validierung.}%
}

\newglossaryentry{duarouter}{%
    name={duarouter},%
    description={Ein SUMO-Tool zur Erzeugung konfliktfreier Fahrzeugrouten basierend auf Nachfrage- und Netzdefinitionen.}%
}

% --- Verkehrs- und Evaluationsmetriken ---

\newglossaryentry{waitingqueue}{%
    name={Warteschlange},%
    description={Eine Gruppe von Fahrzeugen, die an einer Ampel oder durch Verkehrsüberlastung gestoppt sind. In der Arbeit eine zentrale Messgröße für Staus.}%
}

\newglossaryentry{emissions}{%
    name={Emissionen},%
    description={Von Fahrzeugen ausgestoßene Schadstoffe wie CO2, NOx oder Feinstaub. In SUMO über HBEFA-Tabellen modelliert und als Optimierungsziel genutzt.}%
}

\newglossaryentry{reproducibility}{%
    name={Reproduzierbarkeit},%
    description={Die Fähigkeit, Simulationen und Ergebnisse mit denselben Parametern und Seeds identisch wiederholen zu können.}%
}

\newglossaryentry{checkpoint}{%
    name={Checkpoint},%
    description={Ein gespeicherter Zwischenstand im Training eines RL-Agenten, der für Fortsetzung, Vergleich oder Auswertung genutzt werden kann.}%
}

\newglossaryentry{logging}{%
    name={Logging},%
    description={Das systematische Aufzeichnen von Simulations- und Trainingsdaten, um Verlauf, Fehler und Leistungsmetriken nachzuvollziehen.}%
}

% --- Datengrundlage ---


\newglossaryentry{mobidatabw}{%
    name={MobiData BW},%
    description={Eine Datenplattform des Landes Baden-Württemberg, die offene Mobilitäts- und Verkehrsdaten bereitstellt.}%
}
\newglossaryentry{tensorboard}{%
    name={TensorBoard},%
    description={Visualisierungs- und Analysewerkzeug (u.\,a.\ für Stable-Baselines3-Logs) zur Darstellung von Trainingsmetriken wie Reward, Lernrate oder Loss.}%
}

\newglossaryentry{tensorflow}{%
    name={TensorFlow},%
    description={Eine Open-Source-Plattform für maschinelles Lernen, die von Google entwickelt wurde. TensorFlow bietet Werkzeuge für Deep Learning, neuronale Netze und Reinforcement Learning und wird häufig für Forschung und produktive Anwendungen eingesetzt.}%
}

\newglossaryentry{pytorch}{%
    name={PyTorch},%
    description={Eine Open-Source-Deep-Learning-Bibliothek, die von Meta (ehemals Facebook) entwickelt wurde. PyTorch ist besonders durch seine dynamische Rechen-Graph-Struktur und die einfache Integration in Python-Code beliebt und wird häufig für Reinforcement Learning verwendet.}%
}
\newglossaryentry{thread}{%
    name={Thread},%
    description={Ein leichtgewichtiger Ausführungsstrang innerhalb eines Prozesses. Mehrere Threads können parallel ablaufen und gemeinsam Ressourcen nutzen, wodurch parallele Berechnungen oder nebenläufige Abläufe in Programmen ermöglicht werden.}%
}


\newacronym{bast}{BASt}{Bundesanstalt für Straßenwesen}
\newacronym{LUBW}{LUBW}{Landesanstalt für Umwelt Baden-Württemberg}
\newacronym{ppo}{PPO}{Proximal Policy Optimization}
\newacronym{rl}{RL}{Reinforcement Learning}
\newacronym{cpu}{CPU}{Central Processing Unit}
\newacronym{gpu}{GPU}{Graphics Processing Unit}
\newacronym{dtv}{DTV}{Durchschnittlicher Tagesverkehr}
\newacronym{tls}{TLS}{Traffic Light System}
\newacronym{gui}{GUI}{Graphical User Interface}
\newacronym{xml}{XML}{Extensible Markup Language}
\newacronym{api}{API}{Application Programming Interface}
\newacronym{osmxml}{OSM-XML}{OpenStreetMap-XML-Datei}
\newacronym{hbefa}{HBEFA}{Handbook Emission Factors for Road Transport}
\newacronym{sv}{SV}{Schwerverkehr}
\newacronym{kfz}{KFZ}{Kraftfahrzeug}
\newacronym{pkw}{PKW}{Personenkraftwagen}
\newacronym{snfz}{sNfz}{Schwere Nutzfahrzeuge}
\newacronym{v2x}{V2X}{Vehicle-to-Everything Kommunikation}
\newacronym{a2c}{A2C}{Advantage Actor-Critic}
\newacronym{dqn}{DQN}{Deep Q-Network}
\newacronym{sac}{SAC}{Soft Actor-Critic}
\newacronym{ki}{KI}{künstlische Intelligenz}
\newacronym{tcp}{TCP}{Transmission Control Protocol}
\newacronym{sumo}{SUMO}{Simulation of Urban Mobility}
\newacronym{dlr}{dlr}{Deutsches Zentrum für Luft- und Raumfahrt}
\newacronym{traci}{TraCI}{Traffic Control Interface}
\newacronym{sb3}{SB3}{Stable-Baseline3}
\newacronym{rllib}{RLlib}{Reinforcement Learning Library, ein skalierbares RL-Framework basierend auf Ray}
\newacronym{josm}{JOSM}{Java OpenStreetMap Editor, ein Editor zur Bearbeitung von OpenStreetMap-Daten}
\newacronym{od}{OD-Matrix}{Origin--Destination-Matrix}%
\newacronym{SVZBW}{SVZ-BW}{Straßenverkehrszentrale Baden-Württemberg}%

\newglossaryentry{lowtrafficneighbourhoods}{%
    name={Low-Traffic-Neighbourhoods},%
    description={Ein städtebauliches Konzept, bei dem bestimmte Wohngebiete durch verkehrslenkende Maßnahmen wie Durchfahrtssperren oder Einbahnregelungen vom motorisierten Durchgangsverkehr entlastet werden. 
    Ziel ist es, die Lebensqualität in diesen Quartieren zu erhöhen, Sicherheit für Fußgänger und Radfahrer zu schaffen und die Umweltbelastung zu reduzieren.}%
}

\newglossaryentry{supersuit}{%
    name={SuperSuit},%
    description={Sammlung von Wrappern zur Anpassung von PettingZoo-Umgebungen an RL-Frameworks (z.\,B.\ Padding von Beobachtungen/Aktionen, Vektorisierung, Umwandlung in SB3-kompatible VecEnvs).}%
}

\newglossaryentry{vecnormalize}{%
    name={VecNormalize},%
    description={Wrapper aus Stable-Baselines3 zur Normalisierung von Beobachtungen und Rewards; verbessert Stabilität und Vergleichbarkeit zwischen Kreuzungen.}%
}

\newglossaryentry{vecmonitor}{%
    name={VecMonitor},%
    description={Wrapper aus Stable-Baselines3 für das Aufzeichnen von Episodenstatistiken (z.\,B.\ Rewards, Längen) zur späteren Auswertung.}%
}

\newglossaryentry{callback}{%
    name={Callback},%
    description={Erweiterungsmechanismus im Training (z.\,B.\ periodische Checkpoints, Metrik-Logging, Bestmodell-Speicherung), der zu bestimmten Zeitpunkten während des Lernprozesses ausgeführt wird.}%
}

\newglossaryentry{onehot}{%
    name={One-Hot-Kodierung},%
    description={Kodierung diskreter Kategorien als Binärvektoren mit genau einer Eins; verwendet zur Repräsentation der aktiven Ampelphase.}%
}

\newglossaryentry{observationspace}{%
    name={Observationsraum},%
    description={Menge der Merkmale/Zustände, die ein Agent wahrnimmt (z.\,B.\ Phasenstatus, Warteschlangenlänge, Dichte, Geschwindigkeiten).}%
}

\newglossaryentry{actionspace}{%
    name={Aktionsraum},%
    description={Menge der zulässigen Aktionen eines Agenten; bei Ampelsteuerung typischerweise Phasenwahl bzw.\ Fortsetzen/Wechseln einer Phase im festen Intervall.}%
}

\newglossaryentry{mae}{%
    name={Multi-Agent Environment (MAE)},%
    description={Umgebung, in der mehrere Agenten (z.\,B.\ je Lichtsignalanlage einer) gleichzeitig handeln; Interaktion und Training erfolgen parallel.}%
}

% Optional, falls du die SUMO-GUI separat nennen möchtest:
\newglossaryentry{sumogui}{%
    name={SUMO GUI (sumo-gui)},%
    description={Grafische Oberfläche von SUMO zur Visualisierung von Netz, Fahrzeugen, Ampelphasen und Simulationsergebnissen.}%
}

% Optional, sensorisch:
\newglossaryentry{laneareadetector}{%
    name={laneAreaDetector},%
    description={SUMO-Sensor zur Erfassung spurbezogener Größen (z.\,B.\ Dichte, Anzahl wartender Fahrzeuge), genutzt für Zustandsmerkmale im RL.}%
}

\newglossaryentry{visum}{%
    name={VISUM},%
    description={Eine kommerzielle Verkehrsplanungssoftware von PTV, deren Modelle für Netzimporte in SUMO genutzt werden können.}%
}


\newglossaryentry{optuna}{%
    name={Optuna},%
    description={Ein Framework zur Hyperparameter-Optimierung, das effiziente Suchstrategien wie Tree-structured Parzen Estimator (TPE) nutzt.}%
}

\newglossaryentry{raytune}{%
    name={Ray Tune},%
    description={Eine Python-Bibliothek für verteilte Hyperparameter-Optimierung und Experimentmanagement.}%
}

\newglossaryentry{curriculumlearning}{%
    name={Curriculum Learning},%
    description={Eine Trainingsstrategie im maschinellen Lernen, bei der Aufgaben in zunehmender Schwierigkeit präsentiert werden, um Stabilität und Effizienz zu erhöhen.}%
}

\newglossaryentry{multiobjective}{%
    name={Multi-Objective Learning},%
    description={Ein Ansatz, bei dem mehrere Zielgrößen gleichzeitig optimiert werden, wie Wartezeit und Emissionen.}%
}

\newglossaryentry{vecenv}{%
    name={VecEnv},%
    description={Schnittstelle in Stable-Baselines3 zur parallelen Ausführung mehrerer Umgebungen, um das Training effizienter zu machen.}%
}

\newglossaryentry{backlog}{%
    name={Backlog},%
    description={Anzahl der Fahrzeuge, die nicht mehr ins Netz einfahren konnten, weil es überlastet war. Dient als Metrik in der Evaluation.}%
}

\newglossaryentry{shapefiles}{%
    name={Shapefiles},%
    description={Ein von ESRI entwickeltes Geodatenformat (bestehend aus mehreren Dateien wie .shp, .shx, .dbf), das in Geoinformationssystemen (GIS) zur Speicherung und zum Austausch von Vektor-Geodaten wie Straßennetzen verwendet wird.}%
}

\newglossaryentry{parallelenv}{%
    name={parallel\_env},%
    description={Ein Wrapper aus der PettingZoo-Bibliothek, der Multi-Agent-Umgebungen in einer parallelen Schnittstelle bereitstellt und damit kompatibel mit Reinforcement-Learning-Frameworks wie Stable-Baselines3 ist.}%
}


\newglossaryentry{qlearning}{%
    name={Q-Learning},%
    description={Ein wertbasierter Reinforcement-Learning-Algorithmus, der eine sogenannte Q-Funktion erlernt. 
    Diese ordnet jedem Zustand-Aktions-Paar einen erwarteten kumulativen Reward zu und ermöglicht dem Agenten, durch Auswahl der Aktion mit dem höchsten Q-Wert eine optimale Strategie zu entwickeln.}%
}
