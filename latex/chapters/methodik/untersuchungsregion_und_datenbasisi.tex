
\subsection{Untersuchungsregion und Datenbasis}

\subsubsection{Auswahl der Untersuchungsregion}

Für die Anwendung und Evaluation der KI-basierten Verkehrssteuerung wurde ein Ausschnitt des innerstädtischen Straßennetzes von Karlsruhe gewählt. Die Auswahl fiel auf ein \hyperref[fig:zaehlstellenkarte]{Gebiet} rund um die Reinhold-Frank-Straße und das Mühlburger Tor, das durch hohe Verkehrsdichte, komplexe Knotenpunkte und mehrere signalgesteuerte Kreuzungen gekennzeichnet ist. Der gewählte Bereich liegt geografisch zwischen 49{,}006947\,\textdegree{}N und 49{,}015602\,\textdegree{}N sowie 8{,}380176\,\textdegree{}E und 8{,}403887\,\textdegree{}E und deckt mehrere stark frequentierte Hauptachsen ab.

Die Entscheidung für diese Region basiert auf folgenden Kriterien (siehe Kapitel~\ref{sec:datenquellen_und_modellgrundlage}):

\begin{itemize}
    \item \textbf{Hohe Verkehrsbedeutung:} Das Gebiet stellt einen wichtigen innerstädtischen Verkehrsraum dar.
    \item \textbf{Bekanntes Stauaufkommen:} Die Reinhold-Frank-Straße ist in der Stadtbevölkerung für regelmäßige Verkehrsstaus bekannt, insbesondere zu Stoßzeiten. \cite{Stau_in_Karlsruhe}
    \item \textbf{Verfügbarkeit realer Verkehrszähldaten:} Eine automatische Dauerzählstelle erhebt dort täglich Verkehrsdaten. Für den Zeitraum vom 01.01.2025 bis 20.6.2025 wurden durchschnittlich 21.300 Kfz/Tag erfasst. \cite{Verkehrszählungen_Baden-Württemberg,Dauerzählstellen_Ergebnisse}
    \item \textbf{Zusätzliche Zähldaten angrenzender Hauptverkehrsstraßen:} Zählstellen an der B10, B36, L605 und in Durlach liefern ergänzende Werte zur Plausibilisierung des Gesamtverkehrsflusses.
    \item \textbf{Vorhandensein mehrerer Ampelanlagen:} Im Netz befinden sich 17 signalgesteuerte Kreuzungen, geeignet für RL-gesteuerte Steuerungsexperimente (siehe Kapitel \ref{alg:find_valid_tls}).
    \item \textbf{Gute Abgrenzbarkeit:} Das Gebiet ist topologisch geschlossen und in SUMO sauber simulierbar.
    \item \textbf{Verfügbarkeit von Geodaten:} Die Region ist in OpenStreetMap detailliert kartiert. \cite{osm-export}
\end{itemize}

\subsubsection{Verfügbare Verkehrszähldaten}

Für die Kalibrierung und Validierung der Verkehrssimulation wurden verschiedene reale Zähldatenquellen aus dem Raum Karlsruhe herangezogen. Hauptquelle war dabei die offene Mobilitätsdatenplattform des Landes Baden-Württemberg \cite{mobidata_stunden}. Dort werden automatisiert erfasste Stundenwerte stationärer Dauerzählstellen veröffentlicht, die eine fein aufgelöste Analyse von Verkehrsverläufen ermöglichen.

Konkret wurden folgende Datensätze ausgewertet:

\begin{itemize}
    \item \textbf{Dauerzählstelle Reinhold-Frank-Straße:} Erfasst täglich die Anzahl der Kraftfahrzeuge (Kfz), aufgeschlüsselt nach Fahrzeugklassen (KFZ, PKW, sNfz). Für den Zeitraum 01.01.–20.06.2025 lag der durchschnittliche Tagesverkehr bei ca. 21.300 Kfz/Tag.

    \item \textbf{Historische Jahresmittelwerte:} Langzeitdatenreihen von 2008-2024 aus MobiData BW ermöglichen eine Kontextualisierung der aktuellen Verkehrsbelastung. \cite{mobidata_karte}

    \item \textbf{Zählstellen an äußeren Zufahrtsachsen:} Ergänzende Zähldaten aus dem Jahr 2023 an acht stark befahrenen Einfallstraßen (u.\,a. B10, B36, L605)\cite{mobidata_karte} liefern Anhaltspunkte zur Verkehrsstärke an den Netzrändern.
\end{itemize}

Die Kombination dieser Quellen ermöglicht eine robuste, datenbasierte Schätzung realistischer Flussverteilungen für die Simulation, sowohl zeitlich (z.\,B. Spitzenlasten) als auch räumlich (Zufahrtsverteilung).
