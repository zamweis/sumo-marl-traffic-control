\subsection{Aufbau des Simulationsmodells in SUMO}

\subsubsection{Netzgenerierung}

Zur Modellierung des realen Straßennetzes wurde ein Kartenausschnitt des Untersuchungsgebiets über die Exportfunktion von OpenStreetMap \cite{osm-export} heruntergeladen und anschließend mit JOSM \cite{josm} bereinigt. Der Ausschnitt umfasst die Reinhold-Frank-Straße sowie angrenzende Hauptverkehrsachsen im Bereich des Mühlburger Tors. Der bearbeitete Kartenausschnitt wurde mit dem SUMO-Werkzeug \texttt{netconvert} \cite{sumo-tools} in ein XML-Netzwerkformat überführt. Dabei kamen Optionen zur automatischen TLS-Erkennung und zur Verbesserung der Fahrstreifenzuordnung zum Einsatz (z.\,B. \texttt{--tls.guess-signals}, \texttt{--junctions.join}). \cite{sumo-doc}

\subsubsection{Erzeugung von Fahrzeugflüssen}

Für die Simulation wurden zwei Ansätze genutzt. Zunächst entstanden mit dem SUMO-Skript \texttt{randomTrips.py} \cite{randomTrips} einfache Testflüsse, die zur Validierung des Netzmodells dienten. Anschließend wurden auf Basis realer Verkehrszähldaten (siehe Kapitel \ref{sec:zaehlstellen-karlsruhe}) Flussprofile definiert, welche die beobachteten DTV-Werte proportional auf die äußeren Zufahrtskanten verteilten. Hauptachsen wie B10 oder B36 erhielten dabei ein höheres Gewicht. Die daraus erzeugten Routendateien wurden mit \texttt{duarouter} \cite{duarouter} zu konfliktfreien Fahrten verarbeitet. Unterschiedliche Szenarien (Morgen- und Abendspitzen, gleichmäßige Verteilung) erlaubten die Abbildung verschiedener Verkehrslagen.

\subsubsection{Identifikation relevanter Zufahrtskanten}

Die Auswahl geeigneter Zufahrtskanten basierte auf den äußeren Hauptverkehrsachsen (u.\,a. B10, B36, L605, Durlacher Allee). Hierzu wurde ein \hyperref[app:find_relevant_edges]{Python-Skript} eingesetzt, das Kanten mit einem \texttt{name}-Attribut automatisch durchsucht und auf relevante Straßennamen prüfte (z.\,B. \texttt{"B10"}, \texttt{"Reinhold-Frank-Straße"}). Die Ergebnisse wurden manuell überprüft und in \texttt{netedit} \cite{netedit} ergänzt. Die so extrahierten Kanten wurden je Verkehrsachse gruppiert und dienten als Grundlage für die segmentierte Trip-Erzeugung.

\subsubsection{Automatisierte Generierung von Trips}

Die DTV-Werte wurden auf die Zufahrtsgruppen skaliert und auf eine Simulationsdauer von 5000\,s verteilt. Ein eigens entwickeltes Python-Skript erzeugte daraus Fahrzeugeinträge (\texttt{<trip>}), die über die ermittelten Zufahrtskanten ins Netz eingespeist wurden. Die Trips wurden im XML-Format gespeichert und mit \texttt{duarouter} in vollständige, konfliktfreie Routen (\texttt{<route>}) überführt. Dabei wurde die Gesamtanzahl stündlich extrapoliert und stark belastete Achsen (z.\,B. B10, L605) mit entsprechend höherem Anteil berücksichtigt. Die erzeugten Flüsse wurden durch Detektor-Ausgaben (u.\,a. \texttt{laneAreaDetector}) \cite{sumo-doc} überprüft und bei Bedarf angepasst.

\subsubsection{Validierung des Netzmodells}

Vor dem Einsatz des Modells erfolgte eine mehrstufige Validierung:

\begin{itemize}
  \item \textbf{Netzprüfung:} Einsatz von \texttt{netconvert --check-lane-geometry} und \texttt{netcheck} zur Überprüfung der topologischen Konsistenz. \cite{netconvert}
  \item \textbf{Visuelle Kontrolle:} Inspektion kritischer Knoten in der SUMO-GUI und in \texttt{netedit}, um Fehler wie unverbundene Spuren oder falsche Richtungen zu erkennen. \cite{netedit}
  \item \textbf{TLS-Prüfung:} Alle 17 Lichtsignalanlagen wurden in \texttt{netedit} geöffnet. Phasenpläne, gesteuerte Verbindungen und Zustandslängen (\texttt{state}) wurden geprüft und bei Bedarf korrigiert.
\end{itemize}

Die manuelle Nachbearbeitung war zeitaufwändig, da fehlerhafte TLS nicht automatisch erkannt werden. In mehreren Fällen mussten Phasenpläne neu erstellt oder angepasst werden.

\subsubsection{Szenarien und Referenzsimulationen}

Zur Validierung der Verkehrsflüsse wurden unterschiedliche Szenarien umgesetzt:

\begin{itemize}
  \item \textbf{Morgenspitze:} Verstärkte Einträge an Süd- und Westzufahrten.
  \item \textbf{Abendliche Rückstaus:} Höhere Belastung der Ausfallstraßen und des Innenstadtrings.
  \item \textbf{Gleichmäßiger Tagesverlauf:} Homogene Einträge über den Tag verteilt.
  \item \textbf{Zentrumsfokus:} Stärkerer Verkehr über Reinhold-Frank-Straße und Mühlburger Tor.
\end{itemize}

Die Simulationen dienten der Überprüfung der Netzdurchlässigkeit und der Kapazitäten der Knotenpunkte.

\subsubsection{Signalsteuerung und Simulationsparameter}

Für das Reinforcement-Learning-Setup wurden die TLS so konfiguriert, dass sie in SUMO als „aktuiert“\footnote{„Aktuiert“ bedeutet in SUMO, dass die Signalanlagen nicht strikt einem festen Schaltplan folgen, sondern dynamisch mit Hilfe von Detektoren oder externen Eingaben reagieren können.Im Rahmen dieser Arbeit erfolgt die Steuerung ausschließlich über das \texttt{TraCI}-Interface.} \cite{sumo-doc} initialisiert und anschließend direkt durch das RL-Modul über \texttt{TraCI} gesteuert werden konnten \cite{sumo-rl_docs}.


Die Simulation wurde mit folgenden Parametern durchgeführt:

\begin{itemize}
  \item \textbf{Simulationszeitraum:} 5000 Sekunden
  \item \textbf{Zeitschritt (step-length):} 1,0 s
  \item \textbf{Routengenerierung:} deterministisch mit fixer seed zur Reproduzierbarkeit
  \item \textbf{Simulationstyp:} \texttt{meso}-Modus für Training, \texttt{default}-Modus für Evaluation
  \item \textbf{Verkehrsverteilung:} definiert über \texttt{flows.xml} und über Randkanten eingeleitet
\end{itemize}