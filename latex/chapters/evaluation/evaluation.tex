\section{Evaluation und Ergebnisse}
\label{sec:validation}

In diesem Kapitel werden die Ergebnisse der Evaluationsläufe präsentiert.
Alle Modelle, bestehend aus 4 Trainingsseeds pro Reward-Variante, sowie die beiden Baselines, Fixed-Time und Actuated. Diese
wurden in allen vier Szenarien \texttt{morning\_peak}, \texttt{evening\_peak}, \texttt{uniform} und \texttt{random\_heavy}
jeweils über zehn Episoden evaluiert.

\texttt{Anmerkung:} Die Resultate setzten sich aus Mittelwerten pro Kombination aus \emph{Methode} und \emph{Szenario} zusammen. Die Baseline Actuated hat in allen Evaluierungen verhältnismäßig schlecht abgeschlossen und wurde größten teils separat in Diagrammen dargestellt um sicherzustellen, dass die deutlich besser performende Baseline TimeFixed visuell deutlich mit den Modellen vergleichbar bleibt. In SUMO werden Fahrzeuge die sich mit einer Geschwindigkeit kleiner als 0.1 m/s bewegen als gestoppt eingestuft. \cite{sumo-doc}

\subsection{Reward: Diff-Waiting-Time}
Die erste Gruppe von RL-Modellen wurde mit einem Reward trainiert, der die Differenz der Wartezeiten minimiert.

\subsubsection{Mittlere Wartezeiten}
\label{sec:diff-waiting-time-wartezeit}
\begin{figure}[H]
    \centering
    \begin{tikzpicture}
        \begin{axis}[
                ybar,
                bar width=0.25cm,
                width=12cm,
                height=8cm,
                enlarge x limits=0.15,
                ylabel={Mittlere Wartezeit [s]},
                symbolic x coords={evening_peak,morning_peak,random_heavy,uniform},
                xtick=data,
                xticklabels={\text{evening\_peak},\text{morning\_peak},\text{random\_heavy},\text{uniform}},
                x tick label style={rotate=45,anchor=east},
                legend style={at={(1.05,0.5)}, anchor=west},
                ymajorgrids=true,
                grid style=dashed,
                every axis plot post/.append style={thick, fill=.!50}
            ]

            % Baseline FixedTime
            \addplot+[color1, error bars/.cd,
                y dir=minus, y explicit,
                error bar style={line width=1pt, black}] table [
                    x=scenario, y=system_mean_waiting_time_mean, col sep=comma, y error=system_mean_waiting_time_std
                ] {chapters/evaluation/results/diff-waiting-time/Baseline_FixedTime.csv};
            \addlegendentry{Baseline FixedTime}

            % RL Modell 1
            \addplot+[color2, error bars/.cd,
                y dir=minus, y explicit,
                error bar style={line width=1pt, black}] table [
                    x=scenario, y=system_mean_waiting_time_mean, col sep=comma, y error=system_mean_waiting_time_std
                ] {chapters/evaluation/results/diff-waiting-time/ppo_sumo_456_2025-08-18_01-18-32_456.csv};
            \addlegendentry{Model 1}


            % RL Modell 2
            \addplot+[color3, error bars/.cd,
                y dir=minus, y explicit,
                error bar style={line width=1pt, black}] table [
                    x=scenario, y=system_mean_waiting_time_mean, col sep=comma, y error=system_mean_waiting_time_std
                ] {chapters/evaluation/results/diff-waiting-time/ppo_sumo_13755_2025-08-18_04-48-47_13755.csv};
            \addlegendentry{Model 2}

            % RL Modell 3
            \addplot+[color4, error bars/.cd,
                y dir=minus, y explicit,
                error bar style={line width=1pt, black}] table [
                    x=scenario, y=system_mean_waiting_time_mean, col sep=comma, y error=system_mean_waiting_time_std
                ] {chapters/evaluation/results/diff-waiting-time/ppo_sumo_143534_2025-08-17_23-30-08_143534.csv};
            \addlegendentry{Model 3}

            % RL Modell 4
            \addplot+[color5, error bars/.cd,
                y dir=minus, y explicit,
                error bar style={line width=1pt, black}] table [
                    x=scenario, y=system_mean_waiting_time_mean, col sep=comma, y error=system_mean_waiting_time_std
                ] {chapters/evaluation/results/diff-waiting-time/ppo_sumo_635768_2025-08-18_03-03-29_635768.csv};
            \addlegendentry{Model 4}
        \end{axis}
    \end{tikzpicture}
    \caption{Mittlere Wartezeiten}
    \label{fig:diff-waiting-time-wartezeit}
\end{figure}



\begin{figure}[H]
    \centering
    \begin{tikzpicture}
        \begin{axis}[
                ybar,
                bar width=0.25cm,
                width=12cm,
                height=8cm,
                enlarge x limits=0.15,
                ylabel={Mittlere Wartezeit [s]},
                symbolic x coords={evening_peak,morning_peak,random_heavy,uniform},
                xtick=data,
                xticklabels={\text{evening\_peak},\text{morning\_peak},\text{random\_heavy},\text{uniform}},
                x tick label style={rotate=45,anchor=east},
                legend style={at={(1.05,0.5)}, anchor=west},
                ymajorgrids=true,
                grid style=dashed,
                every axis plot post/.append style={thick, fill=.!50}
            ]

            % Baseline FixedTime
            \addplot+[color1, error bars/.cd,
                y dir=minus, y explicit,
                error bar style={line width=1pt, black}] table [
                    x=scenario, y=system_mean_waiting_time_mean, col sep=comma, y error=system_mean_waiting_time_std
                ] {chapters/evaluation/results/diff-waiting-time/Baseline_FixedTime.csv};
            \addlegendentry{Baseline FixedTime}

            % Baseline FixedTime
            \addplot+[color6, error bars/.cd,
                y dir=minus, y explicit,
                error bar style={line width=1pt, black}] table [
                    x=scenario, y=system_mean_waiting_time_mean, col sep=comma, y error=system_mean_waiting_time_std
                ] {chapters/evaluation/results/diff-waiting-time/Baseline_Actuated.csv};
            \addlegendentry{Baseline Actuated}
        \end{axis}
    \end{tikzpicture}
    \caption{Mittlere Wartezeiten}
    \label{fig:diff-waiting-time-wartezeit2}
\end{figure}

Die mittlere Wartezeit für die Diff-Waiting-Time-Rewardfunktion zeigt erneut deutliche Unterschiede zwischen den Baselines und den trainierten Modellen.

Die Fixed-Time-Baseline erreicht im morning\_peak 3.74 s, im evening\_peak 4.20 s, im random\_heavy 4.40 s und im uniform-Szenario 3.90 s. Diese Werte bilden eine stabile Referenz. Die Actuated-Baseline fällt dagegen erneut durch extrem hohe Wartezeiten auf: 964 s, 965 s, 1030 s und 935 s in den vier Szenarien.

Unter den trainierten Modellen zeigen sich differenzierte Muster. Modell 1 erreicht sehr geringe Werte mit 0.20 s, 4.27 s, 8.90 s und 0.21 s. Modell 2 weist im morning\_peak mit 7.9 s sowie im random\_heavy mit 82 s deutlich höhere Wartezeiten auf, während die Werte im evening\_peak (2.1 s) und im uniform (1.5 s) niedrig bleiben. Modell 3 zeigt ebenfalls inkonsistente Ergebnisse: 2.5 s im morning\_peak, aber 95 s im evening\_peak und 77 s im random\_heavy, während das uniform-Szenario mit 0.4 s vergleichsweise gut abschneidet. Modell 4 erzielt insgesamt die besten Resultate mit 0.11 s, 0.18 s, 33 s und 0.10 s, wenngleich auch hier im random\_heavy ein Anstieg erkennbar ist.

Die erhöhten Mittelwerte in einzelnen Szenarien, insbesondere bei Modell 2 und Modell 3, gehen jeweils mit einer hohen Standardabweichung einher. Dies weist auf eine deutliche Instabilität zwischen den Runs hin und deutet darauf, dass die Modelle zwar vereinzelt effiziente Strategien entwickeln konnten, diese jedoch nicht konsistent reproduziert werden.

\subsubsection{Anzahl stoppender Fahrzeuge}
In sumo werden Fahrzeuge die sich mit einer Geschwindigkeit kleiner als 0.1 m/s bewegt.
\begin{figure}[H]
    \centering
    \begin{tikzpicture}
        \begin{axis}[
                ybar,
                bar width=0.25cm,
                width=12cm,
                height=8cm,
                enlarge x limits=0.15,
                ylabel={Anzahl stoppender Fahrzeuge},
                symbolic x coords={evening_peak,morning_peak,random_heavy,uniform},
                xtick=data,
                xticklabels={\text{evening\_peak},\text{morning\_peak},\text{random\_heavy},\text{uniform}},
                x tick label style={rotate=45,anchor=east},
                legend style={at={(1.05,0.5)}, anchor=west},
                ymajorgrids=true,
                grid style=dashed,
                every axis plot post/.append style={thick, fill=.!50}
            ]

            % Baseline FixedTime
            \addplot+[color1, error bars/.cd,
                y dir=minus, y explicit,
                error bar style={line width=1pt, black}] table [
                    x=scenario, y=system_total_stopped_mean, col sep=comma, y error=system_total_stopped_std
                ] {chapters/evaluation/results/diff-waiting-time/Baseline_FixedTime.csv};
            \addlegendentry{Baseline FixedTime}

            % RL Modell 1
            \addplot+[color2, error bars/.cd,
                y dir=minus, y explicit,
                error bar style={line width=1pt, black}] table [
                    x=scenario, y=system_total_stopped_mean, col sep=comma, y error=system_total_stopped_std
                ] {chapters/evaluation/results/diff-waiting-time/ppo_sumo_456_2025-08-18_01-18-32_456.csv};
            \addlegendentry{Model 1}


            % RL Modell 2
            \addplot+[color3, error bars/.cd,
                y dir=minus, y explicit,
                error bar style={line width=1pt, black}] table [
                    x=scenario, y=system_total_stopped_mean, col sep=comma, y error=system_total_stopped_std
                ] {chapters/evaluation/results/diff-waiting-time/ppo_sumo_13755_2025-08-18_04-48-47_13755.csv};
            \addlegendentry{Model 2}

            % RL Modell 3
            \addplot+[color4, error bars/.cd,
                y dir=minus, y explicit,
                error bar style={line width=1pt, black}] table [
                    x=scenario, y=system_total_stopped_mean, col sep=comma, y error=system_total_stopped_std
                ] {chapters/evaluation/results/diff-waiting-time/ppo_sumo_143534_2025-08-17_23-30-08_143534.csv};
            \addlegendentry{Model 3}

            % RL Modell 4
            \addplot+[color5, error bars/.cd,
                y dir=minus, y explicit,
                error bar style={line width=1pt, black}] table [
                    x=scenario, y=system_total_stopped_mean, col sep=comma, y error=system_total_stopped_std
                ] {chapters/evaluation/results/diff-waiting-time/ppo_sumo_635768_2025-08-18_03-03-29_635768.csv};
            \addlegendentry{Model 4}
        \end{axis}
    \end{tikzpicture}
    \caption{Anzahl stoppender Fahrzeuge}
    \label{fig:diff-waiting-time-stopped}
\end{figure}



\begin{figure}[H]
    \centering
    \begin{tikzpicture}
        \begin{axis}[
                ybar,
                bar width=0.25cm,
                width=12cm,
                height=8cm,
                enlarge x limits=0.15,
                ylabel={Anzahl stoppender Fahrzeuge},
                symbolic x coords={evening_peak,morning_peak,random_heavy,uniform},
                xtick=data,
                xticklabels={\text{evening\_peak},\text{morning\_peak},\text{random\_heavy},\text{uniform}},
                x tick label style={rotate=45,anchor=east},
                legend style={at={(1.05,0.5)}, anchor=west},
                ymajorgrids=true,
                grid style=dashed,
                every axis plot post/.append style={thick, fill=.!50}
            ]

            % Baseline FixedTime
            \addplot+[color1, error bars/.cd,
                y dir=minus, y explicit,
                error bar style={line width=1pt, black}] table [
                    x=scenario, y=system_total_stopped_mean, col sep=comma, y error=system_total_stopped_std
                ] {chapters/evaluation/results/diff-waiting-time/Baseline_FixedTime.csv};
            \addlegendentry{Baseline FixedTime}

            % Baseline FixedTime
            \addplot+[color6, error bars/.cd,
                y dir=minus, y explicit,
                error bar style={line width=1pt, black}] table [
                    x=scenario, y=system_total_stopped_mean, col sep=comma, y error=system_total_stopped_std
                ] {chapters/evaluation/results/diff-waiting-time/Baseline_Actuated.csv};
            \addlegendentry{Baseline Actuated}
        \end{axis}
    \end{tikzpicture}
    \caption{Anzahl stoppender Fahrzeuge}
    \label{fig:diff-waiting-time-stopped2}
\end{figure}

Die Ergebnisse zur Anzahl stoppender Fahrzeuge zeigen deutliche Unterschiede zwischen den Baselines und den trainierten Modellen.

Die Fixed-Time-Baseline erreicht im morning\_peak durchschnittlich 12 Fahrzeuge, im evening\_peak 14 Fahrzeuge, im random\_heavy 27 Fahrzeuge und im uniform-Szenario 10 Fahrzeuge. Diese Werte bilden eine stabile und vergleichsweise effiziente Referenz.

Die Actuated-Baseline zeigt dagegen eine massiv erhöhte Zahl an Stopps. So ergeben sich im morning\_peak 539 Fahrzeuge, im evening\_peak 566 Fahrzeuge, im random\_heavy 883 Fahrzeuge und im uniform 442 Fahrzeuge. Damit bestätigt sich auch in dieser Metrik die unzureichende Leistungsfähigkeit der Actuated-Steuerung.

Die trainierten Modelle liefern deutlich geringere Werte. Modell 1 erreicht 1.8, 4.1, 6.6 und 1.5 Fahrzeuge in den vier Szenarien. Modell 2 erzielt mit 2.12, 2.1, 17 und 1.2 Fahrzeugen ebenfalls niedrige Werte, allerdings mit einem deutlichen Anstieg im random\_heavy-Szenario. Modell 3 weist Werte von 2.64, 10, 16 und 1.6 Fahrzeugen auf, wobei insbesondere im evening\_peak und im random\_heavy eine Verschlechterung gegenüber den übrigen Szenarien erkennbar ist. Modell 4 erzielt mit 1.4, 1.7, 9.9 und 1.06 Fahrzeugen die insgesamt besten Resultate und bleibt in allen Szenarien unterhalb der Fixed-Time-Baseline.

Die erhöhten Werte bei Modell 2 und Modell 3 im random\_heavy-Szenario gehen mit einer hohen Standardabweichung einher, was auf starke Schwankungen zwischen den Runs hindeutet. Dies verdeutlicht, dass die Modelle zwar in der Lage sind, den Verkehrsfluss erheblich zu verbessern, ihre Robustheit unter unregelmäßigen Verkehrslasten jedoch eingeschränkt bleibt.

\subsubsection{Anzahl ankommender Fahrzeuge}
\label{sec:diff-waiting-time-ankommend}

\begin{figure}[H]
    \centering
    \begin{tikzpicture}
        \begin{axis}[
                ybar,
                bar width=0.25cm,
                width=12cm,
                height=8cm,
                enlarge x limits=0.15,
                ylabel={Anzahl ankommender Fahrzeuge},
                symbolic x coords={evening_peak,morning_peak,random_heavy,uniform},
                xtick=data,
                xticklabels={\text{evening\_peak},\text{morning\_peak},\text{random\_heavy},\text{uniform}},
                x tick label style={rotate=45,anchor=east},
                legend style={at={(1.05,0.5)}, anchor=west},
                ymajorgrids=true,
                grid style=dashed,
                every axis plot post/.append style={thick, fill=.!50}
            ]

            % Baseline FixedTime
            \addplot+[color1, error bars/.cd,
                y dir=minus, y explicit,
                error bar style={line width=1pt, black}] table [
                    x=scenario, y=system_total_arrived_mean, col sep=comma, y error=system_total_arrived_std
                ] {chapters/evaluation/results/diff-waiting-time/Baseline_FixedTime.csv};
            \addlegendentry{Baseline FixedTime}

            % Baseline FixedTime
            \addplot+[color6, error bars/.cd,
                y dir=minus, y explicit,
                error bar style={line width=1pt, black}] table [
                    x=scenario, y=system_total_arrived_mean, col sep=comma, y error=system_total_arrived_std
                ] {chapters/evaluation/results/diff-waiting-time/Baseline_Actuated.csv};
            \addlegendentry{Baseline Actuated}

            % RL Modell 1
            \addplot+[color2, error bars/.cd,
                y dir=minus, y explicit,
                error bar style={line width=1pt, black}] table [
                    x=scenario, y=system_total_arrived_mean, col sep=comma, y error=system_total_arrived_std
                ] {chapters/evaluation/results/diff-waiting-time/ppo_sumo_456_2025-08-18_01-18-32_456.csv};
            \addlegendentry{Model 1}

            % RL Modell 2
            \addplot+[color3, error bars/.cd,
                y dir=minus, y explicit,
                error bar style={line width=1pt, black}] table [
                    x=scenario, y=system_total_arrived_mean, col sep=comma, y error=system_total_arrived_std
                ] {chapters/evaluation/results/diff-waiting-time/ppo_sumo_13755_2025-08-18_04-48-47_13755.csv};
            \addlegendentry{Model 2}

            % RL Modell 3
            \addplot+[color4, error bars/.cd,
                y dir=minus, y explicit,
                error bar style={line width=1pt, black}] table [
                    x=scenario, y=system_total_arrived_mean, col sep=comma, y error=system_total_arrived_std
                ] {chapters/evaluation/results/diff-waiting-time/ppo_sumo_143534_2025-08-17_23-30-08_143534.csv};
            \addlegendentry{Model 3}

            % RL Modell 4
            \addplot+[color5, error bars/.cd,
                y dir=minus, y explicit,
                error bar style={line width=1pt, black}] table [
                    x=scenario, y=system_total_arrived_mean, col sep=comma, y error=system_total_arrived_std
                ] {chapters/evaluation/results/diff-waiting-time/ppo_sumo_635768_2025-08-18_03-03-29_635768.csv};
            \addlegendentry{Model 4}
        \end{axis}
    \end{tikzpicture}
    \caption{Anzahl ankommender Fahrzeuge}
    \label{fig:diff-waiting-time-arrived}
\end{figure}

Hinsichtlich der Anzahl ankommender Fahrzeuge unterscheiden sich die RL-Modelle kaum von der Fixed-Time-Baseline. Alle Modelle erreichen nahezu identische Werte, was darauf hinweist, dass die Steuerungsstrategien trotz der reduzierten Warte- und Stoppzeiten keine signifikanten Auswirkungen auf die Durchsatzkapazität des Netzes haben. Einzig die Actuated-Baseline zeigt hier, wie auch in den anderen Metriken, ein deutlich schlechteres Abschneiden.



\subsubsection{Durchschnitt fahrender Fahrzeuge}
\label{sec:diff-waiting-time-fahrende}

\begin{figure}[H]
    \centering
    \begin{tikzpicture}
        \begin{axis}[
                ybar,
                bar width=0.25cm,
                width=12cm,
                height=8cm,
                enlarge x limits=0.15,
                ylabel={Durchschnitt fahrender Fahrzeuge},
                symbolic x coords={evening_peak,morning_peak,random_heavy,uniform},
                xtick=data,
                xticklabels={\text{evening\_peak},\text{morning\_peak},\text{random\_heavy},\text{uniform}},
                x tick label style={rotate=45,anchor=east},
                legend style={at={(1.05,0.5)}, anchor=west},
                ymajorgrids=true,
                grid style=dashed,
                every axis plot post/.append style={thick, fill=.!50}
            ]

            % Baseline FixedTime
            \addplot+[color1, error bars/.cd,
                y dir=minus, y explicit,
                error bar style={line width=1pt, black}] table [
                    x=scenario, y=system_total_running_mean, col sep=comma, y error=system_total_running_std
                ] {chapters/evaluation/results/diff-waiting-time/Baseline_FixedTime.csv};
            \addlegendentry{Baseline FixedTime}

            % Baseline FixedTime
            \addplot+[color6, error bars/.cd,
                y dir=minus, y explicit,
                error bar style={line width=1pt, black}] table [
                    x=scenario, y=system_total_running_mean, col sep=comma, y error=system_total_running_std
                ] {chapters/evaluation/results/diff-waiting-time/Baseline_Actuated.csv};
            \addlegendentry{Baseline Actuated}

            % RL Modell 1
            \addplot+[color2, error bars/.cd,
                y dir=minus, y explicit,
                error bar style={line width=1pt, black}] table [
                    x=scenario, y=system_total_running_mean, col sep=comma, y error=system_total_running_std
                ] {chapters/evaluation/results/diff-waiting-time/ppo_sumo_456_2025-08-18_01-18-32_456.csv};
            \addlegendentry{Model 1}

            % RL Modell 2
            \addplot+[color3, error bars/.cd,
                y dir=minus, y explicit,
                error bar style={line width=1pt, black}] table [
                    x=scenario, y=system_total_running_mean, col sep=comma, y error=system_total_running_std
                ] {chapters/evaluation/results/diff-waiting-time/ppo_sumo_13755_2025-08-18_04-48-47_13755.csv};
            \addlegendentry{Model 2}

            % RL Modell 3
            \addplot+[color4, error bars/.cd,
                y dir=minus, y explicit,
                error bar style={line width=1pt, black}] table [
                    x=scenario, y=system_total_running_mean, col sep=comma, y error=system_total_running_std
                ] {chapters/evaluation/results/diff-waiting-time/ppo_sumo_143534_2025-08-17_23-30-08_143534.csv};
            \addlegendentry{Model 3}

            % RL Modell 4
            \addplot+[color5, error bars/.cd,
                y dir=minus, y explicit,
                error bar style={line width=1pt, black}] table [
                    x=scenario, y=system_total_running_mean, col sep=comma, y error=system_total_running_std
                ] {chapters/evaluation/results/diff-waiting-time/ppo_sumo_635768_2025-08-18_03-03-29_635768.csv};
            \addlegendentry{Model 4}
        \end{axis}
    \end{tikzpicture}
    \caption{Durchschnitt fahrender Fahrzeuge}
    \label{fig:diff-waiting-time-running}
\end{figure}

Die Auswertung des Durchschnitts fahrender Fahrzeuge (weniger ist besser) zeigt deutliche Unterschiede zwischen den Baselines und den trainierten Modellen.

Die Fixed-Time-Baseline erreicht im morning\_peak durchschnittlich 63 Fahrzeuge, im evening\_peak 69 Fahrzeuge, im random\_heavy 121 Fahrzeuge und im uniform-Szenario 52 Fahrzeuge. Diese Werte bilden eine stabile Referenz.

Die Actuated-Baseline weist deutlich höhere Werte auf, mit 553 Fahrzeugen im morning\_peak, 481 Fahrzeugen im evening\_peak, 900 Fahrzeugen im random\_heavy und 455 Fahrzeugen im uniform-Szenario. Damit bestätigt sich das schwache Abschneiden dieser Steuerung auch in dieser Metrik.

Die trainierten Modelle erreichen insgesamt Werte, die sehr nahe an der Fixed-Time-Baseline liegen. Modell 1 erzielt 53, 58, 98 und 44 Fahrzeuge. Modell 2 liegt bei 53, 55, 108 und 43 Fahrzeugen. Modell 3 weist 54, 64, 108 und 44 Fahrzeuge auf, während Modell 4 mit 52, 55, 101 und 43 Fahrzeugen die niedrigsten Werte liefert.

Insgesamt zeigen die Modelle konsistent bessere oder vergleichbare Ergebnisse zur Fixed-Time-Baseline. Auffällig ist, dass im random\_heavy-Szenario die Werte von Modell 2 und Modell 3 leicht über der Referenz liegen, was mit einer erhöhten Standardabweichung einhergeht und auf Instabilität in komplexen Verkehrslagen hinweist.

\subsubsection{Durchschnittsgeschwindigkeiten}
\label{sec:diff-waiting-time-geschwindigkeiten}

\begin{figure}[H]
    \centering
    \begin{tikzpicture}
        \begin{axis}[
                ybar,
                bar width=0.25cm,
                width=12cm,
                height=8cm,
                enlarge x limits=0.15,
                ylabel={Durchschnittsgeschwindigkeit [m/s]} ,
                symbolic x coords={evening_peak,morning_peak,random_heavy,uniform},
                xtick=data,
                xticklabels={\text{evening\_peak},\text{morning\_peak},\text{random\_heavy},\text{uniform}},
                x tick label style={rotate=45,anchor=east},
                legend style={at={(1.05,0.5)}, anchor=west},
                ymajorgrids=true,
                grid style=dashed,
                every axis plot post/.append style={thick, fill=.!50}
            ]

            % Baseline FixedTime
            \addplot+[color1, error bars/.cd,
                y dir=minus, y explicit,
                error bar style={line width=1pt, black}] table [
                    x=scenario, y=system_mean_speed_mean, col sep=comma, y error=system_mean_speed_std
                ] {chapters/evaluation/results/diff-waiting-time/Baseline_FixedTime.csv};
            \addlegendentry{Baseline FixedTime}

            % Baseline FixedTime
            \addplot+[color6, error bars/.cd,
                y dir=minus, y explicit,
                error bar style={line width=1pt, black}] table [
                    x=scenario, y=system_mean_speed_mean, col sep=comma, y error=system_mean_speed_std
                ] {chapters/evaluation/results/diff-waiting-time/Baseline_Actuated.csv};
            \addlegendentry{Baseline Actuated}

            % RL Modell 1
            \addplot+[color2, error bars/.cd,
                y dir=minus, y explicit,
                error bar style={line width=1pt, black}] table [
                    x=scenario, y=system_mean_speed_mean, col sep=comma, y error=system_mean_speed_std
                ] {chapters/evaluation/results/diff-waiting-time/ppo_sumo_456_2025-08-18_01-18-32_456.csv};
            \addlegendentry{Model 1}

            % RL Modell 2
            \addplot+[color3, error bars/.cd,
                y dir=minus, y explicit,
                error bar style={line width=1pt, black}] table [
                    x=scenario, y=system_mean_speed_mean, col sep=comma, y error=system_mean_speed_std
                ] {chapters/evaluation/results/diff-waiting-time/ppo_sumo_13755_2025-08-18_04-48-47_13755.csv};
            \addlegendentry{Model 2}

            % RL Modell 3
            \addplot+[color4, error bars/.cd,
                y dir=minus, y explicit,
                error bar style={line width=1pt, black}] table [
                    x=scenario, y=system_mean_speed_mean, col sep=comma, y error=system_mean_speed_std
                ] {chapters/evaluation/results/diff-waiting-time/ppo_sumo_143534_2025-08-17_23-30-08_143534.csv};
            \addlegendentry{Model 3}

            % RL Modell 4
            \addplot+[color5, error bars/.cd,
                y dir=minus, y explicit,
                error bar style={line width=1pt, black}] table [
                    x=scenario, y=system_mean_speed_mean, col sep=comma, y error=system_mean_speed_std
                ] {chapters/evaluation/results/diff-waiting-time/ppo_sumo_635768_2025-08-18_03-03-29_635768.csv};
            \addlegendentry{Model 4}
        \end{axis}
    \end{tikzpicture}
    \caption{Durchschnittsgeschwindigkeiten}
    \label{fig:diff-waiting-time-speed}
\end{figure}

Die Analyse der Durchschnittsgeschwindigkeiten verdeutlicht klare Unterschiede zwischen den Baselines und den trainierten Modellen.

Die Fixed-Time-Baseline erreicht im morning\_peak durchschnittlich 5.9 m/s, im evening\_peak 5.7 m/s, im random\_heavy 5.4 m/s sowie im uniform-Szenario 5.9 m/s. Damit liefert sie konsistente, aber nicht optimale Werte.

Die Actuated-Baseline weist durchgehend extrem niedrige Geschwindigkeiten auf: 0.6, 0.5, 0.5 und 0.7 m/s in den vier Szenarien. Diese Werte liegen um eine Größenordnung unterhalb der Fixed-Time-Baseline und bestätigen die unzureichende Leistungsfähigkeit dieser Steuerung.

Die trainierten Modelle übertreffen die Fixed-Time-Baseline deutlich. Modell 1 erreicht 7.0, 6.8, 6.7 und 7.1 m/s. Modell 2 erzielt sehr stabile Werte mit 7.1, 7.1, 6.2 und 7.2 m/s. Modell 3 liegt mit 6.9, 6.2, 6.1 und 7.1 m/s insgesamt ebenfalls über der Fixed-Time-Baseline, zeigt jedoch im evening\_peak und random\_heavy leicht reduzierte Ergebnisse. Modell 4 liefert mit 7.1, 7.1, 6.5 und 7.2 m/s die besten Resultate, insbesondere durch die hohe Stabilität über alle Szenarien hinweg.

Auffällig ist, dass die Modelle 2 und 4 durchgehend eine nahezu konstante Verbesserung gegenüber der Fixed-Time-Baseline erzielen, während Modell 3 in den Szenarien evening\_peak und random\_heavy leichten Performanceverlust zeigt. In diesen Fällen geht die Verschlechterung mit einer erhöhten Standardabweichung einher, was auf eine weniger stabile Performanz hindeutet.

\subsubsection{Anzahl teleportierender Fahrzeuge}
\label{sec:diff-waiting-time-teleport}

\begin{figure}[H]
    \centering
    \begin{tikzpicture}
        \begin{axis}[
                ybar,
                bar width=0.25cm,
                width=12cm,
                height=5cm,
                enlarge x limits=0.15,
                ylabel={Anzahl teleportierender Fahrzeuge},
                symbolic x coords={evening_peak,morning_peak,random_heavy,uniform},
                xtick=data,
                xticklabels={\text{evening\_peak},\text{morning\_peak},\text{random\_heavy},\text{uniform}},
                x tick label style={rotate=45,anchor=east},
                legend style={at={(1.05,0.5)}, anchor=west},
                ymajorgrids=true,
                grid style=dashed,
                every axis plot post/.append style={thick, fill=.!50}
            ]

            % Baseline FixedTime
            \addplot+[color1, error bars/.cd,
                y dir=minus, y explicit,
                error bar style={line width=1pt, black}] table [
                    x=scenario, y=system_total_teleported_mean, col sep=comma
                ] {chapters/evaluation/results/diff-waiting-time/Baseline_FixedTime.csv};
            \addlegendentry{Baseline FixedTime}

            % Baseline FixedTime
            \addplot+[color6, error bars/.cd,
                y dir=minus, y explicit,
                error bar style={line width=1pt, black}] table [
                    x=scenario, y=system_total_teleported_mean, col sep=comma
                ] {chapters/evaluation/results/diff-waiting-time/Baseline_Actuated.csv};
            \addlegendentry{Baseline Actuated}

            % RL Modell 1
            \addplot+[color2, error bars/.cd,
                y dir=minus, y explicit,
                error bar style={line width=1pt, black}] table [
                    x=scenario, y=system_total_teleported_mean, col sep=comma
                ] {chapters/evaluation/results/diff-waiting-time/ppo_sumo_456_2025-08-18_01-18-32_456.csv};
            \addlegendentry{Model 1}

            % RL Modell 2
            \addplot+[color3, error bars/.cd,
                y dir=minus, y explicit,
                error bar style={line width=1pt, black}] table [
                    x=scenario, y=system_total_teleported_mean, col sep=comma
                ] {chapters/evaluation/results/diff-waiting-time/ppo_sumo_13755_2025-08-18_04-48-47_13755.csv};
            \addlegendentry{Model 2}

            % RL Modell 3
            \addplot+[color4, error bars/.cd,
                y dir=minus, y explicit,
                error bar style={line width=1pt, black}] table [
                    x=scenario, y=system_total_teleported_mean, col sep=comma
                ] {chapters/evaluation/results/diff-waiting-time/ppo_sumo_143534_2025-08-17_23-30-08_143534.csv};
            \addlegendentry{Model 3}

            % RL Modell 4
            \addplot+[color5, error bars/.cd,
                y dir=minus, y explicit,
                error bar style={line width=1pt, black}] table [
                    x=scenario, y=system_total_teleported_mean, col sep=comma
                ] {chapters/evaluation/results/diff-waiting-time/ppo_sumo_635768_2025-08-18_03-03-29_635768.csv};
            \addlegendentry{Model 4}
        \end{axis}
    \end{tikzpicture}
    \caption{Anzahl teleportierender Fahrzeuge}
    \label{fig:diff-waiting-time-teleports}
\end{figure}

Ein Sonderfall ergibt sich bei der Metrik der Teleportationen: Während in fast allen Szenarien keine Teleportationen auftraten, kam es in einzelnen Episoden zu vereinzelten Fällen. Konkret traten bei Modell 3 und Modell 1 im Szenario random\_heavy sowie bei Modell 1 im Szenario uniform jeweils einzelne Teleportationen auf. Angesichts der insgesamt geringen Häufigkeit lassen sich diese Ereignisse als Ausnahmen werten, die die Gesamtbewertung der Modelle kaum beeinflussen.

\subsubsection{Anzahl zurückgehaltener Zahrzeuge}
\label{sec:diff-waiting-time-backlogged}

\begin{figure}[H]
    \centering
    \begin{tikzpicture}
        \begin{axis}[
                ybar,
                bar width=0.25cm,
                width=12cm,
                height=5cm,
                enlarge x limits=0.15,
                ylabel={Anzahl zurückgehaltener Zahrzeuge},
                symbolic x coords={evening_peak,morning_peak,random_heavy,uniform},
                xtick=data,
                xticklabels={\text{evening\_peak},\text{morning\_peak},\text{random\_heavy},\text{uniform}},
                x tick label style={rotate=45,anchor=east},
                legend style={at={(1.05,0.5)}, anchor=west},
                ymajorgrids=true,
                grid style=dashed,
                every axis plot post/.append style={thick, fill=.!50}
            ]

            % Baseline FixedTime
            \addplot+[color1, error bars/.cd,
                y dir=minus, y explicit,
                error bar style={line width=1pt, black}] table [
                    x=scenario, y=system_total_backlogged_mean, col sep=comma
                ] {chapters/evaluation/results/diff-waiting-time/Baseline_FixedTime.csv};
            \addlegendentry{Baseline FixedTime}

            % Baseline FixedTime
            \addplot+[color6, error bars/.cd,
                y dir=minus, y explicit,
                error bar style={line width=1pt, black}] table [
                    x=scenario, y=system_total_backlogged_mean, col sep=comma
                ] {chapters/evaluation/results/diff-waiting-time/Baseline_Actuated.csv};
            \addlegendentry{Baseline Actuated}

            % RL Modell 1
            \addplot+[color2, error bars/.cd,
                y dir=minus, y explicit,
                error bar style={line width=1pt, black}] table [
                    x=scenario, y=system_total_backlogged_mean, col sep=comma
                ] {chapters/evaluation/results/diff-waiting-time/ppo_sumo_456_2025-08-18_01-18-32_456.csv};
            \addlegendentry{Model 1}

            % RL Modell 2
            \addplot+[color3, error bars/.cd,
                y dir=minus, y explicit,
                error bar style={line width=1pt, black}] table [
                    x=scenario, y=system_total_backlogged_mean, col sep=comma
                ] {chapters/evaluation/results/diff-waiting-time/ppo_sumo_13755_2025-08-18_04-48-47_13755.csv};
            \addlegendentry{Model 2}

            % RL Modell 3
            \addplot+[color4, error bars/.cd,
                y dir=minus, y explicit,
                error bar style={line width=1pt, black}] table [
                    x=scenario, y=system_total_backlogged_mean, col sep=comma
                ] {chapters/evaluation/results/diff-waiting-time/ppo_sumo_143534_2025-08-17_23-30-08_143534.csv};
            \addlegendentry{Model 3}

            % RL Modell 4
            \addplot+[color5, error bars/.cd,
                y dir=minus, y explicit,
                error bar style={line width=1pt, black}] table [
                    x=scenario, y=system_total_backlogged_mean, col sep=comma
                ] {chapters/evaluation/results/diff-waiting-time/ppo_sumo_635768_2025-08-18_03-03-29_635768.csv};
            \addlegendentry{Model 4}
        \end{axis}
    \end{tikzpicture}
    \caption{Anzahl zurückgehaltener Zahrzeuge}
    \label{fig:diff-waiting-time-backlogged}
\end{figure}

Es ist klar erkenntbar, dass das Netz nicht an ihre maximal Auslastung gerät. Dies bedeutet jedoch nicht, dass keine Staus oder überlastetet Teilgebiete gibt.

\subsubsection{Einstufung}
\label{sec:diff-waiting-time-einstufung}
Die Modelle, die mit der Diff-Waiting-Time-Rewardfunktion trainiert wurden, zeigen in den meisten Metriken eine klare Überlegenheit gegenüber den Baselines. Die Fixed-Time-Baseline dient als stabile Referenz, wird jedoch in nahezu allen Szenarien von den Modellen deutlich übertroffen. Die Actuated-Baseline bestätigt auch hier ihre Schwäche und liefert durchweg die schlechtesten Ergebnisse.

Besonders auffällig ist die deutliche Reduktion der mittleren Wartezeiten und der Anzahl stoppender Fahrzeuge. Modelle 1 und 4 erreichen dabei die insgesamt besten Resultate und bleiben sowohl in regulären als auch in stark belasteten Szenarien konsistent unterhalb der Fixed-Time-Baseline. Modell 2 und Modell 3 zeigen ebenfalls Verbesserungen, jedoch treten in Szenarien mit hoher Verkehrslast (random\_heavy, evening\_peak) teils deutlich erhöhte Werte und zugleich hohe Standardabweichungen auf. Diese Schwankungen weisen auf eine geringere Stabilität der Strategien hin.

Hinsichtlich der Durchschnittsgeschwindigkeit erzielen alle Modelle höhere Werte als die Fixed-Time-Baseline, wobei insbesondere Modell 2 und Modell 4 durch ihre gleichmäßige Performance hervorstechen. Modell 3 fällt in einzelnen Szenarien leicht zurück, bleibt aber insgesamt dennoch über der Referenz.

Die Metriken zu Teleportationen und zurückgehaltenen Fahrzeugen bestätigen, dass das Netz nicht an seine absolute Kapazitätsgrenze gelangte. Einzelne Teleportationen bei Modell 1 und Modell 3 sind als Randereignisse zu werten und beeinflussen die Gesamteinstufung nicht maßgeblich.

Insgesamt lässt sich festhalten, dass die Diff-Waiting-Time-Rewardfunktion robuste Modelle hervorbringt, die klassische Steuerungen klar übertreffen. Modelle 1 und 4 überzeugen durchgängig mit stabiler Performance, während Modell 2 und Modell 3 unter hoher Verkehrslast anfälliger für Leistungseinbrüche und stärkere Varianz sind.

\subsection{Reward: Queue}
Die zweite Gruppe zielt auf die Minimierung der Warteschlangenlänge.


% Text mit Interpretation: RL reduziert Staus, wirkt sich auf Durchfluss aus etc.


\subsection{Reward: Reale Welt}
Diese Gruppe ziel auf ein kombiniertes Minimieren der Wartezeiten, Anzahl an Phasenwechsel und Staus.

\subsubsection{Mittlere Wartezeiten}
\label{sec:realworld-wartezeit}
\begin{figure}[H]
    \centering
    \begin{tikzpicture}
        \begin{axis}[
                ybar,
                bar width=0.25cm,
                width=12cm,
                height=8cm,
                enlarge x limits=0.15,
                ylabel={Mittlere Wartezeit [s]},
                symbolic x coords={evening_peak,morning_peak,random_heavy,uniform},
                xtick=data,
                xticklabels={\text{evening\_peak},\text{morning\_peak},\text{random\_heavy},\text{uniform}},
                x tick label style={rotate=45,anchor=east},
                legend style={at={(1.05,0.5)}, anchor=west},
                ymajorgrids=true,
                grid style=dashed,
                every axis plot post/.append style={thick, fill=.!50}
            ]

            % Baseline FixedTime
            \addplot+[color1, error bars/.cd,
                y dir=minus, y explicit,
                error bar style={line width=1pt, black}] table [
                    x=scenario, y=system_mean_waiting_time_mean, col sep=comma, y error=system_mean_waiting_time_std
                ] {chapters/evaluation/results/realworld/Baseline_FixedTime.csv};
            \addlegendentry{Baseline FixedTime}

            % RL Modell 1
            \addplot+[color2, error bars/.cd,
                y dir=minus, y explicit,
                error bar style={line width=1pt, black}] table [
                    x=scenario, y=system_mean_waiting_time_mean, col sep=comma, y error=system_mean_waiting_time_std
                ] {chapters/evaluation/results/realworld/ppo_sumo_456_2025-08-18_01-08-35_456.csv};
            \addlegendentry{Model 1}

            % RL Modell 2
            \addplot+[color3, error bars/.cd,
                y dir=minus, y explicit,
                error bar style={line width=1pt, black}] table [
                    x=scenario, y=system_mean_waiting_time_mean, col sep=comma, y error=system_mean_waiting_time_std
                ] {chapters/evaluation/results/realworld/ppo_sumo_13755_2025-08-18_07-38-05_13755.csv};
            \addlegendentry{Model 2}

            % RL Modell 3
            \addplot+[color4, error bars/.cd,
                y dir=minus, y explicit,
                error bar style={line width=1pt, black}] table [
                    x=scenario, y=system_mean_waiting_time_mean, col sep=comma, y error=system_mean_waiting_time_std
                ] {chapters/evaluation/results/realworld/ppo_sumo_143534_2025-08-17_21-54-21_143534.csv};
            \addlegendentry{Model 3}

            % RL Modell 4
            \addplot+[color5, error bars/.cd,
                y dir=minus, y explicit,
                error bar style={line width=1pt, black}] table [
                    x=scenario, y=system_mean_waiting_time_mean, col sep=comma, y error=system_mean_waiting_time_std
                ] {chapters/evaluation/results/realworld/ppo_sumo_635768_2025-08-18_04-23-15_635768.csv};
            \addlegendentry{Model 4}
        \end{axis}
    \end{tikzpicture}
    \caption{Mittlere Wartezeiten}
    \label{fig:realworld-wartezeit}
\end{figure}



\begin{figure}[H]
    \centering
    \begin{tikzpicture}
        \begin{axis}[
                ybar,
                bar width=0.25cm,
                width=12cm,
                height=5cm,
                enlarge x limits=0.15,
                ylabel={Mittlere Wartezeit [s]},
                symbolic x coords={evening_peak,morning_peak,random_heavy,uniform},
                xtick=data,
                xticklabels={\text{evening\_peak},\text{morning\_peak},\text{random\_heavy},\text{uniform}},
                x tick label style={rotate=45,anchor=east},
                legend style={at={(1.05,0.5)}, anchor=west},
                ymajorgrids=true,
                grid style=dashed,
                every axis plot post/.append style={thick, fill=.!50}
            ]

            % Baseline FixedTime
            \addplot+[color1, error bars/.cd,
                y dir=minus, y explicit,
                error bar style={line width=1pt, black}] table [
                    x=scenario, y=system_mean_waiting_time_mean, col sep=comma, y error=system_mean_waiting_time_std
                ] {chapters/evaluation/results/realworld/Baseline_FixedTime.csv};
            \addlegendentry{Baseline FixedTime}

            % Baseline FixedTime
            \addplot+[color6, error bars/.cd,
                y dir=minus, y explicit,
                error bar style={line width=1pt, black}] table [
                    x=scenario, y=system_mean_waiting_time_mean, col sep=comma, y error=system_mean_waiting_time_std
                ] {chapters/evaluation/results/realworld/Baseline_Actuated.csv};
            \addlegendentry{Baseline Actuated}
        \end{axis}
    \end{tikzpicture}
    \caption{Mittlere Wartezeiten}
    \label{fig:realworld-wartezeit2}
\end{figure}

Die Ergebnisse zur mittleren Wartezeit verdeutlichen erneut die starken Unterschiede zwischen den Verfahren.

Die Fixed-Time-Baseline bewegt sich in allen Szenarien auf einem stabilen Niveau von rund vier Sekunden und liefert damit konsistente, wenn auch nicht optimale Werte. Demgegenüber erreicht die Actuated-Baseline extrem hohe Mittelwerte im Bereich von fast eintausend Sekunden, was ihre geringe Effizienz klar unterstreicht.

Die trainierten Modelle erzielen insgesamt deutlich niedrigere Wartezeiten, zeigen jedoch unterschiedliche Stabilität. Modell 1 und Modell 2 erreichen in den meisten Szenarien sehr geringe Werte im Sub-Sekundenbereich und liegen damit weit unterhalb der Fixed-Time-Baseline; nur im evening\_peak und im random\_heavy steigen die Wartezeiten auf einige Dutzend Sekunden. Modell 3 und Modell 4 weisen hingegen in denselben Szenarien deutlich höhere Mittelwerte von mehreren Dutzend bis über hundert Sekunden auf, während sie in den übrigen Fällen ebenfalls sehr niedrige Werte erreichen.

Auffällig ist die hohe Varianz gerade bei Modell 3 und Modell 4 in den schwierigeren Szenarien. Dort schwanken die Ergebnisse stark zwischen einzelnen Episoden, was darauf hinweist, dass die Modelle teils sehr effiziente, teils aber auch deutlich weniger stabile Steuerungsstrategien hervorbringen.

\subsubsection{Anzahl stoppender Fahrzeuge}
\begin{figure}[H]
    \centering
    \begin{tikzpicture}
        \begin{axis}[
                ybar,
                bar width=0.25cm,
                width=12cm,
                height=8cm,
                enlarge x limits=0.15,
                ylabel={Anzahl stoppender Fahrzeuge},
                symbolic x coords={evening_peak,morning_peak,random_heavy,uniform},
                xtick=data,
                xticklabels={\text{evening\_peak},\text{morning\_peak},\text{random\_heavy},\text{uniform}},
                x tick label style={rotate=45,anchor=east},
                legend style={at={(1.05,0.5)}, anchor=west},
                ymajorgrids=true,
                grid style=dashed,
                every axis plot post/.append style={thick, fill=.!50}
            ]

            % Baseline FixedTime
            \addplot+[color1, error bars/.cd,
                y dir=minus, y explicit,
                error bar style={line width=1pt, black}] table [
                    x=scenario, y=system_total_stopped_mean, col sep=comma, y error=system_total_stopped_std
                ] {chapters/evaluation/results/realworld/Baseline_FixedTime.csv};
            \addlegendentry{Baseline FixedTime}

            % RL Modell 1
            \addplot+[color2, error bars/.cd,
                y dir=minus, y explicit,
                error bar style={line width=1pt, black}] table [
                    x=scenario, y=system_total_stopped_mean, col sep=comma, y error=system_total_stopped_std
                ] {chapters/evaluation/results/realworld/ppo_sumo_456_2025-08-18_01-08-35_456.csv};
            \addlegendentry{Model 1}


            % RL Modell 2
            \addplot+[color3, error bars/.cd,
                y dir=minus, y explicit,
                error bar style={line width=1pt, black}] table [
                    x=scenario, y=system_total_stopped_mean, col sep=comma, y error=system_total_stopped_std
                ] {chapters/evaluation/results/realworld/ppo_sumo_13755_2025-08-18_07-38-05_13755.csv};
            \addlegendentry{Model 2}

            % RL Modell 3
            \addplot+[color4, error bars/.cd,
                y dir=minus, y explicit,
                error bar style={line width=1pt, black}] table [
                    x=scenario, y=system_total_stopped_mean, col sep=comma, y error=system_total_stopped_std
                ] {chapters/evaluation/results/realworld/ppo_sumo_143534_2025-08-17_21-54-21_143534.csv};
            \addlegendentry{Model 3}

            % RL Modell 4
            \addplot+[color5, error bars/.cd,
                y dir=minus, y explicit,
                error bar style={line width=1pt, black}] table [
                    x=scenario, y=system_total_stopped_mean, col sep=comma, y error=system_total_stopped_std
                ] {chapters/evaluation/results/realworld/ppo_sumo_635768_2025-08-18_04-23-15_635768.csv};
            \addlegendentry{Model 4}
        \end{axis}
    \end{tikzpicture}
    \caption{Anzahl stoppender Fahrzeuge}
    \label{fig:realworld-stopped}
\end{figure}



\begin{figure}[H]
    \centering
    \begin{tikzpicture}
        \begin{axis}[
                ybar,
                bar width=0.25cm,
                width=12cm,
                height=5cm,
                enlarge x limits=0.15,
                ylabel={Anzahl stoppender Fahrzeuge},
                symbolic x coords={evening_peak,morning_peak,random_heavy,uniform},
                xtick=data,
                xticklabels={\text{evening\_peak},\text{morning\_peak},\text{random\_heavy},\text{uniform}},
                x tick label style={rotate=45,anchor=east},
                legend style={at={(1.05,0.5)}, anchor=west},
                ymajorgrids=true,
                grid style=dashed,
                every axis plot post/.append style={thick, fill=.!50}
            ]

            % Baseline FixedTime
            \addplot+[color1, error bars/.cd,
                y dir=minus, y explicit,
                error bar style={line width=1pt, black}] table [
                    x=scenario, y=system_total_stopped_mean, col sep=comma, y error=system_total_stopped_std
                ] {chapters/evaluation/results/realworld/Baseline_FixedTime.csv};
            \addlegendentry{Baseline FixedTime}

            % Baseline FixedTime
            \addplot+[color6, error bars/.cd,
                y dir=minus, y explicit,
                error bar style={line width=1pt, black}] table [
                    x=scenario, y=system_total_stopped_mean, col sep=comma, y error=system_total_stopped_std
                ] {chapters/evaluation/results/realworld/Baseline_Actuated.csv};
            \addlegendentry{Baseline Actuated}
        \end{axis}
    \end{tikzpicture}
    \caption{Anzahl stoppender Fahrzeuge}
    \label{fig:realworld-stopped2}
\end{figure}
Die mittlere Zahl stoppender Fahrzeuge unterscheidet sich deutlich zwischen den Verfahren.

Die Fixed-Time-Baseline bewegt sich in allen Szenarien auf einem niedrigen zweistelligen Niveau und liefert damit eine solide Referenz. Demgegenüber weist die Actuated-Baseline mit mehreren Hundert bis fast 900 Stopps pro Episode deutlich höhere Werte auf und bestätigt erneut ihre geringe Leistungsfähigkeit.

Die trainierten Modelle zeigen eine deutliche Reduktion: meist liegt die Zahl der Stopps nur bei ein bis wenigen Fahrzeugen. Besonders im morning\_peak und im uniform-Szenario erreichen alle Modelle extrem niedrige Werte. Auffällig ist jedoch, dass Modell 2 bis 4 im random\_heavy-Szenario deutlich höhere Werte aufweisen, die teils mehrere Dutzend Fahrzeuge umfassen. Modell 1 bleibt dagegen auch hier auf einem vergleichsweise niedrigen Niveau.

Die stark erhöhten Werte im random\_heavy gehen mit einer hohen Streuung zwischen den Episoden einher. Dies deutet darauf hin, dass die Modelle zwar häufig sehr effiziente Steuerungsstrategien finden, diese jedoch in einzelnen Durchläufen nicht stabil reproduziert werden.

\subsubsection{Anzahl ankommender Fahrzeuge}
\label{sec:realworld-ankommend}

\begin{figure}[H]
    \centering
    \begin{tikzpicture}
        \begin{axis}[
                ybar,
                bar width=0.25cm,
                width=12cm,
                height=8cm,
                enlarge x limits=0.15,
                ylabel={Anzahl ankommender Fahrzeuge},
                symbolic x coords={evening_peak,morning_peak,random_heavy,uniform},
                xtick=data,
                xticklabels={\text{evening\_peak},\text{morning\_peak},\text{random\_heavy},\text{uniform}},
                x tick label style={rotate=45,anchor=east},
                legend style={at={(1.05,0.5)}, anchor=west},
                ymajorgrids=true,
                grid style=dashed,
                every axis plot post/.append style={thick, fill=.!50}
            ]

            % Baseline FixedTime
            \addplot+[color1, error bars/.cd,
                y dir=minus, y explicit,
                error bar style={line width=1pt, black}] table [
                    x=scenario, y=system_total_arrived_mean, col sep=comma, y error=system_total_arrived_std
                ] {chapters/evaluation/results/realworld/Baseline_FixedTime.csv};
            \addlegendentry{Baseline FixedTime}

            % Baseline FixedTime
            \addplot+[color6, error bars/.cd,
                y dir=minus, y explicit,
                error bar style={line width=1pt, black}] table [
                    x=scenario, y=system_total_arrived_mean, col sep=comma, y error=system_total_arrived_std
                ] {chapters/evaluation/results/realworld/Baseline_Actuated.csv};
            \addlegendentry{Baseline Actuated}

            % RL Modell 1
            \addplot+[color2, error bars/.cd,
                y dir=minus, y explicit,
                error bar style={line width=1pt, black}] table [
                    x=scenario, y=system_total_arrived_mean, col sep=comma, y error=system_total_arrived_std
                ] {chapters/evaluation/results/realworld/ppo_sumo_456_2025-08-18_01-08-35_456.csv};
            \addlegendentry{Model 1}

            % RL Modell 2
            \addplot+[color3, error bars/.cd,
                y dir=minus, y explicit,
                error bar style={line width=1pt, black}] table [
                    x=scenario, y=system_total_arrived_mean, col sep=comma, y error=system_total_arrived_std
                ] {chapters/evaluation/results/realworld/ppo_sumo_13755_2025-08-18_07-38-05_13755.csv};
            \addlegendentry{Model 2}

            % RL Modell 3
            \addplot+[color4, error bars/.cd,
                y dir=minus, y explicit,
                error bar style={line width=1pt, black}] table [
                    x=scenario, y=system_total_arrived_mean, col sep=comma, y error=system_total_arrived_std
                ] {chapters/evaluation/results/realworld/ppo_sumo_143534_2025-08-17_21-54-21_143534.csv};
            \addlegendentry{Model 3}

            % RL Modell 4
            \addplot+[color5, error bars/.cd,
                y dir=minus, y explicit,
                error bar style={line width=1pt, black}] table [
                    x=scenario, y=system_total_arrived_mean, col sep=comma, y error=system_total_arrived_std
                ] {chapters/evaluation/results/realworld/ppo_sumo_635768_2025-08-18_04-23-15_635768.csv};
            \addlegendentry{Model 4}
        \end{axis}
    \end{tikzpicture}
    \caption{Anzahl ankommender Fahrzeuge}
    \label{fig:realworld-arrived}
\end{figure}

Die Auswertung der Anzahl ankommender Fahrzeuge zeigt über alle Szenarien hinweg sehr ähnliche Ergebnisse für die Fixed-Time-Baseline und die trainierten Modelle. Sowohl die Baseline als auch die Modelle erreichen in den meisten Szenarien nahezu das Maximum, was darauf hindeutet, dass der Verkehrsfluss grundsätzlich zuverlässig abgewickelt wird.

Auffällig ist lediglich, dass in random\_heavy einzelne Modelle eine leicht geringere Leistung aufweisen als die Fixed-Time-Baseline. Dieser Rückgang bleibt jedoch moderat, und die Anzahl ankommender Fahrzeuge liegt weiterhin auf einem hohen Niveau. In den übrigen Szenarien (morning\_peak, evening\_peak, uniform) stimmen die Resultate nahezu exakt mit der Fixed-Time-Baseline überein.

Die Actuated-Baseline bestätigt erneut ihre Schwäche und fällt in allen Szenarien deutlich ab. Das deutliche Defizit dieser Steuerungsstrategie kontrastiert stark mit den stabil hohen Werten der Fixed-Time-Baseline und der Modelle.

\subsubsection{Durchschnitt fahrender Fahrzeuge}
\label{sec:realworld-fahrende}

\begin{figure}[H]
    \centering
    \begin{tikzpicture}
        \begin{axis}[
                ybar,
                bar width=0.25cm,
                width=12cm,
                height=8cm,
                enlarge x limits=0.15,
                ylabel={Durchschnitt fahrender Fahrzeuge},
                symbolic x coords={evening_peak,morning_peak,random_heavy,uniform},
                xtick=data,
                xticklabels={\text{evening\_peak},\text{morning\_peak},\text{random\_heavy},\text{uniform}},
                x tick label style={rotate=45,anchor=east},
                legend style={at={(1.05,0.5)}, anchor=west},
                ymajorgrids=true,
                grid style=dashed,
                every axis plot post/.append style={thick, fill=.!50}
            ]

            % Baseline FixedTime
            \addplot+[color1, error bars/.cd,
                y dir=minus, y explicit,
                error bar style={line width=1pt, black}] table [
                    x=scenario, y=system_total_running_mean, col sep=comma, y error=system_total_running_std
                ] {chapters/evaluation/results/realworld/Baseline_FixedTime.csv};
            \addlegendentry{Baseline FixedTime}

            % Baseline FixedTime
            \addplot+[color6, error bars/.cd,
                y dir=minus, y explicit,
                error bar style={line width=1pt, black}] table [
                    x=scenario, y=system_total_running_mean, col sep=comma, y error=system_total_running_std
                ] {chapters/evaluation/results/realworld/Baseline_Actuated.csv};
            \addlegendentry{Baseline Actuated}

            % RL Modell 1
            \addplot+[color2, error bars/.cd,
                y dir=minus, y explicit,
                error bar style={line width=1pt, black}] table [
                    x=scenario, y=system_total_running_mean, col sep=comma, y error=system_total_running_std
                ] {chapters/evaluation/results/realworld/ppo_sumo_456_2025-08-18_01-08-35_456.csv};
            \addlegendentry{Model 1}

            % RL Modell 2
            \addplot+[color3, error bars/.cd,
                y dir=minus, y explicit,
                error bar style={line width=1pt, black}] table [
                    x=scenario, y=system_total_running_mean, col sep=comma, y error=system_total_running_std
                ] {chapters/evaluation/results/realworld/ppo_sumo_13755_2025-08-18_07-38-05_13755.csv};
            \addlegendentry{Model 2}

            % RL Modell 3
            \addplot+[color4, error bars/.cd,
                y dir=minus, y explicit,
                error bar style={line width=1pt, black}] table [
                    x=scenario, y=system_total_running_mean, col sep=comma, y error=system_total_running_std
                ] {chapters/evaluation/results/realworld/ppo_sumo_143534_2025-08-17_21-54-21_143534.csv};
            \addlegendentry{Model 3}

            % RL Modell 4
            \addplot+[color5, error bars/.cd,
                y dir=minus, y explicit,
                error bar style={line width=1pt, black}] table [
                    x=scenario, y=system_total_running_mean, col sep=comma, y error=system_total_running_std
                ] {chapters/evaluation/results/realworld/ppo_sumo_635768_2025-08-18_04-23-15_635768.csv};
            \addlegendentry{Model 4}
        \end{axis}
    \end{tikzpicture}
    \caption{Durchschnitt fahrender Fahrzeuge}
    \label{fig:realworld-running}
\end{figure}

Die Analyse der durchschnittlichen Anzahl an Fahrzeugen im Netz zeigt klare Unterschiede. Ein geringerer Wert bedeutet dabei eine schnellere Abwicklung des Verkehrs.

Die Fixed-Time-Baseline liegt je nach Szenario zwischen rund 50 und 70 Fahrzeugen, im stark belasteten random\_heavy-Fall bei etwa 120. Damit erreicht sie ein insgesamt konsistentes Niveau.

Die Actuated-Baseline schneidet deutlich schlechter ab: mit mehreren hundert Fahrzeugen im Netz liegt sie um ein Vielfaches über der Fixed-Time-Variante und bestätigt ihre geringe Eignung.

Die trainierten Modelle erzielen Werte, die weitgehend auf dem Niveau der Fixed-Time-Baseline liegen. Typischerweise bewegen sie sich bei etwa 40 bis 60 Fahrzeugen, im random\_heavy-Szenario zwischen 100 und 130. Auffällig ist, dass einzelne Modelle in diesem Szenario etwas höhere Werte zeigen, was auch mit einer größeren Streuung einhergeht. In den übrigen Szenarien liefern sie jedoch nahezu identische Ergebnisse zur Fixed-Time-Baseline.

\subsubsection{Durchschnittsgeschwindigkeiten}
\label{sec:realworld-geschwindigkeiten}

\begin{figure}[H]
    \centering
    \begin{tikzpicture}
        \begin{axis}[
                ybar,
                bar width=0.25cm,
                width=12cm,
                height=8cm,
                enlarge x limits=0.15,
                ylabel={Durchschnittsgeschwindigkeit [m/s]} ,
                symbolic x coords={evening_peak,morning_peak,random_heavy,uniform},
                xtick=data,
                xticklabels={\text{evening\_peak},\text{morning\_peak},\text{random\_heavy},\text{uniform}},
                x tick label style={rotate=45,anchor=east},
                legend style={at={(1.05,0.5)}, anchor=west},
                ymajorgrids=true,
                grid style=dashed,
                every axis plot post/.append style={thick, fill=.!50}
            ]

            % Baseline FixedTime
            \addplot+[color1, error bars/.cd,
                y dir=minus, y explicit,
                error bar style={line width=1pt, black}] table [
                    x=scenario, y=system_mean_speed_mean, col sep=comma, y error=system_mean_speed_std
                ] {chapters/evaluation/results/realworld/Baseline_FixedTime.csv};
            \addlegendentry{Baseline FixedTime}

            % Baseline FixedTime
            \addplot+[color6, error bars/.cd,
                y dir=minus, y explicit,
                error bar style={line width=1pt, black}] table [
                    x=scenario, y=system_mean_speed_mean, col sep=comma, y error=system_mean_speed_std
                ] {chapters/evaluation/results/realworld/Baseline_Actuated.csv};
            \addlegendentry{Baseline Actuated}

            % RL Modell 1
            \addplot+[color2, error bars/.cd,
                y dir=minus, y explicit,
                error bar style={line width=1pt, black}] table [
                    x=scenario, y=system_mean_speed_mean, col sep=comma, y error=system_mean_speed_std
                ] {chapters/evaluation/results/realworld/ppo_sumo_456_2025-08-18_01-08-35_456.csv};
            \addlegendentry{Model 1}

            % RL Modell 2
            \addplot+[color3, error bars/.cd,
                y dir=minus, y explicit,
                error bar style={line width=1pt, black}] table [
                    x=scenario, y=system_mean_speed_mean, col sep=comma, y error=system_mean_speed_std
                ] {chapters/evaluation/results/realworld/ppo_sumo_13755_2025-08-18_07-38-05_13755.csv};
            \addlegendentry{Model 2}

            % RL Modell 3
            \addplot+[color4, error bars/.cd,
                y dir=minus, y explicit,
                error bar style={line width=1pt, black}] table [
                    x=scenario, y=system_mean_speed_mean, col sep=comma, y error=system_mean_speed_std
                ] {chapters/evaluation/results/realworld/ppo_sumo_143534_2025-08-17_21-54-21_143534.csv};
            \addlegendentry{Model 3}

            % RL Modell 4
            \addplot+[color5, error bars/.cd,
                y dir=minus, y explicit,
                error bar style={line width=1pt, black}] table [
                    x=scenario, y=system_mean_speed_mean, col sep=comma, y error=system_mean_speed_std
                ] {chapters/evaluation/results/realworld/ppo_sumo_635768_2025-08-18_04-23-15_635768.csv};
            \addlegendentry{Model 4}
        \end{axis}
    \end{tikzpicture}
    \caption{Durchschnittsgeschwindigkeiten}
    \label{fig:realworld-speed}
\end{figure}

Die Analyse der Durchschnittsgeschwindigkeiten macht deutliche Unterschiede sichtbar. Grundsätzlich gilt: höhere Werte bedeuten einen effizienteren Verkehrsfluss.

Die Fixed-Time-Baseline bewegt sich in allen Szenarien auf einem guten Niveau von rund 5½ bis 6 m/s und liegt damit klar über der Actuated-Variante. Diese erreicht nur etwa ½ m/s und ist damit um eine ganze Größenordnung schlechter.

Die trainierten Modelle übertreffen die Fixed-Time-Baseline deutlich und liegen meist zwischen etwa 6½ und 7¼ m/s. Besonders stabil zeigt sich Modell 2, das in allen Szenarien nahe bei 7 m/s bleibt. Modelle 3 und 4 fallen im stark belasteten random\_heavy-Szenario etwas zurück (teils nur knapp über 5½ m/s), während sie in den übrigen Fällen mit den besten Ergebnissen gleichziehen.

Insgesamt bestätigen die Ergebnisse: die trainierten Modelle steigern die Durchschnittsgeschwindigkeit im Vergleich zu beiden Baselines spürbar. Nur im zufällig stark belasteten Szenario zeigen sich Ausreißer und eine höhere Streuung, in allen anderen Fällen ist der Zugewinn stabil und konsistent.
\subsubsection{Anzahl teleportierender Fahrzeuge}

\label{sec:realworld-teleport}

\begin{figure}[H]
    \centering
    \begin{tikzpicture}
        \begin{axis}[
                ybar,
                bar width=0.25cm,
                width=12cm,
                height=5cm,
                enlarge x limits=0.15,
                ylabel={Anzahl teleportierender Fahrzeuge},
                symbolic x coords={evening_peak,morning_peak,random_heavy,uniform},
                xtick=data,
                xticklabels={\text{evening\_peak},\text{morning\_peak},\text{random\_heavy},\text{uniform}},
                x tick label style={rotate=45,anchor=east},
                legend style={at={(1.05,0.5)}, anchor=west},
                ymajorgrids=true,
                grid style=dashed,
                every axis plot post/.append style={thick, fill=.!50}
            ]

            % Baseline FixedTime
            \addplot+[color1, error bars/.cd,
                y dir=minus, y explicit,
                error bar style={line width=1pt, black}] table [
                    x=scenario, y=system_total_teleported_mean, col sep=comma
                ] {chapters/evaluation/results/realworld/Baseline_FixedTime.csv};
            \addlegendentry{Baseline FixedTime}

            % Baseline FixedTime
            \addplot+[color6, error bars/.cd,
                y dir=minus, y explicit,
                error bar style={line width=1pt, black}] table [
                    x=scenario, y=system_total_teleported_mean, col sep=comma
                ] {chapters/evaluation/results/realworld/Baseline_Actuated.csv};
            \addlegendentry{Baseline Actuated}

            % RL Modell 1
            \addplot+[color2, error bars/.cd,
                y dir=minus, y explicit,
                error bar style={line width=1pt, black}] table [
                    x=scenario, y=system_total_teleported_mean, col sep=comma
                ] {chapters/evaluation/results/realworld/ppo_sumo_456_2025-08-18_01-08-35_456.csv};
            \addlegendentry{Model 1}

            % RL Modell 2
            \addplot+[color3, error bars/.cd,
                y dir=minus, y explicit,
                error bar style={line width=1pt, black}] table [
                    x=scenario, y=system_total_teleported_mean, col sep=comma
                ] {chapters/evaluation/results/realworld/ppo_sumo_13755_2025-08-18_07-38-05_13755.csv};
            \addlegendentry{Model 2}

            % RL Modell 3
            \addplot+[color4, error bars/.cd,
                y dir=minus, y explicit,
                error bar style={line width=1pt, black}] table [
                    x=scenario, y=system_total_teleported_mean, col sep=comma
                ] {chapters/evaluation/results/realworld/ppo_sumo_143534_2025-08-17_21-54-21_143534.csv};
            \addlegendentry{Model 3}

            % RL Modell 4
            \addplot+[color5, error bars/.cd,
                y dir=minus, y explicit,
                error bar style={line width=1pt, black}] table [
                    x=scenario, y=system_total_teleported_mean, col sep=comma
                ] {chapters/evaluation/results/realworld/ppo_sumo_635768_2025-08-18_04-23-15_635768.csv};
            \addlegendentry{Model 4}
        \end{axis}
    \end{tikzpicture}
    \caption{Anzahl teleportierender Fahrzeuge}
    \label{fig:realworld-teleports}
\end{figure}

Die Auswertung der Teleportationen zeigt, dass in nahezu allen Szenarien keine Fahrzeuge teleportiert werden mussten. Dies gilt sowohl für die beiden Baselines als auch für die trainierten Modelle. Eine Ausnahme bildet das Szenario random\_heavy bei Modell 3, in dem insgesammt eine Teleportation innerhalb der 10 Episoden auftrat. Dieses Ergebnis steht im Einklang mit den zuvor beobachteten Schwächen desselben Modells in diesem Szenario.

\subsubsection{Anzahl zurückgehaltener Fahrzeuge}
\label{sec:realworld-backlogged}

\begin{figure}[H]
    \centering
    \begin{tikzpicture}
        \begin{axis}[
                ybar,
                bar width=0.25cm,
                width=12cm,
                height=5cm,
                enlarge x limits=0.15,
                ylabel={Anzahl zurückgehaltener Fahrzeuge},
                symbolic x coords={evening_peak,morning_peak,random_heavy,uniform},
                xtick=data,
                xticklabels={\text{evening\_peak},\text{morning\_peak},\text{random\_heavy},\text{uniform}},
                x tick label style={rotate=45,anchor=east},
                legend style={at={(1.05,0.5)}, anchor=west},
                ymajorgrids=true,
                grid style=dashed,
                every axis plot post/.append style={thick, fill=.!50}
            ]

            % Baseline FixedTime
            \addplot+[color1, error bars/.cd,
                y dir=minus, y explicit,
                error bar style={line width=1pt, black}] table [
                    x=scenario, y=system_total_backlogged_mean, col sep=comma
                ] {chapters/evaluation/results/realworld/Baseline_FixedTime.csv};
            \addlegendentry{Baseline FixedTime}

            % Baseline FixedTime
            \addplot+[color6, error bars/.cd,
                y dir=minus, y explicit,
                error bar style={line width=1pt, black}] table [
                    x=scenario, y=system_total_backlogged_mean, col sep=comma
                ] {chapters/evaluation/results/realworld/Baseline_Actuated.csv};
            \addlegendentry{Baseline Actuated}

            % RL Modell 1
            \addplot+[color2, error bars/.cd,
                y dir=minus, y explicit,
                error bar style={line width=1pt, black}] table [
                    x=scenario, y=system_total_backlogged_mean, col sep=comma
                ] {chapters/evaluation/results/realworld/ppo_sumo_456_2025-08-18_01-08-35_456.csv};
            \addlegendentry{Model 1}

            % RL Modell 2
            \addplot+[color3, error bars/.cd,
                y dir=minus, y explicit,
                error bar style={line width=1pt, black}] table [
                    x=scenario, y=system_total_backlogged_mean, col sep=comma
                ] {chapters/evaluation/results/realworld/ppo_sumo_13755_2025-08-18_07-38-05_13755.csv};
            \addlegendentry{Model 2}

            % RL Modell 3
            \addplot+[color4, error bars/.cd,
                y dir=minus, y explicit,
                error bar style={line width=1pt, black}] table [
                    x=scenario, y=system_total_backlogged_mean, col sep=comma
                ] {chapters/evaluation/results/realworld/ppo_sumo_143534_2025-08-17_21-54-21_143534.csv};
            \addlegendentry{Model 3}

            % RL Modell 4
            \addplot+[color5, error bars/.cd,
                y dir=minus, y explicit,
                error bar style={line width=1pt, black}] table [
                    x=scenario, y=system_total_backlogged_mean, col sep=comma
                ] {chapters/evaluation/results/realworld/ppo_sumo_635768_2025-08-18_04-23-15_635768.csv};
            \addlegendentry{Model 4}
        \end{axis}
    \end{tikzpicture}
    \caption{Anzahl zurückgehaltener Fahrzeuge}
    \label{fig:realworld-backlogged}
\end{figure}
\newpage
In allen Szenarien zeigen die vier Modelle sowie die Fixed-Time-Baseline keine zurückgehaltenen Fahrzeuge. Die Actuated-Baseline weist jedoch in sämtlichen Szenarien deutliche Werte auf, die mit der insgesamt schwachen Leistung dieser Methode konsistent sind. Dieses Ergebnis bestätigt, dass ausschließlich die Actuated-Steuerung Fahrzeuge im Netz blockiert, während alle anderen Verfahren einen stabilen Verkehrsfluss ohne Zurückhalten sicherstellen konnten.

\subsubsection{Einstufung}
\label{sec:realworld-einstufung}

Die Auswertung der verschiedenen Metriken zeigt, dass die trainierten Modelle die klassischen Baselines insgesamt deutlich übertreffen. Während die Actuated-Baseline durchgehend schwache Ergebnisse liefert und selbst von der Fixed-Time-Steuerung klar geschlagen wird, gelingt es den Modellen in nahezu allen Szenarien, sowohl die mittlere Wartezeit als auch die Anzahl stoppender Fahrzeuge deutlich zu reduzieren und gleichzeitig höhere Durchschnittsgeschwindigkeiten zu erreichen.

Besonders deutlich wird der Vorteil der Modelle in den Szenarien mit regulärer oder gleichmäßiger Verkehrslast, wo sie konsistent nahe am Optimum operieren. Auch die Anzahl ankommender Fahrzeuge bleibt in diesen Fällen auf dem maximalen Niveau, sodass die Effizienzsteigerung nicht mit einem Verlust an Durchsatz erkauft wird.

Einschränkungen zeigen sich jedoch im random\_heavy-Szenario: hier treten bei mehreren Modellen signifikante Verschlechterungen auf, die zugleich mit einer hohen Standardabweichung verbunden sind. Dies weist auf eine eingeschränkte Robustheit unter komplexeren und schwer vorhersagbaren Verkehrssituationen hin. Besonders ausgeprägt sind diese Schwächen bei einem Modell, das zusätzlich vereinzelt Teleportationen aufweist und damit strukturelle Instabilitäten erkennen lässt.

Insgesamt lässt sich festhalten, dass die lernbasierten Steuerungsansätze das Potenzial besitzen, klassische Verfahren im Hinblick auf Wartezeiten, Staus und Geschwindigkeiten deutlich zu übertreffen. Gleichzeitig verdeutlichen die Ergebnisse, dass die Generalisierungsfähigkeit insbesondere in Szenarien mit unregelmäßiger und schwer prognostizierbarer Verkehrslast eine zentrale Herausforderung bleibt.
\subsection{Reward: Emissionen}
Für die Emissions-basierten Modelle wird zusätzlich die CO\textsubscript{2}-Emission als zentrale Metrik betrachtet.

\subsubsection{Mittlere Wartezeit}

\begin{figure}[H]
    \centering
    \begin{tikzpicture}
        \begin{axis}[
                ybar,
                bar width=0.25cm,
                width=12cm,
                enlarge x limits=0.15,
                ylabel={Mittlere Wartezeit},
                symbolic x coords={evening_peak,morning_peak,random_heavy,uniform},
                xtick=data,
                xticklabels={evening\_peak,morning\_peak,random\_heavy,uniform},
                x tick label style={rotate=45,anchor=east},
                legend style={at={(1.05,0.5)}, anchor=west},
                ymajorgrids=true,
                grid style=dashed,
                every axis plot post/.append style={thick, fill=.!50}
            ]

            % Baseline FixedTime
            \addplot+[color1, error bars/.cd,
                y dir=minus, y explicit,
                error bar style={line width=1pt, black}] table [
                    x=scenario, y=system_mean_waiting_time_mean, col sep=comma, y error=system_mean_waiting_time_std
                ] {chapters/evaluation/results/emissions/Baseline_FixedTime.csv};
            \addlegendentry{Baseline FixedTime}

            % RL Modell 1
            \addplot+[color2, error bars/.cd,
                y dir=minus, y explicit,
                error bar style={line width=1pt, black}] table [
                    x=scenario, y=system_mean_waiting_time_mean, col sep=comma, y error=system_mean_waiting_time_std
                ] {chapters/evaluation/results/emissions/ppo_sumo_456_2025-08-18_16-05-49_456.csv};
            \addlegendentry{Model 1}


            % RL Modell 2
            \addplot+[color3, error bars/.cd,
                y dir=minus, y explicit,
                error bar style={line width=1pt, black}] table [
                    x=scenario, y=system_mean_waiting_time_mean, col sep=comma, y error=system_mean_waiting_time_std
                ] {chapters/evaluation/results/emissions/ppo_sumo_13755_2025-08-18_21-51-17_13755.csv};
            \addlegendentry{Model 2}

            % RL Modell 3
            \addplot+[color4, error bars/.cd,
                y dir=minus, y explicit,
                error bar style={line width=1pt, black}] table [
                    x=scenario, y=system_mean_waiting_time_mean, col sep=comma, y error=system_mean_waiting_time_std
                ] {chapters/evaluation/results/emissions/ppo_sumo_143534_2025-08-18_13-16-03_143534.csv};
            \addlegendentry{Model 3}

            % RL Modell 4
            \addplot+[color5, error bars/.cd,
                y dir=minus, y explicit,
                error bar style={line width=1pt, black}] table [
                    x=scenario, y=system_mean_waiting_time_mean, col sep=comma, y error=system_mean_waiting_time_std
                ] {chapters/evaluation/results/emissions/ppo_sumo_635768_2025-08-18_18-57-01_635768.csv};
            \addlegendentry{Model 4}
        \end{axis}
    \end{tikzpicture}
\end{figure}

\begin{figure}[H]
    \centering
    \begin{tikzpicture}
        \begin{axis}[
                ybar,
                bar width=0.25cm,
                width=12cm,
                enlarge x limits=0.15,
                ylabel={Mittlere Wartezeit},
                symbolic x coords={evening_peak,morning_peak,random_heavy,uniform},
                xtick=data,
                xticklabels={evening\_peak,morning\_peak,random\_heavy,uniform},
                x tick label style={rotate=45,anchor=east},
                legend style={at={(1.05,0.5)}, anchor=west},
                ymajorgrids=true,
                grid style=dashed,
                every axis plot post/.append style={thick, fill=.!50}
            ]


            % Baseline FixedTime
            \addplot+[color1, error bars/.cd,
                y dir=minus, y explicit,
                error bar style={line width=1pt, black}] table [
                    x=scenario, y=system_mean_waiting_time_mean, col sep=comma, y error=system_mean_waiting_time_std
                ] {chapters/evaluation/results/emissions/Baseline_FixedTime.csv};
            \addlegendentry{Baseline FixedTime}
            % Baseline Actuated
            \addplot+[color6, error bars/.cd,
                y dir=minus, y explicit,
                error bar style={line width=1pt, black}] table [
                    x=scenario, y=system_mean_waiting_time_mean, col sep=comma, y error=system_mean_waiting_time_std
                ] {chapters/evaluation/results/emissions/Baseline_Actuated.csv};
            \addlegendentry{Baseline Actuated}
        \end{axis}
    \end{tikzpicture}
\end{figure}

\subsubsection{Anzahl angekommener Fahrzeuge}

\begin{figure}[H]
    \centering
    \begin{tikzpicture}
        \begin{axis}[
                ybar,
                bar width=0.25cm,
                width=12cm,
                enlarge x limits=0.15,
                ylabel={Anzahl angekommener Fahrzeuge},
                symbolic x coords={evening_peak,morning_peak,random_heavy,uniform},
                xtick=data,
                xticklabels={evening\_peak,morning\_peak,random\_heavy,uniform},
                x tick label style={rotate=45,anchor=east},
                legend style={at={(1.05,0.5)}, anchor=west},
                ymajorgrids=true,
                grid style=dashed,
                every axis plot post/.append style={thick, fill=.!50}
            ]

            % Baseline FixedTime
            \addplot+[color1, error bars/.cd,
                y dir=minus, y explicit,
                error bar style={line width=1pt, black}] table [
                    x=scenario, y=system_total_arrived_mean, col sep=comma, y error=system_total_arrived_std
                ] {chapters/evaluation/results/emissions/Baseline_FixedTime.csv};
            \addlegendentry{Baseline FixedTime}

            % RL Modell 1
            \addplot+[color2, error bars/.cd,
                y dir=minus, y explicit,
                error bar style={line width=1pt, black}] table [
                    x=scenario, y=system_total_arrived_mean, col sep=comma, y error=system_total_arrived_std
                ] {chapters/evaluation/results/emissions/ppo_sumo_456_2025-08-18_16-05-49_456.csv};
            \addlegendentry{Model 1}

            % RL Modell 2
            \addplot+[color3, error bars/.cd,
                y dir=minus, y explicit,
                error bar style={line width=1pt, black}] table [
                    x=scenario, y=system_total_arrived_mean, col sep=comma, y error=system_total_arrived_std
                ] {chapters/evaluation/results/emissions/ppo_sumo_13755_2025-08-18_21-51-17_13755.csv};
            \addlegendentry{Model 2}

            % RL Modell 3
            \addplot+[color4, error bars/.cd,
                y dir=minus, y explicit,
                error bar style={line width=1pt, black}] table [
                    x=scenario, y=system_total_arrived_mean, col sep=comma, y error=system_total_arrived_std
                ] {chapters/evaluation/results/emissions/ppo_sumo_143534_2025-08-18_13-16-03_143534.csv};
            \addlegendentry{Model 3}

            % RL Modell 4
            \addplot+[color5, error bars/.cd,
                y dir=minus, y explicit,
                error bar style={line width=1pt, black}] table [
                    x=scenario, y=system_total_arrived_mean, col sep=comma, y error=system_total_arrived_std
                ] {chapters/evaluation/results/emissions/ppo_sumo_635768_2025-08-18_18-57-01_635768.csv};
            \addlegendentry{Model 4}
        \end{axis}
    \end{tikzpicture}
\end{figure}

\begin{figure}[H]
    \centering
    \begin{tikzpicture}
        \begin{axis}[
                ybar,
                bar width=0.25cm,
                width=12cm,
                enlarge x limits=0.15,
                ylabel={Anzahl angekommener Fahrzeuge},
                symbolic x coords={evening_peak,morning_peak,random_heavy,uniform},
                xtick=data,
                xticklabels={evening\_peak,morning\_peak,random\_heavy,uniform},
                x tick label style={rotate=45,anchor=east},
                legend style={at={(1.05,0.5)}, anchor=west},
                ymajorgrids=true,
                grid style=dashed,
                every axis plot post/.append style={thick, fill=.!50}
            ]


            % Baseline FixedTime
            \addplot+[color1, error bars/.cd,
                y dir=minus, y explicit,
                error bar style={line width=1pt, black}] table [
                    x=scenario, y=system_total_arrived_mean, col sep=comma, y error=system_total_arrived_std
                ] {chapters/evaluation/results/emissions/Baseline_FixedTime.csv};
            \addlegendentry{Baseline FixedTime}
            % Baseline Actuated
            \addplot+[color6, error bars/.cd,
                y dir=minus, y explicit,
                error bar style={line width=1pt, black}] table [
                    x=scenario, y=system_total_arrived_mean, col sep=comma, y error=system_total_arrived_std
                ] {chapters/evaluation/results/emissions/Baseline_Actuated.csv};
            \addlegendentry{Baseline Actuated}
        \end{axis}
    \end{tikzpicture}
\end{figure}


\subsubsection{Durchschnitt fahrender Fahrzeuge}

\begin{figure}[H]
    \centering
    \begin{tikzpicture}
        \begin{axis}[
                ybar,
                bar width=0.25cm,
                width=12cm,
                enlarge x limits=0.15,
                ylabel={Durchschnitt fahrender Fahrzeuge},
                symbolic x coords={evening_peak,morning_peak,random_heavy,uniform},
                xtick=data,
                xticklabels={evening\_peak,morning\_peak,random\_heavy,uniform},
                x tick label style={rotate=45,anchor=east},
                legend style={at={(1.05,0.5)}, anchor=west},
                ymajorgrids=true,
                grid style=dashed,
                every axis plot post/.append style={thick, fill=.!50}
            ]

            % Baseline FixedTime
            \addplot+[color1, error bars/.cd,
                y dir=minus, y explicit,
                error bar style={line width=1pt, black}] table [
                    x=scenario, y=system_total_running_mean, col sep=comma, y error=system_total_running_std
                ] {chapters/evaluation/results/emissions/Baseline_FixedTime.csv};
            \addlegendentry{Baseline FixedTime}


            % RL Modell 1
            \addplot+[color2, error bars/.cd,
                y dir=minus, y explicit,
                error bar style={line width=1pt, black}] table [
                    x=scenario, y=system_total_running_mean, col sep=comma, y error=system_total_running_std
                ] {chapters/evaluation/results/emissions/ppo_sumo_456_2025-08-18_16-05-49_456.csv};
            \addlegendentry{Model 1}

            % RL Modell 2
            \addplot+[color3, error bars/.cd,
                y dir=minus, y explicit,
                error bar style={line width=1pt, black}] table [
                    x=scenario, y=system_total_running_mean, col sep=comma, y error=system_total_running_std
                ] {chapters/evaluation/results/emissions/ppo_sumo_13755_2025-08-18_21-51-17_13755.csv};
            \addlegendentry{Model 2}

            % RL Modell 3
            \addplot+[color4, error bars/.cd,
                y dir=minus, y explicit,
                error bar style={line width=1pt, black}] table [
                    x=scenario, y=system_total_running_mean, col sep=comma, y error=system_total_running_std
                ] {chapters/evaluation/results/emissions/ppo_sumo_143534_2025-08-18_13-16-03_143534.csv};
            \addlegendentry{Model 3}

            % RL Modell 4
            \addplot+[color5, error bars/.cd,
                y dir=minus, y explicit,
                error bar style={line width=1pt, black}] table [
                    x=scenario, y=system_total_running_mean, col sep=comma, y error=system_total_running_std
                ] {chapters/evaluation/results/emissions/ppo_sumo_635768_2025-08-18_18-57-01_635768.csv};
            \addlegendentry{Model 4}
        \end{axis}
    \end{tikzpicture}
\end{figure}


\begin{figure}[H]
    \centering
    \begin{tikzpicture}
        \begin{axis}[
                ybar,
                bar width=0.25cm,
                width=12cm,
                enlarge x limits=0.15,
                ylabel={Durchschnitt fahrender Fahrzeuge},
                symbolic x coords={evening_peak,morning_peak,random_heavy,uniform},
                xtick=data,
                xticklabels={evening\_peak,morning\_peak,random\_heavy,uniform},
                x tick label style={rotate=45,anchor=east},
                legend style={at={(1.05,0.5)}, anchor=west},
                ymajorgrids=true,
                grid style=dashed,
                every axis plot post/.append style={thick, fill=.!50}
            ]
            % Baseline FixedTime
            \addplot+[color1, error bars/.cd,
                y dir=minus, y explicit,
                error bar style={line width=1pt, black}] table [
                    x=scenario, y=system_total_running_mean, col sep=comma, y error=system_total_running_std
                ] {chapters/evaluation/results/emissions/Baseline_FixedTime.csv};
            \addlegendentry{Baseline FixedTime}
            % Baseline Actuated
            \addplot+[color6, error bars/.cd,
                y dir=minus, y explicit,
                error bar style={line width=1pt, black}] table [
                    x=scenario, y=system_total_running_mean, col sep=comma, y error=system_total_running_std
                ] {chapters/evaluation/results/emissions/Baseline_Actuated.csv};
            \addlegendentry{Baseline Actuated}
        \end{axis}
    \end{tikzpicture}
\end{figure}


\subsubsection{Durchschnittsgeschwindigkeit}

\begin{figure}[H]
    \centering
    \begin{tikzpicture}
        \begin{axis}[
                ybar,
                bar width=0.25cm,
                width=12cm,
                enlarge x limits=0.15,
                ylabel={Durchschnittsgeschwindigkeit},
                symbolic x coords={evening_peak,morning_peak,random_heavy,uniform},
                xtick=data,
                xticklabels={evening\_peak,morning\_peak,random\_heavy,uniform},
                x tick label style={rotate=45,anchor=east},
                legend style={at={(1.05,0.5)}, anchor=west},
                ymajorgrids=true,
                grid style=dashed,
                every axis plot post/.append style={thick, fill=.!50}
            ]

            % Baseline FixedTime
            \addplot+[color1, error bars/.cd,
                y dir=minus, y explicit,
                error bar style={line width=1pt, black}] table [
                    x=scenario, y=system_mean_co2_mean, col sep=comma, y error=system_mean_co2_std
                ] {chapters/evaluation/results/emissions/Baseline_FixedTime.csv};
            \addlegendentry{Baseline FixedTime}

            % RL Modell 1
            \addplot+[color2, error bars/.cd,
                y dir=minus, y explicit,
                error bar style={line width=1pt, black}] table [
                    x=scenario, y=system_mean_co2_mean, col sep=comma, y error=system_mean_co2_std
                ] {chapters/evaluation/results/emissions/ppo_sumo_456_2025-08-18_16-05-49_456.csv};
            \addlegendentry{Model 1}

            % RL Modell 2
            \addplot+[color3, error bars/.cd,
                y dir=minus, y explicit,
                error bar style={line width=1pt, black}] table [
                    x=scenario, y=system_mean_co2_mean, col sep=comma, y error=system_mean_co2_std
                ] {chapters/evaluation/results/emissions/ppo_sumo_13755_2025-08-18_21-51-17_13755.csv};
            \addlegendentry{Model 2}

            % RL Modell 3
            \addplot+[color4, error bars/.cd,
                y dir=minus, y explicit,
                error bar style={line width=1pt, black}] table [
                    x=scenario, y=system_mean_co2_mean, col sep=comma, y error=system_mean_co2_std
                ] {chapters/evaluation/results/emissions/ppo_sumo_143534_2025-08-18_13-16-03_143534.csv};
            \addlegendentry{Model 3}

            % RL Modell 4
            \addplot+[color5, error bars/.cd,
                y dir=minus, y explicit,
                error bar style={line width=1pt, black}] table [
                    x=scenario, y=system_mean_co2_mean, col sep=comma, y error=system_mean_co2_std
                ] {chapters/evaluation/results/emissions/ppo_sumo_635768_2025-08-18_18-57-01_635768.csv};
            \addlegendentry{Model 4}
        \end{axis}
    \end{tikzpicture}
\end{figure}


\begin{figure}[H]
    \centering
    \begin{tikzpicture}
        \begin{axis}[
                ybar,
                bar width=0.25cm,
                width=12cm,
                enlarge x limits=0.15,
                ylabel={Durchschnittsgeschwindigkeit},
                symbolic x coords={evening_peak,morning_peak,random_heavy,uniform},
                xtick=data,
                xticklabels={evening\_peak,morning\_peak,random\_heavy,uniform},
                x tick label style={rotate=45,anchor=east},
                legend style={at={(1.05,0.5)}, anchor=west},
                ymajorgrids=true,
                grid style=dashed,
                every axis plot post/.append style={thick, fill=.!50}
            ]
            % Baseline FixedTime
            \addplot+[color1, error bars/.cd,
                y dir=minus, y explicit,
                error bar style={line width=1pt, black}] table [
                    x=scenario, y=system_mean_co2_mean, col sep=comma, y error=system_mean_co2_std
                ] {chapters/evaluation/results/emissions/Baseline_FixedTime.csv};
            \addlegendentry{Baseline FixedTime}
            % Baseline Actuated
            \addplot+[color6, error bars/.cd,
                y dir=minus, y explicit,
                error bar style={line width=1pt, black}] table [
                    x=scenario, y=system_mean_co2_mean, col sep=comma, y error=system_mean_co2_std
                ] {chapters/evaluation/results/emissions/Baseline_Actuated.csv};
            \addlegendentry{Baseline Actuated}
        \end{axis}
    \end{tikzpicture}
\end{figure}



\subsubsection{Mittlere CO$_2$-Emissionen pro Fahrzeug}

\begin{figure}[H]
    \centering
    \begin{tikzpicture}
        \begin{axis}[
                ybar,
                bar width=0.25cm,
                width=12cm,
                enlarge x limits=0.15,
                ylabel={CO$_2$ [mg/s]},
                symbolic x coords={evening_peak,morning_peak,random_heavy,uniform},
                xtick=data,
                xticklabels={evening\_peak,morning\_peak,random\_heavy,uniform},
                x tick label style={rotate=45,anchor=east},
                legend style={at={(1.05,0.5)}, anchor=west},
                ymajorgrids=true,
                grid style=dashed,
                every axis plot post/.append style={thick, fill=.!50}
            ]

            % Baseline FixedTime
            \addplot+[color1, error bars/.cd,
                y dir=minus, y explicit,
                error bar style={line width=1pt, black}] table [
                    x=scenario, y=system_mean_co2_mean, col sep=comma, y error=system_mean_co2_std
                ] {chapters/evaluation/results/emissions/Baseline_FixedTime.csv};
            \addlegendentry{Baseline FixedTime}

            % Baseline Actuated
            \addplot+[color6, error bars/.cd,
                y dir=minus, y explicit,
                error bar style={line width=1pt, black}] table [
                    x=scenario, y=system_mean_co2_mean, col sep=comma, y error=system_mean_co2_std
                ] {chapters/evaluation/results/emissions/Baseline_Actuated.csv};
            \addlegendentry{Baseline Actuated}

            % RL Modell 1
            \addplot+[color2, error bars/.cd,
                y dir=minus, y explicit,
                error bar style={line width=1pt, black}] table [
                    x=scenario, y=system_mean_co2_mean, col sep=comma, y error=system_mean_co2_std
                ] {chapters/evaluation/results/emissions/ppo_sumo_456_2025-08-18_16-05-49_456.csv};
            \addlegendentry{Model 1}

            % RL Modell 2
            \addplot+[color3, error bars/.cd,
                y dir=minus, y explicit,
                error bar style={line width=1pt, black}] table [
                    x=scenario, y=system_mean_co2_mean, col sep=comma, y error=system_mean_co2_std
                ] {chapters/evaluation/results/emissions/ppo_sumo_13755_2025-08-18_21-51-17_13755.csv};
            \addlegendentry{Model 2}

            % RL Modell 3
            \addplot+[color4, error bars/.cd,
                y dir=minus, y explicit,
                error bar style={line width=1pt, black}] table [
                    x=scenario, y=system_mean_co2_mean, col sep=comma, y error=system_mean_co2_std
                ] {chapters/evaluation/results/emissions/ppo_sumo_143534_2025-08-18_13-16-03_143534.csv};
            \addlegendentry{Model 3}

            % RL Modell 4
            \addplot+[color5, error bars/.cd,
                y dir=minus, y explicit,
                error bar style={line width=1pt, black}] table [
                    x=scenario, y=system_mean_co2_mean, col sep=comma, y error=system_mean_co2_std
                ] {chapters/evaluation/results/emissions/ppo_sumo_635768_2025-08-18_18-57-01_635768.csv};
            \addlegendentry{Model 4}

        \end{axis}
    \end{tikzpicture}
\end{figure}

% ggf. weitere Metriken: stopped, arrived/departed


\subsection{Robustheit und Replikationsanalyse}

Ein zentrales Ziel der Evaluation bestand darin, nicht nur die absolute Leistungsfähigkeit
der Modelle zu messen, sondern auch deren \textbf{Robustheit} und \textbf{Replikationsfähigkeit}
zu bewerten. Unter Robustheit wird hier die Fähigkeit verstanden, auch unter variierenden
Verkehrsbedingungen (verschiedene Szenarien) und unterschiedlichen Initialisierungen (Seeds)
konsistente Resultate zu erzielen. Replikationsfähigkeit bezeichnet hingegen die Eigenschaft,
dass ein Modell mit gleicher Konfiguration über mehrere Trainingsläufe hinweg vergleichbare
Ergebnisse liefert.

\paragraph{Methodisches Vorgehen}
Für die Robustheitsanalyse wurden alle Modelle sowie die beiden Baselines in den vier Szenarien
(morning\_peak, evening\_peak, uniform, random\_heavy) jeweils über zehn Episoden evaluiert.
Unterschiedliche Seeds stellten dabei sicher, dass sowohl Zufallseinflüsse im Verkehrsfluss
als auch in der Modellinitialisierung berücksichtigt wurden. Pro Reward-Funktion lagen vier
unabhängige Trainingsseeds vor, sodass neben der Szenario-Robustheit auch die Replikationsfähigkeit
geprüft werden konnte.

Die Evaluationsumgebung protokollierte sowohl \textit{mittlere Metriken} (z.\,B. durchschnittliche
Wartezeit, mittlere Geschwindigkeit) als auch \textit{Totals} (z.\,B. Anzahl ankommender Fahrzeuge,
Gesamtemissionen) pro Episode. Diese Auswertung erlaubte es, systematische Leistungsunterschiede
von stochastischen Schwankungen zu trennen. Besonders aufschlussreich war dabei die Betrachtung
der Standardabweichungen: hohe Varianz bei gleicher Reward-Funktion weist auf eine eingeschränkte
Replikationsfähigkeit hin.

\paragraph{Ergebnisse}
Über alle Reward-Funktionen hinweg zeigte sich ein konsistentes Muster:
\begin{itemize}
    \item \textbf{Baselines:} Die Actuated-Baseline erwies sich in allen Szenarien als instabil und deutlich unterlegen.
          Die Fixed-Time-Baseline blieb robust, konnte jedoch nur in Szenarien mit regulärer Last überzeugen.
    \item \textbf{Diff-Waiting-Time:} Modelle auf Basis dieser Reward-Funktion waren insgesamt robust, insbesondere
          Modell~1 und Modell~4. Modelle~2 und~3 zeigten dagegen in random\_heavy erhöhte Varianz und teilweise deutliche
          Leistungseinbrüche, was die eingeschränkte Replikationsfähigkeit einzelner Seeds verdeutlicht.
    \item \textbf{Queue:} Die Queue-basierten Modelle erzielten eine durchgängig hohe Stabilität und reduzierten
          Wartezeiten und Stopps stark. Lediglich Modell~2 zeigte in random\_heavy und morning\_peak eine geringere Robustheit.
    \item \textbf{Real-World:} Diese Reward-Funktion lieferte in allen Szenarien außer random\_heavy sehr stabile
          und nahe am Optimum liegende Ergebnisse. Unter hoher und unregelmäßiger Last traten jedoch erhöhte
          Standardabweichungen auf, was auf eingeschränkte Robustheit hindeutet.
    \item \textbf{CO\textsubscript{2}-Emissionen:} Die Emissions-basierten Modelle kombinierten hohe Effizienz mit
          ökologischer Optimierung und erwiesen sich als besonders stabil. Lediglich in Hochlastszenarien (insb. Modell~2 und~3)
          zeigten sich Replikationsprobleme, die sich jedoch auf ein Niveau im Bereich der Fixed-Time-Baseline einpendelten.
\end{itemize}

\paragraph{Schlussfolgerung}
Die Analyse zeigt, dass die Robustheit und Replikationsfähigkeit stark von der gewählten
Reward-Funktion abhängen. Während Queue- und CO\textsubscript{2}-basierte Ansätze eine besonders
stabile Performanz liefern, neigen Diff-Waiting-Time- und Real-World-Rewards unter Hochlastbedingungen
zu Instabilitäten. In allen Fällen bleiben die Modelle den klassischen Baselines jedoch überlegen.
Damit bestätigt sich, dass Deep-RL-basierte Verkehrssteuerungen grundsätzlich robuste Strategien
hervorbringen, deren Generalisierungsfähigkeit sich jedoch insbesondere in unregelmäßigen
Verkehrssituationen weiter verbessern muss.


\subsection{Gesamtevaluierung und Schlussfolgerung}

Die Evaluation hat gezeigt, dass alle untersuchten Reinforcement-Learning-Ansätze die klassischen Baselines (Fixed-Time und Actuated) in nahezu allen Metriken deutlich übertreffen. Während die Actuated-Baseline durchgängig instabil war, bestätigte sich die Fixed-Time-Strategie als solide Referenz, die jedoch ebenfalls von den RL-Modellen übertroffen werden konnte.

Über alle Reward-Funktionen hinweg führten die Modelle zu einer deutlichen Reduktion der mittleren Wartezeit, der Anzahl stoppender Fahrzeuge und zu höheren Durchschnittsgeschwindigkeiten, ohne Einbußen im Gesamtdurchsatz. Besonders stabil zeigten sich dabei Queue- und CO\textsubscript{2}-basierte Modelle. Diff-Waiting-Time- und Real-World-Ansätze waren zwar oft effizient, reagierten jedoch unter Hochlastbedingungen sensibler auf Zufallseinflüsse.

Insgesamt wird deutlich: Deep-RL kann klassische Steuerungsverfahren nicht nur erreichen, sondern in vielen Szenarien übertreffen. Gleichzeitig bleibt die Robustheit unter unregelmäßigen Bedingungen (z. B. random\_heavy) eine zentrale Herausforderung.