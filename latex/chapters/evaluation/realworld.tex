
\subsection{Reward: Reale Welt}
Diese Gruppe ziel auf ein kombiniertes Minimieren der Wartezeiten, Anzahl an Phasenwechsel und Staus.

\subsubsection{Mittlere Wartezeiten}
\label{sec:realworld-wartezeit}
\begin{figure}[H]
    \centering
    \begin{tikzpicture}
        \begin{axis}[
                ybar,
                bar width=0.25cm,
                width=12cm,
                height=8cm,
                enlarge x limits=0.15,
                ylabel={Mittlere Wartezeit [s]},
                symbolic x coords={evening_peak,morning_peak,random_heavy,uniform},
                xtick=data,
                xticklabels={\text{evening\_peak},\text{morning\_peak},\text{random\_heavy},\text{uniform}},
                x tick label style={rotate=45,anchor=east},
                legend style={at={(1.05,0.5)}, anchor=west},
                ymajorgrids=true,
                grid style=dashed,
                every axis plot post/.append style={thick, fill=.!50}
            ]

            % Baseline FixedTime
            \addplot+[color1, error bars/.cd,
                y dir=minus, y explicit,
                error bar style={line width=1pt, black}] table [
                    x=scenario, y=system_mean_waiting_time_mean, col sep=comma, y error=system_mean_waiting_time_std
                ] {chapters/evaluation/results/realworld/Baseline_FixedTime.csv};
            \addlegendentry{Baseline FixedTime}

            % RL Modell 1
            \addplot+[color2, error bars/.cd,
                y dir=minus, y explicit,
                error bar style={line width=1pt, black}] table [
                    x=scenario, y=system_mean_waiting_time_mean, col sep=comma, y error=system_mean_waiting_time_std
                ] {chapters/evaluation/results/realworld/ppo_sumo_456_2025-08-18_01-08-35_456.csv};
            \addlegendentry{Model 1}

            % RL Modell 2
            \addplot+[color3, error bars/.cd,
                y dir=minus, y explicit,
                error bar style={line width=1pt, black}] table [
                    x=scenario, y=system_mean_waiting_time_mean, col sep=comma, y error=system_mean_waiting_time_std
                ] {chapters/evaluation/results/realworld/ppo_sumo_13755_2025-08-18_07-38-05_13755.csv};
            \addlegendentry{Model 2}

            % RL Modell 3
            \addplot+[color4, error bars/.cd,
                y dir=minus, y explicit,
                error bar style={line width=1pt, black}] table [
                    x=scenario, y=system_mean_waiting_time_mean, col sep=comma, y error=system_mean_waiting_time_std
                ] {chapters/evaluation/results/realworld/ppo_sumo_143534_2025-08-17_21-54-21_143534.csv};
            \addlegendentry{Model 3}

            % RL Modell 4
            \addplot+[color5, error bars/.cd,
                y dir=minus, y explicit,
                error bar style={line width=1pt, black}] table [
                    x=scenario, y=system_mean_waiting_time_mean, col sep=comma, y error=system_mean_waiting_time_std
                ] {chapters/evaluation/results/realworld/ppo_sumo_635768_2025-08-18_04-23-15_635768.csv};
            \addlegendentry{Model 4}
        \end{axis}
    \end{tikzpicture}
    \caption{Mittlere Wartezeiten}
    \label{fig:realworld-wartezeit}
\end{figure}



\begin{figure}[H]
    \centering
    \begin{tikzpicture}
        \begin{axis}[
                ybar,
                bar width=0.25cm,
                width=12cm,
                height=5cm,
                enlarge x limits=0.15,
                ylabel={Mittlere Wartezeit [s]},
                symbolic x coords={evening_peak,morning_peak,random_heavy,uniform},
                xtick=data,
                xticklabels={\text{evening\_peak},\text{morning\_peak},\text{random\_heavy},\text{uniform}},
                x tick label style={rotate=45,anchor=east},
                legend style={at={(1.05,0.5)}, anchor=west},
                ymajorgrids=true,
                grid style=dashed,
                every axis plot post/.append style={thick, fill=.!50}
            ]

            % Baseline FixedTime
            \addplot+[color1, error bars/.cd,
                y dir=minus, y explicit,
                error bar style={line width=1pt, black}] table [
                    x=scenario, y=system_mean_waiting_time_mean, col sep=comma, y error=system_mean_waiting_time_std
                ] {chapters/evaluation/results/realworld/Baseline_FixedTime.csv};
            \addlegendentry{Baseline FixedTime}

            % Baseline FixedTime
            \addplot+[color6, error bars/.cd,
                y dir=minus, y explicit,
                error bar style={line width=1pt, black}] table [
                    x=scenario, y=system_mean_waiting_time_mean, col sep=comma, y error=system_mean_waiting_time_std
                ] {chapters/evaluation/results/realworld/Baseline_Actuated.csv};
            \addlegendentry{Baseline Actuated}
        \end{axis}
    \end{tikzpicture}
    \caption{Mittlere Wartezeiten}
    \label{fig:realworld-wartezeit2}
\end{figure}

Die Ergebnisse zur mittleren Wartezeit verdeutlichen erneut die starken Unterschiede zwischen den Verfahren.

Die Fixed-Time-Baseline bewegt sich in allen Szenarien auf einem stabilen Niveau von rund vier Sekunden und liefert damit konsistente, wenn auch nicht optimale Werte. Demgegenüber erreicht die Actuated-Baseline extrem hohe Mittelwerte im Bereich von fast eintausend Sekunden, was ihre geringe Effizienz klar unterstreicht.

Die trainierten Modelle erzielen insgesamt deutlich niedrigere Wartezeiten, zeigen jedoch unterschiedliche Stabilität. Modell 1 und Modell 2 erreichen in den meisten Szenarien sehr geringe Werte im Sub-Sekundenbereich und liegen damit weit unterhalb der Fixed-Time-Baseline; nur im evening\_peak und im random\_heavy steigen die Wartezeiten auf einige Dutzend Sekunden. Modell 3 und Modell 4 weisen hingegen in denselben Szenarien deutlich höhere Mittelwerte von mehreren Dutzend bis über hundert Sekunden auf, während sie in den übrigen Fällen ebenfalls sehr niedrige Werte erreichen.

Auffällig ist die hohe Varianz gerade bei Modell 3 und Modell 4 in den schwierigeren Szenarien. Dort schwanken die Ergebnisse stark zwischen einzelnen Episoden, was darauf hinweist, dass die Modelle teils sehr effiziente, teils aber auch deutlich weniger stabile Steuerungsstrategien hervorbringen.

\subsubsection{Anzahl stoppender Fahrzeuge}
\begin{figure}[H]
    \centering
    \begin{tikzpicture}
        \begin{axis}[
                ybar,
                bar width=0.25cm,
                width=12cm,
                height=8cm,
                enlarge x limits=0.15,
                ylabel={Anzahl stoppender Fahrzeuge},
                symbolic x coords={evening_peak,morning_peak,random_heavy,uniform},
                xtick=data,
                xticklabels={\text{evening\_peak},\text{morning\_peak},\text{random\_heavy},\text{uniform}},
                x tick label style={rotate=45,anchor=east},
                legend style={at={(1.05,0.5)}, anchor=west},
                ymajorgrids=true,
                grid style=dashed,
                every axis plot post/.append style={thick, fill=.!50}
            ]

            % Baseline FixedTime
            \addplot+[color1, error bars/.cd,
                y dir=minus, y explicit,
                error bar style={line width=1pt, black}] table [
                    x=scenario, y=system_total_stopped_mean, col sep=comma, y error=system_total_stopped_std
                ] {chapters/evaluation/results/realworld/Baseline_FixedTime.csv};
            \addlegendentry{Baseline FixedTime}

            % RL Modell 1
            \addplot+[color2, error bars/.cd,
                y dir=minus, y explicit,
                error bar style={line width=1pt, black}] table [
                    x=scenario, y=system_total_stopped_mean, col sep=comma, y error=system_total_stopped_std
                ] {chapters/evaluation/results/realworld/ppo_sumo_456_2025-08-18_01-08-35_456.csv};
            \addlegendentry{Model 1}


            % RL Modell 2
            \addplot+[color3, error bars/.cd,
                y dir=minus, y explicit,
                error bar style={line width=1pt, black}] table [
                    x=scenario, y=system_total_stopped_mean, col sep=comma, y error=system_total_stopped_std
                ] {chapters/evaluation/results/realworld/ppo_sumo_13755_2025-08-18_07-38-05_13755.csv};
            \addlegendentry{Model 2}

            % RL Modell 3
            \addplot+[color4, error bars/.cd,
                y dir=minus, y explicit,
                error bar style={line width=1pt, black}] table [
                    x=scenario, y=system_total_stopped_mean, col sep=comma, y error=system_total_stopped_std
                ] {chapters/evaluation/results/realworld/ppo_sumo_143534_2025-08-17_21-54-21_143534.csv};
            \addlegendentry{Model 3}

            % RL Modell 4
            \addplot+[color5, error bars/.cd,
                y dir=minus, y explicit,
                error bar style={line width=1pt, black}] table [
                    x=scenario, y=system_total_stopped_mean, col sep=comma, y error=system_total_stopped_std
                ] {chapters/evaluation/results/realworld/ppo_sumo_635768_2025-08-18_04-23-15_635768.csv};
            \addlegendentry{Model 4}
        \end{axis}
    \end{tikzpicture}
    \caption{Anzahl stoppender Fahrzeuge}
    \label{fig:realworld-stopped}
\end{figure}



\begin{figure}[H]
    \centering
    \begin{tikzpicture}
        \begin{axis}[
                ybar,
                bar width=0.25cm,
                width=12cm,
                height=5cm,
                enlarge x limits=0.15,
                ylabel={Anzahl stoppender Fahrzeuge},
                symbolic x coords={evening_peak,morning_peak,random_heavy,uniform},
                xtick=data,
                xticklabels={\text{evening\_peak},\text{morning\_peak},\text{random\_heavy},\text{uniform}},
                x tick label style={rotate=45,anchor=east},
                legend style={at={(1.05,0.5)}, anchor=west},
                ymajorgrids=true,
                grid style=dashed,
                every axis plot post/.append style={thick, fill=.!50}
            ]

            % Baseline FixedTime
            \addplot+[color1, error bars/.cd,
                y dir=minus, y explicit,
                error bar style={line width=1pt, black}] table [
                    x=scenario, y=system_total_stopped_mean, col sep=comma, y error=system_total_stopped_std
                ] {chapters/evaluation/results/realworld/Baseline_FixedTime.csv};
            \addlegendentry{Baseline FixedTime}

            % Baseline FixedTime
            \addplot+[color6, error bars/.cd,
                y dir=minus, y explicit,
                error bar style={line width=1pt, black}] table [
                    x=scenario, y=system_total_stopped_mean, col sep=comma, y error=system_total_stopped_std
                ] {chapters/evaluation/results/realworld/Baseline_Actuated.csv};
            \addlegendentry{Baseline Actuated}
        \end{axis}
    \end{tikzpicture}
    \caption{Anzahl stoppender Fahrzeuge}
    \label{fig:realworld-stopped2}
\end{figure}
Die mittlere Zahl stoppender Fahrzeuge unterscheidet sich deutlich zwischen den Verfahren.

Die Fixed-Time-Baseline bewegt sich in allen Szenarien auf einem niedrigen zweistelligen Niveau und liefert damit eine solide Referenz. Demgegenüber weist die Actuated-Baseline mit mehreren Hundert bis fast 900 Stopps pro Episode deutlich höhere Werte auf und bestätigt erneut ihre geringe Leistungsfähigkeit.

Die trainierten Modelle zeigen eine deutliche Reduktion: meist liegt die Zahl der Stopps nur bei ein bis wenigen Fahrzeugen. Besonders im morning\_peak und im uniform-Szenario erreichen alle Modelle extrem niedrige Werte. Auffällig ist jedoch, dass Modell 2 bis 4 im random\_heavy-Szenario deutlich höhere Werte aufweisen, die teils mehrere Dutzend Fahrzeuge umfassen. Modell 1 bleibt dagegen auch hier auf einem vergleichsweise niedrigen Niveau.

Die stark erhöhten Werte im random\_heavy gehen mit einer hohen Streuung zwischen den Episoden einher. Dies deutet darauf hin, dass die Modelle zwar häufig sehr effiziente Steuerungsstrategien finden, diese jedoch in einzelnen Durchläufen nicht stabil reproduziert werden.

\subsubsection{Anzahl ankommender Fahrzeuge}
\label{sec:realworld-ankommend}

\begin{figure}[H]
    \centering
    \begin{tikzpicture}
        \begin{axis}[
                ybar,
                bar width=0.25cm,
                width=12cm,
                height=8cm,
                enlarge x limits=0.15,
                ylabel={Anzahl ankommender Fahrzeuge},
                symbolic x coords={evening_peak,morning_peak,random_heavy,uniform},
                xtick=data,
                xticklabels={\text{evening\_peak},\text{morning\_peak},\text{random\_heavy},\text{uniform}},
                x tick label style={rotate=45,anchor=east},
                legend style={at={(1.05,0.5)}, anchor=west},
                ymajorgrids=true,
                grid style=dashed,
                every axis plot post/.append style={thick, fill=.!50}
            ]

            % Baseline FixedTime
            \addplot+[color1, error bars/.cd,
                y dir=minus, y explicit,
                error bar style={line width=1pt, black}] table [
                    x=scenario, y=system_total_arrived_mean, col sep=comma, y error=system_total_arrived_std
                ] {chapters/evaluation/results/realworld/Baseline_FixedTime.csv};
            \addlegendentry{Baseline FixedTime}

            % Baseline FixedTime
            \addplot+[color6, error bars/.cd,
                y dir=minus, y explicit,
                error bar style={line width=1pt, black}] table [
                    x=scenario, y=system_total_arrived_mean, col sep=comma, y error=system_total_arrived_std
                ] {chapters/evaluation/results/realworld/Baseline_Actuated.csv};
            \addlegendentry{Baseline Actuated}

            % RL Modell 1
            \addplot+[color2, error bars/.cd,
                y dir=minus, y explicit,
                error bar style={line width=1pt, black}] table [
                    x=scenario, y=system_total_arrived_mean, col sep=comma, y error=system_total_arrived_std
                ] {chapters/evaluation/results/realworld/ppo_sumo_456_2025-08-18_01-08-35_456.csv};
            \addlegendentry{Model 1}

            % RL Modell 2
            \addplot+[color3, error bars/.cd,
                y dir=minus, y explicit,
                error bar style={line width=1pt, black}] table [
                    x=scenario, y=system_total_arrived_mean, col sep=comma, y error=system_total_arrived_std
                ] {chapters/evaluation/results/realworld/ppo_sumo_13755_2025-08-18_07-38-05_13755.csv};
            \addlegendentry{Model 2}

            % RL Modell 3
            \addplot+[color4, error bars/.cd,
                y dir=minus, y explicit,
                error bar style={line width=1pt, black}] table [
                    x=scenario, y=system_total_arrived_mean, col sep=comma, y error=system_total_arrived_std
                ] {chapters/evaluation/results/realworld/ppo_sumo_143534_2025-08-17_21-54-21_143534.csv};
            \addlegendentry{Model 3}

            % RL Modell 4
            \addplot+[color5, error bars/.cd,
                y dir=minus, y explicit,
                error bar style={line width=1pt, black}] table [
                    x=scenario, y=system_total_arrived_mean, col sep=comma, y error=system_total_arrived_std
                ] {chapters/evaluation/results/realworld/ppo_sumo_635768_2025-08-18_04-23-15_635768.csv};
            \addlegendentry{Model 4}
        \end{axis}
    \end{tikzpicture}
    \caption{Anzahl ankommender Fahrzeuge}
    \label{fig:realworld-arrived}
\end{figure}

Die Auswertung der Anzahl ankommender Fahrzeuge zeigt über alle Szenarien hinweg sehr ähnliche Ergebnisse für die Fixed-Time-Baseline und die trainierten Modelle. Sowohl die Baseline als auch die Modelle erreichen in den meisten Szenarien nahezu das Maximum, was darauf hindeutet, dass der Verkehrsfluss grundsätzlich zuverlässig abgewickelt wird.

Auffällig ist lediglich, dass in random\_heavy einzelne Modelle eine leicht geringere Leistung aufweisen als die Fixed-Time-Baseline. Dieser Rückgang bleibt jedoch moderat, und die Anzahl ankommender Fahrzeuge liegt weiterhin auf einem hohen Niveau. In den übrigen Szenarien (morning\_peak, evening\_peak, uniform) stimmen die Resultate nahezu exakt mit der Fixed-Time-Baseline überein.

Die Actuated-Baseline bestätigt erneut ihre Schwäche und fällt in allen Szenarien deutlich ab. Das deutliche Defizit dieser Steuerungsstrategie kontrastiert stark mit den stabil hohen Werten der Fixed-Time-Baseline und der Modelle.

\subsubsection{Durchschnitt fahrender Fahrzeuge}
\label{sec:realworld-fahrende}

\begin{figure}[H]
    \centering
    \begin{tikzpicture}
        \begin{axis}[
                ybar,
                bar width=0.25cm,
                width=12cm,
                height=8cm,
                enlarge x limits=0.15,
                ylabel={Durchschnitt fahrender Fahrzeuge},
                symbolic x coords={evening_peak,morning_peak,random_heavy,uniform},
                xtick=data,
                xticklabels={\text{evening\_peak},\text{morning\_peak},\text{random\_heavy},\text{uniform}},
                x tick label style={rotate=45,anchor=east},
                legend style={at={(1.05,0.5)}, anchor=west},
                ymajorgrids=true,
                grid style=dashed,
                every axis plot post/.append style={thick, fill=.!50}
            ]

            % Baseline FixedTime
            \addplot+[color1, error bars/.cd,
                y dir=minus, y explicit,
                error bar style={line width=1pt, black}] table [
                    x=scenario, y=system_total_running_mean, col sep=comma, y error=system_total_running_std
                ] {chapters/evaluation/results/realworld/Baseline_FixedTime.csv};
            \addlegendentry{Baseline FixedTime}

            % Baseline FixedTime
            \addplot+[color6, error bars/.cd,
                y dir=minus, y explicit,
                error bar style={line width=1pt, black}] table [
                    x=scenario, y=system_total_running_mean, col sep=comma, y error=system_total_running_std
                ] {chapters/evaluation/results/realworld/Baseline_Actuated.csv};
            \addlegendentry{Baseline Actuated}

            % RL Modell 1
            \addplot+[color2, error bars/.cd,
                y dir=minus, y explicit,
                error bar style={line width=1pt, black}] table [
                    x=scenario, y=system_total_running_mean, col sep=comma, y error=system_total_running_std
                ] {chapters/evaluation/results/realworld/ppo_sumo_456_2025-08-18_01-08-35_456.csv};
            \addlegendentry{Model 1}

            % RL Modell 2
            \addplot+[color3, error bars/.cd,
                y dir=minus, y explicit,
                error bar style={line width=1pt, black}] table [
                    x=scenario, y=system_total_running_mean, col sep=comma, y error=system_total_running_std
                ] {chapters/evaluation/results/realworld/ppo_sumo_13755_2025-08-18_07-38-05_13755.csv};
            \addlegendentry{Model 2}

            % RL Modell 3
            \addplot+[color4, error bars/.cd,
                y dir=minus, y explicit,
                error bar style={line width=1pt, black}] table [
                    x=scenario, y=system_total_running_mean, col sep=comma, y error=system_total_running_std
                ] {chapters/evaluation/results/realworld/ppo_sumo_143534_2025-08-17_21-54-21_143534.csv};
            \addlegendentry{Model 3}

            % RL Modell 4
            \addplot+[color5, error bars/.cd,
                y dir=minus, y explicit,
                error bar style={line width=1pt, black}] table [
                    x=scenario, y=system_total_running_mean, col sep=comma, y error=system_total_running_std
                ] {chapters/evaluation/results/realworld/ppo_sumo_635768_2025-08-18_04-23-15_635768.csv};
            \addlegendentry{Model 4}
        \end{axis}
    \end{tikzpicture}
    \caption{Durchschnitt fahrender Fahrzeuge}
    \label{fig:realworld-running}
\end{figure}

Die Analyse der durchschnittlichen Anzahl an Fahrzeugen im Netz zeigt klare Unterschiede. Ein geringerer Wert bedeutet dabei eine schnellere Abwicklung des Verkehrs.

Die Fixed-Time-Baseline liegt je nach Szenario zwischen rund 50 und 70 Fahrzeugen, im stark belasteten random\_heavy-Fall bei etwa 120. Damit erreicht sie ein insgesamt konsistentes Niveau.

Die Actuated-Baseline schneidet deutlich schlechter ab: mit mehreren hundert Fahrzeugen im Netz liegt sie um ein Vielfaches über der Fixed-Time-Variante und bestätigt ihre geringe Eignung.

Die trainierten Modelle erzielen Werte, die weitgehend auf dem Niveau der Fixed-Time-Baseline liegen. Typischerweise bewegen sie sich bei etwa 40 bis 60 Fahrzeugen, im random\_heavy-Szenario zwischen 100 und 130. Auffällig ist, dass einzelne Modelle in diesem Szenario etwas höhere Werte zeigen, was auch mit einer größeren Streuung einhergeht. In den übrigen Szenarien liefern sie jedoch nahezu identische Ergebnisse zur Fixed-Time-Baseline.

\subsubsection{Durchschnittsgeschwindigkeiten}
\label{sec:realworld-geschwindigkeiten}

\begin{figure}[H]
    \centering
    \begin{tikzpicture}
        \begin{axis}[
                ybar,
                bar width=0.25cm,
                width=12cm,
                height=8cm,
                enlarge x limits=0.15,
                ylabel={Durchschnittsgeschwindigkeit [m/s]} ,
                symbolic x coords={evening_peak,morning_peak,random_heavy,uniform},
                xtick=data,
                xticklabels={\text{evening\_peak},\text{morning\_peak},\text{random\_heavy},\text{uniform}},
                x tick label style={rotate=45,anchor=east},
                legend style={at={(1.05,0.5)}, anchor=west},
                ymajorgrids=true,
                grid style=dashed,
                every axis plot post/.append style={thick, fill=.!50}
            ]

            % Baseline FixedTime
            \addplot+[color1, error bars/.cd,
                y dir=minus, y explicit,
                error bar style={line width=1pt, black}] table [
                    x=scenario, y=system_mean_speed_mean, col sep=comma, y error=system_mean_speed_std
                ] {chapters/evaluation/results/realworld/Baseline_FixedTime.csv};
            \addlegendentry{Baseline FixedTime}

            % Baseline FixedTime
            \addplot+[color6, error bars/.cd,
                y dir=minus, y explicit,
                error bar style={line width=1pt, black}] table [
                    x=scenario, y=system_mean_speed_mean, col sep=comma, y error=system_mean_speed_std
                ] {chapters/evaluation/results/realworld/Baseline_Actuated.csv};
            \addlegendentry{Baseline Actuated}

            % RL Modell 1
            \addplot+[color2, error bars/.cd,
                y dir=minus, y explicit,
                error bar style={line width=1pt, black}] table [
                    x=scenario, y=system_mean_speed_mean, col sep=comma, y error=system_mean_speed_std
                ] {chapters/evaluation/results/realworld/ppo_sumo_456_2025-08-18_01-08-35_456.csv};
            \addlegendentry{Model 1}

            % RL Modell 2
            \addplot+[color3, error bars/.cd,
                y dir=minus, y explicit,
                error bar style={line width=1pt, black}] table [
                    x=scenario, y=system_mean_speed_mean, col sep=comma, y error=system_mean_speed_std
                ] {chapters/evaluation/results/realworld/ppo_sumo_13755_2025-08-18_07-38-05_13755.csv};
            \addlegendentry{Model 2}

            % RL Modell 3
            \addplot+[color4, error bars/.cd,
                y dir=minus, y explicit,
                error bar style={line width=1pt, black}] table [
                    x=scenario, y=system_mean_speed_mean, col sep=comma, y error=system_mean_speed_std
                ] {chapters/evaluation/results/realworld/ppo_sumo_143534_2025-08-17_21-54-21_143534.csv};
            \addlegendentry{Model 3}

            % RL Modell 4
            \addplot+[color5, error bars/.cd,
                y dir=minus, y explicit,
                error bar style={line width=1pt, black}] table [
                    x=scenario, y=system_mean_speed_mean, col sep=comma, y error=system_mean_speed_std
                ] {chapters/evaluation/results/realworld/ppo_sumo_635768_2025-08-18_04-23-15_635768.csv};
            \addlegendentry{Model 4}
        \end{axis}
    \end{tikzpicture}
    \caption{Durchschnittsgeschwindigkeiten}
    \label{fig:realworld-speed}
\end{figure}

Die Analyse der Durchschnittsgeschwindigkeiten macht deutliche Unterschiede sichtbar. Grundsätzlich gilt: höhere Werte bedeuten einen effizienteren Verkehrsfluss.

Die Fixed-Time-Baseline bewegt sich in allen Szenarien auf einem guten Niveau von rund 5½ bis 6 m/s und liegt damit klar über der Actuated-Variante. Diese erreicht nur etwa ½ m/s und ist damit um eine ganze Größenordnung schlechter.

Die trainierten Modelle übertreffen die Fixed-Time-Baseline deutlich und liegen meist zwischen etwa 6½ und 7¼ m/s. Besonders stabil zeigt sich Modell 2, das in allen Szenarien nahe bei 7 m/s bleibt. Modelle 3 und 4 fallen im stark belasteten random\_heavy-Szenario etwas zurück (teils nur knapp über 5½ m/s), während sie in den übrigen Fällen mit den besten Ergebnissen gleichziehen.

Insgesamt bestätigen die Ergebnisse: die trainierten Modelle steigern die Durchschnittsgeschwindigkeit im Vergleich zu beiden Baselines spürbar. Nur im zufällig stark belasteten Szenario zeigen sich Ausreißer und eine höhere Streuung, in allen anderen Fällen ist der Zugewinn stabil und konsistent.
\subsubsection{Anzahl teleportierender Fahrzeuge}

\label{sec:realworld-teleport}

\begin{figure}[H]
    \centering
    \begin{tikzpicture}
        \begin{axis}[
                ybar,
                bar width=0.25cm,
                width=12cm,
                height=5cm,
                enlarge x limits=0.15,
                ylabel={Anzahl teleportierender Fahrzeuge},
                symbolic x coords={evening_peak,morning_peak,random_heavy,uniform},
                xtick=data,
                xticklabels={\text{evening\_peak},\text{morning\_peak},\text{random\_heavy},\text{uniform}},
                x tick label style={rotate=45,anchor=east},
                legend style={at={(1.05,0.5)}, anchor=west},
                ymajorgrids=true,
                grid style=dashed,
                every axis plot post/.append style={thick, fill=.!50}
            ]

            % Baseline FixedTime
            \addplot+[color1, error bars/.cd,
                y dir=minus, y explicit,
                error bar style={line width=1pt, black}] table [
                    x=scenario, y=system_total_teleported_mean, col sep=comma
                ] {chapters/evaluation/results/realworld/Baseline_FixedTime.csv};
            \addlegendentry{Baseline FixedTime}

            % Baseline FixedTime
            \addplot+[color6, error bars/.cd,
                y dir=minus, y explicit,
                error bar style={line width=1pt, black}] table [
                    x=scenario, y=system_total_teleported_mean, col sep=comma
                ] {chapters/evaluation/results/realworld/Baseline_Actuated.csv};
            \addlegendentry{Baseline Actuated}

            % RL Modell 1
            \addplot+[color2, error bars/.cd,
                y dir=minus, y explicit,
                error bar style={line width=1pt, black}] table [
                    x=scenario, y=system_total_teleported_mean, col sep=comma
                ] {chapters/evaluation/results/realworld/ppo_sumo_456_2025-08-18_01-08-35_456.csv};
            \addlegendentry{Model 1}

            % RL Modell 2
            \addplot+[color3, error bars/.cd,
                y dir=minus, y explicit,
                error bar style={line width=1pt, black}] table [
                    x=scenario, y=system_total_teleported_mean, col sep=comma
                ] {chapters/evaluation/results/realworld/ppo_sumo_13755_2025-08-18_07-38-05_13755.csv};
            \addlegendentry{Model 2}

            % RL Modell 3
            \addplot+[color4, error bars/.cd,
                y dir=minus, y explicit,
                error bar style={line width=1pt, black}] table [
                    x=scenario, y=system_total_teleported_mean, col sep=comma
                ] {chapters/evaluation/results/realworld/ppo_sumo_143534_2025-08-17_21-54-21_143534.csv};
            \addlegendentry{Model 3}

            % RL Modell 4
            \addplot+[color5, error bars/.cd,
                y dir=minus, y explicit,
                error bar style={line width=1pt, black}] table [
                    x=scenario, y=system_total_teleported_mean, col sep=comma
                ] {chapters/evaluation/results/realworld/ppo_sumo_635768_2025-08-18_04-23-15_635768.csv};
            \addlegendentry{Model 4}
        \end{axis}
    \end{tikzpicture}
    \caption{Anzahl teleportierender Fahrzeuge}
    \label{fig:realworld-teleports}
\end{figure}

Die Auswertung der Teleportationen zeigt, dass in nahezu allen Szenarien keine Fahrzeuge teleportiert werden mussten. Dies gilt sowohl für die beiden Baselines als auch für die trainierten Modelle. Eine Ausnahme bildet das Szenario random\_heavy bei Modell 3, in dem insgesammt eine Teleportation innerhalb der 10 Episoden auftrat. Dieses Ergebnis steht im Einklang mit den zuvor beobachteten Schwächen desselben Modells in diesem Szenario.

\subsubsection{Anzahl zurückgehaltener Fahrzeuge}
\label{sec:realworld-backlogged}

\begin{figure}[H]
    \centering
    \begin{tikzpicture}
        \begin{axis}[
                ybar,
                bar width=0.25cm,
                width=12cm,
                height=5cm,
                enlarge x limits=0.15,
                ylabel={Anzahl zurückgehaltener Fahrzeuge},
                symbolic x coords={evening_peak,morning_peak,random_heavy,uniform},
                xtick=data,
                xticklabels={\text{evening\_peak},\text{morning\_peak},\text{random\_heavy},\text{uniform}},
                x tick label style={rotate=45,anchor=east},
                legend style={at={(1.05,0.5)}, anchor=west},
                ymajorgrids=true,
                grid style=dashed,
                every axis plot post/.append style={thick, fill=.!50}
            ]

            % Baseline FixedTime
            \addplot+[color1, error bars/.cd,
                y dir=minus, y explicit,
                error bar style={line width=1pt, black}] table [
                    x=scenario, y=system_total_backlogged_mean, col sep=comma
                ] {chapters/evaluation/results/realworld/Baseline_FixedTime.csv};
            \addlegendentry{Baseline FixedTime}

            % Baseline FixedTime
            \addplot+[color6, error bars/.cd,
                y dir=minus, y explicit,
                error bar style={line width=1pt, black}] table [
                    x=scenario, y=system_total_backlogged_mean, col sep=comma
                ] {chapters/evaluation/results/realworld/Baseline_Actuated.csv};
            \addlegendentry{Baseline Actuated}

            % RL Modell 1
            \addplot+[color2, error bars/.cd,
                y dir=minus, y explicit,
                error bar style={line width=1pt, black}] table [
                    x=scenario, y=system_total_backlogged_mean, col sep=comma
                ] {chapters/evaluation/results/realworld/ppo_sumo_456_2025-08-18_01-08-35_456.csv};
            \addlegendentry{Model 1}

            % RL Modell 2
            \addplot+[color3, error bars/.cd,
                y dir=minus, y explicit,
                error bar style={line width=1pt, black}] table [
                    x=scenario, y=system_total_backlogged_mean, col sep=comma
                ] {chapters/evaluation/results/realworld/ppo_sumo_13755_2025-08-18_07-38-05_13755.csv};
            \addlegendentry{Model 2}

            % RL Modell 3
            \addplot+[color4, error bars/.cd,
                y dir=minus, y explicit,
                error bar style={line width=1pt, black}] table [
                    x=scenario, y=system_total_backlogged_mean, col sep=comma
                ] {chapters/evaluation/results/realworld/ppo_sumo_143534_2025-08-17_21-54-21_143534.csv};
            \addlegendentry{Model 3}

            % RL Modell 4
            \addplot+[color5, error bars/.cd,
                y dir=minus, y explicit,
                error bar style={line width=1pt, black}] table [
                    x=scenario, y=system_total_backlogged_mean, col sep=comma
                ] {chapters/evaluation/results/realworld/ppo_sumo_635768_2025-08-18_04-23-15_635768.csv};
            \addlegendentry{Model 4}
        \end{axis}
    \end{tikzpicture}
    \caption{Anzahl zurückgehaltener Fahrzeuge}
    \label{fig:realworld-backlogged}
\end{figure}
\newpage
In allen Szenarien zeigen die vier Modelle sowie die Fixed-Time-Baseline keine zurückgehaltenen Fahrzeuge. Die Actuated-Baseline weist jedoch in sämtlichen Szenarien deutliche Werte auf, die mit der insgesamt schwachen Leistung dieser Methode konsistent sind. Dieses Ergebnis bestätigt, dass ausschließlich die Actuated-Steuerung Fahrzeuge im Netz blockiert, während alle anderen Verfahren einen stabilen Verkehrsfluss ohne Zurückhalten sicherstellen konnten.

\subsubsection{Einstufung}
\label{sec:realworld-einstufung}

Die Auswertung der verschiedenen Metriken zeigt, dass die trainierten Modelle die klassischen Baselines insgesamt deutlich übertreffen. Während die Actuated-Baseline durchgehend schwache Ergebnisse liefert und selbst von der Fixed-Time-Steuerung klar geschlagen wird, gelingt es den Modellen in nahezu allen Szenarien, sowohl die mittlere Wartezeit als auch die Anzahl stoppender Fahrzeuge deutlich zu reduzieren und gleichzeitig höhere Durchschnittsgeschwindigkeiten zu erreichen.

Besonders deutlich wird der Vorteil der Modelle in den Szenarien mit regulärer oder gleichmäßiger Verkehrslast, wo sie konsistent nahe am Optimum operieren. Auch die Anzahl ankommender Fahrzeuge bleibt in diesen Fällen auf dem maximalen Niveau, sodass die Effizienzsteigerung nicht mit einem Verlust an Durchsatz erkauft wird.

Einschränkungen zeigen sich jedoch im random\_heavy-Szenario: hier treten bei mehreren Modellen signifikante Verschlechterungen auf, die zugleich mit einer hohen Standardabweichung verbunden sind. Dies weist auf eine eingeschränkte Robustheit unter komplexeren und schwer vorhersagbaren Verkehrssituationen hin. Besonders ausgeprägt sind diese Schwächen bei einem Modell, das zusätzlich vereinzelt Teleportationen aufweist und damit strukturelle Instabilitäten erkennen lässt.

Insgesamt lässt sich festhalten, dass die lernbasierten Steuerungsansätze das Potenzial besitzen, klassische Verfahren im Hinblick auf Wartezeiten, Staus und Geschwindigkeiten deutlich zu übertreffen. Gleichzeitig verdeutlichen die Ergebnisse, dass die Generalisierungsfähigkeit insbesondere in Szenarien mit unregelmäßiger und schwer prognostizierbarer Verkehrslast eine zentrale Herausforderung bleibt.