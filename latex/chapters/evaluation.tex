\section{Evaluation und Ergebnisse}
\label{sec:validation}

\subsection{Vergleich RL vs. Baselines (Gesamtergebnisse)}
Zunächst werden die aggregierten Resultate aller RL-Modelle den beiden Baselines
(\texttt{Fixed-Time}, \texttt{Actuated}) gegenübergestellt.
Betrachtet werden Mittelwerte über alle Seeds, Szenarien und Episoden.
Diese Analyse beantwortet die zentrale Frage, ob die RL-Ansätze gegenüber klassischen Steuerungen einen generellen Vorteil erzielen.

\subsection{Szenarien-spezifische Analyse}
Anschließend erfolgt eine differenzierte Betrachtung nach Verkehrsszenarien
(morning\_peak, evening\_peak, uniform, random\_heavy).
Für jedes Szenario werden die Methoden (\texttt{RL}, \texttt{Baseline\_FixedTime}, \texttt{Baseline\_Actuated})
gegenübergestellt.
So lässt sich bewerten, ob RL insbesondere unter stochastisch hoher Last oder in Spitzenzeiten Vorteile ausspielen kann.

\subsection{Vergleich der Reward-Varianten}
Falls mehrere Reward-Funktionen trainiert wurden, erfolgt hier ein interner Vergleich der RL-Varianten.
Dabei wird untersucht, wie sich unterschiedliche Belohnungsdesigns auf zentrale Kennzahlen
(z.\,B.\ mittlere Geschwindigkeit, Warteschlangenlänge, CO\textsubscript{2}-Emissionen) auswirken.

\subsection{Robustheit und Replikationsanalyse}
Um die Stabilität der Ergebnisse zu bewerten, werden die Verteilungen über:
\begin{itemize}
    \item die 4 Trainingsseeds pro Reward-Funktion,
    \item die 10 Evaluations-Episoden pro Szenario
\end{itemize}
analysiert.
Dies erlaubt Aussagen zur Varianz und zur Reproduzierbarkeit der Ergebnisse.
Boxplots oder Konfidenzintervalle verdeutlichen die Robustheit der Ansätze.

\subsection{Zusammenfassung der Ergebnisse}
Abschließend werden die wichtigsten Beobachtungen zusammengefasst:
\begin{itemize}
    \item Überlegenheit oder Schwächen von RL im Vergleich zu Baselines,
    \item Szenarien, in denen RL besonders profitiert oder Probleme zeigt,
    \item Einflüsse verschiedener Reward-Funktionen,
    \item Stabilität der Ergebnisse über Seeds und Episoden.
\end{itemize}
