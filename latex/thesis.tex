\documentclass[a4paper, ngerman, 10pt]{article}

\usepackage{chemformula}



% code
\usepackage{algorithm}
\usepackage{algpseudocode}

% Colors
\usepackage{xcolor}
\definecolor{hellblau}{RGB}{0,153,255}
\definecolor{darkblue}{RGB}{0, 100, 170}
\usepackage{minted}

% Eingabecodierung und Sprachunterstützung
\usepackage[utf8]{inputenc}
\usepackage[ngerman]{babel}
\usepackage[T1]{fontenc}
\usepackage{float}
\usepackage{lmodern}

% Links und URLs
\usepackage{hyperref}
\hypersetup{
  colorlinks=true,
  citecolor=hellblau,   % Zitate
  urlcolor=darkblue,   % URLs
  linkcolor=darkblue,    % interne Links (TOC, Querverweise)
  linktoc=all,
}


\usepackage[acronym]{glossaries-extra}
\makeglossaries

\setabbreviationstyle[acronym]{long-short}

% glossar.tex

% --- Allgemeine Begriffe ---
\newglossaryentry{reinforcementlearning}{%
    name={Reinforcement Learning (RL)},%
    description={Ein Teilgebiet des maschinellen Lernens, bei dem ein Agent durch Interaktion mit einer Umgebung lernt, Entscheidungen zu treffen, um eine kumulative Belohnung zu maximieren.}%
}

\newglossaryentry{agent}{%
    name={Agent},%
    description={Eine Entscheidungsinstanz im Reinforcement Learning, die in der Umgebung handelt, Beobachtungen verarbeitet und eine Policy befolgt.}%
}

\newglossaryentry{policy}{%
    name={Policy},%
    description={Eine Strategie oder Abbildung, die festlegt, welche Aktion ein Agent in einem gegebenen Zustand ausführt.}%
}

\newglossaryentry{rewardfunction}{%
    name={Reward-Funktion},%
    description={Eine Funktion, die dem Agenten Rückmeldung über die Qualität einer Aktion gibt und das Lernen steuert. In dieser Arbeit wurden verschiedene Varianten untersucht (z.\,B. Diff-Waiting-Time, Queue, Real-World, Emissionen).}%
}

\newglossaryentry{episode}{%
    name={Episode},%
    description={Eine vollständige Simulationseinheit, bestehend aus einer Sequenz von Zeitschritten vom Start bis zur Terminierung.}%
}

\newglossaryentry{seed}{%
    name={Seed},%
    description={Startwert für Zufallszahlengeneratoren, der die Reproduzierbarkeit von Experimenten sicherstellt.}%
}

\newglossaryentry{hyperparameter}{%
    name={Hyperparameter},%
    description={Parameter, die das Lernverhalten eines Modells steuern, z.\,B. Lernrate, Discount-Faktor oder Explorationsparameter.}%
}

% --- Methoden und Algorithmen ---
\newacronym{ppo}{PPO}{Proximal Policy Optimization}
\newacronym{rl}{RL}{Reinforcement Learning}
\newacronym{cpu}{CPU}{Central Processing Unit}
\newacronym{gpu}{GPU}{Graphics Processing Unit}

\newglossaryentry{populationbased}{%
    name={Population-Based Training},%
    description={Ein Verfahren zur automatisierten Anpassung von Hyperparametern, bei dem mehrere Modelle parallel trainiert und Parameter zwischen erfolgreichen Modellen ausgetauscht werden.}%
}

\newglossaryentry{bayesianopt}{%
    name={Bayesian Optimization},%
    description={Ein Verfahren zur Hyperparameter-Optimierung, das auf probabilistischen Modellen basiert und gezielt vielversprechende Konfigurationen auswählt.}%
}

% --- Simulation und Daten ---
\newglossaryentry{SUMO}{%
    name={SUMO},%
    description={Simulation of Urban MObility, eine quelloffene mikroskopische Verkehrssimulationssoftware, die das Verhalten einzelner Fahrzeuge in Straßennetzen abbildet.}%
}

\newglossaryentry{OSM}{%
    name={OpenStreetMap (OSM)},%
    description={Ein kollaboratives Projekt, das Geodaten frei zur Verfügung stellt und als Grundlage für die Straßennetze in dieser Arbeit diente.}%
}

\newglossaryentry{Karlsruhe}{%
    name={Karlsruhe},%
    description={Die in dieser Arbeit verwendete Fallstudien-Stadt. Das Straßennetz basierte auf OpenStreetMap-Daten.}%
}

% --- Verkehrssteuerung ---
\newglossaryentry{trafficlight}{%
    name={Lichtsignalanlage},%
    description={Eine Ampelanlage, die den Verkehr an Knotenpunkten regelt. In dieser Arbeit Zielsystem für die Optimierung.}%
}

\newglossaryentry{fixedtime}{%
    name={Fixed-Time},%
    description={Eine klassische Steuerungsstrategie, bei der feste Signalzeiten für Grün- und Rotphasen verwendet werden. Diente als Baseline in den Experimenten.}%
}

\newglossaryentry{actuated}{%
    name={Actuated Control},%
    description={Eine dynamische Steuerung von Lichtsignalanlagen basierend auf Sensordaten. Diente als zweite Baseline, erwies sich jedoch in den Experimenten als ineffizient.}%
}

\newglossaryentry{queue}{%
    name={Queue},%
    description={Stausituation im Verkehr, gemessen als Anzahl der stoppenden Fahrzeuge. Gleichzeitig eine Reward-Funktion in den Experimenten.}%
}

\newglossaryentry{waitingtime}{%
    name={Wartezeit},%
    description={Die Zeit, die Fahrzeuge im Netz stillstehen oder stark verlangsamt sind.}%
}

\newglossaryentry{teleportation}{%
    name={Teleportation},%
    description={Eine in SUMO vorkommende Notfallmaßnahme, bei der Fahrzeuge versetzt werden, wenn sie in Deadlocks oder unrealistischen Situationen feststecken.}%
}

\newglossaryentry{deadlock}{%
    name={Deadlock},%
    description={Eine Verkehrssituation, in der sich Fahrzeuge gegenseitig blockieren und kein Fortschritt mehr möglich ist.}%
}

% --- Zukunftsperspektiven ---
\newglossaryentry{connectedvehicles}{%
    name={Connected Vehicles},%
    description={Fahrzeuge, die über drahtlose Kommunikation Daten mit Infrastruktur oder anderen Fahrzeugen austauschen können.}%
}

\newglossaryentry{autonomousvehicles}{%
    name={Autonome Fahrzeuge},%
    description={Fahrzeuge, die ohne menschliche Steuerung im Verkehr agieren und Entscheidungen selbstständig treffen.}%
}

\newglossaryentry{smartcity}{%
    name={Smart City},%
    description={Ein Konzept für Städte, die Informations- und Kommunikationstechnologien nutzen, um Effizienz, Nachhaltigkeit und Lebensqualität zu verbessern.}%
}


% Literatur
\usepackage{csquotes}
\usepackage[
  backend=biber,
  style=numeric,
  sorting=nyt,
  url=true,
  doi=true,
  isbn=false
]{biblatex}

\addbibresource{chapters/literatur.bib}

% Typografie und Layout
\usepackage{lmodern}
\usepackage{microtype}

% Bilder und Grafiken
\usepackage{graphicx}
\usepackage{caption}
\usepackage{subcaption}

% Aufzählungen
\usepackage{enumitem}

\usepackage[a4paper, top=4cm, bottom=4cm, left=4cm, right=4cm]{geometry}

% Mathematische Ausdrücke
\usepackage{amsmath}

% TikZ für grafische Elemente
\usepackage{tikz}
\usetikzlibrary{shapes, arrows.meta, positioning}

\usepackage{booktabs}
\usepackage{array}
\usepackage{float}

% Diagramme
\usepackage{pgfplots}
\usepackage{pgfplotstable}
\pgfplotsset{compat=1.18}

\definecolor{color1}{rgb}{0.165,0.294,0.600}   % rgba(42, 75, 153, 1)   #2A4B99
\definecolor{color2}{rgb}{0.196,0.745,0.918}   % rgba(50, 190, 234, 1)  #32BEEA
\definecolor{color3}{rgb}{0.667,0.149,0.235}   % rgba(170, 38, 60, 1)   #AA263C
\definecolor{color4}{rgb}{0.122,0.671,0.506}   % rgba(31, 171, 129, 1)  #1FAB81
\definecolor{color5}{rgb}{0.804,0.373,0.082}   % rgba(205, 95, 21, 1)   #CD5F15
\definecolor{color6}{rgb}{0.624,0.306,0.918}   % rgba(159, 78, 234, 1)  #9F4EEA


\tikzset{
  box/.style={
    rectangle,
    draw,
    thick,
    minimum width=7cm,
    minimum height=1.2cm,
    align=center,
    fill=blue!5
  },
  arrow/.style={
    -{Latex[length=3mm]},
    thick
  }
}

% Image folder
\graphicspath{{images/}}

\title{Optimierung einer Verkehrssimulation mit KI-basierten Agenten in SUMO}
\author{Sam Weiler}
\date{\small \today}


\begin{document}
\begin{titlepage}
	\centering

	% Hochschule und Fakultät
	{\large Hochschule Karlsruhe – Technik und Wirtschaft \\[0.3cm]}
	{\normalsize Fakultät für Informatik und Wirtschaftsinformatik}

	\vspace{2cm}

	% Titel
	{\LARGE\bfseries Multi-Agent Reinforcement Learning \\[0.2cm]
		für adaptive Ampelsteuerung in SUMO \par}

	\vspace{1.8cm}

	% Art der Arbeit
	{\large Bachelorarbeit \\[0.2cm]}
	{\normalsize im Studiengang Informatik}

	\vspace{1.25cm}

	\begin{tikzpicture}[scale=0.8, every node/.style={font=\small}]
		% Straßen
		\fill[gray!20] (-4,-0.6) rectangle (4,0.6);
		\fill[gray!20] (-0.6,-4) rectangle (0.6,4);

		% Fahrbahnmarkierungen (leicht verschoben, damit Striche in der Mitte aufeinander treffen)
		\draw[white, line width=1pt, dash pattern=on 6pt off 6pt]
		(-4.09,0) -- (4,0); % horizontal etwas nach links verschoben

		\draw[white, line width=1pt, dash pattern=on 6pt off 6pt]
		(0,-4.1) -- (0,4); % vertikal etwas nach unten verschoben

		% Ampeln
		\foreach \pos/\col in {(-0.9,0.9)/red, (0.9,-0.9)/red,
				(0.9,0.9)/green, (-0.9,-0.9)/green} {
				\fill[\col] \pos circle (0.15);
			}

		% Autos
		\filldraw[blue!60!black,rounded corners=1pt] (-2,-0.1) rectangle (-1.5,-0.5);
		\draw[-{Latex[length=2mm]},blue!60!black,thick] (-1.75,-0.3) -- ++(0.6,0);

		\filldraw[orange!80!black,rounded corners=1pt] (-0.1,2) rectangle (-0.5,1.5);

		\filldraw[green!60!black,rounded corners=1pt] (1.5,0.5) rectangle (2,0.1);
		\draw[-{Latex[length=2mm]},green!60!black,thick] (1.75,0.3) -- ++(-0.6,0);
	\end{tikzpicture}


	\vfill

	\vspace{1.25cm}

	% Autorendetails
	\begin{tabular}{rl}
		Vorgelegt von:  & Sam Weiler \\
		Studiengang:    & Informatik \\
	\end{tabular}

	\vspace{1cm}

	% Prüfer/Betreuer
	\begin{tabular}{rl}
		Erstprüfer:  & Prof. Dr. rer. nat. Patrick Baier \\
		Zweitprüfer: & Prof. Dr. Heiko Körner \\
	\end{tabular}

	\vspace{1cm}

	% Datum
	{\normalsize Karlsruhe, den \today}

\end{titlepage}


\tableofcontents
\newpage


\section{Einleitung}

\subsection{Motivation und Problemstellung}

Städte stehen zunehmend vor der Herausforderung, mit den wachsenden Anforderungen des urbanen Verkehrs zurechtzukommen. Die Zahl der Fahrzeuge im Individualverkehr steigt kontinuierlich\cite{umweltbundesamt-motorisierungsgrad, Kraftfahrt-Bundesamt}, was zu einer Verdichtung des Verkehrsaufkommens, insbesondere in städtischen Knotenpunkten, führt. Die daraus resultierenden Konsequenzen sind vielfältig: Verkehrsüberlastungen führen zu erhöhten Reisezeiten, steigenden \gls{emissions} und einer verminderten Lebensqualität für die Bevölkerung. \cite{Europäische-Umweltagentur} Darüber hinaus verursacht ineffizienter Verkehr einen erheblichen wirtschaftlichen Schaden durch Zeitverluste und Ressourcenverschwendung. \cite{umweltbundesamt-emissionen, Inrix-Traffic-Scorecard}

Ein zentraler Hebel zur Verbesserung dieser Situation liegt in der intelligenten Steuerung des Verkehrsflusses, insbesondere an Kreuzungen, an denen mehrere Verkehrsströme aufeinandertreffen. Die \gls{trafficlight}, die dort zum Einsatz kommen, arbeiten vielerorts noch nach starren, zeitbasierten Schaltplänen, die selten in Echtzeit auf veränderte Verkehrssituationen reagieren. \cite{baden-wuerttemberg} Auch adaptive Verfahren, wie verkehrsabhängige Steuerungen mittels \gls{induktionsschleifen} oder \gls{kamera}, sind in ihrer Reaktionsfähigkeit beschränkt. Damit bleibt ein enormes Potenzial zur Effizienzsteigerung ungenutzt. \cite{Bundesanstallt}

Vor diesem Hintergrund bietet die Kombination moderner Simulationstechniken mit Methoden der künstlichen Intelligenz, insbesondere dem \gls{rl}, eine vielversprechende Alternative. Reinforcement Learning ist ein lernbasiertes Verfahren, bei dem ein \gls{agent} durch Interaktion mit einer Umgebung eine optimale Strategie zur Maximierung eines definierten Belohnungskriteriums erlernt. Die Anwendung dieses Konzepts auf Ampelsteuerungen erlaubt es, reaktive, datengestützte Systeme zu entwickeln, die dynamisch auf die aktuelle Verkehrssituation reagieren und dabei auf langfristige Effizienz optimiert sind.

Zur Erprobung solcher Verfahren eignet sich die Verkehrssimulationsumgebung \gls{sumo}, eine quelloffene, modular aufgebaute Plattform, die es ermöglicht, Verkehrsflüsse realitätsnah zu modellieren und zu analysieren. In Kombination mit dem Framework \gls{sumo-rl}\cite{sumo-rl}, das eine Brücke zwischen SUMO und gängigen Machine-Learning-Frameworks wie \gls{tensorflow} oder \gls{pytorch} schlägt, lassen sich Reinforcement-Learning-Agenten direkt in die Simulationsumgebung einbetten. Diese können dann die Steuerung einzelner Ampelanlagen übernehmen und ihre Strategien durch wiederholte Simulation iterativ verbessern.

\subsection{Zielsetzung der Arbeit}

Ziel dieser Bachelorarbeit ist es, eine auf Reinforcement Learning basierende Steuerung von Ampelanlagen innerhalb eines realitätsnahen, simulierten städtischen Verkehrsnetzes zu entwickeln, umzusetzen und zu evaluieren. Als Modellregion dient ein ausgewählter, stark befahrener Bereich der Stadt \gls{Karlsruhe}, dessen Straßennetz mithilfe von \gls{osm}-Daten und Verkehrsdaten von Institutionen wie \gls{LUBW}, \gls{mobidatabw} und der \gls{bast} realitätsnah abgebildet wird. \cite{osm,mobidata, lubw}

Die Arbeit verfolgt einen anwendungsorientierten Ansatz: Es wird ein vollständiges System aufgebaut, in dem einzelne Ampelkreuzungen durch RL-Agenten gesteuert werden. Diese erhalten als Eingabe Informationen zur aktuellen Verkehrslage, etwa Fahrzeuganzahl, Wartezeiten oder Stauentwicklungen, und geben als Ausgabe Ampelschaltbefehle zurück. Ziel ist es, durch Training in der Simulation eine Steuerungsstrategie zu entwickeln, die relevante Zielgrößen wie die durchschnittliche Wartezeit, den Verkehrsfluss oder die Anzahl von Fahrzeugstopps optimiert.

Ein positiver Untersuchungsverlauf könnte zeigen, dass bestehende Straßennetze effizienter genutzt werden können, ohne kostspielige Neubauten oder Erweiterungen. Die verbesserte Auslastung bestehender Infrastruktur spart Kosten, reduziert Flächenversiegelung und mindert Umweltbelastung durch Verkehrsvermeidung. Außerdem wäre ein solches adaptive System klimafreundlicher als starre Ampelsteuerungen.

Darüber hinaus soll die Arbeit systematisch untersuchen, wie sich unterschiedliche Modellierungsentscheidungen (z.B. Wahl der Belohnungsfunktion, Anzahl der gesteuerten Agenten, Parametrisierung der Umgebung) auf das Verhalten und die Leistungsfähigkeit der lernenden Agenten auswirken. Die gewonnenen Erkenntnisse sollen kritisch reflektiert und mit konventionellen, nicht-adaptiven Steuerungsstrategien verglichen werden.
\subsection{Begrenzung des Projektumfangs}

Trotz des Anspruchs auf Realitätsnähe handelt es sich bei der vorliegenden Arbeit um ein simulationsbasiertes Projekt mit bewusst gewähltem Fokus. Die Umsetzung erfolgt ausschließlich in der Simulationsumgebung SUMO und basiert auf öffentlich zugänglichen Geodaten (OpenStreetMap) sowie begrenzt verfügbaren Verkehrsdaten von staatlichen und kommunalen Institutionen. Eine vollständige Abbildung aller Aspekte des realen Straßenverkehrs ist damit weder angestrebt noch möglich. \cite{sumo-doc}

Insbesondere ergeben sich folgende Einschränkungen:

\begin{itemize}
    \item \textbf{Eingeschränkte Datenverfügbarkeit:} Nicht alle für eine realitätsnahe Verkehrsmodellierung relevanten Daten liegen in ausreichender Qualität oder Auflösung vor. Exakte Ampelschaltzeiten, Fußgängerfrequenzen oder dynamische Verkehrsdaten zu Stoßzeiten sind teilweise nicht öffentlich zugänglich oder nur unvollständig. Dazu kommt, dass Kommunen teilweise bewusst den Verkehr lenken,etwa durch Zufahrtsbeschränkungen oder Verkehrsberuhigungszonen, was oft nicht öffentlich kommuniziert wird. \cite{umweltbundesamt-umweltzonen}

    \item \textbf{Vereinfachte Modellierung der Umgebung:} In der Simulation wird angenommen, dass alle Verkehrsteilnehmer (Fahrzeuge, Fußgänger, Radfahrer) durch die Agenten präzise erfasst werden können, eine Annahme, die in der Realität durch technische und datenschutzrechtliche Hürden nicht haltbar ist. Moderne Systeme arbeiten hier mit Datenschutz‑mechanismen, aber eine flächendeckende, genaue Erfassung ist unerlässlich, aber derzeit technisch und rechtlich nicht umsetzbar. \cite{DSGVO} Dies wird in der Arbeit berücksichtigt, vor allem bei realiätsnahem Modelltraining.

    \item \textbf{Städtebauliche Verkehrslenkung:} In der Realität regeln Städte Verkehrsflüsse z.B. durch Low-Traffic-Neighbourhoods, Zufahrtsbeschränkungen oder geregelte Zuflusssteuerung, um bestimmte Stadtbereiche zu entlasten. \cite{Low-traffic-Amsterdam} Solche Maßnahmen sind jedoch in der Simulationsumgebung nicht dynamisch abbildbar, da nur externe Ampelagenten kontrollieren und keine zonale Steuerungslogik abgebildet wird.

    \item \textbf{Begrenzter räumlicher und zeitlicher Umfang:} Simuliert wird lediglich ein ausgewählter Ausschnitt des Karlsruher Straßennetzes und nur für definierte Zeitabschnitte. Eine vollständige Tag‑Nacht‑Modellierung liegt außerhalb des Umfangs.

    \item \textbf{Trainings- und Evaluierungsgrenzen:} Reinforcement‑Learning‑Agenten benötigen viele Trainingszyklen. Die in dieser Arbeit verwendete Hardware limitiert Trainingsdauer und Modellkomplexität.
\end{itemize}

Diese bewusste Eingrenzung ermöglicht es, sich auf die technische Umsetzbarkeit und das methodische Vorgehen zu konzentrieren. Dennoch sind die gewonnenen Erkenntnisse relevant, sie liefern zentrale Einsichten in die Wirksamkeit von \gls{ki}‑basierten Verkehrssteuerungssystemen und können als Grundlage für weiterführende Forschung dienen.

\subsection{Wissenschaftliche und gesellschaftliche Relevanz}

Die Kombination von KI und Verkehrssteuerung ist nicht nur ein hochaktuelles Forschungsthema, sondern besitzt auch ein erhebliches Potenzial für den realweltlichen Einsatz. \cite{KI4LSA} Durch die Integration lernfähiger Steuerungssysteme in bestehende Verkehrsmanagementlösungen könnten Städte künftig dynamischer, effizienter und umweltfreundlicher agieren. Die hier behandelte Arbeit leistet einen Beitrag zur Untersuchung der technischen Machbarkeit sowie der Leistungsfähigkeit solcher Systeme unter realitätsnahen Bedingungen.

Gleichzeitig dient die Arbeit als Beispiel für den Einsatz moderner Methoden der Informatik in einem interdisziplinären Anwendungsfeld. Sie schlägt die Brücke zwischen Verkehrsingenieurwesen, Datenanalyse und maschinellem Lernen und eröffnet damit Perspektiven für eine zukunftsweisende Gestaltung urbaner Infrastrukturen.

\subsection{Aufbau der Arbeit}

Die Arbeit ist in sieben Kapitel unterteilt:

\begin{itemize}
    \item Kapitel 2 stellt die theoretischen Grundlagen der Arbeit dar. Es werden die Funktionsweise von SUMO, die Prinzipien des Reinforcement Learning sowie die zugrundeliegenden technischen Komponenten erläutert. Auch verwandte Arbeiten werden kritisch betrachtet.
    \item Kapitel 3 widmet sich den Datenquellen und der Modellierungsgrundlage. Es werden sowohl die verwendeten Geodaten als auch Verkehrszählungen, Ampelschaltpläne und Annahmen beschrieben.
    \item Kapitel 4 beschreibt die methodische Vorgehensweise bei der Erstellung des Simulationsmodells, der Formulierung des Lernproblems, der Wahl der Trainingsstrategie und der technischen Umsetzung.
    \item Kapitel 5 präsentiert die Ergebnisse der Simulationen und stellt sie in Bezug zur gewählten Zielsetzung. Es erfolgt eine quantitative und qualitative Auswertung der Agentenleistung.
    \item Kapitel 6 diskutiert zentrale Herausforderungen und Limitationen der Arbeit, sowohl methodisch als auch datenbezogen.
    \item Kapitel 7 fasst die wesentlichen Erkenntnisse zusammen und gibt einen Ausblick auf weiterführende Forschungsansätze und Anwendungsoptionen.
\end{itemize}

\section{Hintergrund und Stand der Technik}

\subsection{Urbane Verkehrssysteme und Verkehrssteuerung}

Die urbane Verkehrssteuerung umfasst alle Maßnahmen zur Regelung, Lenkung und Optimierung von Verkehrsflüssen innerhalb städtischer Räume. Ziel ist es, den Verkehrsfluss effizient zu gestalten, Staus zu vermeiden, die Sicherheit aller Verkehrsteilnehmer zu erhöhen sowie Emissionen und Lärm zu reduzieren. Klassische Steuerungsmechanismen basieren häufig auf festen Zeitplänen oder einfachen verkehrsabhängigen Regeln, z.\,B. durch Induktionsschleifen oder Detektoren gesteuerte Ampelphasen.

Mit dem Aufkommen neuer Technologien und wachsender Mobilitätsdaten entstehen zunehmend datenbasierte und dynamische Steuerungsansätze. Dazu gehören adaptive Lichtsignalsteuerungen, vernetzte Fahrzeuge (V2X-Kommunikation) und erste Pilotprojekte mit KI-gesteuerten Verkehrsmanagementsystemen. Dennoch sind viele Systeme in der Praxis noch unflexibel oder schwer skalierbar.

\subsection{Simulation urbaner Mobilität mit SUMO}

\textit{Simulation of Urban MObility} (SUMO) ist ein quelloffener, mikroskopischer Verkehrs-Simulator, der ursprünglich vom Deutschen Zentrum für Luft- und Raumfahrt (DLR) entwickelt wurde. SUMO erlaubt die detaillierte Modellierung individueller Fahrzeuge, Straßeninfrastruktur, Ampelschaltungen sowie Fahrverhalten.

Besonders relevant für diese Arbeit sind folgende Merkmale:

\begin{itemize}
    \item \textbf{Mikroskopische Modellierung:} Jedes Fahrzeug wird als individuelles Objekt simuliert. Parameter wie Geschwindigkeit, Abstand oder Spurwechselverhalten sind individuell konfigurierbar.
    \item \textbf{Flexible Netzdefinition:} Verkehrsnetze lassen sich aus OpenStreetMap-Daten sowie aus Shapefiles oder VISUM-Modellen mit dem Tool \texttt{netconvert} erzeugen. Netzdateien können auch mit \texttt{netedit} visuell editiert werden.
    \item \textbf{Nachfragegenerierung:} Fahrpläne und Routen lassen sich mit Tools wie \texttt{activitygen}, \texttt{randomTrips}, \texttt{od2trips} oder \texttt{duarouter} erzeugen – basierend auf statistischen oder echten OD-Matrizen.
    \item \textbf{Multimodalität:} SUMO unterstützt neben Pkw auch Busse, Fahrräder, Fußgänger sowie den öffentlichen Nahverkehr. Ampeln können für alle Verkehrsarten gleichzeitig modelliert werden.
    \item \textbf{Emissionsmodellierung:} Mit Hilfe von integrierten HBEFA-Tabellen (Version 4) kann SUMO CO\textsubscript{2}-, NO\textsubscript{x}- und Feinstaubemissionen simulieren und ausgeben.
    \item \textbf{Steuerbare Ampelanlagen:} Lichtsignalanlagen können sowohl mit festen Programmen als auch dynamisch über die TraCI-Schnittstelle gesteuert werden.
    \item \textbf{Reproduzierbarkeit und Kontrolle:} SUMO ist vollständig deterministisch, was es ideal für kontrollierte Experimente und das Training von KI-Agenten macht.
    \item \textbf{Visualisierung und Debugging:} Die SUMO-GUI und das Tool \texttt{sumo-gui} ermöglichen eine grafische Darstellung von Netz, Fahrzeugen, Ampelphasen und Simulationsergebnissen.
\end{itemize}

Die SUMO-Toolchain bietet damit alle notwendigen Komponenten für die Entwicklung, Analyse und Auswertung urbaner Verkehrsszenarien und stellt eine erprobte Plattform für KI-gestützte Steuerungsexperimente dar.

\subsection{Verstärkendes Lernen (Reinforcement Learning)}

Reinforcement Learning (RL) ist ein Teilgebiet des maschinellen Lernens, bei dem ein Agent durch Interaktion mit einer Umgebung lernt, optimale Handlungen auszuführen. Dabei verfolgt er das Ziel, eine kumulative Belohnung zu maximieren.

Ein RL-Prozess wird typischerweise als Markov Decision Process (MDP) beschrieben und besteht aus folgenden Komponenten:

\begin{itemize}
    \item \textbf{Zustand $s$ (state):} Eine Repräsentation der aktuellen Situation der Umgebung.
    \item \textbf{Aktion $a$ (action):} Eine Entscheidung oder Handlung, die der Agent im Zustand $s$ trifft.
    \item \textbf{Belohnung $r$ (reward):} Ein numerischer Wert, der die Güte der Aktion bewertet.
    \item \textbf{Policy $\pi$:} Eine Strategie, die angibt, welche Aktion in welchem Zustand gewählt wird.
\end{itemize}

Der Agent interagiert mit der Umgebung, beobachtet den Zustand, wählt eine Aktion, erhält eine Belohnung und gelangt in einen neuen Zustand. Durch viele Wiederholungen lernt er, welche Entscheidungen langfristig die besten sind.

Wichtige Algorithmen, die in dieser Arbeit potenziell relevant sind, sind:

\begin{itemize}
    \item \textbf{Q-Learning:} Modellfreies, off-policy Lernverfahren zur Annäherung an optimale Aktionen.
    \item \textbf{DQN (Deep Q-Network):} Kombination von Q-Learning mit neuronalen Netzen.
    \item \textbf{PPO (Proximal Policy Optimization):} Policy-basierter RL-Ansatz mit stabiler Optimierung.
\end{itemize}

\subsection{SUMO-RL: Architektur und Funktionalität}

\texttt{sumo-rl} ist ein Python-Framework, das SUMO mit Reinforcement Learning verbindet. Es basiert auf der \texttt{gymnasium}-Schnittstelle und abstrahiert typische Aufgaben wie die Definition von Beobachtungen, Aktionen und Belohnungen für RL-Agenten. Die Umgebung wird durch die Klasse \texttt{SumoEnvironment} bereitgestellt.

Zentrale Eigenschaften von \texttt{sumo-rl}:\cite{sumo-rl}

\begin{itemize}
    \item \textbf{TraCI-Integration:} Ermöglicht über das Traffic Control Interface zur Laufzeit den Zugriff auf Fahrzeugdaten, Ampelphasen, Fahrzeugwarteschlangen u.\,v.\,m.
    \item \textbf{Ein- und Mehragentenunterstützung:} \texttt{sumo-rl} unterstützt sowohl Single-Agent-Setups als auch Multi-Agent-Steuerung über die PettingZoo-API. Jeder gesteuerte Knoten im Netz kann einem eigenen Agenten zugewiesen werden.
    \item \textbf{Beobachtungen:} Die Umgebung liefert Beobachtungsvektoren mit kodierter Ampelphase, Rückstaulänge, Anzahl wartender Fahrzeuge und Fahrzeugdichte je Spur.
    \item \textbf{Aktionen:} Die Agenten treffen diskrete Entscheidungen über Phasenwechsel, wobei \texttt{delta\_time}, \texttt{yellow\_time} und \texttt{min\_green} die zeitliche Dynamik definieren.
    \item \textbf{Belohnungsfunktionen:} Der Standard-Reward basiert auf der Differenz kumulierter Wartezeiten. Eigene Funktionen können bei Initialisierung übergeben werden.
    \item \textbf{Kompatibilität:} Das Framework ist kompatibel mit Stable-Baselines3, PyTorch, TensorFlow, RLlib und anderen gängigen ML-Frameworks.
\end{itemize}

Beispielhafte Initialisierung:

\begin{verbatim}
env = SumoEnvironment(
    net_file='net.net.xml',
    route_file='routes.rou.xml',
    use_gui=True,
    reward_fn='diff-waiting-time',
    single_agent=True,
    delta_time=5,
    yellow_time=2,
    min_green=5
)
\end{verbatim}

\subsection{Verwandte Arbeiten}

In den letzten Jahren wurden zunehmend Studien veröffentlicht, die KI-Methoden zur Optimierung der Verkehrssteuerung einsetzen. Eine Auswahl relevanter Forschungsansätze:

\begin{itemize}
    \item \textbf{Wei et al. (2019):} Einsatz von Deep Q-Learning zur Optimierung einer einzelnen Ampel in SUMO mit signifikantem Rückgang der Wartezeiten \cite{wei2019}.
    \item \textbf{Chu et al. (2020):} Untersuchung von Multi-Agent-Ansätzen mit Deep RL zur Steuerung großflächiger Ampelnetze \cite{chu2020}.
    \item \textbf{Zheng et al. (2019):} Einführung eines Lernverfahrens zur Koordination konkurrierender Phasen bei Ampelsteuerungen mit Hilfe von SUMO \cite{zheng2019}.
\end{itemize}

Diese Arbeiten zeigen, dass RL-basierte Methoden das Potenzial haben, bestehende Systeme zu übertreffen – sowohl bei einfachen als auch bei komplexeren Szenarien. Die vorliegende Arbeit knüpft an diesen Forschungsstand an und erweitert ihn um eine Anwendung auf reale Geodaten aus Karlsruhe sowie eine methodische Evaluation.



\section{Datenquellen und Modellierungsgrundlage}
\label{sec:datenquellen_und_modellgrundlage}
\subsection{OpenStreetMap als Grundlage für das Verkehrsmodell}

Das Verkehrsnetz für die Simulation basiert auf öffentlich verfügbaren Geodaten der Plattform OpenStreetMap (OSM). OSM bietet eine frei zugängliche, kollaborativ gepflegte Datenbank, die detaillierte Informationen zu Straßenverläufen, Kreuzungen, Fahrspuren, Tempolimits und teilweise zu Ampelanlagen enthält. Diese Eigenschaften machen OSM zu einer geeigneten Grundlage für Verkehrssimulationen mit SUMO. \cite{osm, osm-git}

Zur Erstellung des Netzes wurde ein Ausschnitt des Straßennetzes der Stadt Karlsruhe exportiert, der einen stark frequentierten urbanen Bereich mit mehreren signalgesteuerten Kreuzungen umfasst. Der betrachtete Bereich liegt zwischen 49{,}00738,\textdegree{}N und 49{,}01523,\textdegree{}N sowie 8{,}38589,\textdegree{}E und 8{,}40050,\textdegree{}E und deckt unter anderem die Reinhold-Frank-Straße, das Mühlburger Tor und angrenzende Hauptverkehrsachsen ab. Der Export erfolgte als \texttt{.osm}-Datei über den Geofabrik-Downloaddienst bzw. mit dem Tool \gls{Josm}. \cite{osm-export, josm} Die anschließende Konvertierung in das SUMO-Format erfolgte mit dem Programm \texttt{netconvert} (Version 1.19.0), einem Teil der SUMO-Toolchain. \cite{sumo-tools} Hierbei wurden relevante Parameter wie Straßentypen, Fahrspuren, Prioritäten und erlaubte Abbiegevorgänge berücksichtigt. Als Typemap kam \texttt{osmNetconvert.typ.xml} zum Einsatz, um realitätsnahe Geschwindigkeiten und Fahrspuren zuzuweisen.

Das resultierende Verkehrsnetz umfasst 1.379 definierte Knotenpunkte (\textit{junctions}), 1.919 Straßenkanten (\textit{edges}) sowie insgesamt 5.310 modellierte Fahrstreifen (\textit{lanes}). Darüber hinaus konnten 17 signalgesteuerte Kreuzungen mit Lichtsignalanlagen (\textit{traffic lights}) identifiziert (siehe Algorithmus~\ref{alg:find_valid_tls}) werden, die als Steuerungspunkte für das spätere Training der Reinforcement-Learning-Agenten dienen.

Zusätzliche Informationen wie Ampeldefinitionen und Vorfahrtsregeln manuell über das Tool \texttt{netedit} ergänzt oder angepasst, um die Netzrealität weiter zu verfeinern. Dabei wurden insbesondere fehlerhafte Knotenbeziehungen bereinigt sowie isolierte Netzteile entfernt. Die finale \texttt{.net.xml}-Datei bildet die topologische und funktionale Grundlage für alle weiteren Simulationsschritte.

\begin{figure}[H]
    \centering
    \includegraphics[width=0.7\textwidth]{images/karlsruhe_net.png}
    \caption{Visualisierung des aus OSM generierten SUMO-Netzes (\gls{sumogui}).}
    \label{fig:sumo_network}
\end{figure}

\begin{figure}[H]
    \centering
    \includegraphics[width=0.6\textwidth]{images/karlsruhe_osm.png}
    \caption{Screenshot des ursprünglichen OpenStreetMap-Ausschnitts (OpenStreetMap).}
    \label{fig:osm_screenshot}
\end{figure}

Die Wahl von OpenStreetMap als Datenquelle gewährleistet eine offene, reproduzierbare und erweiterbare Modellierungsbasis. Jedoch bringt die Nutzung von OSM-Daten auch einige Einschränkungen mit sich, die bei der Modellierung berücksichtigt werden müssen:\cite{osm-export, osm, osm-Guide}

\begin{itemize}
    \item \textbf{Uneinheitlicher Detaillierungsgrad:} Die Erfassungstiefe variiert regional stark, was dazu führt, dass z.\,B. Tempolimits, Fahrspuren oder Abbiegebeschränkungen an vielen Stellen fehlen oder unvollständig sind.
    \item \textbf{Fehlende Ampel- und Signalsteuerungsdaten:} OSM enthält in der Regel keine vollständigen Angaben zu Ampelphasen, Umlaufzeiten oder koordinierter Schaltung. SUMO kann zwar aus heuristischen Annahmen Standardampeln generieren, diese weichen jedoch potenziell stark von der realen Steuerung ab.
    \item \textbf{Keine garantierte Netzvollständigkeit:} Besonders kleinere Straßen, private Zufahrten oder temporäre Baustellen sind häufig nicht oder nur unzureichend erfasst. Zudem treten beim Zuschnitt von Kartenausschnitten an den Netzrändern regelmäßig unvollständige Knoten oder isolierte Kanten auf.
    \item \textbf{Abweichende Modellierungskonzepte:} In OSM werden parallele Fahrbahnen oder getrennte Richtungsfahrbahnen oft als unabhängige Wege modelliert. Ohne geeignete Nachbearbeitung kann dies zu unnötigen Knoten und ineffizientem Verkehrsverhalten führen.
    \item \textbf{Abhängig von Typemap- und Importoptionen:} Die Interpretation der OSM-Tags erfolgt in SUMO durch sogenannte Typemaps, die z.\,B. Tempolimits und Spuranzahl je nach Straßentyp zuweisen. Ohne geeignete Typemap kann das Verhalten nicht der Realität entsprechen. \cite{netconvert}
\end{itemize}

\textbf{Fazit:} Insgesamt erlaubt OSM trotz dieser Limitationen den Aufbau eines funktionalen Verkehrsnetzes für mikroskopische Simulationen, sofern der Import sorgfältig konfiguriert und die resultierenden Daten kritisch hinterfragt und gegebenenfalls manuell nachbearbeitet werden.

\subsection{Verfügbare Verkehrsdaten}
Zur Kalibrierung und Validierung der Simulation sind verlässliche Verkehrsdaten unerlässlich. In Baden-Württemberg stehen hierfür mehrere öffentliche sowie kommerzielle Quellen zur Verfügung. Diese umfassen Informationen über Verkehrsstärken, Fahrzeugzusammensetzung, Reisezeiten und Störungen im Straßenverkehr. Im Folgenden werden die wichtigsten Quellen sowie die für das vorliegende Projekt relevanten Verkehrszählungen zusammengefasst.

\subsubsection{Öffentliche Datenquellen: LUBW, MobiData BW, Straßenverkehrszentrale, BASt}
Die LUBW stellt aggregierte Verkehrszählungen im Rahmen automatischer Straßenverkehrszählungen bereit. Diese umfassen Tagesmittelwerte sowie jahreszeitliche Schwankungen für verschiedene Fahrzeugkategorien. Die Daten der \gls{SVZBW} liefern zudem Echtzeitinformationen zu Störungen, Baustellen und Verkehrsfluss.\cite{Verkehrszählungen_Baden-Württemberg,bast,lubw,baden-wuerttemberg,svzbw}

Über die Plattform MobiData BW werden offene Mobilitätsdaten gebündelt bereitgestellt, darunter auch historische Detektordaten und OpenTraffic-Feeds. Die BASt wiederum veröffentlicht bundesweite Zähldaten, insbesondere für überörtliche Straßen. \cite{bast}

Diese öffentlichen Quellen bilden eine solide Grundlage für die realitätsnahe Modellierung des Verkehrsaufkommens, sind jedoch teilweise nur in aggregierter Form oder mit begrenzter räumlicher Auflösung verfügbar.

\subsubsection{Stationäre Zählstellen in Karlsruhe und Umgebung}
\label{sec:zaehlstellen-karlsruhe}
Eine besonders wertvolle Datenquelle zur realitätsnahen Modellierung des Verkehrsaufkommens stellen die stationären Zählstellen des Landes Baden-Württemberg dar. Diese liefern standardisierte Tagesverkehrswerte, getrennt nach Fahrzeugklassen.

Im direkten Untersuchungsgebiet, der Reinhold-Frank-Straße in Karlsruhe, befindet sich eine automatische Dauerzählstelle. Die dort erfassten Werte für den Zeitraum vom 1.1. bis 20.6.2025 lauten: \cite{Verkehrszählungen_Baden-Württemberg}

\begin{itemize}
    \item \textbf{\gls{kfz}:} 21.300 Fahrzeuge/Tag
    \item \textbf{PKW:} 20.500 Fahrzeuge/Tag
    \item \textbf{\gls{snfz}:} 120 Fahrzeuge/Tag
\end{itemize}

Diese Messwerte stimmen gut mit den aus den äußeren Zufahrtsachsen abgeleiteten Schätzungen überein. Um das Verkehrsaufkommen plausibel zu quantifizieren, wurden zusätzlich acht zentrale Zählstellen aus dem Jahr 2023 entlang wichtiger Ein- und Ausfallstraßen berücksichtigt. Sie bilden die Grundlage für die Annahmen über den täglichen Verkehr, der potenziell durch das untersuchte innerstädtische Netz fließt:

\begin{table}[H]
    \centering
    \caption{Verkehrszählungen in und um Karlsruhe (DTV, Jahr 2023) \cite{Dauerzählstellen_Ergebnisse}}
    \begin{tabular}{|l|l|r|r|r|}
        \hline
        \textbf{Zufahrt} & \textbf{Zählstellenbeschreibung}      & \textbf{KFZ/Tag} & \textbf{SV/Tag} & \textbf{Gesamt} \\
        \hline
        B10 West         & Rheinbrücke / Entenfang               & 62.102           & 6.159           & 68.261          \\
        B36 Neureut      & Neureuter Str. / Ausfahrt Neureut Süd & 35.165           & 1.712           & 36.877          \\
        B36 Nord         & Eggenstein / Neureut                  & 28.595           & 1.361           & 29.956          \\
        L605 Nord        & Weißes Haus / Eggenstein              & 14.563           & 220             & 14.783          \\
        B36 Süd          & Rheinstetten / Innenstadt             & 24.239           & 1.487           & 25.726          \\
        B36 Mörsch       & Mörsch / Forchheim                    & 26.841           & 1.531           & 28.372          \\
        L605 Süd         & Ettlingen / Bulacher Kreuz            & 65.816           & 3.474           & 69.290          \\
        B10 Ost          & Durlach (A5) / Innenstadt             & 28.555           & 913             & 29.468          \\
        \hline
    \end{tabular}
    \caption*{\footnotesize Hinweis: KFZ = Leichtverkehr (Pkw, Lieferwagen, Motorräder);
        SV = Schwerverkehr (Lkw, Busse, schwere Nutzfahrzeuge);
        Gesamt = Summe aus KFZ und SV.}
    \label{tab:zaehlstellen}
\end{table}

\begin{figure}[H]
    \centering
    \includegraphics[width=0.95\textwidth]{images/zaehlstellenkarte.png}
    \label{fig:zaehlstellenkarte}
    \vspace{0.3em}
    \begin{minipage}{0.9\linewidth}
        \footnotesize \textbf{Legende:}
        \textcolor{yellow}{\large$\bullet$} Temporäre Zählstellen \quad
        \textcolor{red}{\large$\bullet$} Dauerzählstellen \quad
        \textcolor{gray}{\large$\bullet$} Manuelle Zählstellen
    \end{minipage}
    \caption{Lage der Dauerzählstellen im Raum Karlsruhe (Quelle: MobiData BW \cite{mobidata_karte}).}
\end{figure}

Diese externen Zuflüsse bilden die Grundlage für realistische Eingangsströme in der Simulation. Sie versorgen das Untersuchungsgebiet direkt und ergeben ein plausibles Verkehrsaufkommen von etwa 20.000 bis 40.000 Fahrzeugen pro Tag, was mit den Messungen in der Reinhold-Frank-Straße übereinstimmt.

Die Zähldaten erlauben es, die Fahrzeugströme in SUMO proportional zu den realen Verhältnissen abzubilden und unterstützen zugleich die spätere Kalibrierung und Validierung der Szenarien.
\subsubsection{Kommerzielle APIs: TomTom, Google Maps}

Ergänzend zu den öffentlichen Datenquellen bieten kommerzielle Anbieter wie TomTom und Google über Programmierschnittstellen (APIs) hochaufgelöste Echtzeit- und Historikdaten an. Diese umfassen unter anderem:

\begin{itemize}
    \item Durchschnittliche Fahrgeschwindigkeiten nach Wochentag und Uhrzeit,
    \item Verkehrsdichte und Stauinformationen,
    \item Prognosen basierend auf anonymisierten Bewegungsdaten.
\end{itemize}

Der Zugriff auf diese APIs ist in der Regel kostenpflichtig oder durch Nutzungsbeschränkungen limitiert. Sie ermöglichen eine deutlich feinere zeitliche und räumliche Auflösung, was für die Modellierung und spätere Optimierung des Verkehrsflusses mittels KI von Vorteil sein könnte.

Für die vorliegende Arbeit wurden diese kommerziellen Angebote nicht genutzt. Die Modellierung basiert ausschließlich auf offenen Datenquellen wie OSM sowie auf Google Maps für einzelne Standortrecherchen. \cite{googlemaps, tomtom}

\subsection{Modellierung der Ampelschaltungen}

Für eine realitätsnahe Simulation spielt die Modellierung der Lichtsignalsteuerung eine zentrale Rolle. Ampelanlagen beeinflussen maßgeblich den Verkehrsfluss an Knotenpunkten und sind daher ein zentraler Bestandteil der Simulationslogik. \cite{Sumo-tls}

\subsubsection{Verfügbare Daten und Herausforderungen}

In den öffentlich zugänglichen OSM-Daten sind Ampelanlagen in der Regel lediglich als Punktobjekte an Kreuzungen vermerkt. Informationen zu Phasenplänen, Umlaufzeiten oder koordinierter Schaltung fehlen vollständig. Auch von Seiten der Stadt Karlsruhe oder anderer kommunaler Stellen liegen keine detaillierten Steuerungsdaten vor, da diese in der Regel nicht öffentlich zugänglich sind. \cite{Sumo-osm}

Eine eigene systematische Erfassung der Schaltzeiten wäre zwar prinzipiell möglich, hätte jedoch einen erheblichen Zeitaufwand bedeutet und wäre aufgrund der dynamischen, nicht-statischen Signalsteuerungen (z.\,B. verkehrsabhängige Phasen) methodisch schwer zuverlässig umzusetzen gewesen.

\subsubsection{Vereinfachte Modellierung}

Aus diesen Gründen wurde anfangs auf eine synthetische Modellierung zurückgegriffen. Mittels netgenerate wurde ein synthetisches Netz generiert und abenfalls testweise Modelle trainiert. Dies erwies sich als sehr simpel, wegen geringer Komplexität. SUMO bietet hierfür die Möglichkeit, sogenannte \texttt{tlLogic}-Blöcke manuell oder automatisch zu definieren, die verschiedene Phasenfolgen und Zeitparameter enthalten. In der vorliegenden Arbeit wurde auf Standardampelprogramme zurückgegriffen, wie sie in SUMO generisch verwendet werden, um testweise eine vereinfachte Lichtsignalsteuerung zu modellieren. Diese erlaubt die spätere Umsetzung des realen karlsruher Netzes. \cite{Sumo-tls,netgenerate}

\section{Methodik}


\subsection{Untersuchungsregion und Datenbasis}

\subsubsection{Auswahl der Untersuchungsregion}

Für die Anwendung und Evaluation der KI-basierten Verkehrssteuerung wurde ein Ausschnitt des innerstädtischen Straßennetzes von Karlsruhe gewählt. Die Auswahl fiel auf ein \hyperref[fig:zaehlstellenkarte]{Gebiet} rund um die Reinhold-Frank-Straße und das Mühlburger Tor, das durch hohe Verkehrsdichte, komplexe Knotenpunkte und mehrere signalgesteuerte Kreuzungen gekennzeichnet ist. Der gewählte Bereich liegt geografisch zwischen 49{,}006947\,\textdegree{}N und 49{,}015602\,\textdegree{}N sowie 8{,}380176\,\textdegree{}E und 8{,}403887\,\textdegree{}E und deckt mehrere stark frequentierte Hauptachsen ab.

Die Entscheidung für diese Region basiert auf folgenden Kriterien (siehe Kapitel~\ref{sec:datenquellen_und_modellgrundlage}):

\begin{itemize}
    \item \textbf{Hohe Verkehrsbedeutung:} Das Gebiet stellt einen wichtigen innerstädtischen Verkehrsraum dar.
    \item \textbf{Bekanntes Stauaufkommen:} Die Reinhold-Frank-Straße ist in der Stadtbevölkerung für regelmäßige Verkehrsstaus bekannt, insbesondere zu Stoßzeiten. \cite{Stau_in_Karlsruhe}
    \item \textbf{Verfügbarkeit realer Verkehrszähldaten:} Eine automatische Dauerzählstelle erhebt dort täglich Verkehrsdaten. Für den Zeitraum vom 01.01.2025 bis 20.6.2025 wurden durchschnittlich 21.300 Kfz/Tag erfasst. \cite{Verkehrszählungen_Baden-Württemberg,Dauerzählstellen_Ergebnisse}
    \item \textbf{Zusätzliche Zähldaten angrenzender Hauptverkehrsstraßen:} Zählstellen an der B10, B36, L605 und in Durlach liefern ergänzende Werte zur Plausibilisierung des Gesamtverkehrsflusses.
    \item \textbf{Vorhandensein mehrerer Ampelanlagen:} Im Netz befinden sich 17 signalgesteuerte Kreuzungen, geeignet für RL-gesteuerte Steuerungsexperimente (siehe Kapitel \ref{alg:find_valid_tls}).
    \item \textbf{Gute Abgrenzbarkeit:} Das Gebiet ist topologisch geschlossen und in SUMO sauber simulierbar.
    \item \textbf{Verfügbarkeit von Geodaten:} Die Region ist in OpenStreetMap detailliert kartiert. \cite{osm-export}
\end{itemize}

\subsubsection{Verfügbare Verkehrszähldaten}

Für die Kalibrierung und Validierung der Verkehrssimulation wurden verschiedene reale Zähldatenquellen aus dem Raum Karlsruhe herangezogen. Hauptquelle war dabei die offene Mobilitätsdatenplattform des Landes Baden-Württemberg \cite{mobidata_stunden}. Dort werden automatisiert erfasste Stundenwerte stationärer Dauerzählstellen veröffentlicht, die eine fein aufgelöste Analyse von Verkehrsverläufen ermöglichen.

Konkret wurden folgende Datensätze ausgewertet:

\begin{itemize}
    \item \textbf{Dauerzählstelle Reinhold-Frank-Straße:} Erfasst täglich die Anzahl der Kraftfahrzeuge (Kfz), aufgeschlüsselt nach Fahrzeugklassen (KFZ, PKW, sNfz). Für den Zeitraum 01.01.–20.06.2025 lag der durchschnittliche Tagesverkehr bei ca. 21.300 Kfz/Tag.

    \item \textbf{Historische Jahresmittelwerte:} Langzeitdatenreihen von 2008-2024 aus MobiData BW ermöglichen eine Kontextualisierung der aktuellen Verkehrsbelastung. \cite{mobidata_karte}

    \item \textbf{Zählstellen an äußeren Zufahrtsachsen:} Ergänzende Zähldaten aus dem Jahr 2023 an acht stark befahrenen Einfallstraßen (u.\,a. B10, B36, L605)\cite{mobidata_karte} liefern Anhaltspunkte zur Verkehrsstärke an den Netzrändern.
\end{itemize}

Die Kombination dieser Quellen ermöglicht eine robuste, datenbasierte Schätzung realistischer Flussverteilungen für die Simulation, sowohl zeitlich (z.\,B. Spitzenlasten) als auch räumlich (Zufahrtsverteilung).

\subsection{Aufbau des Simulationsmodells in SUMO}

\subsubsection{Netzgenerierung}

Zur Modellierung des realen Straßennetzes wurde ein Kartenausschnitt des Untersuchungsgebiets über die Exportfunktion von OpenStreetMap \cite{osm-export} heruntergeladen und anschließend mit JOSM \cite{josm} bereinigt. Der Ausschnitt umfasst die Reinhold-Frank-Straße sowie angrenzende Hauptverkehrsachsen im Bereich des Mühlburger Tors. Der bearbeitete Kartenausschnitt wurde mit dem SUMO-Werkzeug \texttt{netconvert} \cite{sumo-tools} in ein XML-Netzwerkformat überführt. Dabei kamen Optionen zur automatischen TLS-Erkennung und zur Verbesserung der Fahrstreifenzuordnung zum Einsatz (z.\,B. \texttt{--tls.guess-signals}, \texttt{--junctions.join}). \cite{sumo-doc}

\subsubsection{Erzeugung von Fahrzeugflüssen}

Für die Simulation wurden zwei Ansätze genutzt. Zunächst entstanden mit dem SUMO-Skript \texttt{randomTrips.py} \cite{randomTrips} einfache Testflüsse, die zur Validierung des Netzmodells dienten. Anschließend wurden auf Basis realer Verkehrszähldaten (siehe Kapitel \ref{sec:zaehlstellen-karlsruhe}) Flussprofile definiert, welche die beobachteten DTV-Werte proportional auf die äußeren Zufahrtskanten verteilten. Hauptachsen wie B10 oder B36 erhielten dabei ein höheres Gewicht. Die daraus erzeugten Routendateien wurden mit \texttt{duarouter} \cite{duarouter} zu konfliktfreien Fahrten verarbeitet. Unterschiedliche Szenarien (Morgen- und Abendspitzen, gleichmäßige Verteilung) erlaubten die Abbildung verschiedener Verkehrslagen.

\subsubsection{Identifikation relevanter Zufahrtskanten}

Die Auswahl geeigneter Zufahrtskanten basierte auf den äußeren Hauptverkehrsachsen (u.\,a. B10, B36, L605, Durlacher Allee). Hierzu wurde ein \hyperref[app:find_relevant_edges]{Python-Skript} eingesetzt, das Kanten mit einem \texttt{name}-Attribut automatisch durchsucht und auf relevante Straßennamen prüfte (z.\,B. \texttt{"B10"}, \texttt{"Reinhold-Frank-Straße"}). Die Ergebnisse wurden manuell überprüft und in \texttt{netedit} \cite{netedit} ergänzt. Die so extrahierten Kanten wurden je Verkehrsachse gruppiert und dienten als Grundlage für die segmentierte Trip-Erzeugung.

\subsubsection{Automatisierte Generierung von Trips}

Die DTV-Werte wurden auf die Zufahrtsgruppen skaliert und auf eine Simulationsdauer von 5000\,s verteilt. Ein eigens entwickeltes Python-Skript erzeugte daraus Fahrzeugeinträge (\texttt{<trip>}), die über die ermittelten Zufahrtskanten ins Netz eingespeist wurden. Die Trips wurden im XML-Format gespeichert und mit \texttt{duarouter} in vollständige, konfliktfreie Routen (\texttt{<route>}) überführt. Dabei wurde die Gesamtanzahl stündlich extrapoliert und stark belastete Achsen (z.\,B. B10, L605) mit entsprechend höherem Anteil berücksichtigt. Die erzeugten Flüsse wurden durch Detektor-Ausgaben (u.\,a. \texttt{laneAreaDetector}) \cite{sumo-doc} überprüft und bei Bedarf angepasst.

\subsubsection{Validierung des Netzmodells}

Vor dem Einsatz des Modells erfolgte eine mehrstufige Validierung:

\begin{itemize}
  \item \textbf{Netzprüfung:} Einsatz von \texttt{netconvert --check-lane-geometry} und \texttt{netcheck} zur Überprüfung der topologischen Konsistenz. \cite{netconvert}
  \item \textbf{Visuelle Kontrolle:} Inspektion kritischer Knoten in der SUMO-GUI und in \texttt{netedit}, um Fehler wie unverbundene Spuren oder falsche Richtungen zu erkennen. \cite{netedit}
  \item \textbf{TLS-Prüfung:} Alle 17 Lichtsignalanlagen wurden in \texttt{netedit} geöffnet. Phasenpläne, gesteuerte Verbindungen und Zustandslängen (\texttt{state}) wurden geprüft und bei Bedarf korrigiert.
\end{itemize}

Die manuelle Nachbearbeitung war zeitaufwändig, da fehlerhafte TLS nicht automatisch erkannt werden. In mehreren Fällen mussten Phasenpläne neu erstellt oder angepasst werden.

\subsubsection{Szenarien und Referenzsimulationen}

Zur Validierung der Verkehrsflüsse wurden unterschiedliche Szenarien umgesetzt:

\begin{itemize}
  \item \textbf{Morgenspitze:} Verstärkte Einträge an Süd- und Westzufahrten.
  \item \textbf{Abendliche Rückstaus:} Höhere Belastung der Ausfallstraßen und des Innenstadtrings.
  \item \textbf{Gleichmäßiger Tagesverlauf:} Homogene Einträge über den Tag verteilt.
  \item \textbf{Zentrumsfokus:} Stärkerer Verkehr über Reinhold-Frank-Straße und Mühlburger Tor.
\end{itemize}

Die Simulationen dienten der Überprüfung der Netzdurchlässigkeit und der Kapazitäten der Knotenpunkte.

\subsubsection{Signalsteuerung und Simulationsparameter}

Für das Reinforcement-Learning-Setup wurden die TLS so konfiguriert, dass sie in SUMO als „aktuiert“\footnote{„Aktuiert“ bedeutet in SUMO, dass die Signalanlagen nicht strikt einem festen Schaltplan folgen, sondern dynamisch mit Hilfe von Detektoren oder externen Eingaben reagieren können.Im Rahmen dieser Arbeit erfolgt die Steuerung ausschließlich über das \texttt{TraCI}-Interface.} \cite{sumo-doc} initialisiert und anschließend direkt durch das RL-Modul über \texttt{TraCI} gesteuert werden konnten \cite{sumo-rl_docs}.


Die Simulation wurde mit folgenden Parametern durchgeführt:

\begin{itemize}
  \item \textbf{Simulationszeitraum:} 5000 Sekunden
  \item \textbf{Zeitschritt (step-length):} 1,0 s
  \item \textbf{Routengenerierung:} deterministisch mit fixer seed zur Reproduzierbarkeit
  \item \textbf{Simulationstyp:} \texttt{meso}-Modus für Training, \texttt{default}-Modus für Evaluation
  \item \textbf{Verkehrsverteilung:} definiert über \texttt{flows.xml} und über Randkanten eingeleitet
\end{itemize}

\subsection{Reinforcement-Learning-Konzept}
\subsubsection{Formulierung des RL-Problems}
Das Problem der Verkehrssteuerung wird als sequentielles Entscheidungsproblem modelliert und mit Hilfe von Reinforcement Learning (RL) gelöst. Ziel ist es, einen Agenten zu trainieren, der durch geeignete Steuerung der Ampelphasen den Verkehrsfluss optimiert. Die Umgebung besteht aus dem simulierten Straßennetz, wie es in SUMO definiert ist. Die Interaktion erfolgt über das TraCI-Interface, das eine Laufzeitsteuerung der Ampelanlagen erlaubt.
\paragraph{Zustände}

Der Zustand \( s_t \) eines Reinforcement-Learning-Agenten beschreibt die Verkehrssituation an einer einzelnen Kreuzung zum Zeitpunkt \( t \). Ziel ist es, dem Agenten ausreichend Informationen über den lokalen Verkehrsfluss zur Verfügung zu stellen, damit er fundierte Entscheidungen über die Steuerung der Lichtsignalanlage treffen kann.

Die Zustandsrepräsentation basiert auf den folgenden Komponenten:

\begin{itemize}
  \item \textbf{Fahrzeuganzahl pro Zufahrtsspur:} Für jede dem Knoten zuführende Fahrspur wird die aktuelle Anzahl an Fahrzeugen ermittelt. Dies geschieht über sogenannte \texttt{laneAreaDetector}, die für jede Spur individuell in SUMO definiert werden. Die Werte werden periodisch über \texttt{TraCI} abgefragt.

  \item \textbf{Warteschlangenlänge (queue length):} Gibt die Anzahl der Fahrzeuge an, die sich auf einer Spur mit Geschwindigkeit \texttt{< 0.1 m/s} befinden. Dies ist ein wichtiges Maß für Rückstaus an Kreuzungen.

  \item \textbf{Durchschnittliche Geschwindigkeit pro Spur:} Diese Kenngröße erlaubt Rückschlüsse auf den Verkehrsfluss pro Richtung und ergänzt die reine Anzahlinformation.

  \item \textbf{Ampelphase (TLS state):} Die aktuell geschaltete Ampelphase wird als diskrete Phase kodiert (z.\,B. 0, 1, 2, ...). In SUMO entspricht dies der Index der aktiven Phase im Phasenplan der TLS.

  \item \textbf{Dauer der aktuellen Phase:} Die Anzahl der Zeitschritte seit Beginn der aktuellen Phase. Diese Information ist notwendig, um Phasenlängen sinnvoll zu steuern (z.\,B. Mindestgrünzeit).

  \item \textbf{Binärmasken zur Phasenwechselbarkeit:} Kodierung, ob ein Wechsel zur nächsten Phase gemäß Übergangsbedingungen (z.\,B. Mindestgrünzeit) aktuell möglich ist. Diese Information ist erforderlich, falls das Aktionsmodell auch direkte Sprünge zwischen nicht direkt benachbarten Phasen erlaubt.

  \item \textbf{Optional – Nachbarschaftszustand:} In Multi-Agent-Settings kann es sinnvoll sein, zusätzlich aggregierte Zustandsinformationen benachbarter Knoten einzubeziehen (z.\,B. Gesamtwarteschlange auf ausgehenden Spuren, die zu benachbarten TLS führen).
\end{itemize}

Die Zustände werden zu einem normierten Merkmalsvektor kombiniert und bilden damit die Eingabe für das neuronale Entscheidungsmodell des Agenten.

\paragraph{Aktionen}

Die Aktionsmenge \( A \) eines Agenten beschreibt die Eingriffsmöglichkeiten in den Steuerungsablauf der jeweiligen Ampelkreuzung. Dabei wird zwischen zwei gängigen Aktionsmodellen unterschieden:

\begin{enumerate}
  \item \textbf{Phasenwechsel-Modell:} Der Agent entscheidet, ob die aktuelle Phase fortgesetzt oder zur nächsten gewechselt werden soll. Es handelt sich um ein binäres Aktionsmodell:
        \[
          A = \{ \texttt{keep},\ \texttt{switch} \}
        \]
        Diese Variante wird häufig in klassischen SUMO-RL-Implementierungen verwendet (z.\,B. `sumo-rl`). Die Reihenfolge der Phasen ist dabei festgelegt (z.\,B. zyklischer Übergang).

  \item \textbf{Direktwahl-Modell:} Der Agent wählt direkt aus allen möglichen Phasen die nächste aus:
        \[
          A = \{ \texttt{phase}_0,\ \texttt{phase}_1,\ \ldots,\ \texttt{phase}_n \}
        \]
        Diese Variante erfordert eine eigene Definition der Übergangslogik in SUMO (z.\,B. über permissive TLS-Ketten), erlaubt aber größere Flexibilität und exploratives Verhalten.
\end{enumerate}

Unabhängig vom Modell gelten folgende Einschränkungen:

\begin{itemize}
  \item \textbf{Mindestgrünzeiten:} Ein Wechsel der Phase darf erst nach einer definierten Mindestgrünzeit erfolgen (z.\,B. 5 s), um realistische Signalisierung und Verkehrssicherheit zu gewährleisten.

  \item \textbf{Sicherheitsbedingte Zwischenphasen:} SUMO erzwingt automatisch Zwischenphasen wie Gelb- oder Räumzeiten. Der Agent gibt nur den Phasenwunsch an, die exakte Ablaufsteuerung erfolgt durch das TLS-Modell in SUMO.

  \item \textbf{Simultane Agentenentscheidung:} Bei mehreren Knoten wird jeder TLS-Agent unabhängig gesteuert, es sei denn, ein zentrales Multi-Agent-Training wird implementiert.
\end{itemize}

Zur Reduktion der Aktionsfrequenz wird häufig ein sogenanntes \textbf{Action Interval} festgelegt (z.\,B. alle 5 s), sodass Entscheidungen nur in bestimmten Zeitschritten getroffen werden können. Dies verhindert zu hektisches Umschalten der Ampeln.

\paragraph{Belohnungsfunktion}

Die Belohnungsfunktion ist zentrales Element des Lernprozesses und bestimmt das Optimierungsziel. Sie wurde so gestaltet, dass sie folgende Aspekte negativ gewichtet:

\begin{itemize}
  \item \textbf{Gesamte Wartezeit aller Fahrzeuge} (minimieren)
  \item \textbf{Länge der Fahrzeugschlangen} (minimieren)
  \item \textbf{Anzahl der Stopps} (minimieren)
\end{itemize}

Die konkrete Belohnung \( r_t \) zum Zeitpunkt \( t \) berechnet sich nach:

\[
  r_t = -\alpha \cdot \sum_{\text{alle Spuren}} \text{queueLength}_i(t) - \beta \cdot \sum_{\text{alle Fahrzeuge}} \text{waitingTime}_j(t)
\]

wobei \( \alpha \) und \( \beta \) Gewichtungsfaktoren darstellen, die im Rahmen der Hyperparameteroptimierung bestimmt werden. In späteren Varianten kann die Belohnung durch zusätzliche Komponenten wie Emissionen oder Energieverbrauch ergänzt werden, um umweltfreundliche Steuerungsstrategien zu fördern.


\subsection{Analyse und Herausforderungen bei der OSM-Netznutzung}

\subsubsection{Grundstruktur von Lichtsignalanlagen in SUMO}
Bevor die Probleme beim OSM-Import analysiert werden, ist es hilfreich, den Aufbau und die Abhängigkeiten der relevanten XML-Elemente in SUMO zu verstehen, insbesondere im Zusammenhang mit der Steuerung von Lichtsignalanlagen. \cite{sumo-doc}
\vspace{0.5cm}
\begin{figure}[H]
  \centering
  \includegraphics[width=0.7\textwidth]{images/junktion.png}
  \caption{Visualisierung einer TLS-Kreuzung (Quelle nededit\cite{netedit}).}
  \label{fig:sumo_karlsruhe}
\end{figure}
\vspace{0.5cm}

\begin{itemize}
  \item \textbf{\texttt{<junction>}} Definiert Knotenpunkte im Netz. Falls eine Ampel gesteuert wird, ist der Typ \texttt{type="traffic\_light"}. Die ID entspricht in der Regel der TLS-ID.

  \item \textbf{\texttt{<connection>}} Verbindet zwei Fahrstreifen (von \texttt{from} nach \texttt{to}). Wenn diese Verbindung durch eine Ampel kontrolliert wird, enthält sie die Attribute \texttt{tl=<tls\_id>} und \texttt{linkIndex}. Die Reihenfolge der \texttt{linkIndex}-Werte bestimmt die Position im Phasen-String.

  \item \textbf{Controlled Link} Jede \texttt{<connection>} mit einem \texttt{tl}-Attribut zählt als „gesteuerte Verbindung“. Die Anzahl solcher Verbindungen bestimmt die Länge des Phasenstrings (\texttt{state}).

  \item \textbf{\texttt{<tlLogic>}} Enthält die Steuerungslogik einer TLS. Jede \texttt{<tlLogic>} hat eine eindeutige ID (i.d.R. identisch zur \texttt{junction}-ID) und eine Liste von \texttt{<phase>}-Elementen.

  \item \textbf{\texttt{<phase>}} Jede Phase ist ein String (z.B. \texttt{"grGr"}), der den Zustand aller \texttt{linkIndex}-Verbindungen kodiert. Jeder Buchstabe (z.B. G = MajorGrün, g = MinorGrün, r = Rot) steht für den Status eines bestimmten kontrollierten Links.

  \item \textbf{\texttt{<request>}} Optionale Anforderungen einzelner Signalgruppen, meist bei aktuierten oder adaptiven TLS. Jeder Eintrag verweist über \texttt{index=} auf einen gesteuerten Link.
\end{itemize}

\begin{figure}[H]
  \centering
  \begin{tikzpicture}[
      font=\small,
      node distance=1.2cm and 3cm,
      every node/.style={align=center},
      box/.style={draw, rounded corners, minimum width=3.2cm, minimum height=1cm}
    ]

    % Hauptknoten
    \node[box] (junction) {\texttt{<junction>}\\ID = J1};
    \node[box, right=4.5cm of junction] (tlLogic) {\texttt{<tlLogic>}\\ID = J1};

    % Verbindungen unterhalb
    \node[box, below left=1.7cm and 0.4cm of junction] (conn1) {\texttt{<connection>}\\\texttt{linkIndex=0}};
    \node[box, below right=1.7cm and 0.4cm of junction] (conn2) {\texttt{<connection>}\\\texttt{linkIndex=1}};

    % Phase
    \node[box, below right=1.7cm and 0.4cm of conn1] (phase) {\texttt{<phase>}\\\texttt{state="Gr"}};

    % Kanten
    \draw[->] (junction.east) -- node[above] {\texttt{tl="J1"}} (tlLogic.west);
    \draw[->] (junction.south west) -- (conn1.north);
    \draw[->] (junction.south east) -- (conn2.north);
    \draw[->] (conn1.south) -- ([xshift=-0.8cm]phase.north);
    \draw[->] (conn2.south) -- ([xshift=+0.8cm]phase.north);

  \end{tikzpicture}
  \caption{Zusammenspiel von Kreuzung, Verbindungen und Ampellogik in SUMO}
  \label{fig:tls_structure}
\end{figure}

\subsubsection{Typische Fehlerquellen nach OSM-Import}
\label{sec:OSM-Import-Fehler}

Die automatische Ableitung von Ampelsteuerungen aus OSM ist unvollständig und fehleranfällig. Im Zusammenspiel mit \texttt{sumo-rl} ergeben sich daraus mehrere konkrete Probleme:

\begin{itemize}
  \item \textbf{Fehlende oder unvollständige TLS-Definitionen:} In OSM sind Ampelanlagen in der Regel lediglich als Punktknoten mit dem Tag \texttt{highway=traffic\_signals} erfasst. Die genaue Schaltlogik (\texttt{tlLogic}), also die Phasen und Zustände, fehlt vollständig. SUMO generiert daher beim Netzimport mit \texttt{--tls.guess-signals} heuristische Ampeldefinitionen, die jedoch oft lückenhaft oder unbrauchbar sind. \cite{OSM-Trafic-signals,Sumo-osm}

  \item \textbf{TLS mit nur einer Phase:} Viele der generierten Ampeln besitzen lediglich eine einzige definierte Phase. Dies entspricht keinem realen Verhalten und führt zu Fehlern beim Training mit \texttt{sumo-rl}, da das Framework mindestens zwei steuerbare Phasen voraussetzt. Die betroffenen Knoten müssen daher identifiziert und aus der Simulation ausgeschlossen oder manuell korrigiert werden. \cite{sumo-rl_docs,sumo-doc,Sumo-osm}

  \item \textbf{Unstimmige Phasenlängen:} Jede Phase in SUMO ist ein Zeichenstring (\texttt{state}), dessen Länge der Anzahl der gesteuerten Verbindungen (sogenannte \textit{controlled links}) entsprechen muss. Bei fehlerhafter Generierung ist diese Bedingung oft verletzt, beispielsweise wenn der \texttt{state} zu kurz oder zu lang ist. Dies führt in \texttt{sumo-rl} zu Indexfehlern oder undefiniertem Verhalten. \cite{sumo-rl_docs}

  \item \textbf{Fehlerhafte oder überzählige \texttt{<request>}-Einträge:} Jede TLS enthält in der Netzdatei zusätzliche Steuerinformationen über \texttt{request}-Elemente. Diese verweisen auf spezifische Signale mittels eines Index. Häufig verweisen diese Einträge jedoch auf nicht vorhandene Verbindungen, da \texttt{netconvert} Signalverknüpfungen nicht korrekt zuordnet. SUMO ignoriert solche Fehler teilweise still, während \texttt{sumo-rl} hingegen bricht mit Ausnahmen ab. \cite{sumo-rl_docs}

  \item \textbf{Mehrdeutige oder verschachtelte Kreuzungen:} In komplexeren innerstädtischen Kreuzungen fasst SUMO mehrere OSM-Knoten zu einem „cluster“ zusammen, um den Verkehrsfluss abzubilden. Dies kann zu sehr großen TLS mit dutzenden Ein- und Ausfahrten führen, die übermäßig viele Phasen oder extrem lange Zustandsdefinitionen erzeugen. Solche TLS sind schwer zu debuggen und häufig inkompatibel mit den Erwartungen von \texttt{sumo-rl}. \cite{sumo-rl_docs,sumo-doc}
\end{itemize}

\paragraph{Folgen für \texttt{sumo-rl}}

Das Framework \texttt{sumo-rl} erwartet für jede zu steuernde TLS: \cite{sumo-rl_docs}

\begin{itemize}
  \item mindestens zwei valide Phasen,
  \item konsistente Phasenzustände (\texttt{state}) mit korrekter Länge,
  \item vollständige Verbindungen zu kontrollierten Links,
  \item eindeutig identifizierbare TLS-IDs.
\end{itemize}

Sind diese Anforderungen nicht erfüllt, führt dies typischerweise zu einer der folgenden Fehlermeldungen:

\begin{itemize}
  \item \texttt{IndexError: string index out of range}
  \item \texttt{ValueError: Invalid phase length}
  \item \texttt{KeyError: TLS not found}
\end{itemize}

Da diese Probleme nicht durch SUMO selbst gemeldet, sondern erst zur Laufzeit in \texttt{sumo-rl} sichtbar werden, ist ein systematischer Debugging- und Reparaturprozess zwingend notwendig. Die Komplexität steigt dabei exponentiell mit der Anzahl der TLS im Netz.
\paragraph{Erkenntnis}

Der direkte Import von OSM-Daten in SUMO erzeugt ein formal nutzbares Verkehrsnetz,jedoch nicht automatisch ein für Reinforcement Learning (RL) geeignetes. Ohne zusätzliche Aufbereitung ist ein stabiler Trainingsbetrieb in \texttt{sumo-rl} nicht möglich. Im Rahmen dieser Arbeit wurde das reale OSM-Netz von Karlsruhe daher gezielt analysiert, bereinigt und überarbeitet, sodass es nun erfolgreich und stabil im RL-Kontext eingesetzt werden kann. Dazu wurden eigene Werkzeuge zur automatisierten Strukturprüfung und Reparatur entwickelt, die im einem folgenden Abschnitt näher beschrieben werden.

\subsubsection{Problematik nicht-motorisierter Verkehrswege im OSM-Modell}

Ein zentrales Problem beim ursprünglichen OSM-Import stellten die Strukturen nicht-motorisierter Verkehrsträger dar, insbesondere Fußwege, Überwege und Fahrradtrassen.. Diese sind im OSM-Modell zwar detailliert erfasst, führen aber in SUMO häufig zu problematischen Simulationseffekten: \cite{Sumo-osm}

\begin{itemize}
  \item \textbf{Separate Fahrspuren für Radverkehr:} Zusätzliche Radstreifen erzeugen neue Kanten mit eigenen Abbiegebeziehungen, die von SUMO automatisch als TLS-relevant eingestuft werden, was häufig zu übermäßig vielen Signalgruppen führt.

  \item \textbf{Fußgängerüberwege mit Konfliktzonen:} \texttt{highway=crossing}-Elemente erzeugen automatisch Übergänge mit Konfliktzonen, die eine Ampelregelung erfordern, selbst wenn sie im Originalnetz nur symbolisch vorhanden sind.

  \item \textbf{Komplexität beim Entfernen:} Die gezielte Entfernung solcher Elemente führte häufig zu inkonsistenten Junctions und Netzfragmentierung. Ein manuelles Vorgehen wäre fehleranfällig und kaum skalierbar gewesen.
\end{itemize}

Diese Herausforderungen machten eine rein automatische Nutzung des OSM-Imports zunächst unmöglich. Erst durch gezielte algorithmische Nachbearbeitung konnte das Karlsruher Netz so transformiert werden, dass es für die RL-Simulation zuverlässig nutzbar wurde.

\subsubsection{Eingesetzte \texttt{netconvert}-Optionen und deren Grenzen}

Zur automatisierten Aufbereitung kamen zahlreiche Optionen von \texttt{netconvert} zum Einsatz, um das aus OSM exportierte Netz anzupassen. Dabei zeigte sich jedoch, dass viele dieser Optionen nicht auf die hohen Anforderungen von RL-Umgebungen zugeschnitten sind: \cite{netconvert}

\begin{itemize}
  \item \texttt{--tls.guess-signals}: Erzeugt Ampeln auf Basis der Netzstruktur, allerdings oft mit unrealistischen oder unbrauchbaren Phasen.

  \item \texttt{--tls.join} und \texttt{--junctions.join}: Reduzieren Komplexität, erzeugen jedoch teils unübersichtliche Cluster, die schwer manuell kontrollierbar sind.

  \item \texttt{--ramps.guess}: Für urbane Netze weitgehend irrelevant oder sogar kontraproduktiv.

  \item \texttt{--remove-edges.isolated}, \texttt{--keep-edges.by-vclass} und \texttt{--discard-simple}: Dienen der Netzvereinfachung, führen aber oft zu strukturellen Problemen oder fehlenden funktionalen Verbindungen.
\end{itemize}

Obwohl diese Optionen wichtige Vorarbeiten leisteten, war ihre Wirkung für das RL-Zielmodell begrenzt. Erst durch zusätzliche Werkzeuge und maßgeschneiderte Filterlogik konnte das Netz gezielt bereinigt und optimiert werden.

\subsubsection{Manuelle Eingriffe und strukturelle Rekonstruktionen}

Neben automatisierten Bereinigungen waren auch gezielte manuelle Anpassungen notwendig. Insbesondere wurden mithilfe von \texttt{netedit} einzelne Kreuzungen vollständig neu aufgebaut, um \textbf{Deadlocks zu vermeiden}, die zwar in der realen Verkehrsführung nicht auftreten, jedoch in SUMO durch implizite Abbiegelogiken und Vorrangregeln entstehen können.

Diese Rekonstruktionen erfolgten unter Beibehaltung der realweltlichen Topologie, jedoch mit einer technisch sauberen Definition aller Fahrbeziehungen und Signalisierungen. Damit konnte sichergestellt werden, dass auch komplexere Kreuzungen reproduzierbar, konfliktfrei und steuerbar bleiben.

\paragraph{Beseitigung von Deadlocks}
Während der initialen Simulationen traten in SUMO wiederholt Deadlocks an komplexen Kreuzungen auf, die in der realen Verkehrsführung nicht vorkommen.
Ursache waren insbesondere von OSM importierte Rampen- und Abbiegebeziehungen, die zu konfliktbehafteten Fahrbeziehungen führten.
Diese Knoten wurden in \texttt{netedit} gezielt angepasst:
\begin{itemize}
  \item Anpassung der Geometrie, um SUMO-konforme Fahrspuren und eindeutige Vorrangregeln zu gewährleisten,
  \item Entfernung redundanter oder fehlerhafter Rampenverbindungen,
  \item Sicherstellung, dass jede Fahrbeziehung durch ein passendes Signal geregelt wird.
\end{itemize}
Die Änderungen wurden so umgesetzt, dass die realweltliche Topologie beibehalten wurde.
Durch diese Eingriffe konnte die Simulation ohne Deadlocks und mit stabilen Verkehrsflüssen betrieben werden.

\paragraph{Visualisierung manueller Anpassungen}
Abbildung~\ref{fig:tls_fix} zeigt exemplarisch eine Kreuzung vor und nach der manuellen Anpassung in \texttt{netedit}.
Links ist die fehlerhafte Geometrie mit konfliktbehafteten Rampenverbindungen zu sehen,
die in SUMO zu Deadlocks führten.
Rechts ist die angepasste Version dargestellt,
bei der alle Fahrbeziehungen eindeutig definiert und den Signalgruppen korrekt zugeordnet sind.
Die Topologie entspricht der realen Verkehrsführung, wurde jedoch so modelliert, dass SUMO sie fehlerfrei simulieren kann.

\begin{figure}[H]
  \centering
  \begin{subfigure}[t]{0.41\textwidth}
    \centering
    \includegraphics[width=\textwidth]{junction_before.PNG}
    \caption*{Vorher}
  \end{subfigure}
  \hfill
  \begin{subfigure}[t]{0.45\textwidth}
    \centering
    \includegraphics[width=\textwidth]{junction_after.PNG}
    \caption*{Nachher}
  \end{subfigure}
  \caption{Vorher-Nachher-Vergleich einer angepassten TLS-Kreuzung (netedit \cite{netedit})}
  \label{fig:tls_fix}
\end{figure}


\subsubsection{Ergebnis: ein realistisches, RL-kompatibles Netz}

Im Gegensatz zu einer synthetischen Umgebung basiert das nun eingesetzte Trainingsnetz auf realen topologischen Daten, wurde jedoch gezielt für den Einsatz mit \texttt{sumo-rl} überarbeitet. Es erfüllt folgende Eigenschaften:

\begin{itemize}
  \item \textbf{Hohe Realitätsnähe bei kontrollierter Komplexität:} Das Netz bildet reale Strukturen ab, wurde jedoch so bereinigt, dass es RL-kompatibel bleibt.

  \item \textbf{Stabile TLS-Struktur:} Alle Kreuzungen mit Lichtsignalanlagen enthalten reproduzierbare und sinnvoll steuerbare Phasen.

  \item \textbf{Fehlerminimierung und Modularität:} Durch gezielte Reduktion und Nachbearbeitung sind Trainingsläufe wiederholbar und ohne strukturelle Störungen durchführbar.

  \item \textbf{Deadlock-Vermeidung durch gezielte Rekonstruktion:} Kritische Junctions wurden manuell so modelliert, dass sie SUMO-spezifische Blockadesituationen vermeiden, ohne die Realität zu verzerren.
\end{itemize}

Der Einsatz dieses verbesserten Karlsruher Netzes stellt einen zentralen methodischen Beitrag dieser Arbeit dar, da er demonstriert, wie reale OSM-Daten erfolgreich für das Reinforcement Learning nutzbar gemacht werden können, trotz ihrer ursprünglichen Limitierungen.


\subsection{Netzprüfung, Reparatur und Toolchain}
Aufgrund der oben beschriebenen strukturellen Schwächen im importierten OSM-Netz (siehe Kapitel \ref{sec:OSM-Import-Fehler}) war eine manuelle Nachbearbeitung ineffizient und fehleranfällig. Daher wurden eigene Werkzeuge entwickelt, um eine systematische und automatisierte Reparatur zu ermöglichen.


\subsubsection{Werkzeuge zur Netzprüfung und Reparatur}

Um die Kompatibilität des aus OpenStreetMap abgeleiteten Verkehrsnetzes mit \texttt{sumo-rl} sicherzustellen, wurde eine Reihe eigenentwickelter Python-Skripte implementiert. Diese Werkzeuge automatisieren die Analyse, Validierung und Korrektur der Netzstruktur mit Fokus auf Lichtsignalanlagen (TLS). Der modulare Aufbau erlaubt es, problematische Netzbestandteile zu identifizieren und gezielt zu bereinigen.

\paragraph{Prüfung der Signalverknüpfungen und Zustandslängen}

Zwei zentrale Tools wurden entwickelt, um die Konsistenz zwischen kontrollierten Verbindungen (\textit{controlled links}) und Phasenzuständen (\texttt{state}) der TLS zu überprüfen:

\begin{itemize}
  \item \textbf{\texttt{check\_tls\_consistency.py}} (siehe Anhang \ref{app:check_tls_consistency}) prüft, ob die Länge jedes \texttt{state}-Strings in den \texttt{<phase>}-Elementen exakt der Anzahl der gesteuerten Signalindizes entspricht. Abweichungen werden detailliert gelistet, inklusive betroffener Phase und TLS-ID.
        \begin{algorithm}[H]
          \caption{CheckTLSLengths – Prüfung inkonsistenter Phasenlängen}
          \begin{algorithmic}[1]
            \Function{CheckTLSLengths}{net.xml}
            \State Lade XML-Baum und extrahiere \texttt{<connection>}-Elemente
            \State Erstelle Dictionary \texttt{tls\_controlled\_links} mit Anzahl gesteuerter Links pro TLS
            \ForAll{\texttt{tlLogic}-Elemente im Netz}
            \State \texttt{expectedLen} $\gets$ Anzahl \texttt{controlledLinks} aus Dictionary
            \If{\texttt{expectedLen} = 0}
            \State Gib Warnung: TLS hat keine gesteuerten Verbindungen
            \State \textbf{continue}
            \EndIf
            \ForAll{Phasen $i$ in \texttt{tlLogic}}
            \State \texttt{actualLen} $\gets$ Länge des \texttt{state}-Strings
            \If{\texttt{actualLen} $\neq$ \texttt{expectedLen}}
            \State Gib Warnung mit TLS-ID, Phase und \texttt{state}-Inhalt aus
            \EndIf
            \EndFor
            \EndFor
            \If{keine Abweichungen gefunden}
            \State Gib Erfolgsmeldung aus
            \EndIf
            \EndFunction
          \end{algorithmic}
        \end{algorithm}

  \item \textbf{\texttt{check\_tls\_requests.py}} (siehe Anhang \ref{app:check_tls_requests}) validiert, ob alle \texttt{<request>}-Indizes innerhalb zulässiger Grenzen liegen. Falsch verknüpfte Einträge – z.\,B. \texttt{index > max(signalIndex)} – werden gemeldet.

        \begin{algorithm}[H]
          \caption{CheckTLSRequests – Prüfung ungültiger \texttt{request}-Indizes}
          \begin{algorithmic}[1]
            \Function{CheckTLSRequests}{net.xml}
            \State Lade XML-Datei und parse Wurzelknoten
            \State Erzeuge Dictionary \texttt{tls\_signal\_indices} mit Signalindizes je TLS aus \texttt{<connection>}-Elementen
            \ForAll{\texttt{junction}-Elemente im Netz}
            \State \texttt{tls\_id} $\gets$ ID der Junction
            \If{\texttt{tls\_id} in \texttt{tls\_signal\_indices}}
            \State \texttt{expected\_max} $\gets$ Länge der Signalindizes für dieses TLS
            \ForAll{\texttt{request}-Elemente in Junction}
            \State \texttt{index} $\gets$ Wert des \texttt{index}-Attributs
            \If{\texttt{index} $\geq$ \texttt{expected\_max}}
            \State Gib Warnung mit \texttt{tls\_id} und \texttt{index} aus
            \EndIf
            \EndFor
            \EndIf
            \EndFor
            \If{keine Warnungen ausgegeben}
            \State Gib Erfolgsmeldung aus
            \EndIf
            \EndFunction
          \end{algorithmic}
        \end{algorithm}
\end{itemize}

\paragraph{Automatische Reparaturwerkzeuge}

Die folgenden Programme wurden zur strukturellen Korrektur entwickelt:

\begin{itemize}
  \item \textbf{\texttt{fix\_requests.py}} (siehe Anhang \ref{app:fix_requests}) entfernt überzählige \texttt{<request>}-Einträge und kürzt \texttt{state}-Strings in Phasen auf die zulässige Länge. Die Bereinigung erfolgt anhand der tatsächlichen Anzahl gesteuerter Signalverbindungen (\texttt{linkIndex}).

        \begin{algorithm}[H]
          \caption{FixRequests – Bereinigung ungültiger \texttt{<request>}-Einträge und Anpassung der Phasen}
          \begin{algorithmic}[1]
            \Function{FixRequests}{net.xml}
            \State Lade XML-Baum mit Netzstruktur
            \State Initialisiere Dictionary \texttt{tls\_max\_index} für maximale Signalindices
            \ForAll{\texttt{connection}-Elemente}
            \If{TLS-ID und \texttt{linkIndex} vorhanden}
            \State Aktualisiere \texttt{tls\_max\_index[tl]} mit höchstem Index
            \EndIf
            \EndFor

            \ForAll{\texttt{junction}-Elemente}
            \State Hole TLS-ID
            \If{TLS nicht in \texttt{tls\_max\_index}}
            \State \textbf{continue}
            \EndIf
            \State Bestimme erlaubten Maximalindex (\texttt{max\_idx})
            \ForAll{\texttt{request}-Einträge}
            \If{Index $>$ \texttt{max\_idx}}
            \State Entferne ungültigen \texttt{request}
            \EndIf
            \EndFor

            \ForAll{\texttt{tlLogic}-Elemente mit passender TLS-ID}
            \ForAll{Phasen}
            \If{\texttt{state}-String ist zu lang}
            \State Kürze \texttt{state} auf \texttt{max\_idx + 1}
            \EndIf
            \EndFor
            \EndFor
            \EndFor

            \State Speichere modifizierte XML-Datei
            \State Gib Statistiken zu entfernten Requests und angepassten Phasen aus
            \EndFunction
          \end{algorithmic}
        \end{algorithm}

        \newpage
  \item \textbf{\texttt{repair-net.py}} (siehe Anhang \ref{app:repair_net}) nutzt ein manuell gepflegtes Dictionary mit TLS-IDs und deren erwarteter Phasenlänge (Anzahl kontrollierter Verbindungen). Alle Phasen, deren Länge abweicht, werden automatisch gekürzt oder aufgefüllt.

        \begin{algorithm}[H]
          \caption{RepairTLSStates – Korrektur der Phasenlängen anhand manuell gepflegter Referenz}
          \begin{algorithmic}[1]
            \Function{RepairTLSStates}{net.xml, referenz\_dictionary}
            \State Lade Netzstruktur aus \texttt{net.xml}
            \ForAll{\texttt{tlLogic}-Elemente im Netz}
            \State \texttt{tls\_id} $\gets$ ID des Ampelknotens
            \If{\texttt{tls\_id} nicht in referenz\_dictionary}
            \State \textbf{continue}
            \EndIf
            \State \texttt{correctLen} $\gets$ erwartete Zustandslänge aus Referenz
            \ForAll{Phasen des Knotens}
            \State \texttt{state} $\gets$ Zeichenkette der Phase
            \If{Länge(\texttt{state}) $\neq$ \texttt{correctLen}}
            \State Kürze oder ergänze \texttt{state} auf \texttt{correctLen}
            \State Markiere Netz als geändert
            \EndIf
            \EndFor
            \EndFor
            \If{Netz wurde geändert}
            \State Speichere bereinigte Netzdatei als \texttt{karlsruhe\_fixed.net.xml}
            \Else
            \State Gib Hinweis: Alle Phasen bereits korrekt
            \EndIf
            \EndFunction
          \end{algorithmic}
        \end{algorithm}

  \item \textbf{\texttt{statecheck.py}} (siehe Anhang \ref{app:statecheck}) gibt eine Liste aller TLS-Phasen mit ungewöhnlichen Längen aus. Dieses Tool wurde verwendet, um bei vereinheitlichten Netzen auf eine Ziel-Zustandslänge zu prüfen.

        \begin{algorithm}[H]
          \caption{StateCheck – Prüfung auf einheitliche Phasenlängen}
          \begin{algorithmic}[1]
            \Function{StateCheck}{net.xml}
            \State Lade XML-Baum aus der Netzdatei
            \ForAll{\texttt{tlLogic}-Elemente im Netz}
            \State \texttt{tl\_id} $\gets$ ID des aktuellen TLS
            \ForAll{Phasen $i$ in \texttt{tlLogic}}
            \State \texttt{state} $\gets$ Zustand der Phase
            \If{\texttt{len(state)} $\neq$ 57}
            \State Gib Warnung mit \texttt{tl\_id}, Phasenindex und tatsächlicher Länge aus
            \EndIf
            \EndFor
            \EndFor
            \EndFunction
          \end{algorithmic}
        \end{algorithm}

\end{itemize}

\newpage
\paragraph{Gültigkeitsprüfung für SUMO-RL}

Zur Vorbereitung des Trainings wurden weitere Programme zur Identifikation funktionaler TLS entwickelt:

\begin{itemize}
  \item \textbf{\texttt{find\_valid\_tls.py}} (siehe Anhang \ref{app:find_valid_tls}) iteriert über alle TLS im Netz und testet jede einzeln in einem minimalen \texttt{sumo-rl}-Lauf. TLS, bei denen die Umgebung erfolgreich initialisiert werden kann, gelten als kompatibel.

        \begin{algorithm}[H]
          \caption{FindValidTLS – Gültigkeitsprüfung aller TLS im Netz}
          \label{alg:find_valid_tls}
          \begin{algorithmic}[1]
            \Function{TestTLS}{\texttt{tls\_id}}
            \State Initialisiere \texttt{SumoEnvironment}
            \State Setze \texttt{ts\_ids} auf \texttt{[tls\_id]}
            \State Versuche: \texttt{env.reset()}
            \If{kein Fehler}
            \State \texttt{env.close()}
            \State \Return \texttt{True}
            \Else
            \State Gib Fehlermeldung aus
            \State \Return \texttt{False}
            \EndIf
            \EndFunction
            \vspace{0.5em}
            \State Initialisiere leere Liste \texttt{all\_tls}
            \State Versuche: Umgebung mit \texttt{SumoEnvironment} zu starten
            \If{erfolgreich}
            \State Lese alle \texttt{ts\_ids}
            \State Schließe Umgebung
            \Else
            \State Gib Fehler aus
            \EndIf
            \vspace{0.5em}
            \State Initialisiere leere Liste \texttt{valid\_tls}
            \ForAll{\texttt{tls\_id} in \texttt{all\_tls}}
            \If{ \Call{TestTLS}{\texttt{tls\_id}} }
            \State Füge \texttt{tls\_id} zu \texttt{valid\_tls} hinzu
            \EndIf
            \EndFor
            \State Gib alle gültigen TLS aus
          \end{algorithmic}
        \end{algorithm}
\end{itemize}

\subsubsection{Auswahl eines bereinigten Netzes}

Nach mehrfacher Iteration und Debugging wurde ein final bereinigtes Netz erzeugt: \texttt{network.net.xml}. Dieses enthält ausschließlich überprüfte TLS mit konsistenten Phasenlängen und steuerbaren Verbindungen. Es bildet die Grundlage für alle nachfolgenden Reinforcement-Learning-Experimente.

\newpage
\subsubsection{Vorteil des automatisierten Workflows}

Die entwickelte Toolchain ermöglicht:

\begin{itemize}
  \item eine strukturierte Diagnose typischer OSM-bedingter Netzprobleme,
  \item reproduzierbare Netzreparaturen ohne reines manuelles Editieren in \texttt{netedit} \cite{netedit},
  \item gezielte Selektion steuerbarer TLS für das Experiment.
\end{itemize}

Der Einsatz dieser Werkzeuge war unerlässlich, um ein funktionales, kompatibles und robusteres Simulationsnetz auf Basis realer OSM-Daten zu etablieren.


\subsection{Einbindung des SUMO-Netzes in die RL-Umgebung}
\label{sec:sumo-rl-architektur}

Nach Abschluss der Netzbereinigung, der strukturellen Validierung und der Identifikation steuerbarer Lichtsignalanlagen (TLS) wurde das finale Verkehrsnetz in eine auf \texttt{sumo-rl} basierende Reinforcement-Learning-Umgebung integriert. Ziel war die Realisierung einer robusten, modularen Multiagentenumgebung, die eine lernbasierte Optimierung der Verkehrssteuerung unter realitätsnahen Bedingungen erlaubt.

\subsubsection{Gesamtsystem und Architektur}

Die Architektur der Lernumgebung ist als verteiltes Multiagentensystem ausgelegt, bei dem jede signalgesteuerte Kreuzung durch einen eigenständigen Agenten repräsentiert wird. Die Interaktion erfolgt über das TraCI-Protokoll von SUMO, das eine Echtzeitkommunikation zwischen Simulator und RL-Agenten ermöglicht. Die zentrale Steuerung und das Training der Agenten basiert auf der RL-Bibliothek \texttt{Stable-Baselines3}, konkret dem Algorithmus \texttt{Proximal Policy Optimization (PPO)}.


\begin{figure}[H]
  \centering
  \begin{tikzpicture}[node distance=1.2cm]

    \node[box] (sim) {Verkehrssimulation\\\textbf{SUMO + Traci}};
    \node[box, below=of sim] (mae) {Multi-Agent Environment (MAE)\\\texttt{sumo-rl.parallel\_env} + \texttt{custom\_reward}};
    \node[box, below=of mae] (wrap) {Supersuit Wrappers\\\texttt{- pad\_obs / pad\_actions}\\\texttt{- pettingzoo\_to\_vec\_env}};
    \node[box, below=of wrap] (vec) {VecNormalize + VecMonitor\\(Stable-Baselines3)};
    \node[box, below=of vec] (ppo) {PPO Agent (Stable-Baselines3)\\\texttt{- MLPPolicy}\\\texttt{- Schedules: LR, Clip, Ent.}};
    \node[box, below=of ppo] (cb) {Callbacks\\\texttt{- TimeBasedCheckpoint}\\\texttt{- BestModelSaver (mean\_reward)}\\\texttt{- LearningRateLogger}};
    \node[box, below=of cb] (log) {Modell- \& Log-Speicherung\\\texttt{(pro Seed in runs/)}};

    % Arrows
    \draw[arrow] (sim) -- (mae);
    \draw[arrow] (mae) -- (wrap);
    \draw[arrow] (wrap) -- (vec);
    \draw[arrow] (vec) -- (ppo);
    \draw[arrow] (ppo) -- (cb);
    \draw[arrow] (cb) -- (log);

  \end{tikzpicture}
  \caption{Architektur der RL-Trainingspipeline mit SUMO, MAE und Stable-Baselines3}
  \label{fig:rl_architektur}
\end{figure}

Zur Vereinheitlichung der Multiagentenumgebung kamen die Bibliotheken \texttt{PettingZoo} und \texttt{SuperSuit} zum Einsatz. \texttt{PettingZoo} stellt ein standardisiertes API für Multiagenten-Umgebungen bereit – analog zu \texttt{Gymnasium}, jedoch speziell für Szenarien mit mehreren Agenten. \texttt{SuperSuit} erweitert diese Umgebungen durch eine Vielzahl an Wrappers, beispielsweise zur Vereinheitlichung von Beobachtungs- und Aktionsräumen (z.\,B. durch Padding) oder zur Umwandlung in vektorisierte Formate, wie sie für paralleles Training mit \texttt{Stable-Baselines3} erforderlich sind.

\subsubsection{Konfiguration der Umgebung}

Die Reinforcement-Learning-Umgebung wurde auf Basis der Klasse \texttt{SumoEnvironment} bzw. \texttt{parallel\_env} aus \texttt{sumo-rl} konfiguriert. Wichtige Parameter umfassen:

\begin{itemize}
  \item \texttt{net\_file}, \texttt{route\_file}: Pfade zum bereinigten Verkehrsnetz und zugehöriger Routendatei.
  \item \texttt{use\_gui}: Aktiviert die grafische Visualisierung von SUMO (zur Laufzeit abschaltbar für Trainingsgeschwindigkeit).
  \item \texttt{num\_seconds}: Dauer einer Simulationsepisode in Sekunden.
  \item \texttt{reward\_fn}: Referenz auf die benutzerdefinierte Belohnungsfunktion.
  \item \texttt{min\_green}: Minimale Grünphasenlänge in Sekunden zur Sicherstellung realistischer Signalzyklen.
  \item \texttt{max\_depart\_delay}: Maximale Verzögerung bei der Einfahrt eines Fahrzeugs (zur Kontrolle der Spawn-Zeit).
  \item \texttt{sumo\_seed}: Zufalls-Seed zur Reproduzierbarkeit von Verkehrsflüssen und Routenentscheidungen.
  \item \texttt{add\_system\_info}: Wenn aktiviert, werden systemweite Kennzahlen (z.\,B. durchschnittliche Wartezeit) in die Beobachtung eingebettet.
  \item \texttt{add\_per\_agent\_info}: Ergänzt die lokale Beobachtung jedes Agenten um zusätzliche Kontextdaten (z.\,B. Verkehrsdichte im Knoten).
  \item \texttt{single\_agent = False}: Aktiviert den Multiagentenmodus, bei dem jede steuerbare Ampel einen separaten Agenten darstellt.
\end{itemize}

Die Umgebung ist vollständig kompatibel mit \texttt{Gymnasium}, \texttt{PettingZoo} sowie den Wrapper-Bibliotheken \texttt{SuperSuit} und \texttt{VecEnv}, wodurch ein standardisiertes Interfacing mit RL-Algorithmen ermöglicht wird.

\subsubsection{Beobachtungen und Aktionsraum}

Jeder Agent erhält eine lokale Beobachtung, die sich aus folgenden Informationen zusammensetzt:

\begin{itemize}
  \item aktuelle Ampelphase (diskreter Index oder One-Hot-Encoding),
  \item Dauer der aktuellen Phase (zur Einhaltung von Mindestzeiten),
  \item für jede anliegende Spur: Anzahl wartender Fahrzeuge, durchschnittliche Geschwindigkeit, Dichte.
\end{itemize}

Der Aktionsraum ist diskret und erlaubt die Auswahl der nächsten Phase. Der Phasenwechsel wird durch SUMO automatisch mit einer Zwischenphase (Gelbphase) ergänzt. Die Entscheidung erfolgt synchron für alle Agenten alle \texttt{delta\_time} Sekunden.

\subsubsection{Belohnungsfunktionen}

\texttt{sumo-rl} unterstützt verschiedene standardisierte Reward-Funktionen:

\begin{itemize}
  \item \texttt{"diff-waiting-time"}: Reduktion der Differenz kumulierter Wartezeiten,
  \item \texttt{\char`\"average-speed"}: Maximierung der mittleren Geschwindigkeit im Netz,
  \item \texttt{"queue"}: Minimierung der Gesamtlänge aller Warteschlangen.
\end{itemize}

Im Rahmen dieser Arbeit wurde zusätzlich eine eigene Reward-Funktion definiert, welche folgende Größen kombiniert:

\begin{itemize}
  \item aktuelle Warteschlangenlänge (negativ),
  \item akkumulierte Wartezeiten (negativ),
  \item Anzahl an Teleportationen und Kollisionen (stark negativ),
  \item Anzahl neu eingetroffener Fahrzeuge (positiv),
  \item Veränderung der Stauhöhe zur Vorperiode (positiv).
\end{itemize}

Die Belohnung wird nach jedem Simulationsschritt einzeln für jeden TLS-Agenten berechnet. Extreme Ereignisse (z.\,B. viele Teleports) führen zu stark negativen Strafwerten, um stabile Lernverläufe zu fördern.

\subsubsection{Trainingsalgorithmus und Hyperparameter}

Das Training der Agenten erfolgte mittels \texttt{PPO}, wobei folgende Hyperparameter eingesetzt wurden:

\begin{itemize}
  \item \textbf{Policy-Architektur:} Zwei Hidden-Layer mit jeweils 128 Neuronen,
  \item \textbf{Batchgröße:} 2048,
  \item \textbf{Lernrate:} linear abnehmend von $3\cdot10^{-4}$,
  \item \textbf{Clip-Range:} dynamisch, linear von 0.2 auf 0.1,
  \item \textbf{Entropiekoeffizient:} $0.005$ zur Förderung explorativen Verhaltens,
  \item \textbf{Discount-Faktor:} $\gamma = 0.99$,
  \item \textbf{GAE-Lambda:} $0.95$ für stabilisierte Vorteilsschätzung.
\end{itemize}

Die Umgebung wurde über \texttt{VecNormalize} normalisiert und mit \texttt{VecMonitor} überwacht. Zusätzlich kamen Wrapper zur Aktion- und Beobachtungsstandardisierung (\texttt{pad\_observations\_v0}, \texttt{pad\_action\_space\_v0}) zum Einsatz, um variable TLS-Strukturen zu harmonisieren.

\subsubsection{Checkpoints, Monitoring und Logging}

Zur Sicherstellung eines robusten Trainingsprozesses wurde eine Reihe von Callback-Mechanismen implementiert:

\begin{itemize}
  \item \textbf{Checkpointing:} Zeitbasierte Sicherung des Modells alle 60 Minuten,
  \item \textbf{Bestmodell-Erkennung:} Automatische Speicherung des jeweils besten Modells (höchste mittlere Reward),
  \item \textbf{Adaptive Schedules:} Dynamische Anpassung von Lernrate und Clip-Range an den Trainingsfortschritt,
  \item \textbf{Logging via TensorBoard:} Visualisierung von Reward-Kurven, Lernraten, Clip-Werten und Modellmetriken.
\end{itemize}

Alle Modellartefakte (\texttt{.zip}, \texttt{vecnormalize.pkl}) sowie die TensorBoard-Logs wurden pro Seed-Version strukturiert gespeichert. Dadurch konnten sowohl Reproduzierbarkeit als auch vergleichende Auswertungen zwischen Trainingsläufen gewährleistet werden.

\subsubsection{Zusammenfassung}

Die konfigurierte RL-Umgebung erlaubt eine modulare und flexible Steuerung realer Verkehrsnetze auf Basis von SUMO. Durch die Kombination aus systematischer TLS-Auswahl, stabiler Reward-Funktion, adaptiven Trainingsparametern und umfassendem Monitoring wurde eine solide Grundlage für die experimentelle Evaluation lernbasierter Verkehrssteuerung geschaffen.


\subsection{Evaluationsstrategie}
\label{sec:evaluation-strategy}

Zur Bewertung der trainierten Modelle wird ein systematischer, skriptbasierter Evaluationsplan verfolgt (Details und Ergebnisse siehe Kapitel~\ref{sec:validation}).
Dabei wird stets dasselbe, bereinigte Verkehrsnetz (\texttt{map.net.xml}) verwendet, um die Vergleichbarkeit zu gewährleisten; variiert werden Signalsteuerung, Routenbelegung und Seeds.
Statische Baselines werden direkt im Evaluationslauf mitgeführt (fester Phasenplan und aktuiert).

\subsubsection{Vergleichsszenarien}
Die Evaluation der RL-Agenten und der Baselines erfolgt vollautomatisch über \texttt{evaluate.py}.
Vier Szenarien decken typische Lastprofile ab:
\begin{itemize}
    \item \textbf{morning\_peak} (\texttt{flows\_morning.rou.xml}): morgendliche Lastspitze,
    \item \textbf{evening\_peak} (\texttt{flows\_evening.rou.xml}): abendliche Lastspitze,
    \item \textbf{uniform} (\texttt{flows\_uniform.rou.xml}): gleichmäßige Zuflüsse,
    \item \textbf{random\_heavy} (\texttt{flows\_random\_heavy.rou.xml}): stochastisch dichter Verkehr.
\end{itemize}

\subsubsection{Bewertete Modelle}
Die zu evaluierenden RL-Modelle werden automatisch aus \texttt{runs/} ermittelt:
\begin{itemize}
    \item Alle Unterordner mit Präfix \texttt{ppo\_sumo\_*}.
    \item Aus jedem Lauf wird das \textbf{beste Modell} \texttt{best\_model.zip} sowie die zugehörige \textbf{Normalisierung} \texttt{vecnormalize.pkl} geladen.
\end{itemize}
Zusätzlich zu den RL-Läufen werden im selben Skript die Baselines (Fixed-Time, Actuated) ausgeführt.

\subsubsection{Reward-Funktionen}
Aktuell wird \texttt{realworld\_reward} verwendet, die den Durchfluss belohnt und Stauaufbau sowie häufige Phasenwechsel bestraft:
\begin{itemize}
    \item EMA-Glättung mit Faktor \(\alpha=0{,}3\),
    \item Normalisierung: \texttt{max\_storage=40}, \texttt{max\_outflow\_per\_step=8},
    \item Gewichtungen: \(w_q=1{,}0\), \(w_{\text{build}}=0{,}8\), \(w_{\text{flow}}=0{,}7\), \(w_{\text{switch}}=0{,}1\),
    \item Reward-Clipping auf \(\pm 5\),
    \item Step-Cache reduziert TraCI-Aufrufe pro Zeitschritt.
\end{itemize}
Weitere Reward-Varianten (z.\,B.\ \texttt{pressure}, \texttt{throughput}) können identisch eingebunden werden; die Evaluationslogik bleibt unverändert.

\subsubsection{Simulations- und Umgebungsparameter}
Für alle Simulationen (RL und Baselines) gelten konsistente Parameter:
\begin{itemize}
    \item Episodenlänge: \texttt{EP\_LENGTH\_S = 3500}~s,
    \item Minimaldauer Grünphase: \texttt{min\_green = 5}~s,
    \item Maximaler Abfahrtsverzug: \texttt{max\_depart\_delay = 100}~s,
    \item Zusätzliche Systemmetriken: \texttt{add\_system\_info = True},
    \item Pro-Agent-Infos: \texttt{add\_per\_agent\_info = False},
    \item SUMO läuft ohne GUI (\texttt{use\_gui=False}).
\end{itemize}

\paragraph{RL-Umgebung.}
Die PettingZoo-Umgebung wird mit SuperSuit vorbereitet (\texttt{pad\_observations\_v0}, \texttt{pad\_action\_space\_v0}, \texttt{pettingzoo\_env\_to\_vec\_env\_v1}, \texttt{concat\_vec\_envs\_v1}) und mit \texttt{VecMonitor} versehen.
Die Normalisierung wird aus \texttt{vecnormalize.pkl} geladen und eingefroren (\texttt{training=False}, \texttt{norm\_reward=False}); die Modelle werden deterministisch ausgeführt.

\paragraph{Baselines.}
Für die Baselines wird die SUMO-interne Steuerung genutzt (\texttt{fixed\_ts=True} für Fixed-Time, \texttt{fixed\_ts=False} für Actuated).
Ein \emph{Dummy-Reward} (\texttt{0.0}) stellt sicher, dass kein RL-Einfluss auf die Steuerung erfolgt; Aktionen sind gültige Platzhalter aus dem Action Space.

\subsubsection{Ablauf je Episode}
Der Ablauf ist für RL und Baselines einheitlich:
\begin{enumerate}
    \item Reset der Umgebung,
    \item Simulation bis zum Terminalzustand,
    \item Pro Schritt: (i) Aktion (bei RL deterministisch via \texttt{model.predict}), (ii) \texttt{env.step}, (iii) Metrik-Sampling aus \texttt{infos},
    \item Episodenende: Mittelwerte aller numerischen Keys sowie \texttt{ep\_rew} (kumuliert) und \texttt{ep\_len} (Schritte).
\end{enumerate}
Baselines führen Dummy-Aktionen aus, die Steuerungslogik stammt jedoch aus SUMO.

\subsubsection{Seeds und Replikationsdesign}
Das Seed-Design ist eindimensional: Es wird eine feste Menge von zehn \emph{Episoden-Seeds} (\texttt{EP\_SEEDS}) verwendet, die vom Training entkoppelt sind.
Pro Kombination aus \emph{Modell}, \emph{Szenario} und \emph{Methode} werden \texttt{N\_EPISODES = 10} Episoden mit diesen Seeds ausgewertet. Damit ergibt sich:
\[
    \#\text{Episoden} = |\texttt{RUNS}| \cdot 4 \cdot 10 \cdot 3.
\]

\subsubsection{Metriken und Logging}
Alle numerischen \texttt{infos}-Keys werden über die Episoden gemittelt.
Für die Visualisierung werden Präfixe teilweise entfernt (\texttt{system\_*} $\rightarrow$ ohne Präfix).
Pro Kombination werden Metriken für \texttt{Baseline\_FixedTime}, \texttt{Baseline\_Actuated} und \texttt{RL} geloggt (\texttt{tensorboard} und \texttt{stdout}); die Logs liegen unter
\texttt{evaluation/logs/eval\_run\{run\_idx\}\_\{run\_dir\}\_\{scenario\}}.
Die aggregierten Ergebnisse werden zusätzlich in \texttt{evaluation/eval\_results.json} persistiert.

\subsubsection{Reproduzierbarkeit}
Die Evaluation ist deterministisch durch:
\begin{enumerate}
    \item feste \texttt{EP\_SEEDS},
    \item deterministische Aktionswahl bei RL (\texttt{deterministic=True}),
    \item eingefrorene \texttt{VecNormalize}-Parameter,
    \item feste Simulationsdauer, identische Netz- und Routen-Dateien.
\end{enumerate}

Diese Strategie stellt sicher, dass die in Kapitel~\ref{sec:validation} präsentierten Ergebnisse reproduzierbar, belastbar und zwischen RL-Varianten sowie Baselines vergleichbar sind.


\section{Evaluation und Ergebnisse}
\label{sec:validation}

In diesem Kapitel werden die Ergebnisse der Evaluationsläufe präsentiert.
Alle Modelle (vier Trainingsseeds pro Reward-Variante) sowie die beiden Baselines (Fixed-Time, Actuated)
wurden in allen vier Szenarien (\texttt{morning\_peak}, \texttt{evening\_peak}, \texttt{uniform}, \texttt{random\_heavy})
jeweils über zehn Episoden evaluiert.
Die Resultate sind Mittelwerte pro Kombination aus \emph{Methode} und \emph{Szenario}.

\subsection{Reward: Diff-Waiting-Time}
Die erste Gruppe von RL-Modellen wurde mit einem Reward trainiert, der die Differenz der Wartezeiten minimiert.

\subsubsection{Mittlere Wartezeiten}
\label{sec:diff-waiting-time-wartezeit}
\begin{figure}[H]
    \centering
    \begin{tikzpicture}
        \begin{axis}[
                ybar,
                bar width=0.25cm,
                width=12cm,
                height=8cm,
                enlarge x limits=0.15,
                ylabel={Mittlere Wartezeit [s]},
                symbolic x coords={evening_peak,morning_peak,random_heavy,uniform},
                xtick=data,
                xticklabels={\text{evening\_peak},\text{morning\_peak},\text{random\_heavy},\text{uniform}},
                x tick label style={rotate=45,anchor=east},
                legend style={at={(1.05,0.5)}, anchor=west},
                ymajorgrids=true,
                grid style=dashed,
                every axis plot post/.append style={thick, fill=.!50}
            ]

            % Baseline FixedTime
            \addplot+[color1, error bars/.cd,
                y dir=minus, y explicit,
                error bar style={line width=1pt, black}] table [
                    x=scenario, y=system_mean_waiting_time_mean, col sep=comma, y error=system_mean_waiting_time_std
                ] {chapters/evaluation/results/diff-waiting-time/Baseline_FixedTime.csv};
            \addlegendentry{Baseline FixedTime}

            % RL Modell 1
            \addplot+[color2, error bars/.cd,
                y dir=minus, y explicit,
                error bar style={line width=1pt, black}] table [
                    x=scenario, y=system_mean_waiting_time_mean, col sep=comma, y error=system_mean_waiting_time_std
                ] {chapters/evaluation/results/diff-waiting-time/ppo_sumo_456_2025-08-18_01-18-32_456.csv};
            \addlegendentry{Model 1}


            % RL Modell 2
            \addplot+[color3, error bars/.cd,
                y dir=minus, y explicit,
                error bar style={line width=1pt, black}] table [
                    x=scenario, y=system_mean_waiting_time_mean, col sep=comma, y error=system_mean_waiting_time_std
                ] {chapters/evaluation/results/diff-waiting-time/ppo_sumo_13755_2025-08-18_04-48-47_13755.csv};
            \addlegendentry{Model 2}

            % RL Modell 3
            \addplot+[color4, error bars/.cd,
                y dir=minus, y explicit,
                error bar style={line width=1pt, black}] table [
                    x=scenario, y=system_mean_waiting_time_mean, col sep=comma, y error=system_mean_waiting_time_std
                ] {chapters/evaluation/results/diff-waiting-time/ppo_sumo_143534_2025-08-17_23-30-08_143534.csv};
            \addlegendentry{Model 3}

            % RL Modell 4
            \addplot+[color5, error bars/.cd,
                y dir=minus, y explicit,
                error bar style={line width=1pt, black}] table [
                    x=scenario, y=system_mean_waiting_time_mean, col sep=comma, y error=system_mean_waiting_time_std
                ] {chapters/evaluation/results/diff-waiting-time/ppo_sumo_635768_2025-08-18_03-03-29_635768.csv};
            \addlegendentry{Model 4}
        \end{axis}
    \end{tikzpicture}
    \caption{Mittlere Wartezeiten}
    \label{fig:diff-waiting-time-wartezeit}
\end{figure}



\begin{figure}[H]
    \centering
    \begin{tikzpicture}
        \begin{axis}[
                ybar,
                bar width=0.25cm,
                width=12cm,
                height=8cm,
                enlarge x limits=0.15,
                ylabel={Mittlere Wartezeit [s]},
                symbolic x coords={evening_peak,morning_peak,random_heavy,uniform},
                xtick=data,
                xticklabels={\text{evening\_peak},\text{morning\_peak},\text{random\_heavy},\text{uniform}},
                x tick label style={rotate=45,anchor=east},
                legend style={at={(1.05,0.5)}, anchor=west},
                ymajorgrids=true,
                grid style=dashed,
                every axis plot post/.append style={thick, fill=.!50}
            ]

            % Baseline FixedTime
            \addplot+[color1, error bars/.cd,
                y dir=minus, y explicit,
                error bar style={line width=1pt, black}] table [
                    x=scenario, y=system_mean_waiting_time_mean, col sep=comma, y error=system_mean_waiting_time_std
                ] {chapters/evaluation/results/diff-waiting-time/Baseline_FixedTime.csv};
            \addlegendentry{Baseline FixedTime}

            % Baseline FixedTime
            \addplot+[color6, error bars/.cd,
                y dir=minus, y explicit,
                error bar style={line width=1pt, black}] table [
                    x=scenario, y=system_mean_waiting_time_mean, col sep=comma, y error=system_mean_waiting_time_std
                ] {chapters/evaluation/results/diff-waiting-time/Baseline_Actuated.csv};
            \addlegendentry{Baseline Actuated}
        \end{axis}
    \end{tikzpicture}
    \caption{Mittlere Wartezeiten}
    \label{fig:diff-waiting-time-wartezeit2}
\end{figure}

Die mittlere Wartezeit für die Diff-Waiting-Time-Rewardfunktion zeigt erneut deutliche Unterschiede zwischen den Baselines und den trainierten Modellen.

Die Fixed-Time-Baseline erreicht im morning\_peak 3.74 s, im evening\_peak 4.20 s, im random\_heavy 4.40 s und im uniform-Szenario 3.90 s. Diese Werte bilden eine stabile Referenz. Die Actuated-Baseline fällt dagegen erneut durch extrem hohe Wartezeiten auf: 964 s, 965 s, 1030 s und 935 s in den vier Szenarien.

Unter den trainierten Modellen zeigen sich differenzierte Muster. Modell 1 erreicht sehr geringe Werte mit 0.20 s, 4.27 s, 8.90 s und 0.21 s. Modell 2 weist im morning\_peak mit 7.9 s sowie im random\_heavy mit 82 s deutlich höhere Wartezeiten auf, während die Werte im evening\_peak (2.1 s) und im uniform (1.5 s) niedrig bleiben. Modell 3 zeigt ebenfalls inkonsistente Ergebnisse: 2.5 s im morning\_peak, aber 95 s im evening\_peak und 77 s im random\_heavy, während das uniform-Szenario mit 0.4 s vergleichsweise gut abschneidet. Modell 4 erzielt insgesamt die besten Resultate mit 0.11 s, 0.18 s, 33 s und 0.10 s, wenngleich auch hier im random\_heavy ein Anstieg erkennbar ist.

Die erhöhten Mittelwerte in einzelnen Szenarien, insbesondere bei Modell 2 und Modell 3, gehen jeweils mit einer hohen Standardabweichung einher. Dies weist auf eine deutliche Instabilität zwischen den Runs hin und deutet darauf, dass die Modelle zwar vereinzelt effiziente Strategien entwickeln konnten, diese jedoch nicht konsistent reproduziert werden.

\subsubsection{Anzahl stoppender Fahrzeuge}
In sumo werden Fahrzeuge die sich mit einer Geschwindigkeit kleiner als 0.1 m/s bewegt.
\begin{figure}[H]
    \centering
    \begin{tikzpicture}
        \begin{axis}[
                ybar,
                bar width=0.25cm,
                width=12cm,
                height=8cm,
                enlarge x limits=0.15,
                ylabel={Anzahl stoppender Fahrzeuge},
                symbolic x coords={evening_peak,morning_peak,random_heavy,uniform},
                xtick=data,
                xticklabels={\text{evening\_peak},\text{morning\_peak},\text{random\_heavy},\text{uniform}},
                x tick label style={rotate=45,anchor=east},
                legend style={at={(1.05,0.5)}, anchor=west},
                ymajorgrids=true,
                grid style=dashed,
                every axis plot post/.append style={thick, fill=.!50}
            ]

            % Baseline FixedTime
            \addplot+[color1, error bars/.cd,
                y dir=minus, y explicit,
                error bar style={line width=1pt, black}] table [
                    x=scenario, y=system_total_stopped_mean, col sep=comma, y error=system_total_stopped_std
                ] {chapters/evaluation/results/diff-waiting-time/Baseline_FixedTime.csv};
            \addlegendentry{Baseline FixedTime}

            % RL Modell 1
            \addplot+[color2, error bars/.cd,
                y dir=minus, y explicit,
                error bar style={line width=1pt, black}] table [
                    x=scenario, y=system_total_stopped_mean, col sep=comma, y error=system_total_stopped_std
                ] {chapters/evaluation/results/diff-waiting-time/ppo_sumo_456_2025-08-18_01-18-32_456.csv};
            \addlegendentry{Model 1}


            % RL Modell 2
            \addplot+[color3, error bars/.cd,
                y dir=minus, y explicit,
                error bar style={line width=1pt, black}] table [
                    x=scenario, y=system_total_stopped_mean, col sep=comma, y error=system_total_stopped_std
                ] {chapters/evaluation/results/diff-waiting-time/ppo_sumo_13755_2025-08-18_04-48-47_13755.csv};
            \addlegendentry{Model 2}

            % RL Modell 3
            \addplot+[color4, error bars/.cd,
                y dir=minus, y explicit,
                error bar style={line width=1pt, black}] table [
                    x=scenario, y=system_total_stopped_mean, col sep=comma, y error=system_total_stopped_std
                ] {chapters/evaluation/results/diff-waiting-time/ppo_sumo_143534_2025-08-17_23-30-08_143534.csv};
            \addlegendentry{Model 3}

            % RL Modell 4
            \addplot+[color5, error bars/.cd,
                y dir=minus, y explicit,
                error bar style={line width=1pt, black}] table [
                    x=scenario, y=system_total_stopped_mean, col sep=comma, y error=system_total_stopped_std
                ] {chapters/evaluation/results/diff-waiting-time/ppo_sumo_635768_2025-08-18_03-03-29_635768.csv};
            \addlegendentry{Model 4}
        \end{axis}
    \end{tikzpicture}
    \caption{Anzahl stoppender Fahrzeuge}
    \label{fig:diff-waiting-time-stopped}
\end{figure}



\begin{figure}[H]
    \centering
    \begin{tikzpicture}
        \begin{axis}[
                ybar,
                bar width=0.25cm,
                width=12cm,
                height=8cm,
                enlarge x limits=0.15,
                ylabel={Anzahl stoppender Fahrzeuge},
                symbolic x coords={evening_peak,morning_peak,random_heavy,uniform},
                xtick=data,
                xticklabels={\text{evening\_peak},\text{morning\_peak},\text{random\_heavy},\text{uniform}},
                x tick label style={rotate=45,anchor=east},
                legend style={at={(1.05,0.5)}, anchor=west},
                ymajorgrids=true,
                grid style=dashed,
                every axis plot post/.append style={thick, fill=.!50}
            ]

            % Baseline FixedTime
            \addplot+[color1, error bars/.cd,
                y dir=minus, y explicit,
                error bar style={line width=1pt, black}] table [
                    x=scenario, y=system_total_stopped_mean, col sep=comma, y error=system_total_stopped_std
                ] {chapters/evaluation/results/diff-waiting-time/Baseline_FixedTime.csv};
            \addlegendentry{Baseline FixedTime}

            % Baseline FixedTime
            \addplot+[color6, error bars/.cd,
                y dir=minus, y explicit,
                error bar style={line width=1pt, black}] table [
                    x=scenario, y=system_total_stopped_mean, col sep=comma, y error=system_total_stopped_std
                ] {chapters/evaluation/results/diff-waiting-time/Baseline_Actuated.csv};
            \addlegendentry{Baseline Actuated}
        \end{axis}
    \end{tikzpicture}
    \caption{Anzahl stoppender Fahrzeuge}
    \label{fig:diff-waiting-time-stopped2}
\end{figure}

Die Ergebnisse zur Anzahl stoppender Fahrzeuge zeigen deutliche Unterschiede zwischen den Baselines und den trainierten Modellen.

Die Fixed-Time-Baseline erreicht im morning\_peak durchschnittlich 12 Fahrzeuge, im evening\_peak 14 Fahrzeuge, im random\_heavy 27 Fahrzeuge und im uniform-Szenario 10 Fahrzeuge. Diese Werte bilden eine stabile und vergleichsweise effiziente Referenz.

Die Actuated-Baseline zeigt dagegen eine massiv erhöhte Zahl an Stopps. So ergeben sich im morning\_peak 539 Fahrzeuge, im evening\_peak 566 Fahrzeuge, im random\_heavy 883 Fahrzeuge und im uniform 442 Fahrzeuge. Damit bestätigt sich auch in dieser Metrik die unzureichende Leistungsfähigkeit der Actuated-Steuerung.

Die trainierten Modelle liefern deutlich geringere Werte. Modell 1 erreicht 1.8, 4.1, 6.6 und 1.5 Fahrzeuge in den vier Szenarien. Modell 2 erzielt mit 2.12, 2.1, 17 und 1.2 Fahrzeugen ebenfalls niedrige Werte, allerdings mit einem deutlichen Anstieg im random\_heavy-Szenario. Modell 3 weist Werte von 2.64, 10, 16 und 1.6 Fahrzeugen auf, wobei insbesondere im evening\_peak und im random\_heavy eine Verschlechterung gegenüber den übrigen Szenarien erkennbar ist. Modell 4 erzielt mit 1.4, 1.7, 9.9 und 1.06 Fahrzeugen die insgesamt besten Resultate und bleibt in allen Szenarien unterhalb der Fixed-Time-Baseline.

Die erhöhten Werte bei Modell 2 und Modell 3 im random\_heavy-Szenario gehen mit einer hohen Standardabweichung einher, was auf starke Schwankungen zwischen den Runs hindeutet. Dies verdeutlicht, dass die Modelle zwar in der Lage sind, den Verkehrsfluss erheblich zu verbessern, ihre Robustheit unter unregelmäßigen Verkehrslasten jedoch eingeschränkt bleibt.

\subsubsection{Anzahl ankommender Fahrzeuge}
\label{sec:diff-waiting-time-ankommend}

\begin{figure}[H]
    \centering
    \begin{tikzpicture}
        \begin{axis}[
                ybar,
                bar width=0.25cm,
                width=12cm,
                height=8cm,
                enlarge x limits=0.15,
                ylabel={Anzahl ankommender Fahrzeuge},
                symbolic x coords={evening_peak,morning_peak,random_heavy,uniform},
                xtick=data,
                xticklabels={\text{evening\_peak},\text{morning\_peak},\text{random\_heavy},\text{uniform}},
                x tick label style={rotate=45,anchor=east},
                legend style={at={(1.05,0.5)}, anchor=west},
                ymajorgrids=true,
                grid style=dashed,
                every axis plot post/.append style={thick, fill=.!50}
            ]

            % Baseline FixedTime
            \addplot+[color1, error bars/.cd,
                y dir=minus, y explicit,
                error bar style={line width=1pt, black}] table [
                    x=scenario, y=system_total_arrived_mean, col sep=comma, y error=system_total_arrived_std
                ] {chapters/evaluation/results/diff-waiting-time/Baseline_FixedTime.csv};
            \addlegendentry{Baseline FixedTime}

            % Baseline FixedTime
            \addplot+[color6, error bars/.cd,
                y dir=minus, y explicit,
                error bar style={line width=1pt, black}] table [
                    x=scenario, y=system_total_arrived_mean, col sep=comma, y error=system_total_arrived_std
                ] {chapters/evaluation/results/diff-waiting-time/Baseline_Actuated.csv};
            \addlegendentry{Baseline Actuated}

            % RL Modell 1
            \addplot+[color2, error bars/.cd,
                y dir=minus, y explicit,
                error bar style={line width=1pt, black}] table [
                    x=scenario, y=system_total_arrived_mean, col sep=comma, y error=system_total_arrived_std
                ] {chapters/evaluation/results/diff-waiting-time/ppo_sumo_456_2025-08-18_01-18-32_456.csv};
            \addlegendentry{Model 1}

            % RL Modell 2
            \addplot+[color3, error bars/.cd,
                y dir=minus, y explicit,
                error bar style={line width=1pt, black}] table [
                    x=scenario, y=system_total_arrived_mean, col sep=comma, y error=system_total_arrived_std
                ] {chapters/evaluation/results/diff-waiting-time/ppo_sumo_13755_2025-08-18_04-48-47_13755.csv};
            \addlegendentry{Model 2}

            % RL Modell 3
            \addplot+[color4, error bars/.cd,
                y dir=minus, y explicit,
                error bar style={line width=1pt, black}] table [
                    x=scenario, y=system_total_arrived_mean, col sep=comma, y error=system_total_arrived_std
                ] {chapters/evaluation/results/diff-waiting-time/ppo_sumo_143534_2025-08-17_23-30-08_143534.csv};
            \addlegendentry{Model 3}

            % RL Modell 4
            \addplot+[color5, error bars/.cd,
                y dir=minus, y explicit,
                error bar style={line width=1pt, black}] table [
                    x=scenario, y=system_total_arrived_mean, col sep=comma, y error=system_total_arrived_std
                ] {chapters/evaluation/results/diff-waiting-time/ppo_sumo_635768_2025-08-18_03-03-29_635768.csv};
            \addlegendentry{Model 4}
        \end{axis}
    \end{tikzpicture}
    \caption{Anzahl ankommender Fahrzeuge}
    \label{fig:diff-waiting-time-arrived}
\end{figure}

Hinsichtlich der Anzahl ankommender Fahrzeuge unterscheiden sich die RL-Modelle kaum von der Fixed-Time-Baseline. Alle Modelle erreichen nahezu identische Werte, was darauf hinweist, dass die Steuerungsstrategien trotz der reduzierten Warte- und Stoppzeiten keine signifikanten Auswirkungen auf die Durchsatzkapazität des Netzes haben. Einzig die Actuated-Baseline zeigt hier, wie auch in den anderen Metriken, ein deutlich schlechteres Abschneiden.



\subsubsection{Durchschnitt fahrender Fahrzeuge}
\label{sec:diff-waiting-time-fahrende}

\begin{figure}[H]
    \centering
    \begin{tikzpicture}
        \begin{axis}[
                ybar,
                bar width=0.25cm,
                width=12cm,
                height=8cm,
                enlarge x limits=0.15,
                ylabel={Durchschnitt fahrender Fahrzeuge},
                symbolic x coords={evening_peak,morning_peak,random_heavy,uniform},
                xtick=data,
                xticklabels={\text{evening\_peak},\text{morning\_peak},\text{random\_heavy},\text{uniform}},
                x tick label style={rotate=45,anchor=east},
                legend style={at={(1.05,0.5)}, anchor=west},
                ymajorgrids=true,
                grid style=dashed,
                every axis plot post/.append style={thick, fill=.!50}
            ]

            % Baseline FixedTime
            \addplot+[color1, error bars/.cd,
                y dir=minus, y explicit,
                error bar style={line width=1pt, black}] table [
                    x=scenario, y=system_total_running_mean, col sep=comma, y error=system_total_running_std
                ] {chapters/evaluation/results/diff-waiting-time/Baseline_FixedTime.csv};
            \addlegendentry{Baseline FixedTime}

            % Baseline FixedTime
            \addplot+[color6, error bars/.cd,
                y dir=minus, y explicit,
                error bar style={line width=1pt, black}] table [
                    x=scenario, y=system_total_running_mean, col sep=comma, y error=system_total_running_std
                ] {chapters/evaluation/results/diff-waiting-time/Baseline_Actuated.csv};
            \addlegendentry{Baseline Actuated}

            % RL Modell 1
            \addplot+[color2, error bars/.cd,
                y dir=minus, y explicit,
                error bar style={line width=1pt, black}] table [
                    x=scenario, y=system_total_running_mean, col sep=comma, y error=system_total_running_std
                ] {chapters/evaluation/results/diff-waiting-time/ppo_sumo_456_2025-08-18_01-18-32_456.csv};
            \addlegendentry{Model 1}

            % RL Modell 2
            \addplot+[color3, error bars/.cd,
                y dir=minus, y explicit,
                error bar style={line width=1pt, black}] table [
                    x=scenario, y=system_total_running_mean, col sep=comma, y error=system_total_running_std
                ] {chapters/evaluation/results/diff-waiting-time/ppo_sumo_13755_2025-08-18_04-48-47_13755.csv};
            \addlegendentry{Model 2}

            % RL Modell 3
            \addplot+[color4, error bars/.cd,
                y dir=minus, y explicit,
                error bar style={line width=1pt, black}] table [
                    x=scenario, y=system_total_running_mean, col sep=comma, y error=system_total_running_std
                ] {chapters/evaluation/results/diff-waiting-time/ppo_sumo_143534_2025-08-17_23-30-08_143534.csv};
            \addlegendentry{Model 3}

            % RL Modell 4
            \addplot+[color5, error bars/.cd,
                y dir=minus, y explicit,
                error bar style={line width=1pt, black}] table [
                    x=scenario, y=system_total_running_mean, col sep=comma, y error=system_total_running_std
                ] {chapters/evaluation/results/diff-waiting-time/ppo_sumo_635768_2025-08-18_03-03-29_635768.csv};
            \addlegendentry{Model 4}
        \end{axis}
    \end{tikzpicture}
    \caption{Durchschnitt fahrender Fahrzeuge}
    \label{fig:diff-waiting-time-running}
\end{figure}

Die Auswertung des Durchschnitts fahrender Fahrzeuge (weniger ist besser) zeigt deutliche Unterschiede zwischen den Baselines und den trainierten Modellen.

Die Fixed-Time-Baseline erreicht im morning\_peak durchschnittlich 63 Fahrzeuge, im evening\_peak 69 Fahrzeuge, im random\_heavy 121 Fahrzeuge und im uniform-Szenario 52 Fahrzeuge. Diese Werte bilden eine stabile Referenz.

Die Actuated-Baseline weist deutlich höhere Werte auf, mit 553 Fahrzeugen im morning\_peak, 481 Fahrzeugen im evening\_peak, 900 Fahrzeugen im random\_heavy und 455 Fahrzeugen im uniform-Szenario. Damit bestätigt sich das schwache Abschneiden dieser Steuerung auch in dieser Metrik.

Die trainierten Modelle erreichen insgesamt Werte, die sehr nahe an der Fixed-Time-Baseline liegen. Modell 1 erzielt 53, 58, 98 und 44 Fahrzeuge. Modell 2 liegt bei 53, 55, 108 und 43 Fahrzeugen. Modell 3 weist 54, 64, 108 und 44 Fahrzeuge auf, während Modell 4 mit 52, 55, 101 und 43 Fahrzeugen die niedrigsten Werte liefert.

Insgesamt zeigen die Modelle konsistent bessere oder vergleichbare Ergebnisse zur Fixed-Time-Baseline. Auffällig ist, dass im random\_heavy-Szenario die Werte von Modell 2 und Modell 3 leicht über der Referenz liegen, was mit einer erhöhten Standardabweichung einhergeht und auf Instabilität in komplexen Verkehrslagen hinweist.

\subsubsection{Durchschnittsgeschwindigkeiten}
\label{sec:diff-waiting-time-geschwindigkeiten}

\begin{figure}[H]
    \centering
    \begin{tikzpicture}
        \begin{axis}[
                ybar,
                bar width=0.25cm,
                width=12cm,
                height=8cm,
                enlarge x limits=0.15,
                ylabel={Durchschnittsgeschwindigkeit [m/s]} ,
                symbolic x coords={evening_peak,morning_peak,random_heavy,uniform},
                xtick=data,
                xticklabels={\text{evening\_peak},\text{morning\_peak},\text{random\_heavy},\text{uniform}},
                x tick label style={rotate=45,anchor=east},
                legend style={at={(1.05,0.5)}, anchor=west},
                ymajorgrids=true,
                grid style=dashed,
                every axis plot post/.append style={thick, fill=.!50}
            ]

            % Baseline FixedTime
            \addplot+[color1, error bars/.cd,
                y dir=minus, y explicit,
                error bar style={line width=1pt, black}] table [
                    x=scenario, y=system_mean_speed_mean, col sep=comma, y error=system_mean_speed_std
                ] {chapters/evaluation/results/diff-waiting-time/Baseline_FixedTime.csv};
            \addlegendentry{Baseline FixedTime}

            % Baseline FixedTime
            \addplot+[color6, error bars/.cd,
                y dir=minus, y explicit,
                error bar style={line width=1pt, black}] table [
                    x=scenario, y=system_mean_speed_mean, col sep=comma, y error=system_mean_speed_std
                ] {chapters/evaluation/results/diff-waiting-time/Baseline_Actuated.csv};
            \addlegendentry{Baseline Actuated}

            % RL Modell 1
            \addplot+[color2, error bars/.cd,
                y dir=minus, y explicit,
                error bar style={line width=1pt, black}] table [
                    x=scenario, y=system_mean_speed_mean, col sep=comma, y error=system_mean_speed_std
                ] {chapters/evaluation/results/diff-waiting-time/ppo_sumo_456_2025-08-18_01-18-32_456.csv};
            \addlegendentry{Model 1}

            % RL Modell 2
            \addplot+[color3, error bars/.cd,
                y dir=minus, y explicit,
                error bar style={line width=1pt, black}] table [
                    x=scenario, y=system_mean_speed_mean, col sep=comma, y error=system_mean_speed_std
                ] {chapters/evaluation/results/diff-waiting-time/ppo_sumo_13755_2025-08-18_04-48-47_13755.csv};
            \addlegendentry{Model 2}

            % RL Modell 3
            \addplot+[color4, error bars/.cd,
                y dir=minus, y explicit,
                error bar style={line width=1pt, black}] table [
                    x=scenario, y=system_mean_speed_mean, col sep=comma, y error=system_mean_speed_std
                ] {chapters/evaluation/results/diff-waiting-time/ppo_sumo_143534_2025-08-17_23-30-08_143534.csv};
            \addlegendentry{Model 3}

            % RL Modell 4
            \addplot+[color5, error bars/.cd,
                y dir=minus, y explicit,
                error bar style={line width=1pt, black}] table [
                    x=scenario, y=system_mean_speed_mean, col sep=comma, y error=system_mean_speed_std
                ] {chapters/evaluation/results/diff-waiting-time/ppo_sumo_635768_2025-08-18_03-03-29_635768.csv};
            \addlegendentry{Model 4}
        \end{axis}
    \end{tikzpicture}
    \caption{Durchschnittsgeschwindigkeiten}
    \label{fig:diff-waiting-time-speed}
\end{figure}

Die Analyse der Durchschnittsgeschwindigkeiten verdeutlicht klare Unterschiede zwischen den Baselines und den trainierten Modellen.

Die Fixed-Time-Baseline erreicht im morning\_peak durchschnittlich 5.9 m/s, im evening\_peak 5.7 m/s, im random\_heavy 5.4 m/s sowie im uniform-Szenario 5.9 m/s. Damit liefert sie konsistente, aber nicht optimale Werte.

Die Actuated-Baseline weist durchgehend extrem niedrige Geschwindigkeiten auf: 0.6, 0.5, 0.5 und 0.7 m/s in den vier Szenarien. Diese Werte liegen um eine Größenordnung unterhalb der Fixed-Time-Baseline und bestätigen die unzureichende Leistungsfähigkeit dieser Steuerung.

Die trainierten Modelle übertreffen die Fixed-Time-Baseline deutlich. Modell 1 erreicht 7.0, 6.8, 6.7 und 7.1 m/s. Modell 2 erzielt sehr stabile Werte mit 7.1, 7.1, 6.2 und 7.2 m/s. Modell 3 liegt mit 6.9, 6.2, 6.1 und 7.1 m/s insgesamt ebenfalls über der Fixed-Time-Baseline, zeigt jedoch im evening\_peak und random\_heavy leicht reduzierte Ergebnisse. Modell 4 liefert mit 7.1, 7.1, 6.5 und 7.2 m/s die besten Resultate, insbesondere durch die hohe Stabilität über alle Szenarien hinweg.

Auffällig ist, dass die Modelle 2 und 4 durchgehend eine nahezu konstante Verbesserung gegenüber der Fixed-Time-Baseline erzielen, während Modell 3 in den Szenarien evening\_peak und random\_heavy leichten Performanceverlust zeigt. In diesen Fällen geht die Verschlechterung mit einer erhöhten Standardabweichung einher, was auf eine weniger stabile Performanz hindeutet.

\subsubsection{Anzahl teleportierender Fahrzeuge}
\label{sec:diff-waiting-time-teleport}

\begin{figure}[H]
    \centering
    \begin{tikzpicture}
        \begin{axis}[
                ybar,
                bar width=0.25cm,
                width=12cm,
                height=5cm,
                enlarge x limits=0.15,
                ylabel={Anzahl teleportierender Fahrzeuge},
                symbolic x coords={evening_peak,morning_peak,random_heavy,uniform},
                xtick=data,
                xticklabels={\text{evening\_peak},\text{morning\_peak},\text{random\_heavy},\text{uniform}},
                x tick label style={rotate=45,anchor=east},
                legend style={at={(1.05,0.5)}, anchor=west},
                ymajorgrids=true,
                grid style=dashed,
                every axis plot post/.append style={thick, fill=.!50}
            ]

            % Baseline FixedTime
            \addplot+[color1, error bars/.cd,
                y dir=minus, y explicit,
                error bar style={line width=1pt, black}] table [
                    x=scenario, y=system_total_teleported_mean, col sep=comma
                ] {chapters/evaluation/results/diff-waiting-time/Baseline_FixedTime.csv};
            \addlegendentry{Baseline FixedTime}

            % Baseline FixedTime
            \addplot+[color6, error bars/.cd,
                y dir=minus, y explicit,
                error bar style={line width=1pt, black}] table [
                    x=scenario, y=system_total_teleported_mean, col sep=comma
                ] {chapters/evaluation/results/diff-waiting-time/Baseline_Actuated.csv};
            \addlegendentry{Baseline Actuated}

            % RL Modell 1
            \addplot+[color2, error bars/.cd,
                y dir=minus, y explicit,
                error bar style={line width=1pt, black}] table [
                    x=scenario, y=system_total_teleported_mean, col sep=comma
                ] {chapters/evaluation/results/diff-waiting-time/ppo_sumo_456_2025-08-18_01-18-32_456.csv};
            \addlegendentry{Model 1}

            % RL Modell 2
            \addplot+[color3, error bars/.cd,
                y dir=minus, y explicit,
                error bar style={line width=1pt, black}] table [
                    x=scenario, y=system_total_teleported_mean, col sep=comma
                ] {chapters/evaluation/results/diff-waiting-time/ppo_sumo_13755_2025-08-18_04-48-47_13755.csv};
            \addlegendentry{Model 2}

            % RL Modell 3
            \addplot+[color4, error bars/.cd,
                y dir=minus, y explicit,
                error bar style={line width=1pt, black}] table [
                    x=scenario, y=system_total_teleported_mean, col sep=comma
                ] {chapters/evaluation/results/diff-waiting-time/ppo_sumo_143534_2025-08-17_23-30-08_143534.csv};
            \addlegendentry{Model 3}

            % RL Modell 4
            \addplot+[color5, error bars/.cd,
                y dir=minus, y explicit,
                error bar style={line width=1pt, black}] table [
                    x=scenario, y=system_total_teleported_mean, col sep=comma
                ] {chapters/evaluation/results/diff-waiting-time/ppo_sumo_635768_2025-08-18_03-03-29_635768.csv};
            \addlegendentry{Model 4}
        \end{axis}
    \end{tikzpicture}
    \caption{Anzahl teleportierender Fahrzeuge}
    \label{fig:diff-waiting-time-teleports}
\end{figure}

Ein Sonderfall ergibt sich bei der Metrik der Teleportationen: Während in fast allen Szenarien keine Teleportationen auftraten, kam es in einzelnen Episoden zu vereinzelten Fällen. Konkret traten bei Modell 3 und Modell 1 im Szenario random\_heavy sowie bei Modell 1 im Szenario uniform jeweils einzelne Teleportationen auf. Angesichts der insgesamt geringen Häufigkeit lassen sich diese Ereignisse als Ausnahmen werten, die die Gesamtbewertung der Modelle kaum beeinflussen.

\subsubsection{Anzahl zurückgehaltener Zahrzeuge}
\label{sec:diff-waiting-time-backlogged}

\begin{figure}[H]
    \centering
    \begin{tikzpicture}
        \begin{axis}[
                ybar,
                bar width=0.25cm,
                width=12cm,
                height=5cm,
                enlarge x limits=0.15,
                ylabel={Anzahl zurückgehaltener Zahrzeuge},
                symbolic x coords={evening_peak,morning_peak,random_heavy,uniform},
                xtick=data,
                xticklabels={\text{evening\_peak},\text{morning\_peak},\text{random\_heavy},\text{uniform}},
                x tick label style={rotate=45,anchor=east},
                legend style={at={(1.05,0.5)}, anchor=west},
                ymajorgrids=true,
                grid style=dashed,
                every axis plot post/.append style={thick, fill=.!50}
            ]

            % Baseline FixedTime
            \addplot+[color1, error bars/.cd,
                y dir=minus, y explicit,
                error bar style={line width=1pt, black}] table [
                    x=scenario, y=system_total_backlogged_mean, col sep=comma
                ] {chapters/evaluation/results/diff-waiting-time/Baseline_FixedTime.csv};
            \addlegendentry{Baseline FixedTime}

            % Baseline FixedTime
            \addplot+[color6, error bars/.cd,
                y dir=minus, y explicit,
                error bar style={line width=1pt, black}] table [
                    x=scenario, y=system_total_backlogged_mean, col sep=comma
                ] {chapters/evaluation/results/diff-waiting-time/Baseline_Actuated.csv};
            \addlegendentry{Baseline Actuated}

            % RL Modell 1
            \addplot+[color2, error bars/.cd,
                y dir=minus, y explicit,
                error bar style={line width=1pt, black}] table [
                    x=scenario, y=system_total_backlogged_mean, col sep=comma
                ] {chapters/evaluation/results/diff-waiting-time/ppo_sumo_456_2025-08-18_01-18-32_456.csv};
            \addlegendentry{Model 1}

            % RL Modell 2
            \addplot+[color3, error bars/.cd,
                y dir=minus, y explicit,
                error bar style={line width=1pt, black}] table [
                    x=scenario, y=system_total_backlogged_mean, col sep=comma
                ] {chapters/evaluation/results/diff-waiting-time/ppo_sumo_13755_2025-08-18_04-48-47_13755.csv};
            \addlegendentry{Model 2}

            % RL Modell 3
            \addplot+[color4, error bars/.cd,
                y dir=minus, y explicit,
                error bar style={line width=1pt, black}] table [
                    x=scenario, y=system_total_backlogged_mean, col sep=comma
                ] {chapters/evaluation/results/diff-waiting-time/ppo_sumo_143534_2025-08-17_23-30-08_143534.csv};
            \addlegendentry{Model 3}

            % RL Modell 4
            \addplot+[color5, error bars/.cd,
                y dir=minus, y explicit,
                error bar style={line width=1pt, black}] table [
                    x=scenario, y=system_total_backlogged_mean, col sep=comma
                ] {chapters/evaluation/results/diff-waiting-time/ppo_sumo_635768_2025-08-18_03-03-29_635768.csv};
            \addlegendentry{Model 4}
        \end{axis}
    \end{tikzpicture}
    \caption{Anzahl zurückgehaltener Zahrzeuge}
    \label{fig:diff-waiting-time-backlogged}
\end{figure}

Es ist klar erkenntbar, dass das Netz nicht an ihre maximal Auslastung gerät. Dies bedeutet jedoch nicht, dass keine Staus oder überlastetet Teilgebiete gibt.

\subsubsection{Einstufung}
\label{sec:diff-waiting-time-einstufung}
Die Modelle, die mit der Diff-Waiting-Time-Rewardfunktion trainiert wurden, zeigen in den meisten Metriken eine klare Überlegenheit gegenüber den Baselines. Die Fixed-Time-Baseline dient als stabile Referenz, wird jedoch in nahezu allen Szenarien von den Modellen deutlich übertroffen. Die Actuated-Baseline bestätigt auch hier ihre Schwäche und liefert durchweg die schlechtesten Ergebnisse.

Besonders auffällig ist die deutliche Reduktion der mittleren Wartezeiten und der Anzahl stoppender Fahrzeuge. Modelle 1 und 4 erreichen dabei die insgesamt besten Resultate und bleiben sowohl in regulären als auch in stark belasteten Szenarien konsistent unterhalb der Fixed-Time-Baseline. Modell 2 und Modell 3 zeigen ebenfalls Verbesserungen, jedoch treten in Szenarien mit hoher Verkehrslast (random\_heavy, evening\_peak) teils deutlich erhöhte Werte und zugleich hohe Standardabweichungen auf. Diese Schwankungen weisen auf eine geringere Stabilität der Strategien hin.

Hinsichtlich der Durchschnittsgeschwindigkeit erzielen alle Modelle höhere Werte als die Fixed-Time-Baseline, wobei insbesondere Modell 2 und Modell 4 durch ihre gleichmäßige Performance hervorstechen. Modell 3 fällt in einzelnen Szenarien leicht zurück, bleibt aber insgesamt dennoch über der Referenz.

Die Metriken zu Teleportationen und zurückgehaltenen Fahrzeugen bestätigen, dass das Netz nicht an seine absolute Kapazitätsgrenze gelangte. Einzelne Teleportationen bei Modell 1 und Modell 3 sind als Randereignisse zu werten und beeinflussen die Gesamteinstufung nicht maßgeblich.

Insgesamt lässt sich festhalten, dass die Diff-Waiting-Time-Rewardfunktion robuste Modelle hervorbringt, die klassische Steuerungen klar übertreffen. Modelle 1 und 4 überzeugen durchgängig mit stabiler Performance, während Modell 2 und Modell 3 unter hoher Verkehrslast anfälliger für Leistungseinbrüche und stärkere Varianz sind.

\subsection{Reward: Queue}
Die zweite Gruppe zielt auf die Minimierung der Warteschlangenlänge.


% Text mit Interpretation: RL reduziert Staus, wirkt sich auf Durchfluss aus etc.


\subsection{Reward: Reale Welt}
Diese Gruppe ziel auf ein kombiniertes Minimieren der Wartezeiten, Anzahl an Phasenwechsel und Staus.

\subsubsection{Mittlere Wartezeiten}
\label{sec:realworld-wartezeit}
\begin{figure}[H]
    \centering
    \begin{tikzpicture}
        \begin{axis}[
                ybar,
                bar width=0.25cm,
                width=12cm,
                height=8cm,
                enlarge x limits=0.15,
                ylabel={Mittlere Wartezeit [s]},
                symbolic x coords={evening_peak,morning_peak,random_heavy,uniform},
                xtick=data,
                xticklabels={\text{evening\_peak},\text{morning\_peak},\text{random\_heavy},\text{uniform}},
                x tick label style={rotate=45,anchor=east},
                legend style={at={(1.05,0.5)}, anchor=west},
                ymajorgrids=true,
                grid style=dashed,
                every axis plot post/.append style={thick, fill=.!50}
            ]

            % Baseline FixedTime
            \addplot+[color1, error bars/.cd,
                y dir=minus, y explicit,
                error bar style={line width=1pt, black}] table [
                    x=scenario, y=system_mean_waiting_time_mean, col sep=comma, y error=system_mean_waiting_time_std
                ] {chapters/evaluation/results/realworld/Baseline_FixedTime.csv};
            \addlegendentry{Baseline FixedTime}

            % RL Modell 1
            \addplot+[color2, error bars/.cd,
                y dir=minus, y explicit,
                error bar style={line width=1pt, black}] table [
                    x=scenario, y=system_mean_waiting_time_mean, col sep=comma, y error=system_mean_waiting_time_std
                ] {chapters/evaluation/results/realworld/ppo_sumo_456_2025-08-18_01-08-35_456.csv};
            \addlegendentry{Model 1}

            % RL Modell 2
            \addplot+[color3, error bars/.cd,
                y dir=minus, y explicit,
                error bar style={line width=1pt, black}] table [
                    x=scenario, y=system_mean_waiting_time_mean, col sep=comma, y error=system_mean_waiting_time_std
                ] {chapters/evaluation/results/realworld/ppo_sumo_13755_2025-08-18_07-38-05_13755.csv};
            \addlegendentry{Model 2}

            % RL Modell 3
            \addplot+[color4, error bars/.cd,
                y dir=minus, y explicit,
                error bar style={line width=1pt, black}] table [
                    x=scenario, y=system_mean_waiting_time_mean, col sep=comma, y error=system_mean_waiting_time_std
                ] {chapters/evaluation/results/realworld/ppo_sumo_143534_2025-08-17_21-54-21_143534.csv};
            \addlegendentry{Model 3}

            % RL Modell 4
            \addplot+[color5, error bars/.cd,
                y dir=minus, y explicit,
                error bar style={line width=1pt, black}] table [
                    x=scenario, y=system_mean_waiting_time_mean, col sep=comma, y error=system_mean_waiting_time_std
                ] {chapters/evaluation/results/realworld/ppo_sumo_635768_2025-08-18_04-23-15_635768.csv};
            \addlegendentry{Model 4}
        \end{axis}
    \end{tikzpicture}
    \caption{Mittlere Wartezeiten}
    \label{fig:realworld-wartezeit}
\end{figure}



\begin{figure}[H]
    \centering
    \begin{tikzpicture}
        \begin{axis}[
                ybar,
                bar width=0.25cm,
                width=12cm,
                height=5cm,
                enlarge x limits=0.15,
                ylabel={Mittlere Wartezeit [s]},
                symbolic x coords={evening_peak,morning_peak,random_heavy,uniform},
                xtick=data,
                xticklabels={\text{evening\_peak},\text{morning\_peak},\text{random\_heavy},\text{uniform}},
                x tick label style={rotate=45,anchor=east},
                legend style={at={(1.05,0.5)}, anchor=west},
                ymajorgrids=true,
                grid style=dashed,
                every axis plot post/.append style={thick, fill=.!50}
            ]

            % Baseline FixedTime
            \addplot+[color1, error bars/.cd,
                y dir=minus, y explicit,
                error bar style={line width=1pt, black}] table [
                    x=scenario, y=system_mean_waiting_time_mean, col sep=comma, y error=system_mean_waiting_time_std
                ] {chapters/evaluation/results/realworld/Baseline_FixedTime.csv};
            \addlegendentry{Baseline FixedTime}

            % Baseline FixedTime
            \addplot+[color6, error bars/.cd,
                y dir=minus, y explicit,
                error bar style={line width=1pt, black}] table [
                    x=scenario, y=system_mean_waiting_time_mean, col sep=comma, y error=system_mean_waiting_time_std
                ] {chapters/evaluation/results/realworld/Baseline_Actuated.csv};
            \addlegendentry{Baseline Actuated}
        \end{axis}
    \end{tikzpicture}
    \caption{Mittlere Wartezeiten}
    \label{fig:realworld-wartezeit2}
\end{figure}

Die Ergebnisse zur mittleren Wartezeit verdeutlichen erneut die starken Unterschiede zwischen den Verfahren.

Die Fixed-Time-Baseline bewegt sich in allen Szenarien auf einem stabilen Niveau von rund vier Sekunden und liefert damit konsistente, wenn auch nicht optimale Werte. Demgegenüber erreicht die Actuated-Baseline extrem hohe Mittelwerte im Bereich von fast eintausend Sekunden, was ihre geringe Effizienz klar unterstreicht.

Die trainierten Modelle erzielen insgesamt deutlich niedrigere Wartezeiten, zeigen jedoch unterschiedliche Stabilität. Modell 1 und Modell 2 erreichen in den meisten Szenarien sehr geringe Werte im Sub-Sekundenbereich und liegen damit weit unterhalb der Fixed-Time-Baseline; nur im evening\_peak und im random\_heavy steigen die Wartezeiten auf einige Dutzend Sekunden. Modell 3 und Modell 4 weisen hingegen in denselben Szenarien deutlich höhere Mittelwerte von mehreren Dutzend bis über hundert Sekunden auf, während sie in den übrigen Fällen ebenfalls sehr niedrige Werte erreichen.

Auffällig ist die hohe Varianz gerade bei Modell 3 und Modell 4 in den schwierigeren Szenarien. Dort schwanken die Ergebnisse stark zwischen einzelnen Episoden, was darauf hinweist, dass die Modelle teils sehr effiziente, teils aber auch deutlich weniger stabile Steuerungsstrategien hervorbringen.

\subsubsection{Anzahl stoppender Fahrzeuge}
\begin{figure}[H]
    \centering
    \begin{tikzpicture}
        \begin{axis}[
                ybar,
                bar width=0.25cm,
                width=12cm,
                height=8cm,
                enlarge x limits=0.15,
                ylabel={Anzahl stoppender Fahrzeuge},
                symbolic x coords={evening_peak,morning_peak,random_heavy,uniform},
                xtick=data,
                xticklabels={\text{evening\_peak},\text{morning\_peak},\text{random\_heavy},\text{uniform}},
                x tick label style={rotate=45,anchor=east},
                legend style={at={(1.05,0.5)}, anchor=west},
                ymajorgrids=true,
                grid style=dashed,
                every axis plot post/.append style={thick, fill=.!50}
            ]

            % Baseline FixedTime
            \addplot+[color1, error bars/.cd,
                y dir=minus, y explicit,
                error bar style={line width=1pt, black}] table [
                    x=scenario, y=system_total_stopped_mean, col sep=comma, y error=system_total_stopped_std
                ] {chapters/evaluation/results/realworld/Baseline_FixedTime.csv};
            \addlegendentry{Baseline FixedTime}

            % RL Modell 1
            \addplot+[color2, error bars/.cd,
                y dir=minus, y explicit,
                error bar style={line width=1pt, black}] table [
                    x=scenario, y=system_total_stopped_mean, col sep=comma, y error=system_total_stopped_std
                ] {chapters/evaluation/results/realworld/ppo_sumo_456_2025-08-18_01-08-35_456.csv};
            \addlegendentry{Model 1}


            % RL Modell 2
            \addplot+[color3, error bars/.cd,
                y dir=minus, y explicit,
                error bar style={line width=1pt, black}] table [
                    x=scenario, y=system_total_stopped_mean, col sep=comma, y error=system_total_stopped_std
                ] {chapters/evaluation/results/realworld/ppo_sumo_13755_2025-08-18_07-38-05_13755.csv};
            \addlegendentry{Model 2}

            % RL Modell 3
            \addplot+[color4, error bars/.cd,
                y dir=minus, y explicit,
                error bar style={line width=1pt, black}] table [
                    x=scenario, y=system_total_stopped_mean, col sep=comma, y error=system_total_stopped_std
                ] {chapters/evaluation/results/realworld/ppo_sumo_143534_2025-08-17_21-54-21_143534.csv};
            \addlegendentry{Model 3}

            % RL Modell 4
            \addplot+[color5, error bars/.cd,
                y dir=minus, y explicit,
                error bar style={line width=1pt, black}] table [
                    x=scenario, y=system_total_stopped_mean, col sep=comma, y error=system_total_stopped_std
                ] {chapters/evaluation/results/realworld/ppo_sumo_635768_2025-08-18_04-23-15_635768.csv};
            \addlegendentry{Model 4}
        \end{axis}
    \end{tikzpicture}
    \caption{Anzahl stoppender Fahrzeuge}
    \label{fig:realworld-stopped}
\end{figure}



\begin{figure}[H]
    \centering
    \begin{tikzpicture}
        \begin{axis}[
                ybar,
                bar width=0.25cm,
                width=12cm,
                height=5cm,
                enlarge x limits=0.15,
                ylabel={Anzahl stoppender Fahrzeuge},
                symbolic x coords={evening_peak,morning_peak,random_heavy,uniform},
                xtick=data,
                xticklabels={\text{evening\_peak},\text{morning\_peak},\text{random\_heavy},\text{uniform}},
                x tick label style={rotate=45,anchor=east},
                legend style={at={(1.05,0.5)}, anchor=west},
                ymajorgrids=true,
                grid style=dashed,
                every axis plot post/.append style={thick, fill=.!50}
            ]

            % Baseline FixedTime
            \addplot+[color1, error bars/.cd,
                y dir=minus, y explicit,
                error bar style={line width=1pt, black}] table [
                    x=scenario, y=system_total_stopped_mean, col sep=comma, y error=system_total_stopped_std
                ] {chapters/evaluation/results/realworld/Baseline_FixedTime.csv};
            \addlegendentry{Baseline FixedTime}

            % Baseline FixedTime
            \addplot+[color6, error bars/.cd,
                y dir=minus, y explicit,
                error bar style={line width=1pt, black}] table [
                    x=scenario, y=system_total_stopped_mean, col sep=comma, y error=system_total_stopped_std
                ] {chapters/evaluation/results/realworld/Baseline_Actuated.csv};
            \addlegendentry{Baseline Actuated}
        \end{axis}
    \end{tikzpicture}
    \caption{Anzahl stoppender Fahrzeuge}
    \label{fig:realworld-stopped2}
\end{figure}
Die mittlere Zahl stoppender Fahrzeuge unterscheidet sich deutlich zwischen den Verfahren.

Die Fixed-Time-Baseline bewegt sich in allen Szenarien auf einem niedrigen zweistelligen Niveau und liefert damit eine solide Referenz. Demgegenüber weist die Actuated-Baseline mit mehreren Hundert bis fast 900 Stopps pro Episode deutlich höhere Werte auf und bestätigt erneut ihre geringe Leistungsfähigkeit.

Die trainierten Modelle zeigen eine deutliche Reduktion: meist liegt die Zahl der Stopps nur bei ein bis wenigen Fahrzeugen. Besonders im morning\_peak und im uniform-Szenario erreichen alle Modelle extrem niedrige Werte. Auffällig ist jedoch, dass Modell 2 bis 4 im random\_heavy-Szenario deutlich höhere Werte aufweisen, die teils mehrere Dutzend Fahrzeuge umfassen. Modell 1 bleibt dagegen auch hier auf einem vergleichsweise niedrigen Niveau.

Die stark erhöhten Werte im random\_heavy gehen mit einer hohen Streuung zwischen den Episoden einher. Dies deutet darauf hin, dass die Modelle zwar häufig sehr effiziente Steuerungsstrategien finden, diese jedoch in einzelnen Durchläufen nicht stabil reproduziert werden.

\subsubsection{Anzahl ankommender Fahrzeuge}
\label{sec:realworld-ankommend}

\begin{figure}[H]
    \centering
    \begin{tikzpicture}
        \begin{axis}[
                ybar,
                bar width=0.25cm,
                width=12cm,
                height=8cm,
                enlarge x limits=0.15,
                ylabel={Anzahl ankommender Fahrzeuge},
                symbolic x coords={evening_peak,morning_peak,random_heavy,uniform},
                xtick=data,
                xticklabels={\text{evening\_peak},\text{morning\_peak},\text{random\_heavy},\text{uniform}},
                x tick label style={rotate=45,anchor=east},
                legend style={at={(1.05,0.5)}, anchor=west},
                ymajorgrids=true,
                grid style=dashed,
                every axis plot post/.append style={thick, fill=.!50}
            ]

            % Baseline FixedTime
            \addplot+[color1, error bars/.cd,
                y dir=minus, y explicit,
                error bar style={line width=1pt, black}] table [
                    x=scenario, y=system_total_arrived_mean, col sep=comma, y error=system_total_arrived_std
                ] {chapters/evaluation/results/realworld/Baseline_FixedTime.csv};
            \addlegendentry{Baseline FixedTime}

            % Baseline FixedTime
            \addplot+[color6, error bars/.cd,
                y dir=minus, y explicit,
                error bar style={line width=1pt, black}] table [
                    x=scenario, y=system_total_arrived_mean, col sep=comma, y error=system_total_arrived_std
                ] {chapters/evaluation/results/realworld/Baseline_Actuated.csv};
            \addlegendentry{Baseline Actuated}

            % RL Modell 1
            \addplot+[color2, error bars/.cd,
                y dir=minus, y explicit,
                error bar style={line width=1pt, black}] table [
                    x=scenario, y=system_total_arrived_mean, col sep=comma, y error=system_total_arrived_std
                ] {chapters/evaluation/results/realworld/ppo_sumo_456_2025-08-18_01-08-35_456.csv};
            \addlegendentry{Model 1}

            % RL Modell 2
            \addplot+[color3, error bars/.cd,
                y dir=minus, y explicit,
                error bar style={line width=1pt, black}] table [
                    x=scenario, y=system_total_arrived_mean, col sep=comma, y error=system_total_arrived_std
                ] {chapters/evaluation/results/realworld/ppo_sumo_13755_2025-08-18_07-38-05_13755.csv};
            \addlegendentry{Model 2}

            % RL Modell 3
            \addplot+[color4, error bars/.cd,
                y dir=minus, y explicit,
                error bar style={line width=1pt, black}] table [
                    x=scenario, y=system_total_arrived_mean, col sep=comma, y error=system_total_arrived_std
                ] {chapters/evaluation/results/realworld/ppo_sumo_143534_2025-08-17_21-54-21_143534.csv};
            \addlegendentry{Model 3}

            % RL Modell 4
            \addplot+[color5, error bars/.cd,
                y dir=minus, y explicit,
                error bar style={line width=1pt, black}] table [
                    x=scenario, y=system_total_arrived_mean, col sep=comma, y error=system_total_arrived_std
                ] {chapters/evaluation/results/realworld/ppo_sumo_635768_2025-08-18_04-23-15_635768.csv};
            \addlegendentry{Model 4}
        \end{axis}
    \end{tikzpicture}
    \caption{Anzahl ankommender Fahrzeuge}
    \label{fig:realworld-arrived}
\end{figure}

Die Auswertung der Anzahl ankommender Fahrzeuge zeigt über alle Szenarien hinweg sehr ähnliche Ergebnisse für die Fixed-Time-Baseline und die trainierten Modelle. Sowohl die Baseline als auch die Modelle erreichen in den meisten Szenarien nahezu das Maximum, was darauf hindeutet, dass der Verkehrsfluss grundsätzlich zuverlässig abgewickelt wird.

Auffällig ist lediglich, dass in random\_heavy einzelne Modelle eine leicht geringere Leistung aufweisen als die Fixed-Time-Baseline. Dieser Rückgang bleibt jedoch moderat, und die Anzahl ankommender Fahrzeuge liegt weiterhin auf einem hohen Niveau. In den übrigen Szenarien (morning\_peak, evening\_peak, uniform) stimmen die Resultate nahezu exakt mit der Fixed-Time-Baseline überein.

Die Actuated-Baseline bestätigt erneut ihre Schwäche und fällt in allen Szenarien deutlich ab. Das deutliche Defizit dieser Steuerungsstrategie kontrastiert stark mit den stabil hohen Werten der Fixed-Time-Baseline und der Modelle.

\subsubsection{Durchschnitt fahrender Fahrzeuge}
\label{sec:realworld-fahrende}

\begin{figure}[H]
    \centering
    \begin{tikzpicture}
        \begin{axis}[
                ybar,
                bar width=0.25cm,
                width=12cm,
                height=8cm,
                enlarge x limits=0.15,
                ylabel={Durchschnitt fahrender Fahrzeuge},
                symbolic x coords={evening_peak,morning_peak,random_heavy,uniform},
                xtick=data,
                xticklabels={\text{evening\_peak},\text{morning\_peak},\text{random\_heavy},\text{uniform}},
                x tick label style={rotate=45,anchor=east},
                legend style={at={(1.05,0.5)}, anchor=west},
                ymajorgrids=true,
                grid style=dashed,
                every axis plot post/.append style={thick, fill=.!50}
            ]

            % Baseline FixedTime
            \addplot+[color1, error bars/.cd,
                y dir=minus, y explicit,
                error bar style={line width=1pt, black}] table [
                    x=scenario, y=system_total_running_mean, col sep=comma, y error=system_total_running_std
                ] {chapters/evaluation/results/realworld/Baseline_FixedTime.csv};
            \addlegendentry{Baseline FixedTime}

            % Baseline FixedTime
            \addplot+[color6, error bars/.cd,
                y dir=minus, y explicit,
                error bar style={line width=1pt, black}] table [
                    x=scenario, y=system_total_running_mean, col sep=comma, y error=system_total_running_std
                ] {chapters/evaluation/results/realworld/Baseline_Actuated.csv};
            \addlegendentry{Baseline Actuated}

            % RL Modell 1
            \addplot+[color2, error bars/.cd,
                y dir=minus, y explicit,
                error bar style={line width=1pt, black}] table [
                    x=scenario, y=system_total_running_mean, col sep=comma, y error=system_total_running_std
                ] {chapters/evaluation/results/realworld/ppo_sumo_456_2025-08-18_01-08-35_456.csv};
            \addlegendentry{Model 1}

            % RL Modell 2
            \addplot+[color3, error bars/.cd,
                y dir=minus, y explicit,
                error bar style={line width=1pt, black}] table [
                    x=scenario, y=system_total_running_mean, col sep=comma, y error=system_total_running_std
                ] {chapters/evaluation/results/realworld/ppo_sumo_13755_2025-08-18_07-38-05_13755.csv};
            \addlegendentry{Model 2}

            % RL Modell 3
            \addplot+[color4, error bars/.cd,
                y dir=minus, y explicit,
                error bar style={line width=1pt, black}] table [
                    x=scenario, y=system_total_running_mean, col sep=comma, y error=system_total_running_std
                ] {chapters/evaluation/results/realworld/ppo_sumo_143534_2025-08-17_21-54-21_143534.csv};
            \addlegendentry{Model 3}

            % RL Modell 4
            \addplot+[color5, error bars/.cd,
                y dir=minus, y explicit,
                error bar style={line width=1pt, black}] table [
                    x=scenario, y=system_total_running_mean, col sep=comma, y error=system_total_running_std
                ] {chapters/evaluation/results/realworld/ppo_sumo_635768_2025-08-18_04-23-15_635768.csv};
            \addlegendentry{Model 4}
        \end{axis}
    \end{tikzpicture}
    \caption{Durchschnitt fahrender Fahrzeuge}
    \label{fig:realworld-running}
\end{figure}

Die Analyse der durchschnittlichen Anzahl an Fahrzeugen im Netz zeigt klare Unterschiede. Ein geringerer Wert bedeutet dabei eine schnellere Abwicklung des Verkehrs.

Die Fixed-Time-Baseline liegt je nach Szenario zwischen rund 50 und 70 Fahrzeugen, im stark belasteten random\_heavy-Fall bei etwa 120. Damit erreicht sie ein insgesamt konsistentes Niveau.

Die Actuated-Baseline schneidet deutlich schlechter ab: mit mehreren hundert Fahrzeugen im Netz liegt sie um ein Vielfaches über der Fixed-Time-Variante und bestätigt ihre geringe Eignung.

Die trainierten Modelle erzielen Werte, die weitgehend auf dem Niveau der Fixed-Time-Baseline liegen. Typischerweise bewegen sie sich bei etwa 40 bis 60 Fahrzeugen, im random\_heavy-Szenario zwischen 100 und 130. Auffällig ist, dass einzelne Modelle in diesem Szenario etwas höhere Werte zeigen, was auch mit einer größeren Streuung einhergeht. In den übrigen Szenarien liefern sie jedoch nahezu identische Ergebnisse zur Fixed-Time-Baseline.

\subsubsection{Durchschnittsgeschwindigkeiten}
\label{sec:realworld-geschwindigkeiten}

\begin{figure}[H]
    \centering
    \begin{tikzpicture}
        \begin{axis}[
                ybar,
                bar width=0.25cm,
                width=12cm,
                height=8cm,
                enlarge x limits=0.15,
                ylabel={Durchschnittsgeschwindigkeit [m/s]} ,
                symbolic x coords={evening_peak,morning_peak,random_heavy,uniform},
                xtick=data,
                xticklabels={\text{evening\_peak},\text{morning\_peak},\text{random\_heavy},\text{uniform}},
                x tick label style={rotate=45,anchor=east},
                legend style={at={(1.05,0.5)}, anchor=west},
                ymajorgrids=true,
                grid style=dashed,
                every axis plot post/.append style={thick, fill=.!50}
            ]

            % Baseline FixedTime
            \addplot+[color1, error bars/.cd,
                y dir=minus, y explicit,
                error bar style={line width=1pt, black}] table [
                    x=scenario, y=system_mean_speed_mean, col sep=comma, y error=system_mean_speed_std
                ] {chapters/evaluation/results/realworld/Baseline_FixedTime.csv};
            \addlegendentry{Baseline FixedTime}

            % Baseline FixedTime
            \addplot+[color6, error bars/.cd,
                y dir=minus, y explicit,
                error bar style={line width=1pt, black}] table [
                    x=scenario, y=system_mean_speed_mean, col sep=comma, y error=system_mean_speed_std
                ] {chapters/evaluation/results/realworld/Baseline_Actuated.csv};
            \addlegendentry{Baseline Actuated}

            % RL Modell 1
            \addplot+[color2, error bars/.cd,
                y dir=minus, y explicit,
                error bar style={line width=1pt, black}] table [
                    x=scenario, y=system_mean_speed_mean, col sep=comma, y error=system_mean_speed_std
                ] {chapters/evaluation/results/realworld/ppo_sumo_456_2025-08-18_01-08-35_456.csv};
            \addlegendentry{Model 1}

            % RL Modell 2
            \addplot+[color3, error bars/.cd,
                y dir=minus, y explicit,
                error bar style={line width=1pt, black}] table [
                    x=scenario, y=system_mean_speed_mean, col sep=comma, y error=system_mean_speed_std
                ] {chapters/evaluation/results/realworld/ppo_sumo_13755_2025-08-18_07-38-05_13755.csv};
            \addlegendentry{Model 2}

            % RL Modell 3
            \addplot+[color4, error bars/.cd,
                y dir=minus, y explicit,
                error bar style={line width=1pt, black}] table [
                    x=scenario, y=system_mean_speed_mean, col sep=comma, y error=system_mean_speed_std
                ] {chapters/evaluation/results/realworld/ppo_sumo_143534_2025-08-17_21-54-21_143534.csv};
            \addlegendentry{Model 3}

            % RL Modell 4
            \addplot+[color5, error bars/.cd,
                y dir=minus, y explicit,
                error bar style={line width=1pt, black}] table [
                    x=scenario, y=system_mean_speed_mean, col sep=comma, y error=system_mean_speed_std
                ] {chapters/evaluation/results/realworld/ppo_sumo_635768_2025-08-18_04-23-15_635768.csv};
            \addlegendentry{Model 4}
        \end{axis}
    \end{tikzpicture}
    \caption{Durchschnittsgeschwindigkeiten}
    \label{fig:realworld-speed}
\end{figure}

Die Analyse der Durchschnittsgeschwindigkeiten macht deutliche Unterschiede sichtbar. Grundsätzlich gilt: höhere Werte bedeuten einen effizienteren Verkehrsfluss.

Die Fixed-Time-Baseline bewegt sich in allen Szenarien auf einem guten Niveau von rund 5½ bis 6 m/s und liegt damit klar über der Actuated-Variante. Diese erreicht nur etwa ½ m/s und ist damit um eine ganze Größenordnung schlechter.

Die trainierten Modelle übertreffen die Fixed-Time-Baseline deutlich und liegen meist zwischen etwa 6½ und 7¼ m/s. Besonders stabil zeigt sich Modell 2, das in allen Szenarien nahe bei 7 m/s bleibt. Modelle 3 und 4 fallen im stark belasteten random\_heavy-Szenario etwas zurück (teils nur knapp über 5½ m/s), während sie in den übrigen Fällen mit den besten Ergebnissen gleichziehen.

Insgesamt bestätigen die Ergebnisse: die trainierten Modelle steigern die Durchschnittsgeschwindigkeit im Vergleich zu beiden Baselines spürbar. Nur im zufällig stark belasteten Szenario zeigen sich Ausreißer und eine höhere Streuung, in allen anderen Fällen ist der Zugewinn stabil und konsistent.
\subsubsection{Anzahl teleportierender Fahrzeuge}

\label{sec:realworld-teleport}

\begin{figure}[H]
    \centering
    \begin{tikzpicture}
        \begin{axis}[
                ybar,
                bar width=0.25cm,
                width=12cm,
                height=5cm,
                enlarge x limits=0.15,
                ylabel={Anzahl teleportierender Fahrzeuge},
                symbolic x coords={evening_peak,morning_peak,random_heavy,uniform},
                xtick=data,
                xticklabels={\text{evening\_peak},\text{morning\_peak},\text{random\_heavy},\text{uniform}},
                x tick label style={rotate=45,anchor=east},
                legend style={at={(1.05,0.5)}, anchor=west},
                ymajorgrids=true,
                grid style=dashed,
                every axis plot post/.append style={thick, fill=.!50}
            ]

            % Baseline FixedTime
            \addplot+[color1, error bars/.cd,
                y dir=minus, y explicit,
                error bar style={line width=1pt, black}] table [
                    x=scenario, y=system_total_teleported_mean, col sep=comma
                ] {chapters/evaluation/results/realworld/Baseline_FixedTime.csv};
            \addlegendentry{Baseline FixedTime}

            % Baseline FixedTime
            \addplot+[color6, error bars/.cd,
                y dir=minus, y explicit,
                error bar style={line width=1pt, black}] table [
                    x=scenario, y=system_total_teleported_mean, col sep=comma
                ] {chapters/evaluation/results/realworld/Baseline_Actuated.csv};
            \addlegendentry{Baseline Actuated}

            % RL Modell 1
            \addplot+[color2, error bars/.cd,
                y dir=minus, y explicit,
                error bar style={line width=1pt, black}] table [
                    x=scenario, y=system_total_teleported_mean, col sep=comma
                ] {chapters/evaluation/results/realworld/ppo_sumo_456_2025-08-18_01-08-35_456.csv};
            \addlegendentry{Model 1}

            % RL Modell 2
            \addplot+[color3, error bars/.cd,
                y dir=minus, y explicit,
                error bar style={line width=1pt, black}] table [
                    x=scenario, y=system_total_teleported_mean, col sep=comma
                ] {chapters/evaluation/results/realworld/ppo_sumo_13755_2025-08-18_07-38-05_13755.csv};
            \addlegendentry{Model 2}

            % RL Modell 3
            \addplot+[color4, error bars/.cd,
                y dir=minus, y explicit,
                error bar style={line width=1pt, black}] table [
                    x=scenario, y=system_total_teleported_mean, col sep=comma
                ] {chapters/evaluation/results/realworld/ppo_sumo_143534_2025-08-17_21-54-21_143534.csv};
            \addlegendentry{Model 3}

            % RL Modell 4
            \addplot+[color5, error bars/.cd,
                y dir=minus, y explicit,
                error bar style={line width=1pt, black}] table [
                    x=scenario, y=system_total_teleported_mean, col sep=comma
                ] {chapters/evaluation/results/realworld/ppo_sumo_635768_2025-08-18_04-23-15_635768.csv};
            \addlegendentry{Model 4}
        \end{axis}
    \end{tikzpicture}
    \caption{Anzahl teleportierender Fahrzeuge}
    \label{fig:realworld-teleports}
\end{figure}

Die Auswertung der Teleportationen zeigt, dass in nahezu allen Szenarien keine Fahrzeuge teleportiert werden mussten. Dies gilt sowohl für die beiden Baselines als auch für die trainierten Modelle. Eine Ausnahme bildet das Szenario random\_heavy bei Modell 3, in dem insgesammt eine Teleportation innerhalb der 10 Episoden auftrat. Dieses Ergebnis steht im Einklang mit den zuvor beobachteten Schwächen desselben Modells in diesem Szenario.

\subsubsection{Anzahl zurückgehaltener Fahrzeuge}
\label{sec:realworld-backlogged}

\begin{figure}[H]
    \centering
    \begin{tikzpicture}
        \begin{axis}[
                ybar,
                bar width=0.25cm,
                width=12cm,
                height=5cm,
                enlarge x limits=0.15,
                ylabel={Anzahl zurückgehaltener Fahrzeuge},
                symbolic x coords={evening_peak,morning_peak,random_heavy,uniform},
                xtick=data,
                xticklabels={\text{evening\_peak},\text{morning\_peak},\text{random\_heavy},\text{uniform}},
                x tick label style={rotate=45,anchor=east},
                legend style={at={(1.05,0.5)}, anchor=west},
                ymajorgrids=true,
                grid style=dashed,
                every axis plot post/.append style={thick, fill=.!50}
            ]

            % Baseline FixedTime
            \addplot+[color1, error bars/.cd,
                y dir=minus, y explicit,
                error bar style={line width=1pt, black}] table [
                    x=scenario, y=system_total_backlogged_mean, col sep=comma
                ] {chapters/evaluation/results/realworld/Baseline_FixedTime.csv};
            \addlegendentry{Baseline FixedTime}

            % Baseline FixedTime
            \addplot+[color6, error bars/.cd,
                y dir=minus, y explicit,
                error bar style={line width=1pt, black}] table [
                    x=scenario, y=system_total_backlogged_mean, col sep=comma
                ] {chapters/evaluation/results/realworld/Baseline_Actuated.csv};
            \addlegendentry{Baseline Actuated}

            % RL Modell 1
            \addplot+[color2, error bars/.cd,
                y dir=minus, y explicit,
                error bar style={line width=1pt, black}] table [
                    x=scenario, y=system_total_backlogged_mean, col sep=comma
                ] {chapters/evaluation/results/realworld/ppo_sumo_456_2025-08-18_01-08-35_456.csv};
            \addlegendentry{Model 1}

            % RL Modell 2
            \addplot+[color3, error bars/.cd,
                y dir=minus, y explicit,
                error bar style={line width=1pt, black}] table [
                    x=scenario, y=system_total_backlogged_mean, col sep=comma
                ] {chapters/evaluation/results/realworld/ppo_sumo_13755_2025-08-18_07-38-05_13755.csv};
            \addlegendentry{Model 2}

            % RL Modell 3
            \addplot+[color4, error bars/.cd,
                y dir=minus, y explicit,
                error bar style={line width=1pt, black}] table [
                    x=scenario, y=system_total_backlogged_mean, col sep=comma
                ] {chapters/evaluation/results/realworld/ppo_sumo_143534_2025-08-17_21-54-21_143534.csv};
            \addlegendentry{Model 3}

            % RL Modell 4
            \addplot+[color5, error bars/.cd,
                y dir=minus, y explicit,
                error bar style={line width=1pt, black}] table [
                    x=scenario, y=system_total_backlogged_mean, col sep=comma
                ] {chapters/evaluation/results/realworld/ppo_sumo_635768_2025-08-18_04-23-15_635768.csv};
            \addlegendentry{Model 4}
        \end{axis}
    \end{tikzpicture}
    \caption{Anzahl zurückgehaltener Fahrzeuge}
    \label{fig:realworld-backlogged}
\end{figure}
\newpage
In allen Szenarien zeigen die vier Modelle sowie die Fixed-Time-Baseline keine zurückgehaltenen Fahrzeuge. Die Actuated-Baseline weist jedoch in sämtlichen Szenarien deutliche Werte auf, die mit der insgesamt schwachen Leistung dieser Methode konsistent sind. Dieses Ergebnis bestätigt, dass ausschließlich die Actuated-Steuerung Fahrzeuge im Netz blockiert, während alle anderen Verfahren einen stabilen Verkehrsfluss ohne Zurückhalten sicherstellen konnten.

\subsubsection{Einstufung}
\label{sec:realworld-einstufung}

Die Auswertung der verschiedenen Metriken zeigt, dass die trainierten Modelle die klassischen Baselines insgesamt deutlich übertreffen. Während die Actuated-Baseline durchgehend schwache Ergebnisse liefert und selbst von der Fixed-Time-Steuerung klar geschlagen wird, gelingt es den Modellen in nahezu allen Szenarien, sowohl die mittlere Wartezeit als auch die Anzahl stoppender Fahrzeuge deutlich zu reduzieren und gleichzeitig höhere Durchschnittsgeschwindigkeiten zu erreichen.

Besonders deutlich wird der Vorteil der Modelle in den Szenarien mit regulärer oder gleichmäßiger Verkehrslast, wo sie konsistent nahe am Optimum operieren. Auch die Anzahl ankommender Fahrzeuge bleibt in diesen Fällen auf dem maximalen Niveau, sodass die Effizienzsteigerung nicht mit einem Verlust an Durchsatz erkauft wird.

Einschränkungen zeigen sich jedoch im random\_heavy-Szenario: hier treten bei mehreren Modellen signifikante Verschlechterungen auf, die zugleich mit einer hohen Standardabweichung verbunden sind. Dies weist auf eine eingeschränkte Robustheit unter komplexeren und schwer vorhersagbaren Verkehrssituationen hin. Besonders ausgeprägt sind diese Schwächen bei einem Modell, das zusätzlich vereinzelt Teleportationen aufweist und damit strukturelle Instabilitäten erkennen lässt.

Insgesamt lässt sich festhalten, dass die lernbasierten Steuerungsansätze das Potenzial besitzen, klassische Verfahren im Hinblick auf Wartezeiten, Staus und Geschwindigkeiten deutlich zu übertreffen. Gleichzeitig verdeutlichen die Ergebnisse, dass die Generalisierungsfähigkeit insbesondere in Szenarien mit unregelmäßiger und schwer prognostizierbarer Verkehrslast eine zentrale Herausforderung bleibt.
\subsection{Reward: Emissionen}
Für die Emissions-basierten Modelle wird zusätzlich die CO\textsubscript{2}-Emission als zentrale Metrik betrachtet.

\subsubsection{Mittlere Wartezeit}

\begin{figure}[H]
    \centering
    \begin{tikzpicture}
        \begin{axis}[
                ybar,
                bar width=0.25cm,
                width=12cm,
                enlarge x limits=0.15,
                ylabel={Mittlere Wartezeit},
                symbolic x coords={evening_peak,morning_peak,random_heavy,uniform},
                xtick=data,
                xticklabels={evening\_peak,morning\_peak,random\_heavy,uniform},
                x tick label style={rotate=45,anchor=east},
                legend style={at={(1.05,0.5)}, anchor=west},
                ymajorgrids=true,
                grid style=dashed,
                every axis plot post/.append style={thick, fill=.!50}
            ]

            % Baseline FixedTime
            \addplot+[color1, error bars/.cd,
                y dir=minus, y explicit,
                error bar style={line width=1pt, black}] table [
                    x=scenario, y=system_mean_waiting_time_mean, col sep=comma, y error=system_mean_waiting_time_std
                ] {chapters/evaluation/results/emissions/Baseline_FixedTime.csv};
            \addlegendentry{Baseline FixedTime}

            % RL Modell 1
            \addplot+[color2, error bars/.cd,
                y dir=minus, y explicit,
                error bar style={line width=1pt, black}] table [
                    x=scenario, y=system_mean_waiting_time_mean, col sep=comma, y error=system_mean_waiting_time_std
                ] {chapters/evaluation/results/emissions/ppo_sumo_456_2025-08-18_16-05-49_456.csv};
            \addlegendentry{Model 1}


            % RL Modell 2
            \addplot+[color3, error bars/.cd,
                y dir=minus, y explicit,
                error bar style={line width=1pt, black}] table [
                    x=scenario, y=system_mean_waiting_time_mean, col sep=comma, y error=system_mean_waiting_time_std
                ] {chapters/evaluation/results/emissions/ppo_sumo_13755_2025-08-18_21-51-17_13755.csv};
            \addlegendentry{Model 2}

            % RL Modell 3
            \addplot+[color4, error bars/.cd,
                y dir=minus, y explicit,
                error bar style={line width=1pt, black}] table [
                    x=scenario, y=system_mean_waiting_time_mean, col sep=comma, y error=system_mean_waiting_time_std
                ] {chapters/evaluation/results/emissions/ppo_sumo_143534_2025-08-18_13-16-03_143534.csv};
            \addlegendentry{Model 3}

            % RL Modell 4
            \addplot+[color5, error bars/.cd,
                y dir=minus, y explicit,
                error bar style={line width=1pt, black}] table [
                    x=scenario, y=system_mean_waiting_time_mean, col sep=comma, y error=system_mean_waiting_time_std
                ] {chapters/evaluation/results/emissions/ppo_sumo_635768_2025-08-18_18-57-01_635768.csv};
            \addlegendentry{Model 4}
        \end{axis}
    \end{tikzpicture}
\end{figure}

\begin{figure}[H]
    \centering
    \begin{tikzpicture}
        \begin{axis}[
                ybar,
                bar width=0.25cm,
                width=12cm,
                enlarge x limits=0.15,
                ylabel={Mittlere Wartezeit},
                symbolic x coords={evening_peak,morning_peak,random_heavy,uniform},
                xtick=data,
                xticklabels={evening\_peak,morning\_peak,random\_heavy,uniform},
                x tick label style={rotate=45,anchor=east},
                legend style={at={(1.05,0.5)}, anchor=west},
                ymajorgrids=true,
                grid style=dashed,
                every axis plot post/.append style={thick, fill=.!50}
            ]


            % Baseline FixedTime
            \addplot+[color1, error bars/.cd,
                y dir=minus, y explicit,
                error bar style={line width=1pt, black}] table [
                    x=scenario, y=system_mean_waiting_time_mean, col sep=comma, y error=system_mean_waiting_time_std
                ] {chapters/evaluation/results/emissions/Baseline_FixedTime.csv};
            \addlegendentry{Baseline FixedTime}
            % Baseline Actuated
            \addplot+[color6, error bars/.cd,
                y dir=minus, y explicit,
                error bar style={line width=1pt, black}] table [
                    x=scenario, y=system_mean_waiting_time_mean, col sep=comma, y error=system_mean_waiting_time_std
                ] {chapters/evaluation/results/emissions/Baseline_Actuated.csv};
            \addlegendentry{Baseline Actuated}
        \end{axis}
    \end{tikzpicture}
\end{figure}

\subsubsection{Anzahl angekommener Fahrzeuge}

\begin{figure}[H]
    \centering
    \begin{tikzpicture}
        \begin{axis}[
                ybar,
                bar width=0.25cm,
                width=12cm,
                enlarge x limits=0.15,
                ylabel={Anzahl angekommener Fahrzeuge},
                symbolic x coords={evening_peak,morning_peak,random_heavy,uniform},
                xtick=data,
                xticklabels={evening\_peak,morning\_peak,random\_heavy,uniform},
                x tick label style={rotate=45,anchor=east},
                legend style={at={(1.05,0.5)}, anchor=west},
                ymajorgrids=true,
                grid style=dashed,
                every axis plot post/.append style={thick, fill=.!50}
            ]

            % Baseline FixedTime
            \addplot+[color1, error bars/.cd,
                y dir=minus, y explicit,
                error bar style={line width=1pt, black}] table [
                    x=scenario, y=system_total_arrived_mean, col sep=comma, y error=system_total_arrived_std
                ] {chapters/evaluation/results/emissions/Baseline_FixedTime.csv};
            \addlegendentry{Baseline FixedTime}

            % RL Modell 1
            \addplot+[color2, error bars/.cd,
                y dir=minus, y explicit,
                error bar style={line width=1pt, black}] table [
                    x=scenario, y=system_total_arrived_mean, col sep=comma, y error=system_total_arrived_std
                ] {chapters/evaluation/results/emissions/ppo_sumo_456_2025-08-18_16-05-49_456.csv};
            \addlegendentry{Model 1}

            % RL Modell 2
            \addplot+[color3, error bars/.cd,
                y dir=minus, y explicit,
                error bar style={line width=1pt, black}] table [
                    x=scenario, y=system_total_arrived_mean, col sep=comma, y error=system_total_arrived_std
                ] {chapters/evaluation/results/emissions/ppo_sumo_13755_2025-08-18_21-51-17_13755.csv};
            \addlegendentry{Model 2}

            % RL Modell 3
            \addplot+[color4, error bars/.cd,
                y dir=minus, y explicit,
                error bar style={line width=1pt, black}] table [
                    x=scenario, y=system_total_arrived_mean, col sep=comma, y error=system_total_arrived_std
                ] {chapters/evaluation/results/emissions/ppo_sumo_143534_2025-08-18_13-16-03_143534.csv};
            \addlegendentry{Model 3}

            % RL Modell 4
            \addplot+[color5, error bars/.cd,
                y dir=minus, y explicit,
                error bar style={line width=1pt, black}] table [
                    x=scenario, y=system_total_arrived_mean, col sep=comma, y error=system_total_arrived_std
                ] {chapters/evaluation/results/emissions/ppo_sumo_635768_2025-08-18_18-57-01_635768.csv};
            \addlegendentry{Model 4}
        \end{axis}
    \end{tikzpicture}
\end{figure}

\begin{figure}[H]
    \centering
    \begin{tikzpicture}
        \begin{axis}[
                ybar,
                bar width=0.25cm,
                width=12cm,
                enlarge x limits=0.15,
                ylabel={Anzahl angekommener Fahrzeuge},
                symbolic x coords={evening_peak,morning_peak,random_heavy,uniform},
                xtick=data,
                xticklabels={evening\_peak,morning\_peak,random\_heavy,uniform},
                x tick label style={rotate=45,anchor=east},
                legend style={at={(1.05,0.5)}, anchor=west},
                ymajorgrids=true,
                grid style=dashed,
                every axis plot post/.append style={thick, fill=.!50}
            ]


            % Baseline FixedTime
            \addplot+[color1, error bars/.cd,
                y dir=minus, y explicit,
                error bar style={line width=1pt, black}] table [
                    x=scenario, y=system_total_arrived_mean, col sep=comma, y error=system_total_arrived_std
                ] {chapters/evaluation/results/emissions/Baseline_FixedTime.csv};
            \addlegendentry{Baseline FixedTime}
            % Baseline Actuated
            \addplot+[color6, error bars/.cd,
                y dir=minus, y explicit,
                error bar style={line width=1pt, black}] table [
                    x=scenario, y=system_total_arrived_mean, col sep=comma, y error=system_total_arrived_std
                ] {chapters/evaluation/results/emissions/Baseline_Actuated.csv};
            \addlegendentry{Baseline Actuated}
        \end{axis}
    \end{tikzpicture}
\end{figure}


\subsubsection{Durchschnitt fahrender Fahrzeuge}

\begin{figure}[H]
    \centering
    \begin{tikzpicture}
        \begin{axis}[
                ybar,
                bar width=0.25cm,
                width=12cm,
                enlarge x limits=0.15,
                ylabel={Durchschnitt fahrender Fahrzeuge},
                symbolic x coords={evening_peak,morning_peak,random_heavy,uniform},
                xtick=data,
                xticklabels={evening\_peak,morning\_peak,random\_heavy,uniform},
                x tick label style={rotate=45,anchor=east},
                legend style={at={(1.05,0.5)}, anchor=west},
                ymajorgrids=true,
                grid style=dashed,
                every axis plot post/.append style={thick, fill=.!50}
            ]

            % Baseline FixedTime
            \addplot+[color1, error bars/.cd,
                y dir=minus, y explicit,
                error bar style={line width=1pt, black}] table [
                    x=scenario, y=system_total_running_mean, col sep=comma, y error=system_total_running_std
                ] {chapters/evaluation/results/emissions/Baseline_FixedTime.csv};
            \addlegendentry{Baseline FixedTime}


            % RL Modell 1
            \addplot+[color2, error bars/.cd,
                y dir=minus, y explicit,
                error bar style={line width=1pt, black}] table [
                    x=scenario, y=system_total_running_mean, col sep=comma, y error=system_total_running_std
                ] {chapters/evaluation/results/emissions/ppo_sumo_456_2025-08-18_16-05-49_456.csv};
            \addlegendentry{Model 1}

            % RL Modell 2
            \addplot+[color3, error bars/.cd,
                y dir=minus, y explicit,
                error bar style={line width=1pt, black}] table [
                    x=scenario, y=system_total_running_mean, col sep=comma, y error=system_total_running_std
                ] {chapters/evaluation/results/emissions/ppo_sumo_13755_2025-08-18_21-51-17_13755.csv};
            \addlegendentry{Model 2}

            % RL Modell 3
            \addplot+[color4, error bars/.cd,
                y dir=minus, y explicit,
                error bar style={line width=1pt, black}] table [
                    x=scenario, y=system_total_running_mean, col sep=comma, y error=system_total_running_std
                ] {chapters/evaluation/results/emissions/ppo_sumo_143534_2025-08-18_13-16-03_143534.csv};
            \addlegendentry{Model 3}

            % RL Modell 4
            \addplot+[color5, error bars/.cd,
                y dir=minus, y explicit,
                error bar style={line width=1pt, black}] table [
                    x=scenario, y=system_total_running_mean, col sep=comma, y error=system_total_running_std
                ] {chapters/evaluation/results/emissions/ppo_sumo_635768_2025-08-18_18-57-01_635768.csv};
            \addlegendentry{Model 4}
        \end{axis}
    \end{tikzpicture}
\end{figure}


\begin{figure}[H]
    \centering
    \begin{tikzpicture}
        \begin{axis}[
                ybar,
                bar width=0.25cm,
                width=12cm,
                enlarge x limits=0.15,
                ylabel={Durchschnitt fahrender Fahrzeuge},
                symbolic x coords={evening_peak,morning_peak,random_heavy,uniform},
                xtick=data,
                xticklabels={evening\_peak,morning\_peak,random\_heavy,uniform},
                x tick label style={rotate=45,anchor=east},
                legend style={at={(1.05,0.5)}, anchor=west},
                ymajorgrids=true,
                grid style=dashed,
                every axis plot post/.append style={thick, fill=.!50}
            ]
            % Baseline FixedTime
            \addplot+[color1, error bars/.cd,
                y dir=minus, y explicit,
                error bar style={line width=1pt, black}] table [
                    x=scenario, y=system_total_running_mean, col sep=comma, y error=system_total_running_std
                ] {chapters/evaluation/results/emissions/Baseline_FixedTime.csv};
            \addlegendentry{Baseline FixedTime}
            % Baseline Actuated
            \addplot+[color6, error bars/.cd,
                y dir=minus, y explicit,
                error bar style={line width=1pt, black}] table [
                    x=scenario, y=system_total_running_mean, col sep=comma, y error=system_total_running_std
                ] {chapters/evaluation/results/emissions/Baseline_Actuated.csv};
            \addlegendentry{Baseline Actuated}
        \end{axis}
    \end{tikzpicture}
\end{figure}


\subsubsection{Durchschnittsgeschwindigkeit}

\begin{figure}[H]
    \centering
    \begin{tikzpicture}
        \begin{axis}[
                ybar,
                bar width=0.25cm,
                width=12cm,
                enlarge x limits=0.15,
                ylabel={Durchschnittsgeschwindigkeit},
                symbolic x coords={evening_peak,morning_peak,random_heavy,uniform},
                xtick=data,
                xticklabels={evening\_peak,morning\_peak,random\_heavy,uniform},
                x tick label style={rotate=45,anchor=east},
                legend style={at={(1.05,0.5)}, anchor=west},
                ymajorgrids=true,
                grid style=dashed,
                every axis plot post/.append style={thick, fill=.!50}
            ]

            % Baseline FixedTime
            \addplot+[color1, error bars/.cd,
                y dir=minus, y explicit,
                error bar style={line width=1pt, black}] table [
                    x=scenario, y=system_mean_co2_mean, col sep=comma, y error=system_mean_co2_std
                ] {chapters/evaluation/results/emissions/Baseline_FixedTime.csv};
            \addlegendentry{Baseline FixedTime}

            % RL Modell 1
            \addplot+[color2, error bars/.cd,
                y dir=minus, y explicit,
                error bar style={line width=1pt, black}] table [
                    x=scenario, y=system_mean_co2_mean, col sep=comma, y error=system_mean_co2_std
                ] {chapters/evaluation/results/emissions/ppo_sumo_456_2025-08-18_16-05-49_456.csv};
            \addlegendentry{Model 1}

            % RL Modell 2
            \addplot+[color3, error bars/.cd,
                y dir=minus, y explicit,
                error bar style={line width=1pt, black}] table [
                    x=scenario, y=system_mean_co2_mean, col sep=comma, y error=system_mean_co2_std
                ] {chapters/evaluation/results/emissions/ppo_sumo_13755_2025-08-18_21-51-17_13755.csv};
            \addlegendentry{Model 2}

            % RL Modell 3
            \addplot+[color4, error bars/.cd,
                y dir=minus, y explicit,
                error bar style={line width=1pt, black}] table [
                    x=scenario, y=system_mean_co2_mean, col sep=comma, y error=system_mean_co2_std
                ] {chapters/evaluation/results/emissions/ppo_sumo_143534_2025-08-18_13-16-03_143534.csv};
            \addlegendentry{Model 3}

            % RL Modell 4
            \addplot+[color5, error bars/.cd,
                y dir=minus, y explicit,
                error bar style={line width=1pt, black}] table [
                    x=scenario, y=system_mean_co2_mean, col sep=comma, y error=system_mean_co2_std
                ] {chapters/evaluation/results/emissions/ppo_sumo_635768_2025-08-18_18-57-01_635768.csv};
            \addlegendentry{Model 4}
        \end{axis}
    \end{tikzpicture}
\end{figure}


\begin{figure}[H]
    \centering
    \begin{tikzpicture}
        \begin{axis}[
                ybar,
                bar width=0.25cm,
                width=12cm,
                enlarge x limits=0.15,
                ylabel={Durchschnittsgeschwindigkeit},
                symbolic x coords={evening_peak,morning_peak,random_heavy,uniform},
                xtick=data,
                xticklabels={evening\_peak,morning\_peak,random\_heavy,uniform},
                x tick label style={rotate=45,anchor=east},
                legend style={at={(1.05,0.5)}, anchor=west},
                ymajorgrids=true,
                grid style=dashed,
                every axis plot post/.append style={thick, fill=.!50}
            ]
            % Baseline FixedTime
            \addplot+[color1, error bars/.cd,
                y dir=minus, y explicit,
                error bar style={line width=1pt, black}] table [
                    x=scenario, y=system_mean_co2_mean, col sep=comma, y error=system_mean_co2_std
                ] {chapters/evaluation/results/emissions/Baseline_FixedTime.csv};
            \addlegendentry{Baseline FixedTime}
            % Baseline Actuated
            \addplot+[color6, error bars/.cd,
                y dir=minus, y explicit,
                error bar style={line width=1pt, black}] table [
                    x=scenario, y=system_mean_co2_mean, col sep=comma, y error=system_mean_co2_std
                ] {chapters/evaluation/results/emissions/Baseline_Actuated.csv};
            \addlegendentry{Baseline Actuated}
        \end{axis}
    \end{tikzpicture}
\end{figure}



\subsubsection{Mittlere CO$_2$-Emissionen pro Fahrzeug}

\begin{figure}[H]
    \centering
    \begin{tikzpicture}
        \begin{axis}[
                ybar,
                bar width=0.25cm,
                width=12cm,
                enlarge x limits=0.15,
                ylabel={CO$_2$ [mg/s]},
                symbolic x coords={evening_peak,morning_peak,random_heavy,uniform},
                xtick=data,
                xticklabels={evening\_peak,morning\_peak,random\_heavy,uniform},
                x tick label style={rotate=45,anchor=east},
                legend style={at={(1.05,0.5)}, anchor=west},
                ymajorgrids=true,
                grid style=dashed,
                every axis plot post/.append style={thick, fill=.!50}
            ]

            % Baseline FixedTime
            \addplot+[color1, error bars/.cd,
                y dir=minus, y explicit,
                error bar style={line width=1pt, black}] table [
                    x=scenario, y=system_mean_co2_mean, col sep=comma, y error=system_mean_co2_std
                ] {chapters/evaluation/results/emissions/Baseline_FixedTime.csv};
            \addlegendentry{Baseline FixedTime}

            % Baseline Actuated
            \addplot+[color6, error bars/.cd,
                y dir=minus, y explicit,
                error bar style={line width=1pt, black}] table [
                    x=scenario, y=system_mean_co2_mean, col sep=comma, y error=system_mean_co2_std
                ] {chapters/evaluation/results/emissions/Baseline_Actuated.csv};
            \addlegendentry{Baseline Actuated}

            % RL Modell 1
            \addplot+[color2, error bars/.cd,
                y dir=minus, y explicit,
                error bar style={line width=1pt, black}] table [
                    x=scenario, y=system_mean_co2_mean, col sep=comma, y error=system_mean_co2_std
                ] {chapters/evaluation/results/emissions/ppo_sumo_456_2025-08-18_16-05-49_456.csv};
            \addlegendentry{Model 1}

            % RL Modell 2
            \addplot+[color3, error bars/.cd,
                y dir=minus, y explicit,
                error bar style={line width=1pt, black}] table [
                    x=scenario, y=system_mean_co2_mean, col sep=comma, y error=system_mean_co2_std
                ] {chapters/evaluation/results/emissions/ppo_sumo_13755_2025-08-18_21-51-17_13755.csv};
            \addlegendentry{Model 2}

            % RL Modell 3
            \addplot+[color4, error bars/.cd,
                y dir=minus, y explicit,
                error bar style={line width=1pt, black}] table [
                    x=scenario, y=system_mean_co2_mean, col sep=comma, y error=system_mean_co2_std
                ] {chapters/evaluation/results/emissions/ppo_sumo_143534_2025-08-18_13-16-03_143534.csv};
            \addlegendentry{Model 3}

            % RL Modell 4
            \addplot+[color5, error bars/.cd,
                y dir=minus, y explicit,
                error bar style={line width=1pt, black}] table [
                    x=scenario, y=system_mean_co2_mean, col sep=comma, y error=system_mean_co2_std
                ] {chapters/evaluation/results/emissions/ppo_sumo_635768_2025-08-18_18-57-01_635768.csv};
            \addlegendentry{Model 4}

        \end{axis}
    \end{tikzpicture}
\end{figure}

% ggf. weitere Metriken: stopped, arrived/departed


\subsection{Robustheit und Replikationsanalyse}
Um die Stabilität der Ergebnisse zu bewerten, werden die Verteilungen über
\begin{itemize}
    \item die 4 Trainingsseeds pro Reward-Variante,
    \item die 10 Evaluations-Episoden pro Szenario
\end{itemize}
analysiert.
Boxplots verdeutlichen die Streuung.
Die RL-Modelle zeigen dabei konsistente Verbesserungen,
auch wenn die Varianz zwischen den Seeds nicht vernachlässigbar ist.

\subsection{Zusammenfassung}

\begin{itemize}
    \item \textbf{Diff-Waiting-Time:} Starke Reduktion der Wartezeit, stabil über Szenarien hinweg.
    \item \textbf{Queue:} Effektive Stauvermeidung, aber teils geringere Geschwindigkeit.
    \item \textbf{Reale Welt:} Ausgewogenes Verhalten, moderate Verbesserungen ohne starke Einbußen.
    \item \textbf{Emissionen:} Reduktion der CO\textsubscript{2}-Emissionen, dafür teils höhere Wartezeiten.
\end{itemize}


\section{Herausforderungen und Limitationen}

Obwohl die Ergebnisse der Evaluation die Leistungsfähigkeit von Deep-RL-Ansätzen zur
Verkehrsflussoptimierung verdeutlichen, bestehen verschiedene technische, methodische und
konzeptionelle Einschränkungen. Im Folgenden werden die zentralen Herausforderungen
systematisch dargestellt.

\subsection{Technische und methodische Hürden}

Ein wesentliches Problem ergab sich aus der verwendeten Simulationsumgebung SUMO.
Da SUMO nicht \textit{thread-sicher} ist, konnte das Training nicht parallel auf mehreren
Instanzen durchgeführt werden. Dies führte dazu, dass ausschließlich CPU-basiertes Training
möglich war, was die Trainingszeiten erheblich verlängerte. Ein GPU-basiertes oder verteiltes
Training hätte hier deutliche Effizienzgewinne ermöglicht.

Darüber hinaus erforderte die Abstimmung der Hyperparameter (z.\,B. Lernrate, Discount-Faktor,
Explorationsparameter) umfangreiche manuelle Experimente. Zwar konnten stabile Konfigurationen
gefunden werden, doch bleibt der Prozess zeitaufwändig und fehleranfällig.
Eine \textit{automatisierte Hyperparameter-Optimierung} (z.\,B. mittels
Bayesian Optimization oder Population-Based Training) könnte diesen Aufwand
in zukünftigen Arbeiten erheblich reduzieren.

\subsection{Repräsentativität und Qualität der Daten}

Die zugrundeliegenden Straßennetze basierten auf OpenStreetMap-Daten.
Diese Daten sind zwar frei verfügbar und weltweit verfügbar, weisen jedoch
erhebliche Schwächen auf. Dazu zählen fehlerhafte oder unvollständige Geometrien,
fehlende Informationen zu Fahrstreifen, Tempolimits oder Ampelphasen sowie
Inkonsistenzen im Datenmodell. Um die Netze nutzbar zu machen, war daher
eine umfangreiche manuelle Nachbearbeitung notwendig, was den Arbeitsaufwand
spürbar erhöhte.

Zudem berücksichtigen die verwendeten Netze ausschließlich den motorisierten
Individualverkehr. Wichtige Verkehrsteilnehmer wie Fußgänger, Radfahrende oder
öffentlicher Nahverkehr (Bus, Bahn) konnten nicht integriert werden, da die
entsprechenden Netzelemente in OSM oft unvollständig vorliegen und in SUMO
nur eingeschränkt realistisch abgebildet werden können. Dies reduziert die
Übertragbarkeit der Ergebnisse auf komplexe urbane Verkehrsszenarien.

\subsection{Realitätsnähe der Simulation}

SUMO bildet den Straßenverkehr stark vereinfacht ab. Fahrzeuge folgen
deterministischen Mikroskopiemodellen und zeigen kein adaptives Verhalten,
wie es in der realen Welt vorkommt. So können Fahrzeuge beispielsweise
bei Hindernissen nicht eigenständig ausweichen oder zurücksetzen, was
teilweise zu \textit{Deadlocks} führt. Auch dynamische Aspekte wie
Baustellen, Haltestellen, kurzfristige Störungen oder Unfallereignisse
sind nicht modelliert.

Darüber hinaus unterscheiden sich die in SUMO verfügbaren Beobachtungsmetriken
von realen Detektoren in Städten. Induktionsschleifen oder Kamerasysteme
liefern in der Realität oftmals unvollständige oder verrauschte Daten,
während die Simulation nahezu perfekte Werte bereitstellt. Die
Übertragbarkeit der Modelle in reale Systeme ist daher eingeschränkt,
solange keine Verfahren zur Modellierung von Messfehlern oder Unsicherheiten
integriert werden.

\subsection{Generalisierbarkeit der Ergebnisse}

Die Modelle wurden auf einem spezifischen Netz (Karlsruhe) und unter
definierten Szenarien (low\_flow, medium\_flow, high\_flow)
trainiert. Zwar zeigen die Ergebnisse deutliche Verbesserungen gegenüber den
Baselines, doch ist nicht gesichert, dass diese Modelle ohne Weiteres auf
andere Städte, Straßennetze oder Verkehrsmuster übertragbar sind.

Besonders das Szenario random\_heavy offenbarte Schwächen einzelner Modelle.
Hier traten teilweise drastisch erhöhte Wartezeiten und Stopps auf, die mit
hohen Standardabweichungen verbunden waren. Dies zeigt, dass die Modelle in
komplexen und unregelmäßigen Lastsituationen weniger robust sind und
Generalisation eine der zentralen Herausforderungen bleibt.

\section{Fazit und Ausblick}
\label{sec:fazit}

\subsection{Zusammenfassung der wichtigsten Erkenntnisse}

Die Arbeit hat gezeigt, dass Deep-Reinforcement-Learning-Ansätze ein erhebliches
Potenzial zur Optimierung der Steuerung von Lichtsignalanlagen besitzen. Über alle
betrachteten Reward-Funktionen hinweg konnten die Modelle die klassischen Baselines
deutlich übertreffen. Insbesondere die Actuated-Baseline erwies sich in allen Szenarien
als instabil und ineffizient, während die Fixed-Time-Baseline, wie aus ihrer weiten
Verbreitung in realen Städten zu erwarten, robuste und solide Ergebnisse lieferte.
Dennoch gelang es den RL-Modellen, selbst diese starke Referenz in den meisten
Metriken zu übertreffen.

Die Evaluation verdeutlichte, dass sich durch lernbasierte Steuerungen die mittlere
Wartezeit reduzieren, die Anzahl stoppender Fahrzeuge verringern und die durch-
schnittliche Reisegeschwindigkeit erhöhen lässt. Gleichzeitig blieb der Gesamtdurch-
satz nahezu immer auf maximalem Niveau, sodass Effizienzsteigerungen nicht mit
einem Verlust an abgefertigten Fahrzeugen einhergingen. Besonders deutlich wurden
die Vorteile in regulären und gleichmäßigen Verkehrsszenarien, in denen die Modelle
konsistent nahe am Optimum operierten.

Gleichzeitig zeigten die Ergebnisse, dass die Robustheit stark von der gewählten
Reward-Funktion abhängt. Während Queue- und CO\textsubscript{2}-basierte Modelle eine sehr
hohe Stabilität und Reproduzierbarkeit aufwiesen, traten bei Diff-Waiting-Time- und
Real-World-basierten Modellen unter Hochlastbedingungen signifikante Schwankungen
auf. Die Analysen unterstrichen zudem, dass besonders in unregelmäßigen Szenarien
\texttt{random\_heavy} die Generalisierungsfähigkeit eine wesentliche Herausforderung bleibt.

\subsection{Mögliche Weiterentwicklungen}


Aus den Ergebnissen ergeben sich mehrere konkrete Ansatzpunkte für weiterführende
Arbeiten:
\begin{itemize}
	\item \textbf{Verbesserte Reward-Funktionen:} Eine Kombination mehrerer Zielgrößen
	      (z.\,B. Wartezeit, Emissionen, Durchsatz) in einem Multi-Objective- oder Curriculum-
	      Learning-Ansatz könnte helfen, die Stärken einzelner Funktionen zu vereinen
	      und die Robustheit zu erhöhen.
	\item \textbf{Automatisierte Hyperparameter-Optimierung:} Der hohe manuelle Ab-
	      stimmungsaufwand könnte durch Frameworks wie \texttt{Optuna} oder \texttt{Ray Tune}
	      deutlich reduziert werden. Diese Werkzeuge erlauben eine strukturierte Suche im
	      Hyperparameter-Raum und steigern die Replizierbarkeit und Effizienz der Modellentwicklung.
	\item \textbf{Integration weiterer Verkehrsteilnehmer:} Zukünftige Arbeiten sollten auch
	      Fußgänger, Radfahrende sowie den öffentlichen Nahverkehr berücksichtigen, um
	      eine ganzheitlichere Optimierung urbaner Mobilität zu ermöglichen.
	\item \textbf{Realistischere Sensordaten:} Die Simulation sollte um Modelle erweitert
	      werden, die Messfehler und Unsicherheiten realer Detektoren abbilden. Dadurch
	      könnten die entwickelten Steuerungsstrategien praxisnäher validiert werden.
	\item \textbf{Skalierbarkeit und Verteilte Systeme:} Ein wichtiger Schritt besteht in
	      der Übertragung der Ansätze auf größere Netze und den Einsatz verteilter
	      Multi-Agent-Systeme, die ganze Stadtteile koordinieren können.
	\item \textbf{Kommunikation zwischen Agenten:} Ein weiterer Forschungspfad ist die
	      Erweiterung hin zu kooperativen Steuerungen, bei denen die einzelnen Agenten
	      Informationen untereinander austauschen. Dies könnte eine bessere globale
	      Koordination ermöglichen und die Systemstabilität in hochdynamischen Situationen
	      deutlich erhöhen.
	\item \textbf{Verbundene und autonome Fahrzeuge:} Mit Blick in die Zukunft könnte die
	      Einbindung von Connected-Vehicle-Technologien und autonomen Fahrzeugen die
	      Steuerung auf eine neue Ebene heben. RL-Agenten könnten direkt mit Fahrzeugen
	      interagieren, Fahrbefehle optimieren und so einen noch effizienteren Verkehrsfluss
	      realisieren.
\end{itemize}

\subsection{Relevanz für reale Verkehrsplanung}

Die Ergebnisse der Arbeit verdeutlichen, dass lernbasierte Steuerungsverfahren das
Potenzial haben, klassische Verfahren in der Verkehrssteuerung nicht nur zu ergänzen,
sondern in vielen Aspekten zu übertreffen. Besonders die Fixed-Time-Steuerung, trotz
ihrer weiten Verbreitung und Robustheit, konnte in zentralen Kennzahlen durch die
RL-Modelle überboten werden. Dies zeigt, dass der Einsatz von Deep Reinforcement
Learning in der Verkehrsplanung eine vielversprechende Perspektive darstellt.

Für die praktische Anwendung ist jedoch entscheidend, dass die Modelle eine
hohe Robustheit und Generalisierungsfähigkeit erreichen. Nur wenn sich die Verfahren
auch unter komplexen, dynamischen und teilweise unvorhersehbaren Bedingungen
zuverlässig verhalten, ist ein Einsatz in urbanen Verkehrsnetzen realistisch. Die in
dieser Arbeit aufgezeigten Schwächen in Szenarien mit hoher Last und unregelmäßigen
Mustern verdeutlichen, dass hierfür noch Forschungs- und Entwicklungsbedarf besteht.

Darüber hinaus eröffnet die Arbeit Perspektiven im Kontext zukünftiger
\textbf{Smart-City-Infrastrukturen}. Mit dem zunehmenden Einsatz von \textit{Connected Vehicles}
und autonomen Fahrzeugsystemen entsteht die Möglichkeit, RL-Agenten direkt mit
Fahrzeugen und Sensornetzen interagieren zu lassen. Eine solche Integration würde
es erlauben, nicht nur Ampelanlagen, sondern das gesamte Verkehrssystem koordiniert
und adaptiv zu steuern. Besonders die Kommunikation zwischen Agenten und
Fahrzeugen könnte einen entscheidenden Beitrag zur Vermeidung von Staus,
zur Reduktion von Emissionen und zur Erhöhung der Verkehrssicherheit leisten.

\textbf{Zusammenfassend} liefert die Arbeit einen klaren Hinweis darauf, dass Deep-RL-
gestützte Verkehrssteuerungen langfristig einen wichtigen Beitrag zur Reduktion von
Staus, Wartezeiten und Emissionen leisten können. Damit stellen sie einen relevanten
Baustein für die Gestaltung zukünftiger, nachhaltiger und effizienter urbaner Mobilität dar.
Zentral bleibt jedoch die Weiterentwicklung in drei Bereichen: \textit{Generalisierungsfähigkeit},
\textit{Hyperparameter-Optimierung} und \textit{praxisnahe Sensordaten}. Erst deren Kombination,
in Verbindung mit der Einbindung von kooperativen Agentensystemen und vernetzten
Fahrzeugen, wird den Übergang von simulationsbasierten Ansätzen zu robusten
Realweltanwendungen im Sinne einer intelligenten und vernetzten Mobilität ermöglichen.

% Anhang
\appendix

Dieser Anhang enthält die vollständigen Python-Skripte, die zur Validierung, Reparatur und Steuerung der SUMO-basierten Reinforcement-Learning-Umgebung eingesetzt wurden. Jedes Unterkapitel dokumentiert ein spezifisches Tool oder Modul aus dem Projekt.

\section{Trainings-Skripte}

\subsection{\texttt{train.py} – Trainingsskript für PPO über mehrere Seeds}
\label{app:train_script}
Das folgende Skript enthält die vollständige Trainingslogik für das Reinforcement Learning mit \texttt{sumo-rl} unter Verwendung von \texttt{Stable-Baselines3}.
\begin{minted}[fontsize=\small, linenos, frame=lines, breaklines, tabsize=4]{python}
# ====== Bibliotheken und Module ======
# Standard-Module für Dateiverwaltung, Zeit und Regex
import os
import re
import time
import datetime
import random

# SUMO-Interface (TraCI) für Simulation
import traci

# Mathematische und numerische Berechnungen
import numpy as np

# PyTorch für neuronale Netze und Reproduzierbarkeit
import torch

# Stable-Baselines3 (RL-Algorithmen, hier PPO)
from stable_baselines3 import PPO
from stable_baselines3.common.vec_env import VecNormalize, VecMonitor
from stable_baselines3.common.callbacks import BaseCallback, CallbackList

# SUMO-RL-Umgebung (PettingZoo-kompatibel)
from sumo_rl.environment.env import parallel_env

# SuperSuit – Hilfsfunktionen, um PettingZoo-Umgebungen mit SB3 zu verwenden
from supersuit import (
    pad_observations_v0,          # Padding für Beobachtungen, um feste Größe zu garantieren
    pad_action_space_v0,          # Padding für Aktionsraum
    pettingzoo_env_to_vec_env_v1, # Konvertierung zu SB3-kompatiblem Vektor-Env
    concat_vec_envs_v1            # Mehrere Envs parallel laufen lassen
)

# Gymnasium für RL-Umgebungs-Schnittstellen
import gymnasium as gym
from gymnasium import Wrapper


# ====== Trainings-Setup ======
SEEDS = [143534, 456, 635768, 13755]  # Verschiedene Zufalls-Seed-Werte für reproduzierbare Runs
ROUTE_FILES = [
    "flows_low.rou.xml",
    "flows_medium.rou.xml",
    "flows_high.rou.xml",
]

# ====== Schedules für Hyperparameter-Anpassung ======
# (Funktionen, die während des Trainings den Wert z. B. von Lernrate oder Clip-Bereich dynamisch anpassen)
def adaptive_entropy_schedule(start=0.01):
    return lambda progress: max(0.001, start * (1 - progress))

def dynamic_clip_range(start=0.2, end=0.1):
    return lambda pr: end + (start - end) * pr

def cosine_clip(start=0.2, end=0.1):
    return lambda pr: end + (start - end) * 0.5 * (1 + np.cos(np.pi * (1 - pr)))

def linear_schedule(start):
    return lambda progress: start * (1 - progress)

def cosine_warmup_floor(start=3e-4, warmup_frac=0.05, min_lr_frac=0.1):
    """
    Lernrate: Erst linear hochfahren (Warmup), dann mit Cosinus auf Minimalwert absenken.
    """
    min_lr = start * min_lr_frac
    warmup_frac = max(0.0, min(0.5, warmup_frac))
    def schedule(progress_remaining: float) -> float:
        t = 1.0 - progress_remaining
        if t < warmup_frac:
            base = start * 0.1 + (start - start * 0.1) * (t / warmup_frac)
        else:
            tt = (t - warmup_frac) / max(1e-8, (1.0 - warmup_frac))
            cos_term = 0.5 * (1 + np.cos(np.pi * tt))
            base = min_lr + (start - min_lr) * cos_term
        return float(base)
    return schedule

# ====== Hilfsfunktionen und Callbacks ======
# (Modelle finden, Checkpoints speichern, Metriken loggen, bestes Modell sichern)
# ====== Letzten vollständigen Run finden ======
def find_latest_complete_run(base_dir="runs", prefix="ppo_sumo_"):
    """
    Sucht im 'runs'-Ordner nach dem neuesten Trainingslauf, der
    - eine gespeicherte VecNormalize-Instanz hat
    - und entweder ein finales Modell oder mindestens einen Checkpoint.
    Gibt die Pfade zu Run-Ordner, Modell und Normalisierungsdatei zurück.
    """
    subdirs = sorted(
        [d for d in os.listdir(base_dir) if d.startswith(prefix)],
        reverse=True
    )
    for d in subdirs:
        dir_path = os.path.join(base_dir, d)
        norm_path = os.path.join(dir_path, "vecnormalize.pkl")
        if not os.path.exists(norm_path):
            continue

        # Prüfe auf finales Modell
        final_model = os.path.join(dir_path, "model.zip")
        if os.path.exists(final_model):
            return dir_path, final_model, norm_path

        # Falls kein finales Modell: Prüfe auf Checkpoints
        checkpoint_models = [
            f for f in os.listdir(dir_path)
            if re.match(r"ppo_sumo_model_(\d+)_steps\.zip", f)
        ]
        if checkpoint_models:
            checkpoint_models.sort(key=lambda x: int(re.findall(r"\d+", x)[0]), reverse=True)
            best_checkpoint = checkpoint_models[0]
            return dir_path, os.path.join(dir_path, best_checkpoint), norm_path

    return None

def make_env(seed, route_files):
    def _init():
        env = parallel_env(
            net_file="map.net.xml",
            route_file=route_files[0],  # Platzhalter
            use_gui=False,
            num_seconds=4096,
            reward_fn="diff-waiting-time",
            min_green=5,
            max_depart_delay=100,
            sumo_seed=seed,
            add_system_info=True,
            add_per_agent_info=False,
        )
        if hasattr(env, "seed"):
            env.seed(seed)

        orig_reset = env.reset
        idx = {"i": -1}  # mutierbares Zähl-Objekt im Closure

        def reset_with_round_robin(**kwargs):
            idx["i"] = (idx["i"] + 1) % len(route_files)
            new_route = route_files[idx["i"]]
            env.route_file = new_route
            if hasattr(env, "sumo_seed"):
                env.sumo_seed = seed
            print(f"\n[DEBUG] Reset → Route: {new_route} | Seed: {seed}\n", flush=True)
            return orig_reset(**kwargs)

        env.reset = reset_with_round_robin
        return env
    return _init


def shorten_key(orig_key: str) -> str:
    return orig_key.replace("system_", "")

# ====== Callback: Zeitbasiertes Speichern ======
class TimeBasedCheckpointCallback(BaseCallback):
    """
    Speichert Modell und Normalisierungsdaten in festen Zeitintervallen (Sekunden).
    """
    def __init__(self, save_interval_sec, save_path, name_prefix="ppo_sumo_model", verbose=0):
        super().__init__(verbose)
        self.save_interval_sec = save_interval_sec
        self.save_path = save_path
        self.name_prefix = name_prefix
        self.last_save_time = time.time()

    def _on_step(self) -> bool:
        return True  # Keine Aktion bei jedem einzelnen Step

    def _on_rollout_end(self) -> bool:
        # Am Ende eines Rollouts prüfen, ob das Zeitintervall abgelaufen ist
        current_time = time.time()
        if current_time - self.last_save_time >= self.save_interval_sec:
            timestep = self.num_timesteps
            filename = f"{self.name_prefix}_{timestep}_steps"
            self.model.save(os.path.join(self.save_path, filename + ".zip"))
            if hasattr(self.training_env, "save"):
                self.training_env.save(os.path.join(self.save_path, f"{filename}_vecnormalize.pkl"))
            print(f"[Checkpoint] Modell gespeichert bei {timestep} Schritten ({filename})")
            self.last_save_time = current_time
        return True


# ====== Callback: Metriken aus der Env loggen ======
class EpisodeMetricsLoggerCallback(BaseCallback):
    def __init__(self, prefix="episode", verbose=0):
        super().__init__(verbose)
        self.prefix = prefix
        self.verbose = verbose
        self.sums = {}
        self.counts = {}
        self.last_totals = {}

    def _on_step(self) -> bool:
        dones = self.locals.get("dones")
        infos = self.locals.get("infos")
        if infos is None:
            return True

        for i, info in enumerate(infos):
            if not isinstance(info, dict):
                continue

            if dones is not None and dones[i]:
                # --- Episode zu Ende ---
                fin = info.get("final_info") or info.get("terminal_info")
                if isinstance(fin, dict):
                    for k, v in fin.items():
                        if not isinstance(v, (int, float)) or not np.isfinite(v):
                            continue
                        if k.startswith("system_mean_"):
                            self.sums[k] = self.sums.get(k, 0.0) + float(v)
                            self.counts[k] = self.counts.get(k, 0) + 1
                        elif k.startswith("system_total_"):
                            self.last_totals[k] = float(v)
            else:
                # --- Nur Zwischenschritt, solange Episode noch läuft ---
                for k, v in info.items():
                    if not isinstance(v, (int, float)) or not np.isfinite(v):
                        continue
                    if k.startswith("system_mean_") or k in [
                        "system_total_waiting_time",
                        "system_total_stopped",
                        "system_total_running",
                    ]:
                        self.sums[k] = self.sums.get(k, 0.0) + float(v)
                        self.counts[k] = self.counts.get(k, 0) + 1
                    elif k.startswith("system_total_"):
                        self.last_totals[k] = float(v)

        # Episode fertig → loggen
        if dones is not None and any(dones):
            for k, total in self.sums.items():
                mean_val = total / max(1, self.counts.get(k, 1))
                short_key = shorten_key(k)
                self.logger.record(f"{self.prefix}/{short_key}", mean_val)
                if self.verbose:
                    print(f"[EpisodeMetrics] {short_key} (mean) = {mean_val:.3f}")

            for k, v in self.last_totals.items():
                short_key = shorten_key(k)
                self.logger.record(f"{self.prefix}/{short_key}", v)
                if self.verbose:
                    print(f"[EpisodeMetrics] {short_key} (total) = {v:.0f}")

            # Reset für nächste Episode
            self.sums.clear()
            self.counts.clear()
            self.last_totals.clear()

        return True

# ====== Callback: Bestes Modell speichern ======
class BestModelSaverCallback(BaseCallback):
    """
    Speichert das Modell mit dem bisher höchsten mittleren Episodenreward.
    """
    def __init__(self, save_path, verbose=0):
        super().__init__(verbose)
        self.best_mean_reward = -float('inf')
        self.save_path = save_path

    def _on_step(self) -> bool:
        return True

    def _on_rollout_end(self):
        ep_info_buffer = self.model.ep_info_buffer
        if len(ep_info_buffer) > 0:
            mean_rew = np.mean([ep_info['r'] for ep_info in ep_info_buffer])
            if mean_rew > self.best_mean_reward:
                self.best_mean_reward = mean_rew
                model_path = os.path.join(self.save_path, "best_model.zip")
                self.model.save(model_path)
                if hasattr(self.model.env, "save"):
                    norm_path = os.path.join(self.save_path, "best_model_vecnormalize.pkl")
                    self.model.env.save(norm_path)
                print(f"[AUTOLOG] Neuer Bestwert {mean_rew:.2f} → best_model gespeichert!", flush=True)


# ====== Haupt-Trainingsschleife ======
for SEED in SEEDS:
    # Reproduzierbarkeit sicherstellen
    np.random.seed(SEED)
    torch.manual_seed(SEED)

    # Log-Verzeichnis erstellen
    now = datetime.datetime.now().strftime("%Y-%m-%d_%H-%M-%S")
    log_dir = os.path.join("runs", f"ppo_sumo_{SEED}_{now}")
    os.makedirs(log_dir, exist_ok=True)

    print(f"\n[INFO] Starte Training mit Seed: {SEED}")

    # SUMO-Umgebung initialisieren
    env = make_env(SEED, ROUTE_FILES)()

    # Falls die Env einen seed()-Aufruf unterstützt
    if hasattr(env, "seed"):
        env.seed(SEED)

    # Anpassung der Beobachtungen und Aktionen an SB3
    env = pad_observations_v0(env)
    env = pad_action_space_v0(env)
    env = pettingzoo_env_to_vec_env_v1(env)

    # WICHTIG: trotzdem concat_vec_envs_v1 mit num_vec_envs=1
    env = concat_vec_envs_v1(env, num_vec_envs=1, num_cpus=1, base_class="stable_baselines3")


    # Logging und Normalisierung
    env = VecMonitor(env, filename=os.path.join(log_dir, "monitor.csv"))
    env = VecNormalize(env, norm_obs=True, norm_reward=True, clip_obs=10.0)

    # PPO-Agent erstellen
    model = PPO(
        policy="MlpPolicy",      # Mehrschicht-Perzeptron-Policy
        env=env,
        verbose=1,               # Ausführliches Logging
        tensorboard_log=log_dir, # TensorBoard-Pfad
        batch_size=256,          # Minibatch-Größe für PPO
        n_steps=2048,            # Rollout-Länge
        learning_rate=cosine_warmup_floor(start=3e-4, warmup_frac=0.05, min_lr_frac=0.1),
        clip_range=cosine_clip(), # Clipping-Range dynamisch
        ent_coef=0.01,            # Entropie-Koeffizient (Exploration)
        gamma=0.99,               # Diskontfaktor
        gae_lambda=0.95,          # Lambda für GAE
        device="cpu",             # Training auf CPU
        policy_kwargs=dict(net_arch=dict(pi=[128, 128], vf=[128, 128])), # Netzarchitekturgit
    )

    # Callback-Liste: Checkpoints, Logging, Best-Model-Speicherung
    callbacks = CallbackList([
        TimeBasedCheckpointCallback(
            save_interval_sec=3600, # Jede Stunde speichern
            save_path=log_dir,
            name_prefix="ppo_sumo_model",
            verbose=1,
        ),
        EpisodeMetricsLoggerCallback(),
        BestModelSaverCallback(save_path=log_dir),
    ])

    # Training starten
    try:
        time.sleep(3) # Kurze Pause für saubere Konsolenlogs
        model.learn(
            total_timesteps=2_000_000,
            callback=callbacks,
        )
        # Nach Abschluss final speichern
        model.save(os.path.join(log_dir, "model.zip"))
        env.save(os.path.join(log_dir, "vecnormalize.pkl"))
        print(f"\n[INFO] Training abgeschlossen für Seed {SEED}. Modell gespeichert unter: {log_dir}")

    # Falls Training manuell abgebrochen wird (Strg+C)
    except KeyboardInterrupt:
        print("[ABBRUCH] Manuelles Beenden erkannt. Speichere aktuellen Stand...")
        model.save(os.path.join(log_dir, "model_interrupt.zip"))
        env.save(os.path.join(log_dir, "vecnormalize_interrupt.pkl"))

    # Generelle Fehlerbehandlung
    except Exception as e:
        print(f"\n[FEHLER] Während des Trainings bei Seed {SEED} aufgetreten: {e}")

    # Cleanup: Env schließen und Normalisierungsdaten sichern
    finally:
        try:
            env.save(os.path.join(log_dir, "vecnormalize.pkl"))
        except Exception as e:
            print(f"[WARNUNG] VecNormalize konnte nicht gespeichert werden: {e}")
        env.close()

\end{minted}

\subsection{\texttt{continuetrain.py} – Trainingsskript zum Weitertrainieren}
\label{app:continuetrain}
Startet für jede einzelne Ampelkreuzung eine Minimalumgebung und überprüft, ob diese in \texttt{sumo-rl} trainierbar ist.
\begin{minted}[fontsize=\small, linenos, frame=lines, breaklines, tabsize=4]{python}
import os
import re
import time
import datetime
import traci
import numpy as np
import torch
import json
from stable_baselines3 import PPO
from stable_baselines3.common.vec_env import VecNormalize, VecMonitor
from stable_baselines3.common.callbacks import BaseCallback, CallbackList
from sumo_rl.environment.env import parallel_env
from supersuit import (
    pad_observations_v0,
    pad_action_space_v0,
    pettingzoo_env_to_vec_env_v1,
    concat_vec_envs_v1
)
from gym import Wrapper

# ==== Seed setzen ====
SEED = 42
np.random.seed(SEED)
torch.manual_seed(SEED)

# ==== Adaptive Parameter-Schedules ====
def dynamic_clip_range(start=0.2):
    return lambda progress: max(0.1, start * (1 - 0.5 * progress))

def linear_schedule(start):
    return lambda progress: start * (1 - progress)

# ==== Finde letzten vollständigen Run ====
def find_latest_complete_run(base_dir="runs", prefix="ppo_sumo_"):
    subdirs = sorted(
        [d for d in os.listdir(base_dir) if d.startswith(prefix)],
        reverse=True
    )
    for d in subdirs:
        dir_path = os.path.join(base_dir, d)
        norm_path = os.path.join(dir_path, "vecnormalize.pkl")
        if not os.path.exists(norm_path):
            continue

        final_model = os.path.join(dir_path, "model.zip")
        if os.path.exists(final_model):
            return dir_path, final_model, norm_path

        checkpoint_models = [
            f for f in os.listdir(dir_path)
            if re.match(r"ppo_sumo_model_(\d+)_steps\.zip", f)
        ]
        if checkpoint_models:
            checkpoint_models.sort(key=lambda x: int(re.findall(r"\d+", x)[0]), reverse=True)
            best_checkpoint = checkpoint_models[0]
            return dir_path, os.path.join(dir_path, best_checkpoint), norm_path

    return None

# ==== Zeitbasierter Checkpoint Callback ====
class TimeBasedCheckpointCallback(BaseCallback):
    def __init__(self, save_interval_sec, save_path, name_prefix="ppo_sumo_model", verbose=0):
        super().__init__(verbose)
        self.save_interval_sec = save_interval_sec
        self.save_path = save_path
        self.name_prefix = name_prefix
        self.last_save_time = time.time()

    def _on_step(self) -> bool:
        return True
        
    def _on_rollout_end(self) -> bool:
        current_time = time.time()
        if current_time - self.last_save_time >= self.save_interval_sec:
            timestep = self.num_timesteps
            filename = f"{self.name_prefix}_{timestep}_steps"
            self.model.save(os.path.join(self.save_path, filename + ".zip"))
            if hasattr(self.training_env, "save"):
                self.training_env.save(os.path.join(self.save_path, f"{filename}_vecnormalize.pkl"))
            print(f"[Checkpoint] Modell gespeichert bei {timestep} Schritten ({filename})")
            self.last_save_time = current_time
        return True

# ==== Learning Rate Logger Callback ====
class LearningRateLoggerCallback(BaseCallback):
    def __init__(self, verbose=0):
        super().__init__(verbose)

    def _on_step(self) -> bool:
        lr = self.model.lr_schedule(self.num_timesteps / self.model._total_timesteps)
        self.logger.record("train/learning_rate", lr)
        return True

# ==== Logging ====
now = datetime.datetime.now().strftime("%Y-%m-%d_%H-%M-%S")
log_dir = os.path.join("runs", f"ppo_sumo_{now}")
os.makedirs(log_dir, exist_ok=True)

# ==== SUMO-RL Umgebung ====
env = parallel_env(
    net_file="network.net.xml",
    route_file="flow.rou.xml",
    use_gui=False,
    num_seconds=4096,
    reward_fn="diff-waiting-time",
    min_green=5,
    max_depart_delay=100,
    sumo_seed=SEED,
    add_system_info=True,
    add_per_agent_info=False,
)

if hasattr(env, "seed"):
    env.seed(SEED)

# ==== Wrapping ====
env = pad_observations_v0(env)
env = pad_action_space_v0(env)
env = pettingzoo_env_to_vec_env_v1(env)
env = concat_vec_envs_v1(env, num_vec_envs=1, num_cpus=8, base_class="stable_baselines3")
env = VecMonitor(env)

# ==== Modell laden oder neu starten ====
result = find_latest_complete_run()
if result:
    latest_run_dir, model_path, normalize_path = result
    print("Fortsetzung wird gestartet mit:")
    print(f"Verzeichnis : {latest_run_dir}")
    print(f"Modell      : {model_path}")
    print(f"Normalize   : {normalize_path}\n")

    env = VecNormalize.load(normalize_path, env)
    env.training = True
    env.norm_reward = True

    model = PPO.load(model_path, env=env, tensorboard_log=log_dir, verbose=1, device="cpu")
    print(f"[INFO] Modell startet bei {model.num_timesteps} Timesteps.")
else:
    print("[INFO] Kein vorheriges Modell gefunden. Starte frisches Training.\n")
    env = VecNormalize(env, norm_obs=True, norm_reward=True, clip_obs=10.0)
    model = PPO(
        policy="MlpPolicy",      # Mehrschicht-Perzeptron-Policy
        env=env,
        verbose=1,               # Ausführliches Logging
        tensorboard_log=log_dir, # TensorBoard-Pfad
        batch_size=256,          # Minibatch-Größe für PPO
        n_steps=2048,            # Rollout-Länge
        learning_rate=cosine_warmup_floor(start=3e-4, warmup_frac=0.05, min_lr_frac=0.1),
        clip_range=cosine_clip(), # Clipping-Range dynamisch
        ent_coef=0.01,            # Entropie-Koeffizient (Exploration)
        gamma=0.99,               # Diskontfaktor
        gae_lambda=0.95,          # Lambda für GAE
        device="cpu",             # Training auf CPU
        policy_kwargs=dict(net_arch=dict(pi=[128, 128], vf=[128, 128])), # Netzarchitekturgit
    )

# ==== Automatisches Speichern bei verbessertem ep_rew_mean ====
class BestModelSaverCallback(BaseCallback):
    def __init__(self, save_path, verbose=0):
        super().__init__(verbose)
        self.best_mean_reward = -float('inf')
        self.save_path = save_path

    def _on_step(self) -> bool:
        # Muss vorhanden sein, selbst wenn sie nichts tut
        return True
        
    def _on_rollout_end(self):
        ep_info_buffer = self.model.ep_info_buffer
        if len(ep_info_buffer) > 0:
            mean_rew = np.mean([ep_info['r'] for ep_info in ep_info_buffer])
            
            if mean_rew > self.best_mean_reward:
                self.best_mean_reward = mean_rew

                model_path = os.path.join(self.save_path, "best_model.zip")
                self.model.save(model_path)

                if hasattr(self.model.env, "save"):
                    norm_path = os.path.join(self.save_path, "best_model_vecnormalize.pkl")
                    self.model.env.save(norm_path)

                print(f"[AUTOLOG] Neuer Bestwert {mean_rew:.2f} → best_model gespeichert!", flush=True)

# ==== Callbacks kombinieren ====
callbacks = CallbackList([
    TimeBasedCheckpointCallback(
        save_interval_sec=3600,
        save_path=log_dir,
        name_prefix="ppo_sumo_model",
        verbose=1,
    ),
    LearningRateLoggerCallback(),
    BestModelSaverCallback(save_path=log_dir),
])

# ==== Training starten ====
try:
    model.learn(
        total_timesteps=1_000_000,
        callback=callbacks,
    )
    model.save(os.path.join(log_dir, "model.zip"))
    env.save(os.path.join(log_dir, "vecnormalize.pkl"))
    print(f"\n[INFO] Training abgeschlossen. Modell gespeichert unter: {log_dir}")

except KeyboardInterrupt:
    print("[ABBRUCH] Manuelles Beenden erkannt. Speichere aktuellen Stand...")
    model.save(os.path.join(log_dir, "model_interrupt.zip"))
    env.save(os.path.join(log_dir, "vecnormalize_interrupt.pkl"))

except Exception as e:
    print(f"\n[FEHLER] Während des Trainings aufgetreten: {e}")

finally:
    try:
        env.save(os.path.join(log_dir, "vecnormalize.pkl"))
    except Exception as e:
        print(f"[WARNUNG] VecNormalize konnte nicht gespeichert werden: {e}")
    env.close()
\end{minted}

\section{Belohnungsfunktionen}
\label{app:rewardfunktionen}

\subsection{\texttt{diff-waiting-time}}
\label{app:reward_diff_waiting_time}
Diese Belohnungsfunktion misst die Differenz der kumulierten Wartezeit zwischen zwei Zeitschritten. Sie belohnt eine Abnahme der Gesamtwartezeit.
\begin{minted}[fontsize=\small, linenos, frame=lines, breaklines, tabsize=4]{python}
def diff_waiting_time_reward(traffic_signal):
    ts_wait = sum(traffic_signal.get_accumulated_waiting_time_per_lane()) / 100.0
    reward = traffic_signal.last_ts_waiting_time - ts_wait
    traffic_signal.last_ts_waiting_time = ts_wait
    return reward
\end{minted}

\subsection{\texttt{queue}}
\label{app:reward_queue}
Hier wird die Anzahl an gestoppten Fahrzeugen direkt als negativer Reward verwendet. Weniger Stau → höherer Reward.
\begin{minted}[fontsize=\small, linenos, frame=lines, breaklines, tabsize=4]{python}
def queue_reward(traffic_signal):
    return -traffic_signal.get_total_queued()

def get_total_queued(traffic_signal) -> int:
    """Returns the total number of vehicles halting in the intersection."""
    return sum(traffic_signal.sumo.lane.getLastStepHaltingNumber(lane) for lane in traffic_signal.lanes)
\end{minted}

\subsection{\texttt{realworld}}
\label{app:reward_realworld}
Diese Funktion kombiniert Geschwindigkeit, Warteschlangenlänge und mittlere Wartezeit in einem additiven Reward.
\begin{minted}[fontsize=\small, linenos, frame=lines, breaklines, tabsize=4]{python}
def realworld_reward(traffic_signal):
    # Speed (0–7.5 m/s -> 0–1)
    avg_speed = traffic_signal.get_average_speed()
    speed_term = min(max(avg_speed, 0.0), 7.5) / 7.5

    # Queue (0–20 Fzg -> 0–1)
    total_queue = traffic_signal.get_total_queued()
    queue_term = min(max(total_queue, 0), 20) / 20.0

    # Mean waiting time (0–10 s -> 0–1)
    waits_per_lane = traffic_signal.get_accumulated_waiting_time_per_lane()
    mean_wait = sum(waits_per_lane) / len(waits_per_lane) if waits_per_lane else 0.0
    wait_term = min(max(mean_wait, 0.0), 20.0) / 10.0

    reward = speed_term - queue_term - wait_term
    return reward
\end{minted}

\subsection{\texttt{emissions}}
\label{app:reward_emissions}
Diese Variante erweitert den Reward zusätzlich um einen Term für die CO\textsubscript{2}-Emissionen, sodass sowohl Verkehrsfluss als auch Nachhaltigkeit berücksichtigt werden.
\begin{minted}[fontsize=\small, linenos, frame=lines, breaklines, tabsize=4]{python}
def emissions_reward(traffic_signal):
    env = getattr(traffic_signal, "env", None)
    if env is None or getattr(env, "sumo", None) is None:
        return 0.0
    if float(env.sim_step) >= float(env.sim_max_time):
        return 0.0

    sumo = env.sumo

    # Speed (0–7.5 m/s → 0..1)
    avg_speed = traffic_signal.get_average_speed()
    speed_term = min(max(avg_speed, 0.0), 7.5) / 7.5

    # Queue (0–20 → 0..1)
    total_queue = traffic_signal.get_total_queued()
    queue_term = min(max(total_queue, 0), 20) / 20.0

    # Wait (0–10 s → 0..1)
    waits = traffic_signal.get_accumulated_waiting_time_per_lane()
    mean_wait = (sum(waits) / len(waits)) if waits else 0.0
    wait_term = min(max(mean_wait, 0.0), 10.0) / 10.0

    # Emissionen
    try:
        lanes = getattr(traffic_signal, "lanes", [])
        total_co2 = sum(sumo.lane.getCO2Emission(lane) for lane in lanes)
        n_veh = sum(sumo.lane.getLastStepVehicleNumber(lane) for lane in lanes)
        BASELINE = 300.0 * max(1, len(lanes))
        CAP      = 2000.0 * max(1, len(lanes))
        co2_term = max(0.0, min(total_co2 - BASELINE, CAP - BASELINE)) / (CAP - BASELINE)
    except Exception:
        co2_term = 0.0

    reward = speed_term - queue_term - wait_term - co2_term
    return reward
\end{minted}

\section{Evaluierungs-Skripte}

\subsection{\texttt{evaluate.py} – Evaluationsskript für PPO-Modelle und Baselines}
\label{app:evaluate_script}
Dieses Skript führt die Evaluation aller trainierten PPO-Modelle in SUMO durch und vergleicht sie mit den Baselines \emph{Fixed-Time} und \emph{Actuated}.
Es rollt mehrere Episoden über verschiedene Seeds aus, extrahiert Metriken (z.\,B. Wartezeit, Geschwindigkeit, Emissionen) und speichert die Ergebnisse als \texttt{eval\_results.json}.
\begin{minted}[fontsize=\small, linenos, frame=lines, breaklines, tabsize=4]{python}
import os, json, numpy as np
import glob
from stable_baselines3 import PPO
from stable_baselines3.common.vec_env import VecNormalize, VecMonitor
from stable_baselines3.common.logger import configure
from sumo_rl.environment.env import parallel_env
from supersuit import pad_observations_v0, pad_action_space_v0
from supersuit import pettingzoo_env_to_vec_env_v1, concat_vec_envs_v1

# ----- Config -----
RUNS = sorted(glob.glob(os.path.join("runs", "ppo_sumo_*")))
MODEL_NAME  = "best_model.zip"
N_EPISODES  = 10
EP_LENGTH_S = 4096
EP_SEEDS    = [12345, 67890, 13579, 24680, 11223, 44556, 77889, 99100, 31415, 27182]
SCENARIOS   = [
    {"name": "morning_peak", "route_file": "flows_morning.rou.xml"},
    {"name": "evening_peak", "route_file": "flows_evening.rou.xml"},
    {"name": "uniform",      "route_file": "flows_uniform.rou.xml"},
    {"name": "random_heavy", "route_file": "flows_random_heavy.rou.xml"},
]

# ----- Env Factory -----
def make_env(route_file, sumo_seed):
    print(f"[DEBUG] Creating SUMO env with route={route_file}, seed={sumo_seed}")
    env = parallel_env(
        net_file="map.net.xml",
        route_file=route_file,
        use_gui=False,
        num_seconds=EP_LENGTH_S,
        reward_fn=dummy_reward,
        min_green=5,
        max_depart_delay=100,
        sumo_seed=sumo_seed,
        add_system_info=True,
        add_per_agent_info=False,
    )
    env = pad_observations_v0(env)
    env = pad_action_space_v0(env)
    env = pettingzoo_env_to_vec_env_v1(env)
    env = concat_vec_envs_v1(env, num_vec_envs=1, num_cpus=1, base_class="stable_baselines3")
    env = VecMonitor(env)
    return env

# ----- Model Loader -----
def load_model_and_norm(env, run_dir):
    vecnorm_path = os.path.join(run_dir, "vecnormalize.pkl")
    model_path   = os.path.join(run_dir, MODEL_NAME)

    #print(f"[DEBUG] Loading VecNormalize from {vecnorm_path}")
    env = VecNormalize.load(vecnorm_path, env)
    env.training = False
    env.norm_reward = False

    #print(f"[DEBUG] Loading PPO model from {model_path}")
    model = PPO.load(model_path, env=env, device="cpu")
    return model, env

# ----- Rollout -----
def rollout(model, env):
    obs = env.reset()
    dones = [False]

    sums = {}
    counts = {}
    last_totals = {}

    while True:
        action, _ = model.predict(obs, deterministic=True)
        obs, rewards, dones, infos = env.step(action)

        info = infos[0] if isinstance(infos, list) else infos
        if not isinstance(info, dict):
            info = {}

        # Wenn Episode zu Ende ist:
        if dones[0]:
            # Falls vorhanden, final_info/terminal_info verwenden
            fin = info.get("final_info") or info.get("terminal_info")
            if isinstance(fin, dict):
                # Mittelwerte vom finalen Step noch einrechnen
                for k, v in fin.items():
                    if k.startswith("system_mean_") and isinstance(v, (int, float)) and np.isfinite(v):
                        sums[k] = sums.get(k, 0.0) + float(v)
                        counts[k] = counts.get(k, 0) + 1
                # Totals aus final_info (echte Endstände)
                for k, v in fin.items():
                    if k.startswith("system_total_") and isinstance(v, (int, float)) and np.isfinite(v):
                        last_totals[k] = float(v)
            break

        # Normaler Zwischenschritt: Mittelwerte sammeln + Totals „letzten gültigen“ merken
        for k, v in info.items():
            if not isinstance(v, (int, float)) or not np.isfinite(v):
                continue
            if k.startswith("system_mean_") or k in ["system_total_waiting_time", "system_total_stopped", "system_total_running"]:
                # momentane Werte mitteln
                sums[k] = sums.get(k, 0.0) + float(v)
                counts[k] = counts.get(k, 0) + 1
            elif k.startswith("system_total_"):
                # Totals: nur letzten Wert merken
                last_totals[k] = float(v)

    mean_metrics = {k: (sums[k] / max(1, counts.get(k, 0))) for k in sums}
    mean_metrics.update(last_totals)
    return mean_metrics


def shorten_key(orig_key: str) -> str:
    return orig_key.replace("system_", "")

# ----- Env Factory für Baselines -----
def make_env_baseline(route_file, sumo_seed, fixed_time=True):
    """
    Erstellt eine SUMO-Umgebung, die den internen Controller verwendet.
    fixed_time=True  -> Fester Phasenplan aus net.xml
    fixed_time=False -> SUMO Actuated Control (falls in net.xml konfiguriert)
    """
    env = parallel_env(
        net_file="map.net.xml",
        route_file=route_file,
        use_gui=False,
        num_seconds=EP_LENGTH_S,
        reward_fn=dummy_reward,            # Kein RL-Reward
        fixed_ts=fixed_time,       # True = fixed, False = actuated
        sumo_seed=sumo_seed,
        add_system_info=True,
        add_per_agent_info=False,
    )
    env = pad_observations_v0(env)
    env = pad_action_space_v0(env)
    env = pettingzoo_env_to_vec_env_v1(env)
    env = concat_vec_envs_v1(env, num_vec_envs=1, num_cpus=1, base_class="stable_baselines3")
    env = VecMonitor(env)
    return env

def dummy_reward(_ts):
    return 0.0

def rollout_baseline(env):
    obs = env.reset()
    dones = [False]

    # Mittelwerte über die Episode
    sums = {}
    counts = {}
    # Letzte gültige Totals (vor Reset)
    last_totals = {}

    # gültige Dummy-Aktion aus dem Action Space
    dummy_action = np.array([env.action_space.sample() for _ in range(env.num_envs)])

    while True:
        obs, rewards, dones, infos = env.step(dummy_action)

        info = infos[0] if isinstance(infos, list) else infos
        if not isinstance(info, dict):
            info = {}

        # Wenn Episode zu Ende ist:
        if dones[0]:
            # Falls vorhanden, final_info/terminal_info verwenden
            fin = info.get("final_info") or info.get("terminal_info")
            if isinstance(fin, dict):
                # Mittelwerte vom finalen Step noch einrechnen
                for k, v in fin.items():
                    if k.startswith("system_mean_") and isinstance(v, (int, float)) and np.isfinite(v):
                        sums[k] = sums.get(k, 0.0) + float(v)
                        counts[k] = counts.get(k, 0) + 1
                # Totals aus final_info (echte Endstände)
                for k, v in fin.items():
                    if k.startswith("system_total_") and isinstance(v, (int, float)) and np.isfinite(v):
                        last_totals[k] = float(v)
            break

        # Normaler Zwischenschritt: Mittelwerte sammeln + Totals „letzten gültigen“ merken
        for k, v in info.items():
            if not isinstance(v, (int, float)) or not np.isfinite(v):
                continue
            if k.startswith("system_mean_") or k in ["system_total_waiting_time", "system_total_stopped", "system_total_running"]:
                # momentane Werte mitteln
                sums[k] = sums.get(k, 0.0) + float(v)
                counts[k] = counts.get(k, 0) + 1
            elif k.startswith("system_total_"):
                # Totals: nur letzten Wert merken
                last_totals[k] = float(v)

    # Mittelwerte berechnen
    mean_metrics = {k: (sums[k] / max(1, counts.get(k, 0))) for k in sums}
    # Letzte gültige Totals übernehmen
    mean_metrics.update(last_totals)

    return mean_metrics


def to_serializable(obj):
    if isinstance(obj, (np.integer,)):
        return int(obj)
    elif isinstance(obj, (np.floating,)):
        return float(obj)
    elif isinstance(obj, (np.ndarray,)):
        return obj.tolist()
    return str(obj)

# ----- Evaluation Loop -----
# ----- Evaluation Loop -----
def evaluate():
    results = []
    log_dir_root = os.path.join("evaluation", "logs")

    # Zählung: 2 Baselines + len(RUNS) RL pro (scenario × episode)
    total_episodes = (2 + len(RUNS)) * len(SCENARIOS) * N_EPISODES
    ep_counter = 0

    for sc in SCENARIOS:
        scen_log_dir = os.path.join(log_dir_root, f"eval_{sc['name']}")
        os.makedirs(scen_log_dir, exist_ok=True)
        logger = configure(scen_log_dir, ["tensorboard", "stdout"])

        print(f"[INFO] Evaluating scenario={sc['name']}")

        for ep in range(N_EPISODES):
            ep_seed = EP_SEEDS[ep]

            # --- 1) Fixed-Time ---
            env = make_env_baseline(sc["route_file"], sumo_seed=ep_seed, fixed_time=True)
            ep_counter += 1
            print(f"[PROGRESS] FixedTime | {sc['name']} | Ep {ep+1}/{N_EPISODES} "
                  f"({ep_counter}/{total_episodes})")
            m = rollout_baseline(env)
            m.update({
                "scenario": sc["name"],
                "episode": ep,
                "method": "Baseline_FixedTime"
            })
            results.append(m)

            # --- 2) Actuated ---
            env = make_env_baseline(sc["route_file"], sumo_seed=ep_seed, fixed_time=False)
            ep_counter += 1
            print(f"[PROGRESS] Actuated | {sc['name']} | Ep {ep+1}/{N_EPISODES} "
                  f"({ep_counter}/{total_episodes})")
            m = rollout_baseline(env)
            m.update({
                "scenario": sc["name"],
                "episode": ep,
                "method": "Baseline_Actuated"
            })
            results.append(m)

            # --- 3) RL-Modelle ---
            for run_dir in RUNS:
                env_raw = make_env(sc["route_file"], sumo_seed=ep_seed)
                model, env = load_model_and_norm(env_raw, run_dir)
                ep_counter += 1
                model_name = os.path.basename(run_dir)
                print(f"[PROGRESS] RL | {sc['name']} | {model_name} "
                      f"| Ep {ep+1}/{N_EPISODES} ({ep_counter}/{total_episodes})")
                m = rollout(model, env)
                
                # Seed extrahieren (3. Teil vom Namen)
                parts = model_name.split("_")
                model_seed = parts[2] if len(parts) > 2 else "unknown"

                m.update({
                    "scenario": sc["name"],
                    "episode": ep,
                    "method": f"{model_name}_{model_seed}"
                })
                results.append(m)

            # --- Logging dieser Episode (Baselines + alle RL) ---
            for entry in results[-(2 + len(RUNS)):]:
                for k, v in entry.items():
                    if isinstance(v, (int, float)) and k not in ["episode", "ep_seed"]:
                        short_key = shorten_key(k)
                        logger.record(f"{entry['method']}/{short_key}", v)
            logger.dump(step=ep)

    results_path = os.path.join("evaluation", "eval_results.json")
    os.makedirs(os.path.dirname(results_path), exist_ok=True)
    with open(results_path, "w") as f:
        json.dump(results, f, indent=2, default=to_serializable)

    print(f"[INFO] Evaluation abgeschlossen. Ergebnisse: {results_path}")

if __name__ == "__main__":
    evaluate()

\end{minted}

\section{Postprocessing der Evaluationsergebnisse}

\subsection{\texttt{json2csv.py} – Konvertierung und Aggregation von Evaluationsergebnissen}
\label{app:json2csv_script}
Dieses Skript verarbeitet die von \texttt{evaluate.py} erzeugte JSON-Datei \texttt{eval\_results.json}.
Es wandelt die Rohdaten zunächst in ein CSV-Format um, berechnet anschließend Mittelwerte und Standardabweichungen pro \emph{Scenario × Methode} und erzeugt sowohl eine aggregierte Gesamttabelle als auch separate CSV-Dateien pro Methode.
\begin{minted}[fontsize=\small, linenos, frame=lines, breaklines, tabsize=4]{python}
import json
import pandas as pd
import os

# Pfade
json_path = "evaluation/eval_results.json"
raw_csv_path = "evaluation/eval_results_raw.csv"
agg_csv_path = "evaluation/eval_results_agg.csv"

# -----------------------------
# Schritt 1: JSON -> Raw CSV
# -----------------------------
print(f"Lese JSON-Datei: {json_path}")
with open(json_path, "r") as f:
    data = json.load(f)

df = pd.DataFrame(data)
df.to_csv(raw_csv_path, index=False)
print(f"Raw CSV geschrieben: {raw_csv_path}")

# -----------------------------
# Schritt 2: Aggregation
# -----------------------------
# numerische Spalten automatisch finden (alles außer scenario, method, episode)
numeric_cols = df.select_dtypes(include="number").columns.tolist()
numeric_cols = [c for c in numeric_cols if c not in ["episode"]]  # episode nicht mitteln

# Aggregationsdict
agg_dict = {}
for col in numeric_cols:
    agg_dict[f"{col}_mean"] = (col, "mean")
    agg_dict[f"{col}_std"] = (col, "std")

# Gruppieren nach Szenario + Methode
agg = df.groupby(["scenario", "method"]).agg(**agg_dict).reset_index()

# Gesamte Aggregation speichern
agg.to_csv(agg_csv_path, index=False)
print(f"Aggregierte Datei geschrieben: {agg_csv_path}")
print("Zeilen:", len(agg))

# -----------------------------
# Schritt 3: Pro-Methode CSVs
# -----------------------------
for method, df_method in agg.groupby("method"):
    safe_name = method.replace(" ", "_").replace("/", "_")
    out_path = f"evaluation/{safe_name}.csv"
    df_method.to_csv(out_path, index=False)
    print(f"Datei für Methode '{method}' geschrieben: {out_path}")

\end{minted}

\section{Netzwerk-Skripte}

\subsection{\texttt{check\_tls\_consistency.py} – Prüfung inkonsistenter Phasenlängen}
\label{app:check_tls_consistency}
Dieses Tool analysiert alle TLS im SUMO-Netz und prüft, ob die Länge des \texttt{state}-Strings mit der Anzahl der kontrollierten Verbindungen übereinstimmt.
\begin{minted}[fontsize=\small, linenos, frame=lines, breaklines, tabsize=4]{python}
import xml.etree.ElementTree as ET

# === Konfiguration ===
net_file = "map.net.xml"

# === Einlesen ===
tree = ET.parse(net_file)
root = tree.getroot()

# === Alle controlledLinks zählen ===
tls_controlled_links = {}
for connection in root.findall("connection"):
    if "tl" in connection.attrib and "linkIndex" in connection.attrib:
        tls_id = connection.attrib["tl"]
        tls_controlled_links.setdefault(tls_id, set()).add(int(connection.attrib["linkIndex"]))

# === Alle Phasen prüfen ===
def check_tls_lengths():
    print("Überprüfe alle TLS auf inkonsistente Phasenlängen...\n")
    any_issues = False
    for logic in root.findall("tlLogic"):
        tls_id = logic.attrib["id"]
        expected_len = len(tls_controlled_links.get(tls_id, []))

        if expected_len == 0:
            print(f" TLS '{tls_id}' hat keine controlledLinks (wird evtl. nicht gesteuert)")
            continue

        for i, phase in enumerate(logic.findall("phase")):
            actual_len = len(phase.attrib["state"])
            if actual_len != expected_len:
                print(f" Phase {i} von TLS '{tls_id}' hat Länge {actual_len}, erwartet: {expected_len}")
                print(f"    → state=\"{phase.attrib['state']}\"")
                any_issues = True

    if not any_issues:
        print(" Alle TLS-Phasen stimmen mit ihren controlledLinks überein!")

check_tls_lengths()
\end{minted}

\subsection{\texttt{check\_tls\_requests.py} – Prüfung ungültiger \texttt{<request>}-Indizes}
\label{app:check_tls_requests}
Prüft, ob alle \texttt{request}-Indizes innerhalb der zulässigen Grenzen liegen, um Laufzeitfehler in \texttt{sumo-rl} zu vermeiden.

\begin{minted}[fontsize=\small, linenos, frame=lines, breaklines, tabsize=4]{python}
import xml.etree.ElementTree as ET

net_file = "map.net.xml"
tree = ET.parse(net_file)
root = tree.getroot()

# Zähle für jedes TLS wie viele signal indices es gibt (controlled links)
tls_signal_indices = {}
for conn in root.findall("connection"):
    if "tl" in conn.attrib and "linkIndex" in conn.attrib:
        tls_id = conn.attrib["tl"]
        tls_signal_indices.setdefault(tls_id, set()).add(int(conn.attrib["linkIndex"]))

# Vergleiche mit den request-Elementen
print("Überprüfe request-Indizes gegen Signalindizes...\n")
any_issues = False
for junction in root.findall("junction"):
    tls_id = junction.attrib.get("id")
    requests = junction.findall("request")
    if tls_id in tls_signal_indices:
        expected_max = len(tls_signal_indices[tls_id])
        for req in requests:
            index = int(req.attrib["index"])
            if index >= expected_max:
                print(f"Junction '{tls_id}': request index {index} > max signal index {expected_max - 1}")
                any_issues = True

if not any_issues:
    print("Alle request-Indizes passen zu den TLS-Signalindizes!")
\end{minted}

\subsection{\texttt{fix\_requests.py} – Automatische Korrektur von Requests und Phasen}
\label{app:fix_requests}
Dieses Skript bereinigt überzählige \texttt{<request>}-Einträge und passt \texttt{state}-Strings in den Phasenlängen an.
\begin{minted}[fontsize=\small, linenos, frame=lines, breaklines, tabsize=4]{python}
import xml.etree.ElementTree as ET

net_file = "map.net.xml"
output_file = "map_fixed_tls.net.xml"

tree = ET.parse(net_file)
root = tree.getroot()

# Finde maximal verwendete Signal-Indices pro TLS
tls_max_index = {}
for conn in root.findall("connection"):
    tl = conn.get("tl")
    idx = conn.get("linkIndex")
    if tl and idx:
        idx = int(idx)
        tls_max_index[tl] = max(tls_max_index.get(tl, -1), idx)

# Bereinigung
total_removed_requests = 0
total_adjusted_phases = 0
changed_tls = []

for junction in root.findall("junction"):
    tls_id = junction.get("id")
    if tls_id not in tls_max_index:
        continue

    max_idx = tls_max_index[tls_id]
    requests = list(junction.findall("request"))
    removed = 0

    for req in requests:
        req_idx = int(req.get("index"))
        if req_idx > max_idx:
            junction.remove(req)
            removed += 1

    if removed > 0:
        print(f"TLS '{tls_id}': {removed} ungültige <request>-Einträge entfernt.")
        total_removed_requests += removed
        changed_tls.append(tls_id)

    # Kürze zugehörige Phasen
    for tl in root.findall("tlLogic"):
        if tl.get("id") == tls_id:
            adjusted = 0
            for phase in tl.findall("phase"):
                state = phase.get("state")
                if len(state) > max_idx + 1:
                    old_len = len(state)
                    phase.set("state", state[:max_idx + 1])
                    adjusted += 1
            if adjusted > 0:
                print(f" TLS '{tls_id}': {adjusted} <phase>-Strings auf Länge {max_idx + 1} gekürzt.")
                total_adjusted_phases += adjusted
                if tls_id not in changed_tls:
                    changed_tls.append(tls_id)

# Speichern
tree.write(output_file, encoding="utf-8")
print("\n Reparatur abgeschlossen.")
print(f" Gesamt entfernte <request>-Einträge: {total_removed_requests}")
print(f" Gesamt angepasste <phase>-Einträge: {total_adjusted_phases}")
print(f" Betroffene TLS-IDs: {len(changed_tls)} Stück")
for tls in changed_tls:
    print(f"  - {tls}")
print(f"\n Bereinigte Datei gespeichert unter: {output_file}")
\end{minted}

\subsection{\texttt{repair\_net.py} – manuelle TLS-Reparatur auf Basis eines Referenz-Dictionaries}
\label{app:repair_net}
Repariert TLS-Definitionen durch Abgleich mit einer vordefinierten Mapping-Tabelle von korrekten Phasenlängen.
\begin{minted}[fontsize=\small, linenos, frame=lines, breaklines, tabsize=4]{python}
from xml.etree import ElementTree as ET

# Manuell gepflegte Dictionary mit {TLS-ID: Anzahl controlledLinks}
controlled_links = {
    "1720933516": 6,
    "3538953167": 2,
    "3664415977": 10,
    "cluster_14795187_1720919996_2670370290_2670370291": 11,
    "cluster_14795804_55474925_6655074904_765746891_#1more": 49,
    "cluster_15431428_1719671850_1720917935": 20,
    "cluster_1590912233_3664415976_5083348337_5083348350": 11,
    "cluster_1692973685_1692973722_1718084055_1718084058_#11more": 36,
    "cluster_1729190097_3687504105": 8,
    "cluster_1744031943_5131521735": 10,
    "joinedS_1623835169_cluster_1137679587_1626739216_1728272870_1728272909_#17more": 33,
    "joinedS_309108716_cluster_11001804363_1125509937_12515596172_1784859792_#5more": 14,
    "joinedS_5092985445_cluster_1590912226_2911376263": 10,
    # ggf. mehr hinzufügen
}

tree = ET.parse("map.net.xml")
root = tree.getroot()
changed = False

for logic in root.findall("tlLogic"):
    tl_id = logic.attrib["id"]
    if tl_id not in controlled_links:
        continue

    correct_len = controlled_links[tl_id]
    for phase in logic.findall("phase"):
        state = phase.attrib["state"]
        if len(state) != correct_len:
            new_state = state[:correct_len].ljust(correct_len, 'r')
            print(f" Fixing {tl_id}: {len(state)} → {correct_len}")
            phase.attrib["state"] = new_state
            changed = True

if changed:
    tree.write("karlsruhe_fixed.net.xml")
    print(" Bereinigte Datei gespeichert: karlsruhe_fixed.net.xml")
else:
    print(" Alle Phasen bereits korrekt.")

\end{minted}

\subsection{\texttt{statecheck.py} – Prüfung auf Ziel-Phasenlänge}
\label{app:statecheck}
Hilft bei der Kontrolle einheitlicher Phasenlängen über das gesamte Netz hinweg (z.\,B. Zielwert = 57).
\begin{minted}[fontsize=\small, linenos, frame=lines, breaklines, tabsize=4]{python}
from xml.etree import ElementTree as ET

tree = ET.parse("map.net.xml")
root = tree.getroot()

for logic in root.findall("tlLogic"):
    tl_id = logic.attrib["id"]
    for i, phase in enumerate(logic.findall("phase")):
        state = phase.attrib["state"]
        if len(state) != 57:
            print(f" Phase {i} of TLS '{tl_id}' has length {len(state)}")
\end{minted}

\subsection{\texttt{find\_valid\_tls.py} – Validierung lauffähiger TLS für SUMO-RL}
\label{app:find_valid_tls}
Startet für jede einzelne Ampelkreuzung eine Minimalumgebung und überprüft, ob diese in \texttt{sumo-rl} trainierbar ist.
\begin{minted}[fontsize=\small, linenos, frame=lines, breaklines, tabsize=4]{python}
from sumo_rl import SumoEnvironment
import traci
import os

def test_tls(tls_id):
    try:
        env = SumoEnvironment(
            net_file="map.net.xml",
            route_file="map.rou.xml",
            use_gui=False,
            single_agent=True
        )
        env.ts_ids = [tls_id]
        env.reset()
        env.close()
        return True
    except Exception as e:
        print(f" TLS {tls_id} nicht gültig: {e}")
        return False

# Alle TLS holen
try:
    env = SumoEnvironment(
        net_file="map.net.xml",
        route_file="map.rou.xml",
        use_gui=False,
        single_agent=True
    )
    all_tls = env.ts_ids
    env.close()
except Exception as e:
    print(" Konnte TLS nicht auslesen:", e)
    all_tls = []

print(f" Teste {len(all_tls)} TLS auf Gültigkeit...\n")
valid_tls = []

for tls_id in all_tls:
    if test_tls(tls_id):
        valid_tls.append(tls_id)

print("\n Gültige TLS:")
print(valid_tls)
\end{minted}


\subsection{\texttt{find\_relevant\_edges.py} – Suche alle relevanten edges}
\label{app:find_relevant_edges}
\begin{minted}[fontsize=\small, linenos, frame=lines, breaklines, tabsize=4]{python}
import xml.etree.ElementTree as ET

# Konfiguration: Pfad zur .net.xml-Datei und Suchbegriffe
NET_FILE = "network.net.xml"
SUCHBEGRIFFE = ["B10", "B36", "L605", "Durlacher Allee", "Reinhold-Frank-Straße"]

# Ausgabe-Datei für gefundene Kanten
OUTPUT_FILE = "edges.txt"

def finde_relevante_kanten(net_file, suchbegriffe):
    tree = ET.parse(net_file)
    root = tree.getroot()
    
    relevante_kanten = []
    
    for edge in root.findall("edge"):
        name = edge.get("name")
        if name:
            for begriff in suchbegriffe:
                if begriff.lower() in name.lower():
                    relevante_kanten.append((edge.get("id"), name))
                    break  # nicht doppelt eintragen, falls mehrere Begriffe passen
                    
    return relevante_kanten

if __name__ == "__main__":
    kanten = finde_relevante_kanten(NET_FILE, SUCHBEGRIFFE)
    
    # Ergebnisse speichern
    with open(OUTPUT_FILE, "w", encoding="utf-8") as f:
        for edge_id, name in kanten:
            f.write(f"{edge_id}\t{name}\n")
    
    print(f"{len(kanten)} relevante Kanten gefunden.")
    print(f"Ergebnisse in '{OUTPUT_FILE}' gespeichert.")

\end{minted}

\section{Sumo-Konfiguration}
\subsection{\texttt{sumoconfig\_.sumocfg}}
\label{app:sumocfg}
Die folgende Konfigurationsdatei definiert die zentralen Eingaben und
Parameter für die Simulation in SUMO. Sie verweist auf die zu ladende
Netzdatei und die zugehörigen Routendateien sowie auf den zu simulierenden
Zeitraum.
\begin{minted}[fontsize=\small, linenos, frame=lines, breaklines, tabsize=4]{xml}
<configuration>
  <input>
    <net-file value="network.net.xml"/>
    <route-files value="routes.xml"/>
  </input>
  <time>
    <begin value="0"/>
    <end value="5000"/>
  </time>
</configuration>
\end{minted}

\printbibliography

\printglossaries


\end{document}
