\documentclass[a4paper, ngerman, 10pt]{article}

% code
\usepackage{algorithm}
\usepackage{algpseudocode}
\usepackage{xcolor}
\usepackage{minted}

% Eingabecodierung und Sprachunterstützung
\usepackage[utf8]{inputenc}
\usepackage[ngerman]{babel}
\usepackage[T1]{fontenc}
\usepackage{float}
\usepackage{lmodern}

% Typografie und Layout
\usepackage{lmodern}
\usepackage{microtype}

% Links und URLs
\usepackage{hyperref}

% Bilder und Grafiken
\usepackage{graphicx}
\usepackage{caption}

% Aufzählungen
\usepackage{enumitem}

\usepackage[a4paper, top=4cm, bottom=4cm, left=4cm, right=4cm]{geometry}

% Mathematische Ausdrücke
\usepackage{amsmath}

% TikZ für grafische Elemente
\usepackage{tikz}
\usetikzlibrary{shapes, arrows.meta, positioning}

\tikzset{
  box/.style={
    rectangle,
    draw,
    thick,
    minimum width=7cm,
    minimum height=1.2cm,
    align=center,
    fill=blue!5
  },
  arrow/.style={
    -{Latex[length=3mm]},
    thick
  }
}

% Image folder
\graphicspath{{images/}}

% Hyperlink setup
\hypersetup{
  colorlinks=true,
  linkcolor=blue,
  filecolor=magenta,      
  urlcolor=cyan,
}

\title{Optimierung einer Verkehrssimulation mit KI-basierten Agenten in SUMO}
\author{Sam Weiler}
\date{\small \today}
\begin{document}

\begin{titlepage}
  \centering
  \includegraphics[width=5cm]{HKA_IWI_Wortmarke_RGB.jpg}
  \hspace{2cm}
  \includegraphics[width=3cm]{HKA_IWI_Bildmarke_RGB.jpg}
  \vspace{2cm}

  \Large
  \textbf{Optimierung einer Verkehrssimulation mit KI-basierten Agenten in SUMO}

  \vspace{2cm}

  Studiengang Informatik\\

  Sam Weiler\\
  Matr. Nr. 73640\\
\end{titlepage}

\tableofcontents
\newpage


\section{Einleitung}

\subsection{Motivation und Problemstellung}

Städte stehen zunehmend vor der Herausforderung, mit den wachsenden Anforderungen des urbanen Verkehrs zurechtzukommen. Die Zahl der Fahrzeuge im Individualverkehr steigt kontinuierlich\cite{umweltbundesamt-motorisierungsgrad, Kraftfahrt-Bundesamt}, was zu einer Verdichtung des Verkehrsaufkommens, insbesondere in städtischen Knotenpunkten, führt. Die daraus resultierenden Konsequenzen sind vielfältig: Verkehrsüberlastungen führen zu erhöhten Reisezeiten, steigenden \gls{emissions} und einer verminderten Lebensqualität für die Bevölkerung. \cite{Europäische-Umweltagentur} Darüber hinaus verursacht ineffizienter Verkehr einen erheblichen wirtschaftlichen Schaden durch Zeitverluste und Ressourcenverschwendung. \cite{umweltbundesamt-emissionen, Inrix-Traffic-Scorecard}

Ein zentraler Hebel zur Verbesserung dieser Situation liegt in der intelligenten Steuerung des Verkehrsflusses, insbesondere an Kreuzungen, an denen mehrere Verkehrsströme aufeinandertreffen. Die \gls{trafficlight}, die dort zum Einsatz kommen, arbeiten vielerorts noch nach starren, zeitbasierten Schaltplänen, die selten in Echtzeit auf veränderte Verkehrssituationen reagieren. \cite{baden-wuerttemberg} Auch adaptive Verfahren, wie verkehrsabhängige Steuerungen mittels \gls{induktionsschleifen} oder \gls{kamera}, sind in ihrer Reaktionsfähigkeit beschränkt. Damit bleibt ein enormes Potenzial zur Effizienzsteigerung ungenutzt. \cite{Bundesanstallt}

Vor diesem Hintergrund bietet die Kombination moderner Simulationstechniken mit Methoden der künstlichen Intelligenz, insbesondere dem \gls{rl}, eine vielversprechende Alternative. Reinforcement Learning ist ein lernbasiertes Verfahren, bei dem ein \gls{agent} durch Interaktion mit einer Umgebung eine optimale Strategie zur Maximierung eines definierten Belohnungskriteriums erlernt. Die Anwendung dieses Konzepts auf Ampelsteuerungen erlaubt es, reaktive, datengestützte Systeme zu entwickeln, die dynamisch auf die aktuelle Verkehrssituation reagieren und dabei auf langfristige Effizienz optimiert sind.

Zur Erprobung solcher Verfahren eignet sich die Verkehrssimulationsumgebung \gls{sumo}, eine quelloffene, modular aufgebaute Plattform, die es ermöglicht, Verkehrsflüsse realitätsnah zu modellieren und zu analysieren. In Kombination mit dem Framework \gls{sumo-rl}\cite{sumo-rl}, das eine Brücke zwischen SUMO und gängigen Machine-Learning-Frameworks wie \gls{tensorflow} oder \gls{pytorch} schlägt, lassen sich Reinforcement-Learning-Agenten direkt in die Simulationsumgebung einbetten. Diese können dann die Steuerung einzelner Ampelanlagen übernehmen und ihre Strategien durch wiederholte Simulation iterativ verbessern.

\subsection{Zielsetzung der Arbeit}

Ziel dieser Bachelorarbeit ist es, eine auf Reinforcement Learning basierende Steuerung von Ampelanlagen innerhalb eines realitätsnahen, simulierten städtischen Verkehrsnetzes zu entwickeln, umzusetzen und zu evaluieren. Als Modellregion dient ein ausgewählter, stark befahrener Bereich der Stadt \gls{Karlsruhe}, dessen Straßennetz mithilfe von \gls{osm}-Daten und Verkehrsdaten von Institutionen wie \gls{LUBW}, \gls{mobidatabw} und der \gls{bast} realitätsnah abgebildet wird. \cite{osm,mobidata, lubw}

Die Arbeit verfolgt einen anwendungsorientierten Ansatz: Es wird ein vollständiges System aufgebaut, in dem einzelne Ampelkreuzungen durch RL-Agenten gesteuert werden. Diese erhalten als Eingabe Informationen zur aktuellen Verkehrslage, etwa Fahrzeuganzahl, Wartezeiten oder Stauentwicklungen, und geben als Ausgabe Ampelschaltbefehle zurück. Ziel ist es, durch Training in der Simulation eine Steuerungsstrategie zu entwickeln, die relevante Zielgrößen wie die durchschnittliche Wartezeit, den Verkehrsfluss oder die Anzahl von Fahrzeugstopps optimiert.

Ein positiver Untersuchungsverlauf könnte zeigen, dass bestehende Straßennetze effizienter genutzt werden können, ohne kostspielige Neubauten oder Erweiterungen. Die verbesserte Auslastung bestehender Infrastruktur spart Kosten, reduziert Flächenversiegelung und mindert Umweltbelastung durch Verkehrsvermeidung. Außerdem wäre ein solches adaptive System klimafreundlicher als starre Ampelsteuerungen.

Darüber hinaus soll die Arbeit systematisch untersuchen, wie sich unterschiedliche Modellierungsentscheidungen (z.B. Wahl der Belohnungsfunktion, Anzahl der gesteuerten Agenten, Parametrisierung der Umgebung) auf das Verhalten und die Leistungsfähigkeit der lernenden Agenten auswirken. Die gewonnenen Erkenntnisse sollen kritisch reflektiert und mit konventionellen, nicht-adaptiven Steuerungsstrategien verglichen werden.
\subsection{Begrenzung des Projektumfangs}

Trotz des Anspruchs auf Realitätsnähe handelt es sich bei der vorliegenden Arbeit um ein simulationsbasiertes Projekt mit bewusst gewähltem Fokus. Die Umsetzung erfolgt ausschließlich in der Simulationsumgebung SUMO und basiert auf öffentlich zugänglichen Geodaten (OpenStreetMap) sowie begrenzt verfügbaren Verkehrsdaten von staatlichen und kommunalen Institutionen. Eine vollständige Abbildung aller Aspekte des realen Straßenverkehrs ist damit weder angestrebt noch möglich. \cite{sumo-doc}

Insbesondere ergeben sich folgende Einschränkungen:

\begin{itemize}
    \item \textbf{Eingeschränkte Datenverfügbarkeit:} Nicht alle für eine realitätsnahe Verkehrsmodellierung relevanten Daten liegen in ausreichender Qualität oder Auflösung vor. Exakte Ampelschaltzeiten, Fußgängerfrequenzen oder dynamische Verkehrsdaten zu Stoßzeiten sind teilweise nicht öffentlich zugänglich oder nur unvollständig. Dazu kommt, dass Kommunen teilweise bewusst den Verkehr lenken,etwa durch Zufahrtsbeschränkungen oder Verkehrsberuhigungszonen, was oft nicht öffentlich kommuniziert wird. \cite{umweltbundesamt-umweltzonen}

    \item \textbf{Vereinfachte Modellierung der Umgebung:} In der Simulation wird angenommen, dass alle Verkehrsteilnehmer (Fahrzeuge, Fußgänger, Radfahrer) durch die Agenten präzise erfasst werden können, eine Annahme, die in der Realität durch technische und datenschutzrechtliche Hürden nicht haltbar ist. Moderne Systeme arbeiten hier mit Datenschutz‑mechanismen, aber eine flächendeckende, genaue Erfassung ist unerlässlich, aber derzeit technisch und rechtlich nicht umsetzbar. \cite{DSGVO} Dies wird in der Arbeit berücksichtigt, vor allem bei realiätsnahem Modelltraining.

    \item \textbf{Städtebauliche Verkehrslenkung:} In der Realität regeln Städte Verkehrsflüsse z.B. durch Low-Traffic-Neighbourhoods, Zufahrtsbeschränkungen oder geregelte Zuflusssteuerung, um bestimmte Stadtbereiche zu entlasten. \cite{Low-traffic-Amsterdam} Solche Maßnahmen sind jedoch in der Simulationsumgebung nicht dynamisch abbildbar, da nur externe Ampelagenten kontrollieren und keine zonale Steuerungslogik abgebildet wird.

    \item \textbf{Begrenzter räumlicher und zeitlicher Umfang:} Simuliert wird lediglich ein ausgewählter Ausschnitt des Karlsruher Straßennetzes und nur für definierte Zeitabschnitte. Eine vollständige Tag‑Nacht‑Modellierung liegt außerhalb des Umfangs.

    \item \textbf{Trainings- und Evaluierungsgrenzen:} Reinforcement‑Learning‑Agenten benötigen viele Trainingszyklen. Die in dieser Arbeit verwendete Hardware limitiert Trainingsdauer und Modellkomplexität.
\end{itemize}

Diese bewusste Eingrenzung ermöglicht es, sich auf die technische Umsetzbarkeit und das methodische Vorgehen zu konzentrieren. Dennoch sind die gewonnenen Erkenntnisse relevant, sie liefern zentrale Einsichten in die Wirksamkeit von \gls{ki}‑basierten Verkehrssteuerungssystemen und können als Grundlage für weiterführende Forschung dienen.

\subsection{Wissenschaftliche und gesellschaftliche Relevanz}

Die Kombination von KI und Verkehrssteuerung ist nicht nur ein hochaktuelles Forschungsthema, sondern besitzt auch ein erhebliches Potenzial für den realweltlichen Einsatz. \cite{KI4LSA} Durch die Integration lernfähiger Steuerungssysteme in bestehende Verkehrsmanagementlösungen könnten Städte künftig dynamischer, effizienter und umweltfreundlicher agieren. Die hier behandelte Arbeit leistet einen Beitrag zur Untersuchung der technischen Machbarkeit sowie der Leistungsfähigkeit solcher Systeme unter realitätsnahen Bedingungen.

Gleichzeitig dient die Arbeit als Beispiel für den Einsatz moderner Methoden der Informatik in einem interdisziplinären Anwendungsfeld. Sie schlägt die Brücke zwischen Verkehrsingenieurwesen, Datenanalyse und maschinellem Lernen und eröffnet damit Perspektiven für eine zukunftsweisende Gestaltung urbaner Infrastrukturen.

\subsection{Aufbau der Arbeit}

Die Arbeit ist in sieben Kapitel unterteilt:

\begin{itemize}
    \item Kapitel 2 stellt die theoretischen Grundlagen der Arbeit dar. Es werden die Funktionsweise von SUMO, die Prinzipien des Reinforcement Learning sowie die zugrundeliegenden technischen Komponenten erläutert. Auch verwandte Arbeiten werden kritisch betrachtet.
    \item Kapitel 3 widmet sich den Datenquellen und der Modellierungsgrundlage. Es werden sowohl die verwendeten Geodaten als auch Verkehrszählungen, Ampelschaltpläne und Annahmen beschrieben.
    \item Kapitel 4 beschreibt die methodische Vorgehensweise bei der Erstellung des Simulationsmodells, der Formulierung des Lernproblems, der Wahl der Trainingsstrategie und der technischen Umsetzung.
    \item Kapitel 5 präsentiert die Ergebnisse der Simulationen und stellt sie in Bezug zur gewählten Zielsetzung. Es erfolgt eine quantitative und qualitative Auswertung der Agentenleistung.
    \item Kapitel 6 diskutiert zentrale Herausforderungen und Limitationen der Arbeit, sowohl methodisch als auch datenbezogen.
    \item Kapitel 7 fasst die wesentlichen Erkenntnisse zusammen und gibt einen Ausblick auf weiterführende Forschungsansätze und Anwendungsoptionen.
\end{itemize}

\section{Hintergrund und Stand der Technik}

\subsection{Urbane Verkehrssysteme und Verkehrssteuerung}

Die urbane Verkehrssteuerung umfasst alle Maßnahmen zur Regelung, Lenkung und Optimierung von Verkehrsflüssen innerhalb städtischer Räume. Ziel ist es, den Verkehrsfluss effizient zu gestalten, Staus zu vermeiden, die Sicherheit aller Verkehrsteilnehmer zu erhöhen sowie Emissionen und Lärm zu reduzieren. Klassische Steuerungsmechanismen basieren häufig auf festen Zeitplänen oder einfachen verkehrsabhängigen Regeln, z.\,B. durch Induktionsschleifen oder Detektoren gesteuerte Ampelphasen.

Mit dem Aufkommen neuer Technologien und wachsender Mobilitätsdaten entstehen zunehmend datenbasierte und dynamische Steuerungsansätze. Dazu gehören adaptive Lichtsignalsteuerungen, vernetzte Fahrzeuge (V2X-Kommunikation) und erste Pilotprojekte mit KI-gesteuerten Verkehrsmanagementsystemen. Dennoch sind viele Systeme in der Praxis noch unflexibel oder schwer skalierbar.

\subsection{Simulation urbaner Mobilität mit SUMO}

\textit{Simulation of Urban MObility} (SUMO) ist ein quelloffener, mikroskopischer Verkehrs-Simulator, der ursprünglich vom Deutschen Zentrum für Luft- und Raumfahrt (DLR) entwickelt wurde. SUMO erlaubt die detaillierte Modellierung individueller Fahrzeuge, Straßeninfrastruktur, Ampelschaltungen sowie Fahrverhalten.

Besonders relevant für diese Arbeit sind folgende Merkmale:

\begin{itemize}
    \item \textbf{Mikroskopische Modellierung:} Jedes Fahrzeug wird als individuelles Objekt simuliert. Parameter wie Geschwindigkeit, Abstand oder Spurwechselverhalten sind individuell konfigurierbar.
    \item \textbf{Flexible Netzdefinition:} Verkehrsnetze lassen sich aus OpenStreetMap-Daten sowie aus Shapefiles oder VISUM-Modellen mit dem Tool \texttt{netconvert} erzeugen. Netzdateien können auch mit \texttt{netedit} visuell editiert werden.
    \item \textbf{Nachfragegenerierung:} Fahrpläne und Routen lassen sich mit Tools wie \texttt{activitygen}, \texttt{randomTrips}, \texttt{od2trips} oder \texttt{duarouter} erzeugen – basierend auf statistischen oder echten OD-Matrizen.
    \item \textbf{Multimodalität:} SUMO unterstützt neben Pkw auch Busse, Fahrräder, Fußgänger sowie den öffentlichen Nahverkehr. Ampeln können für alle Verkehrsarten gleichzeitig modelliert werden.
    \item \textbf{Emissionsmodellierung:} Mit Hilfe von integrierten HBEFA-Tabellen (Version 4) kann SUMO CO\textsubscript{2}-, NO\textsubscript{x}- und Feinstaubemissionen simulieren und ausgeben.
    \item \textbf{Steuerbare Ampelanlagen:} Lichtsignalanlagen können sowohl mit festen Programmen als auch dynamisch über die TraCI-Schnittstelle gesteuert werden.
    \item \textbf{Reproduzierbarkeit und Kontrolle:} SUMO ist vollständig deterministisch, was es ideal für kontrollierte Experimente und das Training von KI-Agenten macht.
    \item \textbf{Visualisierung und Debugging:} Die SUMO-GUI und das Tool \texttt{sumo-gui} ermöglichen eine grafische Darstellung von Netz, Fahrzeugen, Ampelphasen und Simulationsergebnissen.
\end{itemize}

Die SUMO-Toolchain bietet damit alle notwendigen Komponenten für die Entwicklung, Analyse und Auswertung urbaner Verkehrsszenarien und stellt eine erprobte Plattform für KI-gestützte Steuerungsexperimente dar.

\subsection{Verstärkendes Lernen (Reinforcement Learning)}

Reinforcement Learning (RL) ist ein Teilgebiet des maschinellen Lernens, bei dem ein Agent durch Interaktion mit einer Umgebung lernt, optimale Handlungen auszuführen. Dabei verfolgt er das Ziel, eine kumulative Belohnung zu maximieren.

Ein RL-Prozess wird typischerweise als Markov Decision Process (MDP) beschrieben und besteht aus folgenden Komponenten:

\begin{itemize}
    \item \textbf{Zustand $s$ (state):} Eine Repräsentation der aktuellen Situation der Umgebung.
    \item \textbf{Aktion $a$ (action):} Eine Entscheidung oder Handlung, die der Agent im Zustand $s$ trifft.
    \item \textbf{Belohnung $r$ (reward):} Ein numerischer Wert, der die Güte der Aktion bewertet.
    \item \textbf{Policy $\pi$:} Eine Strategie, die angibt, welche Aktion in welchem Zustand gewählt wird.
\end{itemize}

Der Agent interagiert mit der Umgebung, beobachtet den Zustand, wählt eine Aktion, erhält eine Belohnung und gelangt in einen neuen Zustand. Durch viele Wiederholungen lernt er, welche Entscheidungen langfristig die besten sind.

Wichtige Algorithmen, die in dieser Arbeit potenziell relevant sind, sind:

\begin{itemize}
    \item \textbf{Q-Learning:} Modellfreies, off-policy Lernverfahren zur Annäherung an optimale Aktionen.
    \item \textbf{DQN (Deep Q-Network):} Kombination von Q-Learning mit neuronalen Netzen.
    \item \textbf{PPO (Proximal Policy Optimization):} Policy-basierter RL-Ansatz mit stabiler Optimierung.
\end{itemize}

\subsection{SUMO-RL: Architektur und Funktionalität}

\texttt{sumo-rl} ist ein Python-Framework, das SUMO mit Reinforcement Learning verbindet. Es basiert auf der \texttt{gymnasium}-Schnittstelle und abstrahiert typische Aufgaben wie die Definition von Beobachtungen, Aktionen und Belohnungen für RL-Agenten. Die Umgebung wird durch die Klasse \texttt{SumoEnvironment} bereitgestellt.

Zentrale Eigenschaften von \texttt{sumo-rl}:\cite{sumo-rl}

\begin{itemize}
    \item \textbf{TraCI-Integration:} Ermöglicht über das Traffic Control Interface zur Laufzeit den Zugriff auf Fahrzeugdaten, Ampelphasen, Fahrzeugwarteschlangen u.\,v.\,m.
    \item \textbf{Ein- und Mehragentenunterstützung:} \texttt{sumo-rl} unterstützt sowohl Single-Agent-Setups als auch Multi-Agent-Steuerung über die PettingZoo-API. Jeder gesteuerte Knoten im Netz kann einem eigenen Agenten zugewiesen werden.
    \item \textbf{Beobachtungen:} Die Umgebung liefert Beobachtungsvektoren mit kodierter Ampelphase, Rückstaulänge, Anzahl wartender Fahrzeuge und Fahrzeugdichte je Spur.
    \item \textbf{Aktionen:} Die Agenten treffen diskrete Entscheidungen über Phasenwechsel, wobei \texttt{delta\_time}, \texttt{yellow\_time} und \texttt{min\_green} die zeitliche Dynamik definieren.
    \item \textbf{Belohnungsfunktionen:} Der Standard-Reward basiert auf der Differenz kumulierter Wartezeiten. Eigene Funktionen können bei Initialisierung übergeben werden.
    \item \textbf{Kompatibilität:} Das Framework ist kompatibel mit Stable-Baselines3, PyTorch, TensorFlow, RLlib und anderen gängigen ML-Frameworks.
\end{itemize}

Beispielhafte Initialisierung:

\begin{verbatim}
env = SumoEnvironment(
    net_file='net.net.xml',
    route_file='routes.rou.xml',
    use_gui=True,
    reward_fn='diff-waiting-time',
    single_agent=True,
    delta_time=5,
    yellow_time=2,
    min_green=5
)
\end{verbatim}

\subsection{Verwandte Arbeiten}

In den letzten Jahren wurden zunehmend Studien veröffentlicht, die KI-Methoden zur Optimierung der Verkehrssteuerung einsetzen. Eine Auswahl relevanter Forschungsansätze:

\begin{itemize}
    \item \textbf{Wei et al. (2019):} Einsatz von Deep Q-Learning zur Optimierung einer einzelnen Ampel in SUMO mit signifikantem Rückgang der Wartezeiten \cite{wei2019}.
    \item \textbf{Chu et al. (2020):} Untersuchung von Multi-Agent-Ansätzen mit Deep RL zur Steuerung großflächiger Ampelnetze \cite{chu2020}.
    \item \textbf{Zheng et al. (2019):} Einführung eines Lernverfahrens zur Koordination konkurrierender Phasen bei Ampelsteuerungen mit Hilfe von SUMO \cite{zheng2019}.
\end{itemize}

Diese Arbeiten zeigen, dass RL-basierte Methoden das Potenzial haben, bestehende Systeme zu übertreffen – sowohl bei einfachen als auch bei komplexeren Szenarien. Die vorliegende Arbeit knüpft an diesen Forschungsstand an und erweitert ihn um eine Anwendung auf reale Geodaten aus Karlsruhe sowie eine methodische Evaluation.



\section{Datenquellen und Modellierungsgrundlage}
\label{sec:datenquellen_und_modellgrundlage}
\subsection{OpenStreetMap als Grundlage für das Verkehrsmodell}

Das Verkehrsnetz für die Simulation basiert auf öffentlich verfügbaren Geodaten der Plattform OpenStreetMap (OSM). OSM bietet eine frei zugängliche, kollaborativ gepflegte Datenbank, die detaillierte Informationen zu Straßenverläufen, Kreuzungen, Fahrspuren, Tempolimits und teilweise zu Ampelanlagen enthält. Diese Eigenschaften machen OSM zu einer geeigneten Grundlage für Verkehrssimulationen mit SUMO. \cite{osm, osm-git}

Zur Erstellung des Netzes wurde ein Ausschnitt des Straßennetzes der Stadt Karlsruhe exportiert, der einen stark frequentierten urbanen Bereich mit mehreren signalgesteuerten Kreuzungen umfasst. Der betrachtete Bereich liegt zwischen 49{,}00738,\textdegree{}N und 49{,}01523,\textdegree{}N sowie 8{,}38589,\textdegree{}E und 8{,}40050,\textdegree{}E und deckt unter anderem die Reinhold-Frank-Straße, das Mühlburger Tor und angrenzende Hauptverkehrsachsen ab. Der Export erfolgte als \texttt{.osm}-Datei über den Geofabrik-Downloaddienst bzw. mit dem Tool \gls{Josm}. \cite{osm-export, josm} Die anschließende Konvertierung in das SUMO-Format erfolgte mit dem Programm \texttt{netconvert} (Version 1.19.0), einem Teil der SUMO-Toolchain. \cite{sumo-tools} Hierbei wurden relevante Parameter wie Straßentypen, Fahrspuren, Prioritäten und erlaubte Abbiegevorgänge berücksichtigt. Als Typemap kam \texttt{osmNetconvert.typ.xml} zum Einsatz, um realitätsnahe Geschwindigkeiten und Fahrspuren zuzuweisen.

Das resultierende Verkehrsnetz umfasst 1.379 definierte Knotenpunkte (\textit{junctions}), 1.919 Straßenkanten (\textit{edges}) sowie insgesamt 5.310 modellierte Fahrstreifen (\textit{lanes}). Darüber hinaus konnten 17 signalgesteuerte Kreuzungen mit Lichtsignalanlagen (\textit{traffic lights}) identifiziert (siehe Algorithmus~\ref{alg:find_valid_tls}) werden, die als Steuerungspunkte für das spätere Training der Reinforcement-Learning-Agenten dienen.

Zusätzliche Informationen wie Ampeldefinitionen und Vorfahrtsregeln manuell über das Tool \texttt{netedit} ergänzt oder angepasst, um die Netzrealität weiter zu verfeinern. Dabei wurden insbesondere fehlerhafte Knotenbeziehungen bereinigt sowie isolierte Netzteile entfernt. Die finale \texttt{.net.xml}-Datei bildet die topologische und funktionale Grundlage für alle weiteren Simulationsschritte.

\begin{figure}[H]
    \centering
    \includegraphics[width=0.7\textwidth]{images/karlsruhe_net.png}
    \caption{Visualisierung des aus OSM generierten SUMO-Netzes (\gls{sumogui}).}
    \label{fig:sumo_network}
\end{figure}

\begin{figure}[H]
    \centering
    \includegraphics[width=0.6\textwidth]{images/karlsruhe_osm.png}
    \caption{Screenshot des ursprünglichen OpenStreetMap-Ausschnitts (OpenStreetMap).}
    \label{fig:osm_screenshot}
\end{figure}

Die Wahl von OpenStreetMap als Datenquelle gewährleistet eine offene, reproduzierbare und erweiterbare Modellierungsbasis. Jedoch bringt die Nutzung von OSM-Daten auch einige Einschränkungen mit sich, die bei der Modellierung berücksichtigt werden müssen:\cite{osm-export, osm, osm-Guide}

\begin{itemize}
    \item \textbf{Uneinheitlicher Detaillierungsgrad:} Die Erfassungstiefe variiert regional stark, was dazu führt, dass z.\,B. Tempolimits, Fahrspuren oder Abbiegebeschränkungen an vielen Stellen fehlen oder unvollständig sind.
    \item \textbf{Fehlende Ampel- und Signalsteuerungsdaten:} OSM enthält in der Regel keine vollständigen Angaben zu Ampelphasen, Umlaufzeiten oder koordinierter Schaltung. SUMO kann zwar aus heuristischen Annahmen Standardampeln generieren, diese weichen jedoch potenziell stark von der realen Steuerung ab.
    \item \textbf{Keine garantierte Netzvollständigkeit:} Besonders kleinere Straßen, private Zufahrten oder temporäre Baustellen sind häufig nicht oder nur unzureichend erfasst. Zudem treten beim Zuschnitt von Kartenausschnitten an den Netzrändern regelmäßig unvollständige Knoten oder isolierte Kanten auf.
    \item \textbf{Abweichende Modellierungskonzepte:} In OSM werden parallele Fahrbahnen oder getrennte Richtungsfahrbahnen oft als unabhängige Wege modelliert. Ohne geeignete Nachbearbeitung kann dies zu unnötigen Knoten und ineffizientem Verkehrsverhalten führen.
    \item \textbf{Abhängig von Typemap- und Importoptionen:} Die Interpretation der OSM-Tags erfolgt in SUMO durch sogenannte Typemaps, die z.\,B. Tempolimits und Spuranzahl je nach Straßentyp zuweisen. Ohne geeignete Typemap kann das Verhalten nicht der Realität entsprechen. \cite{netconvert}
\end{itemize}

\textbf{Fazit:} Insgesamt erlaubt OSM trotz dieser Limitationen den Aufbau eines funktionalen Verkehrsnetzes für mikroskopische Simulationen, sofern der Import sorgfältig konfiguriert und die resultierenden Daten kritisch hinterfragt und gegebenenfalls manuell nachbearbeitet werden.

\subsection{Verfügbare Verkehrsdaten}
Zur Kalibrierung und Validierung der Simulation sind verlässliche Verkehrsdaten unerlässlich. In Baden-Württemberg stehen hierfür mehrere öffentliche sowie kommerzielle Quellen zur Verfügung. Diese umfassen Informationen über Verkehrsstärken, Fahrzeugzusammensetzung, Reisezeiten und Störungen im Straßenverkehr. Im Folgenden werden die wichtigsten Quellen sowie die für das vorliegende Projekt relevanten Verkehrszählungen zusammengefasst.

\subsubsection{Öffentliche Datenquellen: LUBW, MobiData BW, Straßenverkehrszentrale, BASt}
Die LUBW stellt aggregierte Verkehrszählungen im Rahmen automatischer Straßenverkehrszählungen bereit. Diese umfassen Tagesmittelwerte sowie jahreszeitliche Schwankungen für verschiedene Fahrzeugkategorien. Die Daten der \gls{SVZBW} liefern zudem Echtzeitinformationen zu Störungen, Baustellen und Verkehrsfluss.\cite{Verkehrszählungen_Baden-Württemberg,bast,lubw,baden-wuerttemberg,svzbw}

Über die Plattform MobiData BW werden offene Mobilitätsdaten gebündelt bereitgestellt, darunter auch historische Detektordaten und OpenTraffic-Feeds. Die BASt wiederum veröffentlicht bundesweite Zähldaten, insbesondere für überörtliche Straßen. \cite{bast}

Diese öffentlichen Quellen bilden eine solide Grundlage für die realitätsnahe Modellierung des Verkehrsaufkommens, sind jedoch teilweise nur in aggregierter Form oder mit begrenzter räumlicher Auflösung verfügbar.

\subsubsection{Stationäre Zählstellen in Karlsruhe und Umgebung}
\label{sec:zaehlstellen-karlsruhe}
Eine besonders wertvolle Datenquelle zur realitätsnahen Modellierung des Verkehrsaufkommens stellen die stationären Zählstellen des Landes Baden-Württemberg dar. Diese liefern standardisierte Tagesverkehrswerte, getrennt nach Fahrzeugklassen.

Im direkten Untersuchungsgebiet, der Reinhold-Frank-Straße in Karlsruhe, befindet sich eine automatische Dauerzählstelle. Die dort erfassten Werte für den Zeitraum vom 1.1. bis 20.6.2025 lauten: \cite{Verkehrszählungen_Baden-Württemberg}

\begin{itemize}
    \item \textbf{\gls{kfz}:} 21.300 Fahrzeuge/Tag
    \item \textbf{PKW:} 20.500 Fahrzeuge/Tag
    \item \textbf{\gls{snfz}:} 120 Fahrzeuge/Tag
\end{itemize}

Diese Messwerte stimmen gut mit den aus den äußeren Zufahrtsachsen abgeleiteten Schätzungen überein. Um das Verkehrsaufkommen plausibel zu quantifizieren, wurden zusätzlich acht zentrale Zählstellen aus dem Jahr 2023 entlang wichtiger Ein- und Ausfallstraßen berücksichtigt. Sie bilden die Grundlage für die Annahmen über den täglichen Verkehr, der potenziell durch das untersuchte innerstädtische Netz fließt:

\begin{table}[H]
    \centering
    \caption{Verkehrszählungen in und um Karlsruhe (DTV, Jahr 2023) \cite{Dauerzählstellen_Ergebnisse}}
    \begin{tabular}{|l|l|r|r|r|}
        \hline
        \textbf{Zufahrt} & \textbf{Zählstellenbeschreibung}      & \textbf{KFZ/Tag} & \textbf{SV/Tag} & \textbf{Gesamt} \\
        \hline
        B10 West         & Rheinbrücke / Entenfang               & 62.102           & 6.159           & 68.261          \\
        B36 Neureut      & Neureuter Str. / Ausfahrt Neureut Süd & 35.165           & 1.712           & 36.877          \\
        B36 Nord         & Eggenstein / Neureut                  & 28.595           & 1.361           & 29.956          \\
        L605 Nord        & Weißes Haus / Eggenstein              & 14.563           & 220             & 14.783          \\
        B36 Süd          & Rheinstetten / Innenstadt             & 24.239           & 1.487           & 25.726          \\
        B36 Mörsch       & Mörsch / Forchheim                    & 26.841           & 1.531           & 28.372          \\
        L605 Süd         & Ettlingen / Bulacher Kreuz            & 65.816           & 3.474           & 69.290          \\
        B10 Ost          & Durlach (A5) / Innenstadt             & 28.555           & 913             & 29.468          \\
        \hline
    \end{tabular}
    \caption*{\footnotesize Hinweis: KFZ = Leichtverkehr (Pkw, Lieferwagen, Motorräder);
        SV = Schwerverkehr (Lkw, Busse, schwere Nutzfahrzeuge);
        Gesamt = Summe aus KFZ und SV.}
    \label{tab:zaehlstellen}
\end{table}

\begin{figure}[H]
    \centering
    \includegraphics[width=0.95\textwidth]{images/zaehlstellenkarte.png}
    \label{fig:zaehlstellenkarte}
    \vspace{0.3em}
    \begin{minipage}{0.9\linewidth}
        \footnotesize \textbf{Legende:}
        \textcolor{yellow}{\large$\bullet$} Temporäre Zählstellen \quad
        \textcolor{red}{\large$\bullet$} Dauerzählstellen \quad
        \textcolor{gray}{\large$\bullet$} Manuelle Zählstellen
    \end{minipage}
    \caption{Lage der Dauerzählstellen im Raum Karlsruhe (Quelle: MobiData BW \cite{mobidata_karte}).}
\end{figure}

Diese externen Zuflüsse bilden die Grundlage für realistische Eingangsströme in der Simulation. Sie versorgen das Untersuchungsgebiet direkt und ergeben ein plausibles Verkehrsaufkommen von etwa 20.000 bis 40.000 Fahrzeugen pro Tag, was mit den Messungen in der Reinhold-Frank-Straße übereinstimmt.

Die Zähldaten erlauben es, die Fahrzeugströme in SUMO proportional zu den realen Verhältnissen abzubilden und unterstützen zugleich die spätere Kalibrierung und Validierung der Szenarien.
\subsubsection{Kommerzielle APIs: TomTom, Google Maps}

Ergänzend zu den öffentlichen Datenquellen bieten kommerzielle Anbieter wie TomTom und Google über Programmierschnittstellen (APIs) hochaufgelöste Echtzeit- und Historikdaten an. Diese umfassen unter anderem:

\begin{itemize}
    \item Durchschnittliche Fahrgeschwindigkeiten nach Wochentag und Uhrzeit,
    \item Verkehrsdichte und Stauinformationen,
    \item Prognosen basierend auf anonymisierten Bewegungsdaten.
\end{itemize}

Der Zugriff auf diese APIs ist in der Regel kostenpflichtig oder durch Nutzungsbeschränkungen limitiert. Sie ermöglichen eine deutlich feinere zeitliche und räumliche Auflösung, was für die Modellierung und spätere Optimierung des Verkehrsflusses mittels KI von Vorteil sein könnte.

Für die vorliegende Arbeit wurden diese kommerziellen Angebote nicht genutzt. Die Modellierung basiert ausschließlich auf offenen Datenquellen wie OSM sowie auf Google Maps für einzelne Standortrecherchen. \cite{googlemaps, tomtom}

\subsection{Modellierung der Ampelschaltungen}

Für eine realitätsnahe Simulation spielt die Modellierung der Lichtsignalsteuerung eine zentrale Rolle. Ampelanlagen beeinflussen maßgeblich den Verkehrsfluss an Knotenpunkten und sind daher ein zentraler Bestandteil der Simulationslogik. \cite{Sumo-tls}

\subsubsection{Verfügbare Daten und Herausforderungen}

In den öffentlich zugänglichen OSM-Daten sind Ampelanlagen in der Regel lediglich als Punktobjekte an Kreuzungen vermerkt. Informationen zu Phasenplänen, Umlaufzeiten oder koordinierter Schaltung fehlen vollständig. Auch von Seiten der Stadt Karlsruhe oder anderer kommunaler Stellen liegen keine detaillierten Steuerungsdaten vor, da diese in der Regel nicht öffentlich zugänglich sind. \cite{Sumo-osm}

Eine eigene systematische Erfassung der Schaltzeiten wäre zwar prinzipiell möglich, hätte jedoch einen erheblichen Zeitaufwand bedeutet und wäre aufgrund der dynamischen, nicht-statischen Signalsteuerungen (z.\,B. verkehrsabhängige Phasen) methodisch schwer zuverlässig umzusetzen gewesen.

\subsubsection{Vereinfachte Modellierung}

Aus diesen Gründen wurde anfangs auf eine synthetische Modellierung zurückgegriffen. Mittels netgenerate wurde ein synthetisches Netz generiert und abenfalls testweise Modelle trainiert. Dies erwies sich als sehr simpel, wegen geringer Komplexität. SUMO bietet hierfür die Möglichkeit, sogenannte \texttt{tlLogic}-Blöcke manuell oder automatisch zu definieren, die verschiedene Phasenfolgen und Zeitparameter enthalten. In der vorliegenden Arbeit wurde auf Standardampelprogramme zurückgegriffen, wie sie in SUMO generisch verwendet werden, um testweise eine vereinfachte Lichtsignalsteuerung zu modellieren. Diese erlaubt die spätere Umsetzung des realen karlsruher Netzes. \cite{Sumo-tls,netgenerate}

\section{Methodik}

\subsection{Untersuchungsregion und Datenbasis}

\subsubsection{Auswahl der Untersuchungsregion}

Für die Anwendung und Evaluation der KI-basierten Verkehrssteuerung wurde ein Ausschnitt des innerstädtischen Straßennetzes von Karlsruhe gewählt. Die Auswahl fiel auf ein Gebiet rund um die Reinhold-Frank-Straße und das Mühlburger Tor, das durch hohe Verkehrsdichte, komplexe Knotenpunkte und mehrere signalgesteuerte Kreuzungen gekennzeichnet ist. Der gewählte Bereich liegt geografisch zwischen 49{,}006947\,\textdegree{}N und 49{,}015602\,\textdegree{}N sowie 8{,}380176\,\textdegree{}E und 8{,}403887\,\textdegree{}E und deckt mehrere stark frequentierte Hauptachsen ab.

Die Entscheidung für diese Region basiert auf folgenden Kriterien:

\begin{itemize}
    \item \textbf{Hohe Verkehrsbedeutung:} Das Gebiet stellt einen wichtigen innerstädtischen Verkehrsraum dar, der sowohl Pendlerverkehr als auch lokalen Individualverkehr aufnimmt.
    \item \textbf{Bekanntes Stauaufkommen:} Die Reinhold-Frank-Straße ist in der Stadtbevölkerung für regelmäßige Verkehrsstaus bekannt, insbesondere zu Stoßzeiten.
    \item \textbf{Verfügbarkeit realer Verkehrszähldaten:} Eine automatische Dauerzählstelle erhebt dort täglich Verkehrsdaten. Für den Zeitraum vom 1.1. bis 20.6.2025 wurden durchschnittlich 21.300 Kfz/Tag erfasst.
    \item \textbf{Zusätzliche Zähldaten angrenzender Hauptverkehrsstraßen:} Zählstellen an der B10, B36, L605 und in Durlach liefern ergänzende Werte zur Plausibilisierung des Gesamtverkehrsflusses.
    \item \textbf{Vorhandensein mehrerer Ampelanlagen:} Im Netz befinden sich 17 signalgesteuerte Kreuzungen, geeignet für RL-gesteuerte Steuerungsexperimente.
    \item \textbf{Gute Abgrenzbarkeit:} Das Gebiet ist topologisch geschlossen und in SUMO sauber simulierbar.
    \item \textbf{Verfügbarkeit von Geodaten:} Die Region ist in OpenStreetMap detailliert kartiert.
\end{itemize}

\subsubsection{Verfügbare Verkehrszähldaten}

Für die Kalibrierung und Validierung der Verkehrssimulation wurden verschiedene reale Zähldatenquellen aus dem Raum Karlsruhe herangezogen. Hauptquelle war dabei die offene Mobilitätsdatenplattform des Landes Baden-Württemberg (MobiData BW)\cite{mobidata_stunden}. Dort werden automatisiert erfasste Stundenwerte stationärer Dauerzählstellen veröffentlicht, die eine fein aufgelöste Analyse von Verkehrsverläufen ermöglichen.

Konkret wurden folgende Datensätze ausgewertet:

\begin{itemize}
    \item \textbf{Dauerzählstelle Reinhold-Frank-Straße:} Erfasst täglich die Anzahl der Kraftfahrzeuge (Kfz), aufgeschlüsselt nach Fahrzeugklassen (PKW, lNfz, sNfz). Für den Zeitraum 01.01.–20.06.2025 lag der durchschnittliche Tagesverkehr (DTV\footnote{DTV = durchschnittlicher Tagesverkehr: Durchschnittliche Anzahl an Fahrzeugen pro Tag über einen bestimmten Zeitraum hinweg.}) bei ca. 21.300 Kfz/Tag.
    
    \item \textbf{Historische Jahresmittelwerte:} Langzeitdatenreihen von 2008–2024 aus MobiData BW ermöglichen eine Kontextualisierung der aktuellen Verkehrsbelastung.

    \item \textbf{Zählstellen an äußeren Zufahrtsachsen:} Ergänzende Zähldaten aus dem Jahr 2023 an acht stark befahrenen Einfallstraßen (u.\,a. B10, B36, L605)\cite{mobidata_karte} liefern Anhaltspunkte zur Verkehrsstärke an den Netzrändern.
\end{itemize}

Die Kombination dieser Quellen ermöglicht eine robuste, datenbasierte Schätzung realistischer Flussverteilungen für die Simulation – sowohl zeitlich (z.\,B. Spitzenlasten) als auch räumlich (Zufahrtsverteilung).

\subsection{Aufbau des Simulationsmodells in SUMO}

\subsubsection{Netzgenerierung und Verkehrsflussmodellierung}

Zur Modellierung des realen Straßennetzes wurde ein Kartenausschnitt des Untersuchungsgebiets über die Exportfunktion von OpenStreetMap\cite{osm-export} heruntergeladen und anschließend mit JOSM\cite{josm} bearbeitet. Die Auswahl des Ausschnitts orientierte sich an der geografischen Abgrenzung rund um die Reinhold-Frank-Straße sowie die angrenzenden Hauptverkehrsachsen im Bereich des Mühlburger Tors. Der bereinigte Kartenausschnitt wurde anschließend mit dem SUMO-Werkzeug \texttt{netconvert}\cite{sumo-tools} in ein netzwerkkompatibles XML-Format überführt. Dabei kamen zusätzliche Optionen zur Verbesserung der Ampelmodellierung und Fahrstreifenzuordnung zum Einsatz (z.\,B. \texttt{--tls.guess-signals} und \texttt{--junctions.join}).

Die Erzeugung der Fahrzeugbewegungen erfolgte auf zwei Wegen: Zum einen wurden mit dem SUMO-Skript \texttt{randomTrips.py} initiale Testflüsse erzeugt, um die Simulation zu validieren. Zum anderen wurden auf Basis der analysierten Verkehrszähldaten realitätsnahe Flussprofile definiert, welche die beobachteten DTV-Werte auf die Randkanten des simulierten Netzes verteilen. Dabei wurde darauf geachtet, dass die Hauptverkehrsachsen wie die B10 oder B36 als primäre Zufahrtsrouten fungieren und mit einer höheren Fahrzeugdichte gewichtet werden.

Die resultierenden Routendateien wurden anschließend mit \texttt{duarouter} verarbeitet, um konfliktfreie Fahrten über das simulierte Netz zu erzeugen. Um unterschiedliche Verkehrssituationen abzubilden, wurden Szenarien mit variierender Verkehrsdichte simuliert – beispielsweise für Morgen- und Abendspitzen sowie für gleichmäßige Durchfahrtsverteilung.

Die erzeugten Flüsse orientieren sich dabei an der realen Kapazität der Knoten und Straßen und wurden mit Hilfe von Detektor-Ausgaben (u.\,a. \texttt{laneAreaDetector}) überprüft und bei Bedarf nachjustiert.

\subsubsection{Identifikation relevanter Zufahrtskanten}

Die Generierung realistischer Verkehrseinträge in das simulierte Netz basiert auf der systematischen Ermittlung geeigneter Zufahrtskanten. Diese stellen die äußeren Einfallstraßen dar, über die der Verkehr gemäß den Zähldaten in das Untersuchungsgebiet einfließt.

Zur Auswahl wurden zunächst bekannte Hauptverkehrsachsen wie die B10, B36, L605 oder die Durlacher Allee herangezogen. Anschließend erfolgte eine semiautomatische Zuordnung der SUMO-Kanten (\texttt{<edge>}) auf Basis der in OpenStreetMap vergebenen Straßennamen. Hierzu wurde ein Python-Skript eingesetzt, das alle Kanten mit einem \texttt{name}-Attribut durchsuchte und auf relevante Teilstrings prüfte (z.\,B. \texttt{"B10"}, \texttt{"Reinhold-Frank-Straße"}). Die Ergebnisse wurden manuell geprüft und bei Bedarf durch visuelle Kontrolle in \texttt{netedit} ergänzt.

Die so extrahierten Kanten wurden je Verkehrsachse gruppiert und bilden die Grundlage für die segmentierte Trip-Erzeugung.

\subsubsection{Automatisierte Generierung von Trips auf Basis realer Zählerdaten}

Zur Simulation realitätsnaher Verkehrsströme wurden die durchschnittlichen Tagesverkehre (DTV) aus Abschnitt~\ref{sec:zaehlstellen-karlsruhe} auf die jeweiligen Zufahrtsgruppen skaliert und anschließend auf die Simulationsdauer von 3600\,s verteilt. Ein eigens entwickeltes Python-Skript erzeugte aus diesen Daten Fahrzeugeinträge (\texttt{<trip>}), die mithilfe von \texttt{randomTrips.py} über die identifizierten Randkanten eingespielt wurden.

Die erzeugten Trips wurden im XML-Format gespeichert und anschließend mit \texttt{duarouter} zu vollständigen, konfliktfreien Routen (\texttt{<route>}) konvertiert. Die Gesamtanzahl und Verteilung der Fahrzeuge orientierte sich dabei an den stündlich extrapolierten DTV-Werten. Es wurde darauf geachtet, dass insbesondere stark belastete Zufahrten (z.\,B. B10, L605) mit höherem Gewicht berücksichtigt wurden.

\subsubsection{Visuelle und technische Validierung des Netzmodells}

Vor dem eigentlichen Einsatz des Modells wurde das gesamte Netz sowohl strukturell als auch funktional validiert. Die Prüfung erfolgte in mehreren Stufen:

\begin{itemize}
    \item \textbf{Netzprüfung:} Einsatz von \texttt{netconvert --check-lane-geometry} und \texttt{netcheck} zur Überprüfung der topologischen Konsistenz.
    \item \textbf{Visuelle Kontrolle:} Mit der SUMO-GUI sowie \texttt{netedit} wurden kritische Knoten visuell inspiziert, um Fehler wie unverbundene Spuren, widersprüchliche Geometrien oder falsche Richtungen zu identifizieren.
    \item \textbf{Manuelle Überprüfung aller TLS:} Jede einzelne Lichtsignalanlage wurde manuell in \texttt{netedit} geöffnet. Die Phasenpläne, gesteuerten Verbindungen und Zustandslängen (\texttt{state}) wurden dabei geprüft und bei Bedarf korrigiert.
\end{itemize}

Diese manuelle Validierung war äußerst zeitaufwändig, da fehlerhafte TLS nicht automatisch von SUMO erkannt werden. Die Identifikation von Problemen wie unpassenden \texttt{linkIndex}-Werten oder inkonsistenten Zustandslängen erforderte intensives Testen und systematisches Debugging. In mehreren Fällen mussten TLS-Definitionen komplett neu erstellt oder aufgelöst werden, was ein hohes Maß an Modellierungsverständnis und Geduld erforderte.

\subsubsection{Szenarien und Referenzsimulationen}

Zur Validierung der Verkehrsflüsse wurden unterschiedliche Simulationsszenarien implementiert:

\begin{itemize}
    \item \textbf{Morgenspitze (Rush Hour):} Verstärkte Einträge an den Süd- und Westzufahrten mit hoher Verkehrsdichte.
    \item \textbf{Abendliche Rückstaus:} Stärkere Belastung der Ausfallstraßen und des Innenstadtrings.
    \item \textbf{Gleichmäßiger Tagesverlauf:} Homogene Einträge mit ca. 1000 Fahrzeugen/h pro Richtung.
    \item \textbf{Zentrumsfokus:} Fokus auf Verkehrsströme über die Reinhold-Frank-Straße und das Mühlburger Tor.
\end{itemize}

Die resultierenden Simulationen dienten der Überprüfung der Netzdurchlässigkeit und der Validierung der physischen Kapazität der Kreuzungen. Dazu wurden Heatmaps, Fahrzeugzählungen sowie Detektor-Ausgaben (z.\,B. \texttt{laneAreaDetector}) ausgewertet. Die gewonnenen Erkenntnisse flossen in die finale Konfiguration der Flüsse und Phasenpläne ein.


\subsubsection{Signalsteuerung und Simulationsparameter}

Das untersuchte Verkehrsnetz umfasst insgesamt 17 signalgesteuerte Kreuzungen. Diese wurden aus dem OpenStreetMap-Datensatz automatisch erkannt und im Rahmen der Netzkonvertierung mit \texttt{netconvert} anhand der Option \texttt{--tls.guess-signals} initial als Lichtsignalanlagen (Traffic Light Systems, TLS) modelliert. Die generierten Ampelphasen wurden anschließend manuell überprüft und bei Bedarf über die mit SUMO mitgelieferte GUI \texttt{netedit} angepasst.

Für die initiale Simulation wurde eine feste Phasenlogik (fixed-time control) verwendet, um ein Grundverhalten zu etablieren. Dabei erhielten alle Kreuzungen definierte Signalphasen mit festen Umlaufzeiten zwischen 30 und 60 Sekunden, abhängig von der Knotentopologie. Zur Vorbereitung des Reinforcement-Learning-Trainings wurden alle relevanten Kreuzungen so konfiguriert, dass sie in SUMO als „aktuiert“ geschaltet werden konnten. Dies ist erforderlich, damit die Steuerung durch externe Agenten via \texttt{TraCI} möglich ist.

Die Simulation wurde mit folgenden Parametern durchgeführt:

\begin{itemize}
    \item \textbf{Simulationszeitraum:} 3600 Sekunden (entspricht 1 Stunde Echtzeit)
    \item \textbf{Zeitschritt (step-length):} 1,0 s
    \item \textbf{Routengenerierung:} deterministisch mit fixer seed zur Reproduzierbarkeit
    \item \textbf{Simulationstyp:} \texttt{meso}-Modus zur Beschleunigung des Trainings (später auch \texttt{default}-Modus für Evaluation)
    \item \textbf{Verkehrsverteilung:} über \texttt{flows.xml} definiert und über Randkanten eingeleitet
\end{itemize}

Die Definition und Verwaltung der TLS-Systeme erfolgt in separaten Dateien (\texttt{*.add.xml}), welche in der SUMO-Konfigurationsdatei eingebunden werden. Für die spätere Anbindung an das Reinforcement-Learning-Modul wurden alle zu steuernden Ampelanlagen mit eindeutigen IDs versehen und überprüft.

\subsection{Reinforcement-Learning-Konzept}
\subsubsection{Formulierung des RL-Problems}
Das Problem der Verkehrssteuerung wird als sequentielles Entscheidungsproblem modelliert und mit Hilfe von Reinforcement Learning (RL) gelöst. Ziel ist es, einen Agenten zu trainieren, der durch geeignete Steuerung der Ampelphasen den Verkehrsfluss optimiert. Die Umgebung besteht aus dem simulierten Straßennetz, wie es in SUMO definiert ist. Die Interaktion erfolgt über das TraCI-Interface, das eine Laufzeitsteuerung der Ampelanlagen erlaubt.
\paragraph{Zustände}

Der Zustand \( s_t \) eines Reinforcement-Learning-Agenten beschreibt die Verkehrssituation an einer einzelnen Kreuzung zum Zeitpunkt \( t \). Ziel ist es, dem Agenten ausreichend Informationen über den lokalen Verkehrsfluss zur Verfügung zu stellen, damit er fundierte Entscheidungen über die Steuerung der Lichtsignalanlage treffen kann.

Die Zustandsrepräsentation basiert auf den folgenden Komponenten:

\begin{itemize}
    \item \textbf{Fahrzeuganzahl pro Zufahrtsspur:} Für jede dem Knoten zuführende Fahrspur wird die aktuelle Anzahl an Fahrzeugen ermittelt. Dies geschieht über sogenannte \texttt{laneAreaDetector}, die für jede Spur individuell in SUMO definiert werden. Die Werte werden periodisch über \texttt{TraCI} abgefragt.
    
    \item \textbf{Warteschlangenlänge (queue length):} Gibt die Anzahl der Fahrzeuge an, die sich auf einer Spur mit Geschwindigkeit \texttt{< 0.1 m/s} befinden. Dies ist ein wichtiges Maß für Rückstaus an Kreuzungen.
    
    \item \textbf{Durchschnittliche Geschwindigkeit pro Spur:} Diese Kenngröße erlaubt Rückschlüsse auf den Verkehrsfluss pro Richtung und ergänzt die reine Anzahlinformation. 
    
    \item \textbf{Ampelphase (TLS state):} Die aktuell geschaltete Ampelphase wird als diskrete Phase kodiert (z.\,B. 0, 1, 2, ...). In SUMO entspricht dies der Index der aktiven Phase im Phasenplan der TLS.
    
    \item \textbf{Dauer der aktuellen Phase:} Die Anzahl der Zeitschritte seit Beginn der aktuellen Phase. Diese Information ist notwendig, um Phasenlängen sinnvoll zu steuern (z.\,B. Mindestgrünzeit).
    
    \item \textbf{Binärmasken zur Phasenwechselbarkeit:} Kodierung, ob ein Wechsel zur nächsten Phase gemäß Übergangsbedingungen (z.\,B. Mindestgrünzeit) aktuell möglich ist. Diese Information ist erforderlich, falls das Aktionsmodell auch direkte Sprünge zwischen nicht direkt benachbarten Phasen erlaubt.
    
    \item \textbf{Optional – Nachbarschaftszustand:} In Multi-Agent-Settings kann es sinnvoll sein, zusätzlich aggregierte Zustandsinformationen benachbarter Knoten einzubeziehen (z.\,B. Gesamtwarteschlange auf ausgehenden Spuren, die zu benachbarten TLS führen).
\end{itemize}

Die Zustände werden zu einem normierten Merkmalsvektor kombiniert und bilden damit die Eingabe für das neuronale Entscheidungsmodell des Agenten.

\paragraph{Aktionen}

Die Aktionsmenge \( A \) eines Agenten beschreibt die Eingriffsmöglichkeiten in den Steuerungsablauf der jeweiligen Ampelkreuzung. Dabei wird zwischen zwei gängigen Aktionsmodellen unterschieden:

\begin{enumerate}
    \item \textbf{Phasenwechsel-Modell:} Der Agent entscheidet, ob die aktuelle Phase fortgesetzt oder zur nächsten gewechselt werden soll. Es handelt sich um ein binäres Aktionsmodell:
    \[
    A = \{ \texttt{keep},\ \texttt{switch} \}
    \]
    Diese Variante wird häufig in klassischen SUMO-RL-Implementierungen verwendet (z.\,B. `sumo-rl`). Die Reihenfolge der Phasen ist dabei festgelegt (z.\,B. zyklischer Übergang).
    
    \item \textbf{Direktwahl-Modell:} Der Agent wählt direkt aus allen möglichen Phasen die nächste aus:
    \[
    A = \{ \texttt{phase}_0,\ \texttt{phase}_1,\ \ldots,\ \texttt{phase}_n \}
    \]
    Diese Variante erfordert eine eigene Definition der Übergangslogik in SUMO (z.\,B. über permissive TLS-Ketten), erlaubt aber größere Flexibilität und exploratives Verhalten.
\end{enumerate}

Unabhängig vom Modell gelten folgende Einschränkungen:

\begin{itemize}
    \item \textbf{Mindestgrünzeiten:} Ein Wechsel der Phase darf erst nach einer definierten Mindestgrünzeit erfolgen (z.\,B. 5 s), um realistische Signalisierung und Verkehrssicherheit zu gewährleisten.
    
    \item \textbf{Sicherheitsbedingte Zwischenphasen:} SUMO erzwingt automatisch Zwischenphasen wie Gelb- oder Räumzeiten. Der Agent gibt nur den Phasenwunsch an, die exakte Ablaufsteuerung erfolgt durch das TLS-Modell in SUMO.
    
    \item \textbf{Simultane Agentenentscheidung:} Bei mehreren Knoten wird jeder TLS-Agent unabhängig gesteuert, es sei denn, ein zentrales Multi-Agent-Training wird implementiert.
\end{itemize}

Zur Reduktion der Aktionsfrequenz wird häufig ein sogenanntes \textbf{Action Interval} festgelegt (z.\,B. alle 5 s), sodass Entscheidungen nur in bestimmten Zeitschritten getroffen werden können. Dies verhindert zu hektisches Umschalten der Ampeln.

\paragraph{Belohnungsfunktion}

Die Belohnungsfunktion ist zentrales Element des Lernprozesses und bestimmt das Optimierungsziel. Sie wurde so gestaltet, dass sie folgende Aspekte negativ gewichtet:

\begin{itemize}
    \item \textbf{Gesamte Wartezeit aller Fahrzeuge} (minimieren)
    \item \textbf{Länge der Fahrzeugschlangen} (minimieren)
    \item \textbf{Anzahl der Stopps} (minimieren)
\end{itemize}

Die konkrete Belohnung \( r_t \) zum Zeitpunkt \( t \) berechnet sich nach:

\[
r_t = -\alpha \cdot \sum_{\text{alle Spuren}} \text{queueLength}_i(t) - \beta \cdot \sum_{\text{alle Fahrzeuge}} \text{waitingTime}_j(t)
\]

wobei \( \alpha \) und \( \beta \) Gewichtungsfaktoren darstellen, die im Rahmen der Hyperparameteroptimierung bestimmt werden. In späteren Varianten kann die Belohnung durch zusätzliche Komponenten wie Emissionen oder Energieverbrauch ergänzt werden, um umweltfreundliche Steuerungsstrategien zu fördern.

\subsection{Analyse und Herausforderungen bei der OSM-Netznutzung}


\subsubsection{Grundstruktur von Lichtsignalanlagen (TLS) in SUMO}
Bevor die Probleme beim OSM-Import analysiert werden, ist es hilfreich, den Aufbau und die Abhängigkeiten der relevanten XML-Elemente in SUMO zu verstehen, insbesondere im Zusammenhang mit der Steuerung von Lichtsignalanlagen (\textit{Traffic Light Systems, TLS}).

\begin{figure}[H]
\centering
\includegraphics[width=0.9\textwidth]{images/junktion.png}
\caption{Visualisierung einer TLS-Kreuzung (eigene Darstellung).}
\label{fig:sumo_karlsruhe}
\end{figure}

\begin{itemize}
    \item \textbf{\texttt{<junction>}} – Definiert Knotenpunkte im Netz. Falls eine Ampel gesteuert wird, ist der Typ \texttt{type="traffic\_light"}. Die ID entspricht in der Regel der TLS-ID.
    
    \item \textbf{\texttt{<connection>}} – Verbindet zwei Fahrstreifen (von \texttt{from} nach \texttt{to}). Wenn diese Verbindung durch eine Ampel kontrolliert wird, enthält sie die Attribute \texttt{tl="<tls\_id>"} und \texttt{linkIndex}. Die Reihenfolge der \texttt{linkIndex}-Werte bestimmt die Position im Phasen-String.
    
    \item \textbf{Controlled Link} – Jede \texttt{<connection>} mit einem \texttt{tl}-Attribut zählt als „gesteuerte Verbindung“. Die Anzahl solcher Verbindungen bestimmt die Länge des Phasenstrings (\texttt{state}).
    
    \item \textbf{\texttt{<tlLogic>}} – Enthält die Steuerungslogik einer TLS. Jede \texttt{<tlLogic>} hat eine eindeutige ID (i.d.R. identisch zur \texttt{junction}-ID) und eine Liste von \texttt{<phase>}-Elementen.
    
    \item \textbf{\texttt{<phase>}} – Jede Phase ist ein String (z.B. \texttt{"Grgr"}), der den Zustand aller \texttt{linkIndex}-Verbindungen kodiert. Jeder Buchstabe (z.B. G = Grün, r = Rot) steht für den Status eines bestimmten kontrollierten Links.
    
    \item \textbf{\texttt{<request>}} – Optionale Anforderungen einzelner Signalgruppen, meist bei aktuierten oder adaptiven TLS. Jeder Eintrag verweist über \texttt{index=} auf einen gesteuerten Link.
\end{itemize}

\begin{figure}[H]
\centering
\begin{tikzpicture}[
  font=\small,
  node distance=1.2cm and 3cm,
  every node/.style={align=center},
  box/.style={draw, rounded corners, minimum width=3.2cm, minimum height=1cm}
]

% Hauptknoten
\node[box] (junction) {\texttt{<junction>}\\ID = J1};
\node[box, right=4.5cm of junction] (tlLogic) {\texttt{<tlLogic>}\\ID = J1};

% Verbindungen unterhalb
\node[box, below left=1.7cm and 0.4cm of junction] (conn1) {\texttt{<connection>}\\\texttt{linkIndex=0}};
\node[box, below right=1.7cm and 0.4cm of junction] (conn2) {\texttt{<connection>}\\\texttt{linkIndex=1}};

% Phase
\node[box, below right=1.7cm and 0.4cm of conn1] (phase) {\texttt{<phase>}\\\texttt{state="Gr"}};

% Kanten
\draw[->] (junction.east) -- node[above] {\texttt{tl="J1"}} (tlLogic.west);
\draw[->] (junction.south west) -- (conn1.north);
\draw[->] (junction.south east) -- (conn2.north);
\draw[->] (conn1.south) -- ([xshift=-0.8cm]phase.north);
\draw[->] (conn2.south) -- ([xshift=+0.8cm]phase.north);

\end{tikzpicture}
\caption{Zusammenspiel von Kreuzung, Verbindungen und Ampellogik in SUMO}
\label{fig:tls_structure}
\end{figure}


\subsubsection{Typische Fehlerquellen nach OSM-Import}

Die automatische Ableitung von Ampelsteuerungen aus OSM ist unvollständig und fehleranfällig. Im Zusammenspiel mit \texttt{sumo-rl} ergeben sich daraus mehrere konkrete Probleme:

\begin{itemize}
    \item \textbf{Fehlende oder unvollständige TLS-Definitionen:} In OSM sind Ampelanlagen in der Regel lediglich als Punktknoten mit dem Tag \texttt{highway=traffic\_signals} erfasst. Die genaue Schaltlogik (\texttt{tlLogic}) – inklusive Phasen und Zustände – fehlt vollständig. SUMO generiert daher beim Netzimport mit \texttt{--tls.guess-signals} heuristische Ampeldefinitionen, die jedoch oft lückenhaft oder unbrauchbar sind.
    
    \item \textbf{TLS mit nur einer Phase:} Viele der generierten Ampeln besitzen lediglich eine einzige definierte Phase. Dies entspricht keinem realen Verhalten und führt zu Fehlern beim Training mit \texttt{sumo-rl}, da das Framework mindestens zwei steuerbare Phasen voraussetzt. Die betroffenen Knoten müssen daher identifiziert und aus der Simulation ausgeschlossen oder manuell korrigiert werden.
    
    \item \textbf{Unstimmige Phasenlängen:} Jede Phase in SUMO ist ein Zeichenstring (\texttt{state}), dessen Länge der Anzahl der gesteuerten Verbindungen (sogenannte \textit{controlled links}) entsprechen muss. Bei fehlerhafter Generierung ist diese Bedingung oft verletzt – etwa wenn der \texttt{state} zu kurz oder zu lang ist. Dies führt in \texttt{sumo-rl} zu Indexfehlern oder undefiniertem Verhalten.
    
    \item \textbf{Fehlerhafte oder überzählige \texttt{<request>}-Einträge:} Jede TLS enthält in der Netzdatei zusätzliche Steuerinformationen über \texttt{request}-Elemente. Diese verweisen auf spezifische Signale mittels eines Index. Häufig verweisen diese Einträge jedoch auf nicht vorhandene Verbindungen, da \texttt{netconvert} Signalverknüpfungen nicht korrekt zuordnet. SUMO ignoriert solche Fehler teilweise still – \texttt{sumo-rl} hingegen bricht mit Ausnahmen ab.
    
    \item \textbf{Mehrdeutige oder verschachtelte Kreuzungen:} In komplexeren innerstädtischen Kreuzungen fasst SUMO mehrere OSM-Knoten zu einem „cluster“ zusammen, um den Verkehrsfluss abzubilden. Dies kann zu sehr großen TLS mit dutzenden Ein- und Ausfahrten führen, die übermäßig viele Phasen oder extrem lange Zustandsdefinitionen erzeugen. Solche TLS sind schwer zu debuggen und häufig inkompatibel mit den Erwartungen von \texttt{sumo-rl}.
\end{itemize}

\paragraph{Folgen für \texttt{sumo-rl}}

Das Framework \texttt{sumo-rl} erwartet für jede zu steuernde TLS:

\begin{itemize}
    \item mindestens zwei valide Phasen,
    \item konsistente Phasenzustände (\texttt{state}) mit korrekter Länge,
    \item vollständige Verbindungen zu kontrollierten Links,
    \item eindeutig identifizierbare TLS-IDs.
\end{itemize}

Sind diese Anforderungen nicht erfüllt, führt dies typischerweise zu einer der folgenden Fehlermeldungen:

\begin{itemize}
    \item \texttt{IndexError: string index out of range}
    \item \texttt{ValueError: Invalid phase length}
    \item \texttt{KeyError: TLS not found}
\end{itemize}

Da diese Probleme nicht durch SUMO selbst gemeldet, sondern erst zur Laufzeit in \texttt{sumo-rl} sichtbar werden, ist ein systematischer Debugging- und Reparaturprozess zwingend notwendig. Die Komplexität steigt dabei exponentiell mit der Anzahl der TLS im Netz.

\paragraph{Erkenntnis}

Der direkte Import von OSM-Daten in SUMO erzeugt ein formal nutzbares Netz – jedoch nicht automatisch ein für Reinforcement Learning geeignetes. Ohne zusätzliche Aufbereitung ist ein stabiler Trainingsbetrieb in \texttt{sumo-rl} nicht möglich. Deshalb wurden im Rahmen dieser Arbeit eigene Tools zur automatisierten Analyse und Reparatur des Netzes entwickelt, die im folgenden Abschnitt näher beschrieben werden.

\subsubsection{Problematik nicht-motorisierter Verkehrswege im OSM-Modell}

Ein besonders gravierendes Problem beim OSM-Import waren die Strukturen nicht-motorisierter Verkehrsträger – vor allem Fußwege, Überwege und Fahrradtrassen. Diese sind im OSM-Datenmodell oft detailliert, aber aus simulationslogischer Sicht problematisch integriert:

\begin{itemize}
    \item \textbf{Separate Fahrspuren für Radverkehr:} Diese führen zu zusätzlichen Fahrstreifen mit eigenen Abbiegebeziehungen, die SUMO automatisch als kontrollierungsbedürftig klassifiziert. Häufig entstehen dadurch extrem lange TLS-Zustände mit über 50 Signalgruppen.
    
    \item \textbf{Fußgängerüberwege mit Konfliktzonen:} Viele \texttt{highway=crossing}-Elemente erzeugen automatisierte Verbindungen in SUMO, die mit Ampelphasen abgesichert werden müssen – selbst wenn sie im Originalnetz rein logisch sind. Dies bläht die Netzstruktur zusätzlich auf.
    
    \item \textbf{Entfernung kaum möglich:} Ein gezieltes Entfernen nicht-motorisierter Wege führt zu inkonsistenten Junctions, zerrissenen Knotenverbindungen und Netzfragmentierung. Eine selektive Bereinigung dieser Elemente hätte manuell erfolgen müssen – mit hohem Fehlerpotenzial.
\end{itemize}

Diese Phänomene führten letztlich zu der Entscheidung, auf eine synthetische Umgebung umzusteigen, in der alle Verkehrsteilnehmer und Verkehrsbeziehungen explizit und gezielt modelliert werden können.


\subsubsection{Eingesetzte \texttt{netconvert}-Optionen und deren Grenzen}

Im Rahmen der Netzgenerierung wurden zahlreiche Optionen des SUMO-Tools \texttt{netconvert} genutzt, um die aus OSM exportierten Daten zu verbessern und automatisiert für die Simulation aufzubereiten. Dabei kamen insbesondere folgende Konvertierungsoptionen zum Einsatz:

\begin{itemize}
    \item \texttt{--tls.guess-signals}: Automatische Erkennung und Erzeugung von Ampelanlagen basierend auf der Netzstruktur und OSM-Tags. Zwar werden damit grundlegend steuerbare TLS erstellt, jedoch mit oft unrealistischen Phasenkombinationen oder lediglich einer einzigen Phase – was im Kontext von \texttt{sumo-rl} unbrauchbar ist.

    \item \texttt{--tls.join}: Versucht, benachbarte Ampelanlagen zu einer gemeinsamen Steuerungseinheit zu verschmelzen. Dies führte in der Praxis zu unübersichtlichen und kaum nachvollziehbaren TLS-Clustern mit dutzenden Signalgruppen, die sich weder debuggen noch sinnvoll manuell reparieren ließen.

    \item \texttt{--junctions.join}: Fügt benachbarte Knotenpunkte zusammen, um komplexe Kreuzungen zu vereinfachen. In der Umsetzung erzeugt diese Option allerdings häufig übermäßig große Junction-Cluster mit einer Vielzahl von Ein- und Ausfahrten. Die daraus resultierenden Junctions waren nicht mehr sinnvoll steuerbar.

    \item \texttt{--ramps.guess}: Wird bei Autobahnimporten verwendet, um Ein- und Ausfahrten korrekt zu erkennen. In urbanen Netzen wie dem Karlsruher Modell zeigte diese Option jedoch keine spürbare Verbesserung oder sogar fehlerhafte Zuweisungen von Abbiegebeziehungen.

    \item \texttt{--remove-edges.isolated} und \texttt{--keep-edges.by-vclass}: Weitere Optionen zur Bereinigung überflüssiger oder nicht-kfz-tauglicher Straßenabschnitte zeigten nur begrenzte Wirkung, da viele Fuß- und Radverbindungen dennoch als relevante Kanten im Netz verblieben.

    \item \texttt{--discard-simple}: Entfernt automatisch alle einfachen Kreuzungen (d.\,h. Knoten mit nur zwei verbundenen Kanten) aus dem Netz. Ziel ist es, das Netz zu vereinfachen und unnötige Zwischenknoten zu reduzieren. In der Praxis zeigte sich jedoch, dass durch diese Option auch technisch relevante Einmündungen und Übergänge entfernt wurden, was zu fehlerhaften Verbindungen oder fehlenden Ampelanlagen führte. Darüber hinaus erschwert die Option eine saubere Reproduktion der realen Verkehrsführung, da sie die Netztopologie unkontrollierbar verändert.

\end{itemize}

Obwohl diese Optionen teilweise strukturelle Verbesserungen erzielen konnten, blieben die generierten Netze in ihrer Gesamtheit unzuverlässig: entweder zu komplex, unvollständig, inkonsistent oder inkompatibel mit dem Reinforcement-Learning-Framework. Auch die Kombination mehrerer Optionen führte oft zu unvorhersehbaren Seiteneffekten – etwa widersprüchlichen TLS-Zuständen oder unvollständig verbundenen Junctions. 

Die automatisierten Heuristiken von \texttt{netconvert} sind auf generische OSM-Netze ausgelegt, nicht jedoch auf die hohen Anforderungen an Konsistenz, Steuerbarkeit und Reproduzierbarkeit, wie sie ein RL-basiertes Trainingsumfeld erfordert.

\subsubsection{Grenzen der OSM-basierten Netzmodellierung}

Trotz erheblicher Anstrengungen bei der Bereinigung und Validierung des OSM-basierten Karlsruher Verkehrsnetzes erwiesen sich einige strukturelle Einschränkungen als nicht zuverlässig behebbar. Insbesondere die starke Präsenz und topologische Einbindung von Fuß- und Radverkehrswegen führte zu nachhaltigen Problemen im Modell:

\begin{itemize}
    \item \textbf{Übermäßige Knotenvernetzung:} Fuß- und Radwege erzeugen eine Vielzahl zusätzlicher Knoten und Verbindungen, die beim Import aus OSM oft zu komplex verschachtelten Kreuzungsclustern führen. Diese lassen sich weder zuverlässig entwirren noch stabil mit funktionalen TLS versehen.
    
    \item \textbf{Unkontrollierbare Verbindungsstruktur:} Beim Versuch, Fuß- und Radverbindungen selektiv aus dem Netz zu entfernen, entstehen häufig inkonsistente oder gar fehlerhafte Verbindungen zwischen verbleibenden Fahrbahnen. Dies führt zu nicht steuerbaren Junctions, überzähligen Abbiegebeziehungen und unbrauchbaren Ampelschaltungen.

    \item \textbf{Unvereinbarkeit mit RL-Steuerung:} Selbst nach umfangreicher Reparatur verbleiben viele Knoten, die strukturell oder logisch nicht mit \texttt{sumo-rl} kompatibel sind. Die dafür nötigen manuellen Eingriffe überschreiten den praktikablen Aufwand für ein robustes, reproduzierbares Modell.
\end{itemize}

\subsubsection{Übergang zu einem synthetischen Verkehrsnetz}

Nach mehreren Iterationszyklen wurde daher entschieden, den Fokus dieser Arbeit auf ein \textbf{synthetisch generiertes Verkehrsnetz} zu verlagern. Dieses basiert nicht auf realen Geodaten, sondern wird gezielt so konstruiert, dass es die Anforderungen an eine kontrollierte Umgebung für das Training von Reinforcement-Learning-Agenten erfüllt:

\begin{itemize}
    \item \textbf{Reduzierte Komplexität bei hoher Steuerbarkeit:} Das generierte Netz vermeidet absichtlich komplexe Kreuzungstopologien und erlaubt gezielte Definition steuerbarer TLS.
    
    \item \textbf{Fehlerresistenz und Modularität:} Durch vollständige Kontrolle über Struktur, Phasenpläne und Verkehrsverteilung ist eine konsistente und wiederholbare Simulation möglich.
    
    \item \textbf{RL-geeignete Gestaltung:} Alle im Netz enthaltenen Kreuzungen sind mit mindestens zwei sinnvollen Ampelphasen modelliert, beinhalten konsistente Signaldefinitionen und erfüllen die technischen Anforderungen von \texttt{sumo-rl}.
\end{itemize}

Der Einsatz des synthetischen Netzes ermöglicht es, sich auf die eigentliche Forschungsfrage – die Optimierung von Ampelsteuerung durch RL – zu konzentrieren, ohne durch externe Störfaktoren behindert zu werden. Die im vorigen Abschnitt dokumentierten Schritte bleiben dennoch ein zentraler Bestandteil dieser Arbeit, da sie den praktischen Aufwand und die Limitationen realer OSM-Netze in SUMO transparent machen.

\subsection{Netzprüfung, Reparatur und Toolchain}
Aufgrund der oben beschriebenen strukturellen Schwächen im importierten OSM-Netz war eine manuelle Nachbearbeitung ineffizient und fehleranfällig. Daher wurden eigene Werkzeuge entwickelt, um eine systematische und automatisierte Reparatur zu ermöglichen.


\subsubsection{Werkzeuge zur Netzprüfung und Reparatur}

Um die Kompatibilität des aus OpenStreetMap abgeleiteten Verkehrsnetzes mit \texttt{sumo-rl} sicherzustellen, wurde eine Reihe eigenentwickelter Python-Skripte implementiert. Diese Werkzeuge automatisieren die Analyse, Validierung und Korrektur der Netzstruktur mit Fokus auf Lichtsignalanlagen (TLS). Der modulare Aufbau erlaubt es, problematische Netzbestandteile zu identifizieren und gezielt zu bereinigen.

\paragraph{Prüfung der Signalverknüpfungen und Zustandslängen}

Zwei zentrale Tools wurden entwickelt, um die Konsistenz zwischen kontrollierten Verbindungen (\textit{controlled links}) und Phasenzuständen (\texttt{state}) der TLS zu überprüfen:

\begin{itemize}
    \item \textbf{\texttt{check\_tls\_consistency.py}} prüft, ob die Länge jedes \texttt{state}-Strings in den \texttt{<phase>}-Elementen exakt der Anzahl der gesteuerten Signalindizes entspricht. Abweichungen werden detailliert gelistet, inklusive betroffener Phase und TLS-ID.
        
        \begin{algorithm}[H]
        \caption{CheckTLSLengths – Prüfung inkonsistenter Phasenlängen}
        \begin{algorithmic}[1]
        \Function{CheckTLSLengths}{net.xml}
          \State Lade XML-Baum und extrahiere \texttt{<connection>}-Elemente
          \State Erstelle Dictionary \texttt{tls\_controlled\_links} mit Anzahl gesteuerter Links pro TLS
          \ForAll{\texttt{tlLogic}-Elemente im Netz}
            \State \texttt{expectedLen} $\gets$ Anzahl \texttt{controlledLinks} aus Dictionary
            \If{\texttt{expectedLen} = 0}
              \State Gib Warnung: TLS hat keine gesteuerten Verbindungen
              \State \textbf{continue}
            \EndIf
            \ForAll{Phasen $i$ in \texttt{tlLogic}}
              \State \texttt{actualLen} $\gets$ Länge des \texttt{state}-Strings
              \If{\texttt{actualLen} $\neq$ \texttt{expectedLen}}
                \State Gib Warnung mit TLS-ID, Phase und \texttt{state}-Inhalt aus
              \EndIf
            \EndFor
          \EndFor
          \If{keine Abweichungen gefunden}
            \State Gib Erfolgsmeldung aus
          \EndIf
        \EndFunction
        \end{algorithmic}
        \end{algorithm}
        
        \newpage
\noindent
\textbf{Originalcode in Python (\texttt{check\_tls\_consistency.py}):}

\begin{minted}[fontsize=\small, linenos, frame=lines, breaklines, tabsize=4]{python}
import xml.etree.ElementTree as ET

# === Konfiguration ===
net_file = "karlsruhe.net.xml"

# === Einlesen ===
tree = ET.parse(net_file)
root = tree.getroot()

# === Alle controlledLinks zählen ===
tls_controlled_links = {}
for connection in root.findall("connection"):
    if "tl" in connection.attrib and "linkIndex" in connection.attrib:
        tls_id = connection.attrib["tl"]
        tls_controlled_links.setdefault(tls_id, set()).add(int(connection.attrib["linkIndex"]))

# === Alle Phasen prüfen ===
def check_tls_lengths():
    print("Überprüfe alle TLS auf inkonsistente Phasenlängen...\n")
    any_issues = False
    for logic in root.findall("tlLogic"):
        tls_id = logic.attrib["id"]
        expected_len = len(tls_controlled_links.get(tls_id, []))

        if expected_len == 0:
            print(f" TLS '{tls_id}' hat keine controlledLinks (wird evtl. nicht gesteuert)")
            continue

        for i, phase in enumerate(logic.findall("phase")):
            actual_len = len(phase.attrib["state"])
            if actual_len != expected_len:
                print(f" Phase {i} von TLS '{tls_id}' hat Länge {actual_len}, erwartet: {expected_len}")
                print(f"    → state=\"{phase.attrib['state']}\"")
                any_issues = True

    if not any_issues:
        print(" Alle TLS-Phasen stimmen mit ihren controlledLinks überein!")

check_tls_lengths()
\end{minted}

    \newpage
    \item \textbf{\texttt{check\_tls\_requests.py}} validiert, ob alle \texttt{<request>}-Indizes innerhalb zulässiger Grenzen liegen. Falsch verknüpfte Einträge – z.\,B. \texttt{index > max(signalIndex)} – werden gemeldet.

    \begin{algorithm}[H]
    \caption{CheckTLSRequests – Prüfung ungültiger \texttt{request}-Indizes}
    \begin{algorithmic}[1]
    \Function{CheckTLSRequests}{net.xml}
      \State Lade XML-Datei und parse Wurzelknoten
      \State Erzeuge Dictionary \texttt{tls\_signal\_indices} mit Signalindizes je TLS aus \texttt{<connection>}-Elementen
      \ForAll{\texttt{junction}-Elemente im Netz}
        \State \texttt{tls\_id} $\gets$ ID der Junction
        \If{\texttt{tls\_id} in \texttt{tls\_signal\_indices}}
          \State \texttt{expected\_max} $\gets$ Länge der Signalindizes für dieses TLS
          \ForAll{\texttt{request}-Elemente in Junction}
            \State \texttt{index} $\gets$ Wert des \texttt{index}-Attributs
            \If{\texttt{index} $\geq$ \texttt{expected\_max}}
              \State Gib Warnung mit \texttt{tls\_id} und \texttt{index} aus
            \EndIf
          \EndFor
        \EndIf
      \EndFor
      \If{keine Warnungen ausgegeben}
        \State Gib Erfolgsmeldung aus
      \EndIf
    \EndFunction
    \end{algorithmic}
    \end{algorithm}


\newpage
\noindent
\textbf{Originalcode in Python (\texttt{check\_tls\_requests.py}):}
\begin{minted}[fontsize=\small, linenos, frame=lines, breaklines, tabsize=4]{python}
import xml.etree.ElementTree as ET

net_file = "karlsruhe.net.xml"
tree = ET.parse(net_file)
root = tree.getroot()

# Zähle für jedes TLS wie viele signal indices es gibt (controlled links)
tls_signal_indices = {}
for conn in root.findall("connection"):
    if "tl" in conn.attrib and "linkIndex" in conn.attrib:
        tls_id = conn.attrib["tl"]
        tls_signal_indices.setdefault(tls_id, set()).add(int(conn.attrib["linkIndex"]))

# Vergleiche mit den request-Elementen
print("Überprüfe request-Indizes gegen Signalindizes...\n")
any_issues = False
for junction in root.findall("junction"):
    tls_id = junction.attrib.get("id")
    requests = junction.findall("request")
    if tls_id in tls_signal_indices:
        expected_max = len(tls_signal_indices[tls_id])
        for req in requests:
            index = int(req.attrib["index"])
            if index >= expected_max:
                print(f"Junction '{tls_id}': request index {index} > max signal index {expected_max - 1}")
                any_issues = True

if not any_issues:
    print("Alle request-Indizes passen zu den TLS-Signalindizes!")
\end{minted}
\end{itemize}

\newpage
\paragraph{Automatische Reparaturwerkzeuge}

Die folgenden Programme wurden zur strukturellen Korrektur entwickelt:

\begin{itemize}
    \item \textbf{\texttt{fix\_requests.py}} entfernt überzählige \texttt{<request>}-Einträge und kürzt \texttt{state}-Strings in Phasen auf die zulässige Länge. Die Bereinigung erfolgt anhand der tatsächlichen Anzahl gesteuerter Signalverbindungen (\texttt{linkIndex}).

    \begin{algorithm}[H]
\caption{FixRequests – Bereinigung ungültiger \texttt{<request>}-Einträge und Anpassung der Phasen}
\begin{algorithmic}[1]
\Function{FixRequests}{net.xml}
  \State Lade XML-Baum mit Netzstruktur
  \State Initialisiere Dictionary \texttt{tls\_max\_index} für maximale Signalindices
  \ForAll{\texttt{connection}-Elemente}
    \If{TLS-ID und \texttt{linkIndex} vorhanden}
      \State Aktualisiere \texttt{tls\_max\_index[tl]} mit höchstem Index
    \EndIf
  \EndFor

  \ForAll{\texttt{junction}-Elemente}
    \State Hole TLS-ID
    \If{TLS nicht in \texttt{tls\_max\_index}}
      \State \textbf{continue}
    \EndIf
    \State Bestimme erlaubten Maximalindex (\texttt{max\_idx})
    \ForAll{\texttt{request}-Einträge}
      \If{Index $>$ \texttt{max\_idx}}
        \State Entferne ungültigen \texttt{request}
      \EndIf
    \EndFor

    \ForAll{\texttt{tlLogic}-Elemente mit passender TLS-ID}
      \ForAll{Phasen}
        \If{\texttt{state}-String ist zu lang}
          \State Kürze \texttt{state} auf \texttt{max\_idx + 1}
        \EndIf
      \EndFor
    \EndFor
  \EndFor

  \State Speichere modifizierte XML-Datei
  \State Gib Statistiken zu entfernten Requests und angepassten Phasen aus
\EndFunction
\end{algorithmic}
\end{algorithm}

\newpage
\noindent
\textbf{Originalcode in Python (\texttt{fix\_requests.py}):}
\begin{minted}[fontsize=\small, linenos, frame=lines, breaklines, tabsize=4]{python}
import xml.etree.ElementTree as ET

net_file = "karlsruhe.net.xml"
output_file = "karlsruhe_fixed_tls.net.xml"

tree = ET.parse(net_file)
root = tree.getroot()

# Finde maximal verwendete Signal-Indices pro TLS
tls_max_index = {}
for conn in root.findall("connection"):
    tl = conn.get("tl")
    idx = conn.get("linkIndex")
    if tl and idx:
        idx = int(idx)
        tls_max_index[tl] = max(tls_max_index.get(tl, -1), idx)

# Bereinigung
total_removed_requests = 0
total_adjusted_phases = 0
changed_tls = []

for junction in root.findall("junction"):
    tls_id = junction.get("id")
    if tls_id not in tls_max_index:
        continue

    max_idx = tls_max_index[tls_id]
    requests = list(junction.findall("request"))
    removed = 0

    for req in requests:
        req_idx = int(req.get("index"))
        if req_idx > max_idx:
            junction.remove(req)
            removed += 1

    if removed > 0:
        print(f"TLS '{tls_id}': {removed} ungültige <request>-Einträge entfernt.")
        total_removed_requests += removed
        changed_tls.append(tls_id)

    # Kürze zugehörige Phasen
    for tl in root.findall("tlLogic"):
        if tl.get("id") == tls_id:
            adjusted = 0
            for phase in tl.findall("phase"):
                state = phase.get("state")
                if len(state) > max_idx + 1:
                    old_len = len(state)
                    phase.set("state", state[:max_idx + 1])
                    adjusted += 1
            if adjusted > 0:
                print(f" TLS '{tls_id}': {adjusted} <phase>-Strings auf Länge {max_idx + 1} gekürzt.")
                total_adjusted_phases += adjusted
                if tls_id not in changed_tls:
                    changed_tls.append(tls_id)

# Speichern
tree.write(output_file, encoding="utf-8")
print("\n Reparatur abgeschlossen.")
print(f" Gesamt entfernte <request>-Einträge: {total_removed_requests}")
print(f" Gesamt angepasste <phase>-Einträge: {total_adjusted_phases}")
print(f" Betroffene TLS-IDs: {len(changed_tls)} Stück")
for tls in changed_tls:
    print(f"  - {tls}")
print(f"\n Bereinigte Datei gespeichert unter: {output_file}")
\end{minted}
\vspace{1cm}
    \item \textbf{\texttt{repair-net.py}} nutzt ein manuell gepflegtes Dictionary mit TLS-IDs und deren erwarteter Phasenlänge (Anzahl kontrollierter Verbindungen). Alle Phasen, deren Länge abweicht, werden automatisch gekürzt oder aufgefüllt.

    \begin{algorithm}[H]
\caption{RepairTLSStates – Korrektur der Phasenlängen anhand manuell gepflegter Referenz}
\begin{algorithmic}[1]
\Function{RepairTLSStates}{net.xml, referenz\_dictionary}
  \State Lade Netzstruktur aus \texttt{net.xml}
  \ForAll{\texttt{tlLogic}-Elemente im Netz}
    \State \texttt{tls\_id} $\gets$ ID des Ampelknotens
    \If{\texttt{tls\_id} nicht in referenz\_dictionary}
      \State \textbf{continue}
    \EndIf
    \State \texttt{correctLen} $\gets$ erwartete Zustandslänge aus Referenz
    \ForAll{Phasen des Knotens}
      \State \texttt{state} $\gets$ Zeichenkette der Phase
      \If{Länge(\texttt{state}) $\neq$ \texttt{correctLen}}
        \State Kürze oder ergänze \texttt{state} auf \texttt{correctLen}
        \State Markiere Netz als geändert
      \EndIf
    \EndFor
  \EndFor
  \If{Netz wurde geändert}
    \State Speichere bereinigte Netzdatei als \texttt{karlsruhe\_fixed.net.xml}
  \Else
    \State Gib Hinweis: Alle Phasen bereits korrekt
  \EndIf
\EndFunction
\end{algorithmic}
\end{algorithm}


\newpage
\noindent
\textbf{Originalcode in Python (\texttt{repair\_net.py}):}
\begin{minted}[fontsize=\small, linenos, frame=lines, breaklines, tabsize=4]{python}
from xml.etree import ElementTree as ET

# Manuell gepflegte Dictionary mit {TLS-ID: Anzahl controlledLinks}
controlled_links = {
    "1720933516": 6,
    "3538953167": 2,
    "3664415977": 10,
    "cluster_14795187_1720919996_2670370290_2670370291": 11,
    "cluster_14795804_55474925_6655074904_765746891_#1more": 49,
    "cluster_15431428_1719671850_1720917935": 20,
    "cluster_1590912233_3664415976_5083348337_5083348350": 11,
    "cluster_1692973685_1692973722_1718084055_1718084058_#11more": 36,
    "cluster_1729190097_3687504105": 8,
    "cluster_1744031943_5131521735": 10,
    "joinedS_1623835169_cluster_1137679587_1626739216_1728272870_1728272909_#17more": 33,
    "joinedS_309108716_cluster_11001804363_1125509937_12515596172_1784859792_#5more": 14,
    "joinedS_5092985445_cluster_1590912226_2911376263": 10,
    # ggf. mehr hinzufügen
}

tree = ET.parse("karlsruhe.net.xml")
root = tree.getroot()
changed = False

for logic in root.findall("tlLogic"):
    tl_id = logic.attrib["id"]
    if tl_id not in controlled_links:
        continue

    correct_len = controlled_links[tl_id]
    for phase in logic.findall("phase"):
        state = phase.attrib["state"]
        if len(state) != correct_len:
            new_state = state[:correct_len].ljust(correct_len, 'r')
            print(f" Fixing {tl_id}: {len(state)} → {correct_len}")
            phase.attrib["state"] = new_state
            changed = True

if changed:
    tree.write("karlsruhe_fixed.net.xml")
    print(" Bereinigte Datei gespeichert: karlsruhe_fixed.net.xml")
else:
    print(" Alle Phasen bereits korrekt.")

\end{minted}

    \newpage
    \item \textbf{\texttt{statecheck.py}} gibt eine Liste aller TLS-Phasen mit ungewöhnlichen Längen aus. Dieses Tool wurde verwendet, um bei vereinheitlichten Netzen auf eine Ziel-Zustandslänge zu prüfen.

    \begin{algorithm}[H]
\caption{StateCheck – Prüfung auf einheitliche Phasenlängen}
\begin{algorithmic}[1]
\Function{StateCheck}{net.xml}
  \State Lade XML-Baum aus der Netzdatei
  \ForAll{\texttt{tlLogic}-Elemente im Netz}
    \State \texttt{tl\_id} $\gets$ ID des aktuellen TLS
    \ForAll{Phasen $i$ in \texttt{tlLogic}}
      \State \texttt{state} $\gets$ Zustand der Phase
      \If{\texttt{len(state)} $\neq$ 57}
        \State Gib Warnung mit \texttt{tl\_id}, Phasenindex und tatsächlicher Länge aus
      \EndIf
    \EndFor
  \EndFor
\EndFunction
\end{algorithmic}
\end{algorithm}

\noindent
\textbf{Originalcode in Python (\texttt{statecheck.py}):}
\begin{minted}[fontsize=\small, linenos, frame=lines, breaklines, tabsize=4]{python}
from xml.etree import ElementTree as ET

tree = ET.parse("karlsruhe.net.xml")
root = tree.getroot()

for logic in root.findall("tlLogic"):
    tl_id = logic.attrib["id"]
    for i, phase in enumerate(logic.findall("phase")):
        state = phase.attrib["state"]
        if len(state) != 57:
            print(f" Phase {i} of TLS '{tl_id}' has length {len(state)}")
\end{minted}

\end{itemize}

\newpage
\paragraph{Gültigkeitsprüfung für SUMO-RL}

Zur Vorbereitung des Trainings wurden weitere Programme zur Identifikation funktionaler TLS entwickelt:

\begin{itemize}
    \item \textbf{\texttt{find\_valid\_tls.py}} iteriert über alle TLS im Netz und testet jede einzeln in einem minimalen \texttt{sumo-rl}-Lauf. TLS, bei denen die Umgebung erfolgreich initialisiert werden kann, gelten als kompatibel.

    \begin{algorithm}[H]
\caption{FindValidTLS – Gültigkeitsprüfung aller TLS im Netz}
\begin{algorithmic}[1]
\Function{TestTLS}{\texttt{tls\_id}}
  \State Initialisiere \texttt{SumoEnvironment}
  \State Setze \texttt{ts\_ids} auf \texttt{[tls\_id]}
  \State Versuche: \texttt{env.reset()}
  \If{kein Fehler}
    \State \texttt{env.close()}
    \State \Return \texttt{True}
  \Else
    \State Gib Fehlermeldung aus
    \State \Return \texttt{False}
  \EndIf
\EndFunction
\vspace{0.5em}
\State Initialisiere leere Liste \texttt{all\_tls}
\State Versuche: Umgebung mit \texttt{SumoEnvironment} zu starten
\If{erfolgreich}
  \State Lese alle \texttt{ts\_ids}
  \State Schließe Umgebung
\Else
  \State Gib Fehler aus
\EndIf
\vspace{0.5em}
\State Initialisiere leere Liste \texttt{valid\_tls}
\ForAll{\texttt{tls\_id} in \texttt{all\_tls}}
  \If{ \Call{TestTLS}{\texttt{tls\_id}} }
    \State Füge \texttt{tls\_id} zu \texttt{valid\_tls} hinzu
  \EndIf
\EndFor
\State Gib alle gültigen TLS aus
\end{algorithmic}
\end{algorithm}

\newpage
\noindent
\textbf{Originalcode in Python (\texttt{find\_valid\_tls.py}):}
\begin{minted}[fontsize=\small, linenos, frame=lines, breaklines, tabsize=4]{python}
from sumo_rl import SumoEnvironment
import traci
import os

def test_tls(tls_id):
    try:
        env = SumoEnvironment(
            net_file="karlsruhe.net.xml",
            route_file="karlsruhe.rou.xml",
            use_gui=False,
            single_agent=True
        )
        env.ts_ids = [tls_id]
        env.reset()
        env.close()
        return True
    except Exception as e:
        print(f" TLS {tls_id} nicht gültig: {e}")
        return False

# Alle TLS holen
try:
    env = SumoEnvironment(
        net_file="karlsruhe.net.xml",
        route_file="karlsruhe.rou.xml",
        use_gui=False,
        single_agent=True
    )
    all_tls = env.ts_ids
    env.close()
except Exception as e:
    print(" Konnte TLS nicht auslesen:", e)
    all_tls = []

print(f" Teste {len(all_tls)} TLS auf Gültigkeit...\n")
valid_tls = []

for tls_id in all_tls:
    if test_tls(tls_id):
        valid_tls.append(tls_id)

print("\n Gültige TLS:")
print(valid_tls)
\end{minted}
    
\end{itemize}

\subsubsection{Auswahl eines bereinigten Netzes}

Nach mehrfacher Iteration und Debugging wurde ein final bereinigtes Netz erzeugt: \texttt{karlsruhe.net.xml}. Dieses enthält ausschließlich überprüfte TLS mit konsistenten Phasenlängen und steuerbaren Verbindungen. Es bildet die Grundlage für alle nachfolgenden Reinforcement-Learning-Experimente.

\subsubsection{Vorteil des automatisierten Workflows}

Die entwickelte Toolchain ermöglicht:

\begin{itemize}
    \item eine strukturierte Diagnose typischer OSM-bedingter Netzprobleme,
    \item reproduzierbare Netzreparaturen ohne manuelles Editieren in \texttt{netedit},
    \item automatisierte Validierung vor dem Training in \texttt{sumo-rl},
    \item gezielte Selektion steuerbarer TLS für das Experiment.
\end{itemize}

Der Einsatz dieser Werkzeuge war unerlässlich, um ein funktionales, kompatibles und robusteres Simulationsnetz auf Basis realer OSM-Daten zu etablieren.
\subsubsection{Einbindung des SUMO-Netzes in die RL-Umgebung}

Nach Bereinigung, struktureller Prüfung und gezielter Auswahl geeigneter Lichtsignalanlagen (TLS) wurde das resultierende SUMO-Verkehrsnetz in die Reinforcement-Learning-Umgebung \texttt{sumo-rl} integriert. Die Umgebung basiert auf dem Open-Source-Projekt \texttt{sumo-rl}\footnote{\url{https://lucasalegre.github.io/sumo-rl/}}, das eine Schnittstelle zwischen dem Verkehrssimulator SUMO und modernen RL-Frameworks bietet.

Ziel war es, eine stabil lauffähige Multiagentensimulation zu realisieren, bei der mehrere Ampelknoten unabhängig voneinander durch separate Agenten gesteuert werden. \texttt{sumo-rl} nutzt das \texttt{TraCI}-Protokoll, um die Steuerung einzelner TLS-Knoten durch externe Agenten in Echtzeit zu ermöglichen.

\paragraph{Filterung und Vorbereitung der TLS}

Im Vorfeld der Einbindung wurde eine automatisierte Vorverarbeitung durchgeführt, um nur technisch einwandfreie TLS-Knoten zu berücksichtigen. Dabei wurden folgende Kriterien geprüft:

\begin{itemize}
    \item \textbf{Knotentyp:} Nur TLS vom Typ \texttt{traffic\_light} wurden berücksichtigt.
    \item \textbf{Phasenstruktur:} Jeder Knoten musste mindestens zwei steuerbare \texttt{<phase>}-Einträge enthalten.
    \item \textbf{Signaldefinition:} Alle Phasen mussten syntaktisch korrekt und vollständig sein (korrekte Anzahl an Zeichen, gültige Zustände, keine verweisten Connections).
\end{itemize}

Die Überprüfung erfolgte über das Python-Modul \texttt{sumolib} in Kombination mit XML-Parsing mittels \texttt{ElementTree}. Fehlerhafte oder unvollständige TLS wurden automatisch ausgeschlossen. Die so bereinigte Liste \texttt{valid\_tls\_ids} bildet die Grundlage für die nachfolgende RL-Umgebung.

\paragraph{Konfiguration der \texttt{sumo-rl} Umgebung}

Die zentrale Steuereinheit ist das Objekt \texttt{SumoEnvironment}, das aus der \texttt{sumo-rl}-Bibliothek importiert wird. Für den Multiagentenbetrieb wird der Parameter \texttt{single\_agent=False} gesetzt. Die wichtigsten Parameter sind:

\begin{itemize}
    \item \texttt{net\_file}: Pfad zur bereinigten SUMO-Netzdatei (\texttt{.net.xml}),
    \item \texttt{route\_file}: Pfad zur Datei mit generierten Fahrzeugrouten (\texttt{.rou.xml}),
    \item \texttt{ts\_ids}: Liste gültiger TLS-Knoten zur Steuerung,
    \item \texttt{use\_gui}: Aktiviert das SUMO-GUI zur Laufzeitbeobachtung,
    \item \texttt{delta\_time}: Zeitabstand zwischen zwei Agentenentscheidungen (z.\,B. 5\,s),
    \item \texttt{yellow\_time}: Dauer der Gelbphase bei Phasenwechsel,
    \item \texttt{min\_green}: Mindestgrünzeit einer Phase vor erneutem Wechsel,
    \item \texttt{reward\_fn}: Belohnungsfunktion (z.\,B. \texttt{"diff-waiting-time"}),
    \item \texttt{fixed\_ts}: Wenn \texttt{True}, werden Aktionen ignoriert und feste Signalpläne genutzt.
\end{itemize}

\begin{minted}[fontsize=\small, linenos, frame=lines, breaklines, tabsize=4]{python}
env = SumoEnvironment(
    net_file="network.net.xml",
    route_file="routes.rou.xml",
    use_gui=True,
    single_agent=False,
    reward_fn="diff-waiting-time",
    delta_time=5,
    yellow_time=2,
    min_green=5,
    fixed_ts=False,
    ts_ids=valid_tls_ids
)
\end{minted}

\paragraph{Parallele Steuerung im Multiagentensystem}

Im Multiagentenbetrieb erzeugt \texttt{sumo-rl} intern ein Mapping von TLS-IDs auf einzelne Agenteninstanzen. Jeder Agent erhält eine isolierte Beobachtung, trifft lokal eine Entscheidung und erhält eine eigene Belohnung. Die Entscheidungsfindung erfolgt synchron: Alle Agenten geben gleichzeitig eine Aktion ab, bevor die SUMO-Simulation fortgesetzt wird.

\begin{minted}[fontsize=\small, linenos, frame=lines, breaklines, tabsize=4]{python}
obs = env.reset()
done = False
while not done:
    actions = {tls: env.action_space[tls].sample() for tls in env.ts_ids}
    obs, rewards, done, _, infos = env.step(actions)
\end{minted}

\paragraph{Beobachtungen und Aktionen}

Standardmäßig umfasst die Beobachtung eines TLS-Agenten:

\begin{itemize}
    \item die aktuelle Phase (als One-Hot-Encoding),
    \item die Zeit seit der letzten Phase,
    \item für jede anliegende Spur: normierte Dichte und Fahrzeugwarteschlange.
\end{itemize}

Der Action-Space ist diskret und erlaubt das Umschalten zwischen vordefinierten Phasen. Ein Wechsel löst automatisch die Gelbphase (\texttt{yellow\_time}) aus, bevor die neue Grünphase aktiv wird. Optional können auch eigene Beobachtungsklassen über \texttt{observation\_class=} übergeben werden.

\paragraph{Verfügbare Belohnungsfunktionen}

\texttt{sumo-rl} unterstützt eine Reihe vordefinierter Belohnungsfunktionen, z.\,B.:

\begin{itemize}
    \item \texttt{"diff-waiting-time"}: Reduktion der kumulierten Wartezeit (Standard),
    \item \texttt{"average-speed"}: Maximierung der Durchschnittsgeschwindigkeit,
    \item \texttt{"queue"}: Minimierung der Gesamtlänge aller Fahrzeugwarteschlangen,
    \item benutzerdefiniert: Übergabe eigener Reward-Funktionen möglich.
\end{itemize}

Diese Funktionen lassen sich über \texttt{reward\_fn} und optional über \texttt{reward\_weights} kombinieren. Dadurch können beispielsweise komplexe Metriken mit mehreren Zielen (z.\,B. Fahrzeit + Emissionen) implementiert werden.

\paragraph{Kompatibilität mit RL-Frameworks}

Die Umgebung ist vollständig kompatibel mit gängigen RL-Bibliotheken wie:

\begin{itemize}
    \item \textbf{Stable-Baselines3}: Direkt einsetzbar für PPO, DQN, A2C etc.
    \item \textbf{Gymnasium}: Standardisiertes Interface (\texttt{env.step()}, \texttt{env.reset()}).
    \item \textbf{PettingZoo}: Unterstützung für koordinierte oder unabhängige Multiagentenszenarien.
    \item \textbf{SuperSuit}: Wrapper zur Vereinheitlichung, Padding, Flattening etc.
\end{itemize}

\paragraph{Zusammenfassung}

Mit der Konfiguration von \texttt{sumo-rl} wurde eine robuste, modular erweiterbare RL-Trainingsumgebung geschaffen, in der mehrere Ampelknoten unabhängig voneinander agieren. Die durchgeführte TLS-Filterung sowie die explizite Kontrolle über Beobachtung, Belohnung und Aktionsauswahl ermöglichen gezielte Experimente zur lernbasierten Optimierung von Verkehrssteuerung unter realitätsnahen Bedingungen.


\subsection{Trainingsstrategie und Hyperparameterwahl}


\section{Evaluation und Ergebnisse}
\subsection{Vergleichsszenarien}
\subsection{Leistungsmetriken}
\subsection{Simulationsergebnisse}
\subsection{Interpretation und Diskussion der Ergebnisse}

\section{Herausforderungen und Limitationen}
\subsection{Technische und methodische Hürden}
\subsection{Repräsentativität und Qualität der Daten}
\subsection{Generalisierbarkeit der Ergebnisse}

\section{Fazit und Ausblick}
\subsection{Zusammenfassung der wichtigsten Erkenntnisse}
\subsection{Mögliche Weiterentwicklungen}
\subsection{Relevanz für reale Verkehrsplanung}

% Anhang
\appendix

Dieser Anhang enthält die vollständigen Python-Skripte, die zur Validierung, Reparatur und Steuerung der SUMO-basierten Reinforcement-Learning-Umgebung eingesetzt wurden. Jedes Unterkapitel dokumentiert ein spezifisches Tool oder Modul aus dem Projekt.

\section{Trainings-Skripte}

\subsection{\texttt{train.py} – Trainingsskript für PPO über mehrere Seeds}
\label{app:train_script}
Das folgende Skript enthält die vollständige Trainingslogik für das Reinforcement Learning mit \texttt{sumo-rl} unter Verwendung von \texttt{Stable-Baselines3}.
\begin{minted}[fontsize=\small, linenos, frame=lines, breaklines, tabsize=4]{python}
# ====== Bibliotheken und Module ======
# Standard-Module für Dateiverwaltung, Zeit und Regex
import os
import re
import time
import datetime
import random

# SUMO-Interface (TraCI) für Simulation
import traci

# Mathematische und numerische Berechnungen
import numpy as np

# PyTorch für neuronale Netze und Reproduzierbarkeit
import torch

# Stable-Baselines3 (RL-Algorithmen, hier PPO)
from stable_baselines3 import PPO
from stable_baselines3.common.vec_env import VecNormalize, VecMonitor
from stable_baselines3.common.callbacks import BaseCallback, CallbackList

# SUMO-RL-Umgebung (PettingZoo-kompatibel)
from sumo_rl.environment.env import parallel_env

# SuperSuit – Hilfsfunktionen, um PettingZoo-Umgebungen mit SB3 zu verwenden
from supersuit import (
    pad_observations_v0,          # Padding für Beobachtungen, um feste Größe zu garantieren
    pad_action_space_v0,          # Padding für Aktionsraum
    pettingzoo_env_to_vec_env_v1, # Konvertierung zu SB3-kompatiblem Vektor-Env
    concat_vec_envs_v1            # Mehrere Envs parallel laufen lassen
)

# Gymnasium für RL-Umgebungs-Schnittstellen
import gymnasium as gym
from gymnasium import Wrapper


# ====== Trainings-Setup ======
SEEDS = [143534, 456, 635768, 13755]  # Verschiedene Zufalls-Seed-Werte für reproduzierbare Runs
ROUTE_FILES = [
    "flows_low.rou.xml",
    "flows_medium.rou.xml",
    "flows_high.rou.xml",
]

# ====== Schedules für Hyperparameter-Anpassung ======
# (Funktionen, die während des Trainings den Wert z. B. von Lernrate oder Clip-Bereich dynamisch anpassen)
def adaptive_entropy_schedule(start=0.01):
    return lambda progress: max(0.001, start * (1 - progress))

def dynamic_clip_range(start=0.2, end=0.1):
    return lambda pr: end + (start - end) * pr

def cosine_clip(start=0.2, end=0.1):
    return lambda pr: end + (start - end) * 0.5 * (1 + np.cos(np.pi * (1 - pr)))

def linear_schedule(start):
    return lambda progress: start * (1 - progress)

def cosine_warmup_floor(start=3e-4, warmup_frac=0.05, min_lr_frac=0.1):
    """
    Lernrate: Erst linear hochfahren (Warmup), dann mit Cosinus auf Minimalwert absenken.
    """
    min_lr = start * min_lr_frac
    warmup_frac = max(0.0, min(0.5, warmup_frac))
    def schedule(progress_remaining: float) -> float:
        t = 1.0 - progress_remaining
        if t < warmup_frac:
            base = start * 0.1 + (start - start * 0.1) * (t / warmup_frac)
        else:
            tt = (t - warmup_frac) / max(1e-8, (1.0 - warmup_frac))
            cos_term = 0.5 * (1 + np.cos(np.pi * tt))
            base = min_lr + (start - min_lr) * cos_term
        return float(base)
    return schedule

# ====== Hilfsfunktionen und Callbacks ======
# (Modelle finden, Checkpoints speichern, Metriken loggen, bestes Modell sichern)
# ====== Letzten vollständigen Run finden ======
def find_latest_complete_run(base_dir="runs", prefix="ppo_sumo_"):
    """
    Sucht im 'runs'-Ordner nach dem neuesten Trainingslauf, der
    - eine gespeicherte VecNormalize-Instanz hat
    - und entweder ein finales Modell oder mindestens einen Checkpoint.
    Gibt die Pfade zu Run-Ordner, Modell und Normalisierungsdatei zurück.
    """
    subdirs = sorted(
        [d for d in os.listdir(base_dir) if d.startswith(prefix)],
        reverse=True
    )
    for d in subdirs:
        dir_path = os.path.join(base_dir, d)
        norm_path = os.path.join(dir_path, "vecnormalize.pkl")
        if not os.path.exists(norm_path):
            continue

        # Prüfe auf finales Modell
        final_model = os.path.join(dir_path, "model.zip")
        if os.path.exists(final_model):
            return dir_path, final_model, norm_path

        # Falls kein finales Modell: Prüfe auf Checkpoints
        checkpoint_models = [
            f for f in os.listdir(dir_path)
            if re.match(r"ppo_sumo_model_(\d+)_steps\.zip", f)
        ]
        if checkpoint_models:
            checkpoint_models.sort(key=lambda x: int(re.findall(r"\d+", x)[0]), reverse=True)
            best_checkpoint = checkpoint_models[0]
            return dir_path, os.path.join(dir_path, best_checkpoint), norm_path

    return None

def make_env(seed, route_files):
    def _init():
        env = parallel_env(
            net_file="map.net.xml",
            route_file=route_files[0],  # Platzhalter
            use_gui=False,
            num_seconds=4096,
            reward_fn="diff-waiting-time",
            min_green=5,
            max_depart_delay=100,
            sumo_seed=seed,
            add_system_info=True,
            add_per_agent_info=False,
        )
        if hasattr(env, "seed"):
            env.seed(seed)

        orig_reset = env.reset
        idx = {"i": -1}  # mutierbares Zähl-Objekt im Closure

        def reset_with_round_robin(**kwargs):
            idx["i"] = (idx["i"] + 1) % len(route_files)
            new_route = route_files[idx["i"]]
            env.route_file = new_route
            if hasattr(env, "sumo_seed"):
                env.sumo_seed = seed
            print(f"\n[DEBUG] Reset → Route: {new_route} | Seed: {seed}\n", flush=True)
            return orig_reset(**kwargs)

        env.reset = reset_with_round_robin
        return env
    return _init


def shorten_key(orig_key: str) -> str:
    return orig_key.replace("system_", "")

# ====== Callback: Zeitbasiertes Speichern ======
class TimeBasedCheckpointCallback(BaseCallback):
    """
    Speichert Modell und Normalisierungsdaten in festen Zeitintervallen (Sekunden).
    """
    def __init__(self, save_interval_sec, save_path, name_prefix="ppo_sumo_model", verbose=0):
        super().__init__(verbose)
        self.save_interval_sec = save_interval_sec
        self.save_path = save_path
        self.name_prefix = name_prefix
        self.last_save_time = time.time()

    def _on_step(self) -> bool:
        return True  # Keine Aktion bei jedem einzelnen Step

    def _on_rollout_end(self) -> bool:
        # Am Ende eines Rollouts prüfen, ob das Zeitintervall abgelaufen ist
        current_time = time.time()
        if current_time - self.last_save_time >= self.save_interval_sec:
            timestep = self.num_timesteps
            filename = f"{self.name_prefix}_{timestep}_steps"
            self.model.save(os.path.join(self.save_path, filename + ".zip"))
            if hasattr(self.training_env, "save"):
                self.training_env.save(os.path.join(self.save_path, f"{filename}_vecnormalize.pkl"))
            print(f"[Checkpoint] Modell gespeichert bei {timestep} Schritten ({filename})")
            self.last_save_time = current_time
        return True


# ====== Callback: Metriken aus der Env loggen ======
class EpisodeMetricsLoggerCallback(BaseCallback):
    def __init__(self, prefix="episode", verbose=0):
        super().__init__(verbose)
        self.prefix = prefix
        self.verbose = verbose
        self.sums = {}
        self.counts = {}
        self.last_totals = {}

    def _on_step(self) -> bool:
        dones = self.locals.get("dones")
        infos = self.locals.get("infos")
        if infos is None:
            return True

        for i, info in enumerate(infos):
            if not isinstance(info, dict):
                continue

            if dones is not None and dones[i]:
                # --- Episode zu Ende ---
                fin = info.get("final_info") or info.get("terminal_info")
                if isinstance(fin, dict):
                    for k, v in fin.items():
                        if not isinstance(v, (int, float)) or not np.isfinite(v):
                            continue
                        if k.startswith("system_mean_"):
                            self.sums[k] = self.sums.get(k, 0.0) + float(v)
                            self.counts[k] = self.counts.get(k, 0) + 1
                        elif k.startswith("system_total_"):
                            self.last_totals[k] = float(v)
            else:
                # --- Nur Zwischenschritt, solange Episode noch läuft ---
                for k, v in info.items():
                    if not isinstance(v, (int, float)) or not np.isfinite(v):
                        continue
                    if k.startswith("system_mean_") or k in [
                        "system_total_waiting_time",
                        "system_total_stopped",
                        "system_total_running",
                    ]:
                        self.sums[k] = self.sums.get(k, 0.0) + float(v)
                        self.counts[k] = self.counts.get(k, 0) + 1
                    elif k.startswith("system_total_"):
                        self.last_totals[k] = float(v)

        # Episode fertig → loggen
        if dones is not None and any(dones):
            for k, total in self.sums.items():
                mean_val = total / max(1, self.counts.get(k, 1))
                short_key = shorten_key(k)
                self.logger.record(f"{self.prefix}/{short_key}", mean_val)
                if self.verbose:
                    print(f"[EpisodeMetrics] {short_key} (mean) = {mean_val:.3f}")

            for k, v in self.last_totals.items():
                short_key = shorten_key(k)
                self.logger.record(f"{self.prefix}/{short_key}", v)
                if self.verbose:
                    print(f"[EpisodeMetrics] {short_key} (total) = {v:.0f}")

            # Reset für nächste Episode
            self.sums.clear()
            self.counts.clear()
            self.last_totals.clear()

        return True

# ====== Callback: Bestes Modell speichern ======
class BestModelSaverCallback(BaseCallback):
    """
    Speichert das Modell mit dem bisher höchsten mittleren Episodenreward.
    """
    def __init__(self, save_path, verbose=0):
        super().__init__(verbose)
        self.best_mean_reward = -float('inf')
        self.save_path = save_path

    def _on_step(self) -> bool:
        return True

    def _on_rollout_end(self):
        ep_info_buffer = self.model.ep_info_buffer
        if len(ep_info_buffer) > 0:
            mean_rew = np.mean([ep_info['r'] for ep_info in ep_info_buffer])
            if mean_rew > self.best_mean_reward:
                self.best_mean_reward = mean_rew
                model_path = os.path.join(self.save_path, "best_model.zip")
                self.model.save(model_path)
                if hasattr(self.model.env, "save"):
                    norm_path = os.path.join(self.save_path, "best_model_vecnormalize.pkl")
                    self.model.env.save(norm_path)
                print(f"[AUTOLOG] Neuer Bestwert {mean_rew:.2f} → best_model gespeichert!", flush=True)


# ====== Haupt-Trainingsschleife ======
for SEED in SEEDS:
    # Reproduzierbarkeit sicherstellen
    np.random.seed(SEED)
    torch.manual_seed(SEED)

    # Log-Verzeichnis erstellen
    now = datetime.datetime.now().strftime("%Y-%m-%d_%H-%M-%S")
    log_dir = os.path.join("runs", f"ppo_sumo_{SEED}_{now}")
    os.makedirs(log_dir, exist_ok=True)

    print(f"\n[INFO] Starte Training mit Seed: {SEED}")

    # SUMO-Umgebung initialisieren
    env = make_env(SEED, ROUTE_FILES)()

    # Falls die Env einen seed()-Aufruf unterstützt
    if hasattr(env, "seed"):
        env.seed(SEED)

    # Anpassung der Beobachtungen und Aktionen an SB3
    env = pad_observations_v0(env)
    env = pad_action_space_v0(env)
    env = pettingzoo_env_to_vec_env_v1(env)

    # WICHTIG: trotzdem concat_vec_envs_v1 mit num_vec_envs=1
    env = concat_vec_envs_v1(env, num_vec_envs=1, num_cpus=1, base_class="stable_baselines3")


    # Logging und Normalisierung
    env = VecMonitor(env, filename=os.path.join(log_dir, "monitor.csv"))
    env = VecNormalize(env, norm_obs=True, norm_reward=True, clip_obs=10.0)

    # PPO-Agent erstellen
    model = PPO(
        policy="MlpPolicy",      # Mehrschicht-Perzeptron-Policy
        env=env,
        verbose=1,               # Ausführliches Logging
        tensorboard_log=log_dir, # TensorBoard-Pfad
        batch_size=256,          # Minibatch-Größe für PPO
        n_steps=2048,            # Rollout-Länge
        learning_rate=cosine_warmup_floor(start=3e-4, warmup_frac=0.05, min_lr_frac=0.1),
        clip_range=cosine_clip(), # Clipping-Range dynamisch
        ent_coef=0.01,            # Entropie-Koeffizient (Exploration)
        gamma=0.99,               # Diskontfaktor
        gae_lambda=0.95,          # Lambda für GAE
        device="cpu",             # Training auf CPU
        policy_kwargs=dict(net_arch=dict(pi=[128, 128], vf=[128, 128])), # Netzarchitekturgit
    )

    # Callback-Liste: Checkpoints, Logging, Best-Model-Speicherung
    callbacks = CallbackList([
        TimeBasedCheckpointCallback(
            save_interval_sec=3600, # Jede Stunde speichern
            save_path=log_dir,
            name_prefix="ppo_sumo_model",
            verbose=1,
        ),
        EpisodeMetricsLoggerCallback(),
        BestModelSaverCallback(save_path=log_dir),
    ])

    # Training starten
    try:
        time.sleep(3) # Kurze Pause für saubere Konsolenlogs
        model.learn(
            total_timesteps=2_000_000,
            callback=callbacks,
        )
        # Nach Abschluss final speichern
        model.save(os.path.join(log_dir, "model.zip"))
        env.save(os.path.join(log_dir, "vecnormalize.pkl"))
        print(f"\n[INFO] Training abgeschlossen für Seed {SEED}. Modell gespeichert unter: {log_dir}")

    # Falls Training manuell abgebrochen wird (Strg+C)
    except KeyboardInterrupt:
        print("[ABBRUCH] Manuelles Beenden erkannt. Speichere aktuellen Stand...")
        model.save(os.path.join(log_dir, "model_interrupt.zip"))
        env.save(os.path.join(log_dir, "vecnormalize_interrupt.pkl"))

    # Generelle Fehlerbehandlung
    except Exception as e:
        print(f"\n[FEHLER] Während des Trainings bei Seed {SEED} aufgetreten: {e}")

    # Cleanup: Env schließen und Normalisierungsdaten sichern
    finally:
        try:
            env.save(os.path.join(log_dir, "vecnormalize.pkl"))
        except Exception as e:
            print(f"[WARNUNG] VecNormalize konnte nicht gespeichert werden: {e}")
        env.close()

\end{minted}

\subsection{\texttt{continuetrain.py} – Trainingsskript zum Weitertrainieren}
\label{app:continuetrain}
Startet für jede einzelne Ampelkreuzung eine Minimalumgebung und überprüft, ob diese in \texttt{sumo-rl} trainierbar ist.
\begin{minted}[fontsize=\small, linenos, frame=lines, breaklines, tabsize=4]{python}
import os
import re
import time
import datetime
import traci
import numpy as np
import torch
import json
from stable_baselines3 import PPO
from stable_baselines3.common.vec_env import VecNormalize, VecMonitor
from stable_baselines3.common.callbacks import BaseCallback, CallbackList
from sumo_rl.environment.env import parallel_env
from supersuit import (
    pad_observations_v0,
    pad_action_space_v0,
    pettingzoo_env_to_vec_env_v1,
    concat_vec_envs_v1
)
from gym import Wrapper

# ==== Seed setzen ====
SEED = 42
np.random.seed(SEED)
torch.manual_seed(SEED)

# ==== Adaptive Parameter-Schedules ====
def dynamic_clip_range(start=0.2):
    return lambda progress: max(0.1, start * (1 - 0.5 * progress))

def linear_schedule(start):
    return lambda progress: start * (1 - progress)

# ==== Finde letzten vollständigen Run ====
def find_latest_complete_run(base_dir="runs", prefix="ppo_sumo_"):
    subdirs = sorted(
        [d for d in os.listdir(base_dir) if d.startswith(prefix)],
        reverse=True
    )
    for d in subdirs:
        dir_path = os.path.join(base_dir, d)
        norm_path = os.path.join(dir_path, "vecnormalize.pkl")
        if not os.path.exists(norm_path):
            continue

        final_model = os.path.join(dir_path, "model.zip")
        if os.path.exists(final_model):
            return dir_path, final_model, norm_path

        checkpoint_models = [
            f for f in os.listdir(dir_path)
            if re.match(r"ppo_sumo_model_(\d+)_steps\.zip", f)
        ]
        if checkpoint_models:
            checkpoint_models.sort(key=lambda x: int(re.findall(r"\d+", x)[0]), reverse=True)
            best_checkpoint = checkpoint_models[0]
            return dir_path, os.path.join(dir_path, best_checkpoint), norm_path

    return None

# ==== Zeitbasierter Checkpoint Callback ====
class TimeBasedCheckpointCallback(BaseCallback):
    def __init__(self, save_interval_sec, save_path, name_prefix="ppo_sumo_model", verbose=0):
        super().__init__(verbose)
        self.save_interval_sec = save_interval_sec
        self.save_path = save_path
        self.name_prefix = name_prefix
        self.last_save_time = time.time()

    def _on_step(self) -> bool:
        return True
        
    def _on_rollout_end(self) -> bool:
        current_time = time.time()
        if current_time - self.last_save_time >= self.save_interval_sec:
            timestep = self.num_timesteps
            filename = f"{self.name_prefix}_{timestep}_steps"
            self.model.save(os.path.join(self.save_path, filename + ".zip"))
            if hasattr(self.training_env, "save"):
                self.training_env.save(os.path.join(self.save_path, f"{filename}_vecnormalize.pkl"))
            print(f"[Checkpoint] Modell gespeichert bei {timestep} Schritten ({filename})")
            self.last_save_time = current_time
        return True

# ==== Learning Rate Logger Callback ====
class LearningRateLoggerCallback(BaseCallback):
    def __init__(self, verbose=0):
        super().__init__(verbose)

    def _on_step(self) -> bool:
        lr = self.model.lr_schedule(self.num_timesteps / self.model._total_timesteps)
        self.logger.record("train/learning_rate", lr)
        return True

# ==== Logging ====
now = datetime.datetime.now().strftime("%Y-%m-%d_%H-%M-%S")
log_dir = os.path.join("runs", f"ppo_sumo_{now}")
os.makedirs(log_dir, exist_ok=True)

# ==== SUMO-RL Umgebung ====
env = parallel_env(
    net_file="network.net.xml",
    route_file="flow.rou.xml",
    use_gui=False,
    num_seconds=4096,
    reward_fn="diff-waiting-time",
    min_green=5,
    max_depart_delay=100,
    sumo_seed=SEED,
    add_system_info=True,
    add_per_agent_info=False,
)

if hasattr(env, "seed"):
    env.seed(SEED)

# ==== Wrapping ====
env = pad_observations_v0(env)
env = pad_action_space_v0(env)
env = pettingzoo_env_to_vec_env_v1(env)
env = concat_vec_envs_v1(env, num_vec_envs=1, num_cpus=8, base_class="stable_baselines3")
env = VecMonitor(env)

# ==== Modell laden oder neu starten ====
result = find_latest_complete_run()
if result:
    latest_run_dir, model_path, normalize_path = result
    print("Fortsetzung wird gestartet mit:")
    print(f"Verzeichnis : {latest_run_dir}")
    print(f"Modell      : {model_path}")
    print(f"Normalize   : {normalize_path}\n")

    env = VecNormalize.load(normalize_path, env)
    env.training = True
    env.norm_reward = True

    model = PPO.load(model_path, env=env, tensorboard_log=log_dir, verbose=1, device="cpu")
    print(f"[INFO] Modell startet bei {model.num_timesteps} Timesteps.")
else:
    print("[INFO] Kein vorheriges Modell gefunden. Starte frisches Training.\n")
    env = VecNormalize(env, norm_obs=True, norm_reward=True, clip_obs=10.0)
    model = PPO(
        policy="MlpPolicy",      # Mehrschicht-Perzeptron-Policy
        env=env,
        verbose=1,               # Ausführliches Logging
        tensorboard_log=log_dir, # TensorBoard-Pfad
        batch_size=256,          # Minibatch-Größe für PPO
        n_steps=2048,            # Rollout-Länge
        learning_rate=cosine_warmup_floor(start=3e-4, warmup_frac=0.05, min_lr_frac=0.1),
        clip_range=cosine_clip(), # Clipping-Range dynamisch
        ent_coef=0.01,            # Entropie-Koeffizient (Exploration)
        gamma=0.99,               # Diskontfaktor
        gae_lambda=0.95,          # Lambda für GAE
        device="cpu",             # Training auf CPU
        policy_kwargs=dict(net_arch=dict(pi=[128, 128], vf=[128, 128])), # Netzarchitekturgit
    )

# ==== Automatisches Speichern bei verbessertem ep_rew_mean ====
class BestModelSaverCallback(BaseCallback):
    def __init__(self, save_path, verbose=0):
        super().__init__(verbose)
        self.best_mean_reward = -float('inf')
        self.save_path = save_path

    def _on_step(self) -> bool:
        # Muss vorhanden sein, selbst wenn sie nichts tut
        return True
        
    def _on_rollout_end(self):
        ep_info_buffer = self.model.ep_info_buffer
        if len(ep_info_buffer) > 0:
            mean_rew = np.mean([ep_info['r'] for ep_info in ep_info_buffer])
            
            if mean_rew > self.best_mean_reward:
                self.best_mean_reward = mean_rew

                model_path = os.path.join(self.save_path, "best_model.zip")
                self.model.save(model_path)

                if hasattr(self.model.env, "save"):
                    norm_path = os.path.join(self.save_path, "best_model_vecnormalize.pkl")
                    self.model.env.save(norm_path)

                print(f"[AUTOLOG] Neuer Bestwert {mean_rew:.2f} → best_model gespeichert!", flush=True)

# ==== Callbacks kombinieren ====
callbacks = CallbackList([
    TimeBasedCheckpointCallback(
        save_interval_sec=3600,
        save_path=log_dir,
        name_prefix="ppo_sumo_model",
        verbose=1,
    ),
    LearningRateLoggerCallback(),
    BestModelSaverCallback(save_path=log_dir),
])

# ==== Training starten ====
try:
    model.learn(
        total_timesteps=1_000_000,
        callback=callbacks,
    )
    model.save(os.path.join(log_dir, "model.zip"))
    env.save(os.path.join(log_dir, "vecnormalize.pkl"))
    print(f"\n[INFO] Training abgeschlossen. Modell gespeichert unter: {log_dir}")

except KeyboardInterrupt:
    print("[ABBRUCH] Manuelles Beenden erkannt. Speichere aktuellen Stand...")
    model.save(os.path.join(log_dir, "model_interrupt.zip"))
    env.save(os.path.join(log_dir, "vecnormalize_interrupt.pkl"))

except Exception as e:
    print(f"\n[FEHLER] Während des Trainings aufgetreten: {e}")

finally:
    try:
        env.save(os.path.join(log_dir, "vecnormalize.pkl"))
    except Exception as e:
        print(f"[WARNUNG] VecNormalize konnte nicht gespeichert werden: {e}")
    env.close()
\end{minted}

\section{Belohnungsfunktionen}
\label{app:rewardfunktionen}

\subsection{\texttt{diff-waiting-time}}
\label{app:reward_diff_waiting_time}
Diese Belohnungsfunktion misst die Differenz der kumulierten Wartezeit zwischen zwei Zeitschritten. Sie belohnt eine Abnahme der Gesamtwartezeit.
\begin{minted}[fontsize=\small, linenos, frame=lines, breaklines, tabsize=4]{python}
def diff_waiting_time_reward(traffic_signal):
    ts_wait = sum(traffic_signal.get_accumulated_waiting_time_per_lane()) / 100.0
    reward = traffic_signal.last_ts_waiting_time - ts_wait
    traffic_signal.last_ts_waiting_time = ts_wait
    return reward
\end{minted}

\subsection{\texttt{queue}}
\label{app:reward_queue}
Hier wird die Anzahl an gestoppten Fahrzeugen direkt als negativer Reward verwendet. Weniger Stau → höherer Reward.
\begin{minted}[fontsize=\small, linenos, frame=lines, breaklines, tabsize=4]{python}
def queue_reward(traffic_signal):
    return -traffic_signal.get_total_queued()

def get_total_queued(traffic_signal) -> int:
    """Returns the total number of vehicles halting in the intersection."""
    return sum(traffic_signal.sumo.lane.getLastStepHaltingNumber(lane) for lane in traffic_signal.lanes)
\end{minted}

\subsection{\texttt{realworld}}
\label{app:reward_realworld}
Diese Funktion kombiniert Geschwindigkeit, Warteschlangenlänge und mittlere Wartezeit in einem additiven Reward.
\begin{minted}[fontsize=\small, linenos, frame=lines, breaklines, tabsize=4]{python}
def realworld_reward(traffic_signal):
    # Speed (0–7.5 m/s -> 0–1)
    avg_speed = traffic_signal.get_average_speed()
    speed_term = min(max(avg_speed, 0.0), 7.5) / 7.5

    # Queue (0–20 Fzg -> 0–1)
    total_queue = traffic_signal.get_total_queued()
    queue_term = min(max(total_queue, 0), 20) / 20.0

    # Mean waiting time (0–10 s -> 0–1)
    waits_per_lane = traffic_signal.get_accumulated_waiting_time_per_lane()
    mean_wait = sum(waits_per_lane) / len(waits_per_lane) if waits_per_lane else 0.0
    wait_term = min(max(mean_wait, 0.0), 20.0) / 10.0

    reward = speed_term - queue_term - wait_term
    return reward
\end{minted}

\subsection{\texttt{emissions}}
\label{app:reward_emissions}
Diese Variante erweitert den Reward zusätzlich um einen Term für die CO\textsubscript{2}-Emissionen, sodass sowohl Verkehrsfluss als auch Nachhaltigkeit berücksichtigt werden.
\begin{minted}[fontsize=\small, linenos, frame=lines, breaklines, tabsize=4]{python}
def emissions_reward(traffic_signal):
    env = getattr(traffic_signal, "env", None)
    if env is None or getattr(env, "sumo", None) is None:
        return 0.0
    if float(env.sim_step) >= float(env.sim_max_time):
        return 0.0

    sumo = env.sumo

    # Speed (0–7.5 m/s → 0..1)
    avg_speed = traffic_signal.get_average_speed()
    speed_term = min(max(avg_speed, 0.0), 7.5) / 7.5

    # Queue (0–20 → 0..1)
    total_queue = traffic_signal.get_total_queued()
    queue_term = min(max(total_queue, 0), 20) / 20.0

    # Wait (0–10 s → 0..1)
    waits = traffic_signal.get_accumulated_waiting_time_per_lane()
    mean_wait = (sum(waits) / len(waits)) if waits else 0.0
    wait_term = min(max(mean_wait, 0.0), 10.0) / 10.0

    # Emissionen
    try:
        lanes = getattr(traffic_signal, "lanes", [])
        total_co2 = sum(sumo.lane.getCO2Emission(lane) for lane in lanes)
        n_veh = sum(sumo.lane.getLastStepVehicleNumber(lane) for lane in lanes)
        BASELINE = 300.0 * max(1, len(lanes))
        CAP      = 2000.0 * max(1, len(lanes))
        co2_term = max(0.0, min(total_co2 - BASELINE, CAP - BASELINE)) / (CAP - BASELINE)
    except Exception:
        co2_term = 0.0

    reward = speed_term - queue_term - wait_term - co2_term
    return reward
\end{minted}

\section{Evaluierungs-Skripte}

\subsection{\texttt{evaluate.py} – Evaluationsskript für PPO-Modelle und Baselines}
\label{app:evaluate_script}
Dieses Skript führt die Evaluation aller trainierten PPO-Modelle in SUMO durch und vergleicht sie mit den Baselines \emph{Fixed-Time} und \emph{Actuated}.
Es rollt mehrere Episoden über verschiedene Seeds aus, extrahiert Metriken (z.\,B. Wartezeit, Geschwindigkeit, Emissionen) und speichert die Ergebnisse als \texttt{eval\_results.json}.
\begin{minted}[fontsize=\small, linenos, frame=lines, breaklines, tabsize=4]{python}
import os, json, numpy as np
import glob
from stable_baselines3 import PPO
from stable_baselines3.common.vec_env import VecNormalize, VecMonitor
from stable_baselines3.common.logger import configure
from sumo_rl.environment.env import parallel_env
from supersuit import pad_observations_v0, pad_action_space_v0
from supersuit import pettingzoo_env_to_vec_env_v1, concat_vec_envs_v1

# ----- Config -----
RUNS = sorted(glob.glob(os.path.join("runs", "ppo_sumo_*")))
MODEL_NAME  = "best_model.zip"
N_EPISODES  = 10
EP_LENGTH_S = 4096
EP_SEEDS    = [12345, 67890, 13579, 24680, 11223, 44556, 77889, 99100, 31415, 27182]
SCENARIOS   = [
    {"name": "morning_peak", "route_file": "flows_morning.rou.xml"},
    {"name": "evening_peak", "route_file": "flows_evening.rou.xml"},
    {"name": "uniform",      "route_file": "flows_uniform.rou.xml"},
    {"name": "random_heavy", "route_file": "flows_random_heavy.rou.xml"},
]

# ----- Env Factory -----
def make_env(route_file, sumo_seed):
    print(f"[DEBUG] Creating SUMO env with route={route_file}, seed={sumo_seed}")
    env = parallel_env(
        net_file="map.net.xml",
        route_file=route_file,
        use_gui=False,
        num_seconds=EP_LENGTH_S,
        reward_fn=dummy_reward,
        min_green=5,
        max_depart_delay=100,
        sumo_seed=sumo_seed,
        add_system_info=True,
        add_per_agent_info=False,
    )
    env = pad_observations_v0(env)
    env = pad_action_space_v0(env)
    env = pettingzoo_env_to_vec_env_v1(env)
    env = concat_vec_envs_v1(env, num_vec_envs=1, num_cpus=1, base_class="stable_baselines3")
    env = VecMonitor(env)
    return env

# ----- Model Loader -----
def load_model_and_norm(env, run_dir):
    vecnorm_path = os.path.join(run_dir, "vecnormalize.pkl")
    model_path   = os.path.join(run_dir, MODEL_NAME)

    #print(f"[DEBUG] Loading VecNormalize from {vecnorm_path}")
    env = VecNormalize.load(vecnorm_path, env)
    env.training = False
    env.norm_reward = False

    #print(f"[DEBUG] Loading PPO model from {model_path}")
    model = PPO.load(model_path, env=env, device="cpu")
    return model, env

# ----- Rollout -----
def rollout(model, env):
    obs = env.reset()
    dones = [False]

    sums = {}
    counts = {}
    last_totals = {}

    while True:
        action, _ = model.predict(obs, deterministic=True)
        obs, rewards, dones, infos = env.step(action)

        info = infos[0] if isinstance(infos, list) else infos
        if not isinstance(info, dict):
            info = {}

        # Wenn Episode zu Ende ist:
        if dones[0]:
            # Falls vorhanden, final_info/terminal_info verwenden
            fin = info.get("final_info") or info.get("terminal_info")
            if isinstance(fin, dict):
                # Mittelwerte vom finalen Step noch einrechnen
                for k, v in fin.items():
                    if k.startswith("system_mean_") and isinstance(v, (int, float)) and np.isfinite(v):
                        sums[k] = sums.get(k, 0.0) + float(v)
                        counts[k] = counts.get(k, 0) + 1
                # Totals aus final_info (echte Endstände)
                for k, v in fin.items():
                    if k.startswith("system_total_") and isinstance(v, (int, float)) and np.isfinite(v):
                        last_totals[k] = float(v)
            break

        # Normaler Zwischenschritt: Mittelwerte sammeln + Totals „letzten gültigen“ merken
        for k, v in info.items():
            if not isinstance(v, (int, float)) or not np.isfinite(v):
                continue
            if k.startswith("system_mean_") or k in ["system_total_waiting_time", "system_total_stopped", "system_total_running"]:
                # momentane Werte mitteln
                sums[k] = sums.get(k, 0.0) + float(v)
                counts[k] = counts.get(k, 0) + 1
            elif k.startswith("system_total_"):
                # Totals: nur letzten Wert merken
                last_totals[k] = float(v)

    mean_metrics = {k: (sums[k] / max(1, counts.get(k, 0))) for k in sums}
    mean_metrics.update(last_totals)
    return mean_metrics


def shorten_key(orig_key: str) -> str:
    return orig_key.replace("system_", "")

# ----- Env Factory für Baselines -----
def make_env_baseline(route_file, sumo_seed, fixed_time=True):
    """
    Erstellt eine SUMO-Umgebung, die den internen Controller verwendet.
    fixed_time=True  -> Fester Phasenplan aus net.xml
    fixed_time=False -> SUMO Actuated Control (falls in net.xml konfiguriert)
    """
    env = parallel_env(
        net_file="map.net.xml",
        route_file=route_file,
        use_gui=False,
        num_seconds=EP_LENGTH_S,
        reward_fn=dummy_reward,            # Kein RL-Reward
        fixed_ts=fixed_time,       # True = fixed, False = actuated
        sumo_seed=sumo_seed,
        add_system_info=True,
        add_per_agent_info=False,
    )
    env = pad_observations_v0(env)
    env = pad_action_space_v0(env)
    env = pettingzoo_env_to_vec_env_v1(env)
    env = concat_vec_envs_v1(env, num_vec_envs=1, num_cpus=1, base_class="stable_baselines3")
    env = VecMonitor(env)
    return env

def dummy_reward(_ts):
    return 0.0

def rollout_baseline(env):
    obs = env.reset()
    dones = [False]

    # Mittelwerte über die Episode
    sums = {}
    counts = {}
    # Letzte gültige Totals (vor Reset)
    last_totals = {}

    # gültige Dummy-Aktion aus dem Action Space
    dummy_action = np.array([env.action_space.sample() for _ in range(env.num_envs)])

    while True:
        obs, rewards, dones, infos = env.step(dummy_action)

        info = infos[0] if isinstance(infos, list) else infos
        if not isinstance(info, dict):
            info = {}

        # Wenn Episode zu Ende ist:
        if dones[0]:
            # Falls vorhanden, final_info/terminal_info verwenden
            fin = info.get("final_info") or info.get("terminal_info")
            if isinstance(fin, dict):
                # Mittelwerte vom finalen Step noch einrechnen
                for k, v in fin.items():
                    if k.startswith("system_mean_") and isinstance(v, (int, float)) and np.isfinite(v):
                        sums[k] = sums.get(k, 0.0) + float(v)
                        counts[k] = counts.get(k, 0) + 1
                # Totals aus final_info (echte Endstände)
                for k, v in fin.items():
                    if k.startswith("system_total_") and isinstance(v, (int, float)) and np.isfinite(v):
                        last_totals[k] = float(v)
            break

        # Normaler Zwischenschritt: Mittelwerte sammeln + Totals „letzten gültigen“ merken
        for k, v in info.items():
            if not isinstance(v, (int, float)) or not np.isfinite(v):
                continue
            if k.startswith("system_mean_") or k in ["system_total_waiting_time", "system_total_stopped", "system_total_running"]:
                # momentane Werte mitteln
                sums[k] = sums.get(k, 0.0) + float(v)
                counts[k] = counts.get(k, 0) + 1
            elif k.startswith("system_total_"):
                # Totals: nur letzten Wert merken
                last_totals[k] = float(v)

    # Mittelwerte berechnen
    mean_metrics = {k: (sums[k] / max(1, counts.get(k, 0))) for k in sums}
    # Letzte gültige Totals übernehmen
    mean_metrics.update(last_totals)

    return mean_metrics


def to_serializable(obj):
    if isinstance(obj, (np.integer,)):
        return int(obj)
    elif isinstance(obj, (np.floating,)):
        return float(obj)
    elif isinstance(obj, (np.ndarray,)):
        return obj.tolist()
    return str(obj)

# ----- Evaluation Loop -----
# ----- Evaluation Loop -----
def evaluate():
    results = []
    log_dir_root = os.path.join("evaluation", "logs")

    # Zählung: 2 Baselines + len(RUNS) RL pro (scenario × episode)
    total_episodes = (2 + len(RUNS)) * len(SCENARIOS) * N_EPISODES
    ep_counter = 0

    for sc in SCENARIOS:
        scen_log_dir = os.path.join(log_dir_root, f"eval_{sc['name']}")
        os.makedirs(scen_log_dir, exist_ok=True)
        logger = configure(scen_log_dir, ["tensorboard", "stdout"])

        print(f"[INFO] Evaluating scenario={sc['name']}")

        for ep in range(N_EPISODES):
            ep_seed = EP_SEEDS[ep]

            # --- 1) Fixed-Time ---
            env = make_env_baseline(sc["route_file"], sumo_seed=ep_seed, fixed_time=True)
            ep_counter += 1
            print(f"[PROGRESS] FixedTime | {sc['name']} | Ep {ep+1}/{N_EPISODES} "
                  f"({ep_counter}/{total_episodes})")
            m = rollout_baseline(env)
            m.update({
                "scenario": sc["name"],
                "episode": ep,
                "method": "Baseline_FixedTime"
            })
            results.append(m)

            # --- 2) Actuated ---
            env = make_env_baseline(sc["route_file"], sumo_seed=ep_seed, fixed_time=False)
            ep_counter += 1
            print(f"[PROGRESS] Actuated | {sc['name']} | Ep {ep+1}/{N_EPISODES} "
                  f"({ep_counter}/{total_episodes})")
            m = rollout_baseline(env)
            m.update({
                "scenario": sc["name"],
                "episode": ep,
                "method": "Baseline_Actuated"
            })
            results.append(m)

            # --- 3) RL-Modelle ---
            for run_dir in RUNS:
                env_raw = make_env(sc["route_file"], sumo_seed=ep_seed)
                model, env = load_model_and_norm(env_raw, run_dir)
                ep_counter += 1
                model_name = os.path.basename(run_dir)
                print(f"[PROGRESS] RL | {sc['name']} | {model_name} "
                      f"| Ep {ep+1}/{N_EPISODES} ({ep_counter}/{total_episodes})")
                m = rollout(model, env)
                
                # Seed extrahieren (3. Teil vom Namen)
                parts = model_name.split("_")
                model_seed = parts[2] if len(parts) > 2 else "unknown"

                m.update({
                    "scenario": sc["name"],
                    "episode": ep,
                    "method": f"{model_name}_{model_seed}"
                })
                results.append(m)

            # --- Logging dieser Episode (Baselines + alle RL) ---
            for entry in results[-(2 + len(RUNS)):]:
                for k, v in entry.items():
                    if isinstance(v, (int, float)) and k not in ["episode", "ep_seed"]:
                        short_key = shorten_key(k)
                        logger.record(f"{entry['method']}/{short_key}", v)
            logger.dump(step=ep)

    results_path = os.path.join("evaluation", "eval_results.json")
    os.makedirs(os.path.dirname(results_path), exist_ok=True)
    with open(results_path, "w") as f:
        json.dump(results, f, indent=2, default=to_serializable)

    print(f"[INFO] Evaluation abgeschlossen. Ergebnisse: {results_path}")

if __name__ == "__main__":
    evaluate()

\end{minted}

\section{Postprocessing der Evaluationsergebnisse}

\subsection{\texttt{json2csv.py} – Konvertierung und Aggregation von Evaluationsergebnissen}
\label{app:json2csv_script}
Dieses Skript verarbeitet die von \texttt{evaluate.py} erzeugte JSON-Datei \texttt{eval\_results.json}.
Es wandelt die Rohdaten zunächst in ein CSV-Format um, berechnet anschließend Mittelwerte und Standardabweichungen pro \emph{Scenario × Methode} und erzeugt sowohl eine aggregierte Gesamttabelle als auch separate CSV-Dateien pro Methode.
\begin{minted}[fontsize=\small, linenos, frame=lines, breaklines, tabsize=4]{python}
import json
import pandas as pd
import os

# Pfade
json_path = "evaluation/eval_results.json"
raw_csv_path = "evaluation/eval_results_raw.csv"
agg_csv_path = "evaluation/eval_results_agg.csv"

# -----------------------------
# Schritt 1: JSON -> Raw CSV
# -----------------------------
print(f"Lese JSON-Datei: {json_path}")
with open(json_path, "r") as f:
    data = json.load(f)

df = pd.DataFrame(data)
df.to_csv(raw_csv_path, index=False)
print(f"Raw CSV geschrieben: {raw_csv_path}")

# -----------------------------
# Schritt 2: Aggregation
# -----------------------------
# numerische Spalten automatisch finden (alles außer scenario, method, episode)
numeric_cols = df.select_dtypes(include="number").columns.tolist()
numeric_cols = [c for c in numeric_cols if c not in ["episode"]]  # episode nicht mitteln

# Aggregationsdict
agg_dict = {}
for col in numeric_cols:
    agg_dict[f"{col}_mean"] = (col, "mean")
    agg_dict[f"{col}_std"] = (col, "std")

# Gruppieren nach Szenario + Methode
agg = df.groupby(["scenario", "method"]).agg(**agg_dict).reset_index()

# Gesamte Aggregation speichern
agg.to_csv(agg_csv_path, index=False)
print(f"Aggregierte Datei geschrieben: {agg_csv_path}")
print("Zeilen:", len(agg))

# -----------------------------
# Schritt 3: Pro-Methode CSVs
# -----------------------------
for method, df_method in agg.groupby("method"):
    safe_name = method.replace(" ", "_").replace("/", "_")
    out_path = f"evaluation/{safe_name}.csv"
    df_method.to_csv(out_path, index=False)
    print(f"Datei für Methode '{method}' geschrieben: {out_path}")

\end{minted}

\section{Netzwerk-Skripte}

\subsection{\texttt{check\_tls\_consistency.py} – Prüfung inkonsistenter Phasenlängen}
\label{app:check_tls_consistency}
Dieses Tool analysiert alle TLS im SUMO-Netz und prüft, ob die Länge des \texttt{state}-Strings mit der Anzahl der kontrollierten Verbindungen übereinstimmt.
\begin{minted}[fontsize=\small, linenos, frame=lines, breaklines, tabsize=4]{python}
import xml.etree.ElementTree as ET

# === Konfiguration ===
net_file = "map.net.xml"

# === Einlesen ===
tree = ET.parse(net_file)
root = tree.getroot()

# === Alle controlledLinks zählen ===
tls_controlled_links = {}
for connection in root.findall("connection"):
    if "tl" in connection.attrib and "linkIndex" in connection.attrib:
        tls_id = connection.attrib["tl"]
        tls_controlled_links.setdefault(tls_id, set()).add(int(connection.attrib["linkIndex"]))

# === Alle Phasen prüfen ===
def check_tls_lengths():
    print("Überprüfe alle TLS auf inkonsistente Phasenlängen...\n")
    any_issues = False
    for logic in root.findall("tlLogic"):
        tls_id = logic.attrib["id"]
        expected_len = len(tls_controlled_links.get(tls_id, []))

        if expected_len == 0:
            print(f" TLS '{tls_id}' hat keine controlledLinks (wird evtl. nicht gesteuert)")
            continue

        for i, phase in enumerate(logic.findall("phase")):
            actual_len = len(phase.attrib["state"])
            if actual_len != expected_len:
                print(f" Phase {i} von TLS '{tls_id}' hat Länge {actual_len}, erwartet: {expected_len}")
                print(f"    → state=\"{phase.attrib['state']}\"")
                any_issues = True

    if not any_issues:
        print(" Alle TLS-Phasen stimmen mit ihren controlledLinks überein!")

check_tls_lengths()
\end{minted}

\subsection{\texttt{check\_tls\_requests.py} – Prüfung ungültiger \texttt{<request>}-Indizes}
\label{app:check_tls_requests}
Prüft, ob alle \texttt{request}-Indizes innerhalb der zulässigen Grenzen liegen, um Laufzeitfehler in \texttt{sumo-rl} zu vermeiden.

\begin{minted}[fontsize=\small, linenos, frame=lines, breaklines, tabsize=4]{python}
import xml.etree.ElementTree as ET

net_file = "map.net.xml"
tree = ET.parse(net_file)
root = tree.getroot()

# Zähle für jedes TLS wie viele signal indices es gibt (controlled links)
tls_signal_indices = {}
for conn in root.findall("connection"):
    if "tl" in conn.attrib and "linkIndex" in conn.attrib:
        tls_id = conn.attrib["tl"]
        tls_signal_indices.setdefault(tls_id, set()).add(int(conn.attrib["linkIndex"]))

# Vergleiche mit den request-Elementen
print("Überprüfe request-Indizes gegen Signalindizes...\n")
any_issues = False
for junction in root.findall("junction"):
    tls_id = junction.attrib.get("id")
    requests = junction.findall("request")
    if tls_id in tls_signal_indices:
        expected_max = len(tls_signal_indices[tls_id])
        for req in requests:
            index = int(req.attrib["index"])
            if index >= expected_max:
                print(f"Junction '{tls_id}': request index {index} > max signal index {expected_max - 1}")
                any_issues = True

if not any_issues:
    print("Alle request-Indizes passen zu den TLS-Signalindizes!")
\end{minted}

\subsection{\texttt{fix\_requests.py} – Automatische Korrektur von Requests und Phasen}
\label{app:fix_requests}
Dieses Skript bereinigt überzählige \texttt{<request>}-Einträge und passt \texttt{state}-Strings in den Phasenlängen an.
\begin{minted}[fontsize=\small, linenos, frame=lines, breaklines, tabsize=4]{python}
import xml.etree.ElementTree as ET

net_file = "map.net.xml"
output_file = "map_fixed_tls.net.xml"

tree = ET.parse(net_file)
root = tree.getroot()

# Finde maximal verwendete Signal-Indices pro TLS
tls_max_index = {}
for conn in root.findall("connection"):
    tl = conn.get("tl")
    idx = conn.get("linkIndex")
    if tl and idx:
        idx = int(idx)
        tls_max_index[tl] = max(tls_max_index.get(tl, -1), idx)

# Bereinigung
total_removed_requests = 0
total_adjusted_phases = 0
changed_tls = []

for junction in root.findall("junction"):
    tls_id = junction.get("id")
    if tls_id not in tls_max_index:
        continue

    max_idx = tls_max_index[tls_id]
    requests = list(junction.findall("request"))
    removed = 0

    for req in requests:
        req_idx = int(req.get("index"))
        if req_idx > max_idx:
            junction.remove(req)
            removed += 1

    if removed > 0:
        print(f"TLS '{tls_id}': {removed} ungültige <request>-Einträge entfernt.")
        total_removed_requests += removed
        changed_tls.append(tls_id)

    # Kürze zugehörige Phasen
    for tl in root.findall("tlLogic"):
        if tl.get("id") == tls_id:
            adjusted = 0
            for phase in tl.findall("phase"):
                state = phase.get("state")
                if len(state) > max_idx + 1:
                    old_len = len(state)
                    phase.set("state", state[:max_idx + 1])
                    adjusted += 1
            if adjusted > 0:
                print(f" TLS '{tls_id}': {adjusted} <phase>-Strings auf Länge {max_idx + 1} gekürzt.")
                total_adjusted_phases += adjusted
                if tls_id not in changed_tls:
                    changed_tls.append(tls_id)

# Speichern
tree.write(output_file, encoding="utf-8")
print("\n Reparatur abgeschlossen.")
print(f" Gesamt entfernte <request>-Einträge: {total_removed_requests}")
print(f" Gesamt angepasste <phase>-Einträge: {total_adjusted_phases}")
print(f" Betroffene TLS-IDs: {len(changed_tls)} Stück")
for tls in changed_tls:
    print(f"  - {tls}")
print(f"\n Bereinigte Datei gespeichert unter: {output_file}")
\end{minted}

\subsection{\texttt{repair\_net.py} – manuelle TLS-Reparatur auf Basis eines Referenz-Dictionaries}
\label{app:repair_net}
Repariert TLS-Definitionen durch Abgleich mit einer vordefinierten Mapping-Tabelle von korrekten Phasenlängen.
\begin{minted}[fontsize=\small, linenos, frame=lines, breaklines, tabsize=4]{python}
from xml.etree import ElementTree as ET

# Manuell gepflegte Dictionary mit {TLS-ID: Anzahl controlledLinks}
controlled_links = {
    "1720933516": 6,
    "3538953167": 2,
    "3664415977": 10,
    "cluster_14795187_1720919996_2670370290_2670370291": 11,
    "cluster_14795804_55474925_6655074904_765746891_#1more": 49,
    "cluster_15431428_1719671850_1720917935": 20,
    "cluster_1590912233_3664415976_5083348337_5083348350": 11,
    "cluster_1692973685_1692973722_1718084055_1718084058_#11more": 36,
    "cluster_1729190097_3687504105": 8,
    "cluster_1744031943_5131521735": 10,
    "joinedS_1623835169_cluster_1137679587_1626739216_1728272870_1728272909_#17more": 33,
    "joinedS_309108716_cluster_11001804363_1125509937_12515596172_1784859792_#5more": 14,
    "joinedS_5092985445_cluster_1590912226_2911376263": 10,
    # ggf. mehr hinzufügen
}

tree = ET.parse("map.net.xml")
root = tree.getroot()
changed = False

for logic in root.findall("tlLogic"):
    tl_id = logic.attrib["id"]
    if tl_id not in controlled_links:
        continue

    correct_len = controlled_links[tl_id]
    for phase in logic.findall("phase"):
        state = phase.attrib["state"]
        if len(state) != correct_len:
            new_state = state[:correct_len].ljust(correct_len, 'r')
            print(f" Fixing {tl_id}: {len(state)} → {correct_len}")
            phase.attrib["state"] = new_state
            changed = True

if changed:
    tree.write("karlsruhe_fixed.net.xml")
    print(" Bereinigte Datei gespeichert: karlsruhe_fixed.net.xml")
else:
    print(" Alle Phasen bereits korrekt.")

\end{minted}

\subsection{\texttt{statecheck.py} – Prüfung auf Ziel-Phasenlänge}
\label{app:statecheck}
Hilft bei der Kontrolle einheitlicher Phasenlängen über das gesamte Netz hinweg (z.\,B. Zielwert = 57).
\begin{minted}[fontsize=\small, linenos, frame=lines, breaklines, tabsize=4]{python}
from xml.etree import ElementTree as ET

tree = ET.parse("map.net.xml")
root = tree.getroot()

for logic in root.findall("tlLogic"):
    tl_id = logic.attrib["id"]
    for i, phase in enumerate(logic.findall("phase")):
        state = phase.attrib["state"]
        if len(state) != 57:
            print(f" Phase {i} of TLS '{tl_id}' has length {len(state)}")
\end{minted}

\subsection{\texttt{find\_valid\_tls.py} – Validierung lauffähiger TLS für SUMO-RL}
\label{app:find_valid_tls}
Startet für jede einzelne Ampelkreuzung eine Minimalumgebung und überprüft, ob diese in \texttt{sumo-rl} trainierbar ist.
\begin{minted}[fontsize=\small, linenos, frame=lines, breaklines, tabsize=4]{python}
from sumo_rl import SumoEnvironment
import traci
import os

def test_tls(tls_id):
    try:
        env = SumoEnvironment(
            net_file="map.net.xml",
            route_file="map.rou.xml",
            use_gui=False,
            single_agent=True
        )
        env.ts_ids = [tls_id]
        env.reset()
        env.close()
        return True
    except Exception as e:
        print(f" TLS {tls_id} nicht gültig: {e}")
        return False

# Alle TLS holen
try:
    env = SumoEnvironment(
        net_file="map.net.xml",
        route_file="map.rou.xml",
        use_gui=False,
        single_agent=True
    )
    all_tls = env.ts_ids
    env.close()
except Exception as e:
    print(" Konnte TLS nicht auslesen:", e)
    all_tls = []

print(f" Teste {len(all_tls)} TLS auf Gültigkeit...\n")
valid_tls = []

for tls_id in all_tls:
    if test_tls(tls_id):
        valid_tls.append(tls_id)

print("\n Gültige TLS:")
print(valid_tls)
\end{minted}


\subsection{\texttt{find\_relevant\_edges.py} – Suche alle relevanten edges}
\label{app:find_relevant_edges}
\begin{minted}[fontsize=\small, linenos, frame=lines, breaklines, tabsize=4]{python}
import xml.etree.ElementTree as ET

# Konfiguration: Pfad zur .net.xml-Datei und Suchbegriffe
NET_FILE = "network.net.xml"
SUCHBEGRIFFE = ["B10", "B36", "L605", "Durlacher Allee", "Reinhold-Frank-Straße"]

# Ausgabe-Datei für gefundene Kanten
OUTPUT_FILE = "edges.txt"

def finde_relevante_kanten(net_file, suchbegriffe):
    tree = ET.parse(net_file)
    root = tree.getroot()
    
    relevante_kanten = []
    
    for edge in root.findall("edge"):
        name = edge.get("name")
        if name:
            for begriff in suchbegriffe:
                if begriff.lower() in name.lower():
                    relevante_kanten.append((edge.get("id"), name))
                    break  # nicht doppelt eintragen, falls mehrere Begriffe passen
                    
    return relevante_kanten

if __name__ == "__main__":
    kanten = finde_relevante_kanten(NET_FILE, SUCHBEGRIFFE)
    
    # Ergebnisse speichern
    with open(OUTPUT_FILE, "w", encoding="utf-8") as f:
        for edge_id, name in kanten:
            f.write(f"{edge_id}\t{name}\n")
    
    print(f"{len(kanten)} relevante Kanten gefunden.")
    print(f"Ergebnisse in '{OUTPUT_FILE}' gespeichert.")

\end{minted}

\section{Sumo-Konfiguration}
\subsection{\texttt{sumoconfig\_.sumocfg}}
\label{app:sumocfg}
Die folgende Konfigurationsdatei definiert die zentralen Eingaben und
Parameter für die Simulation in SUMO. Sie verweist auf die zu ladende
Netzdatei und die zugehörigen Routendateien sowie auf den zu simulierenden
Zeitraum.
\begin{minted}[fontsize=\small, linenos, frame=lines, breaklines, tabsize=4]{xml}
<configuration>
  <input>
    <net-file value="network.net.xml"/>
    <route-files value="routes.xml"/>
  </input>
  <time>
    <begin value="0"/>
    <end value="5000"/>
  </time>
</configuration>
\end{minted}


\newpage
\begin{thebibliography}{12}
    \bibitem{bast}
    Bundesanstalt für Straßenwesen (BASt). \url{https://www.bast.de}, Zugriff am 21.05.2025.

    \bibitem{googlemaps}
    Google Maps Traffic API.
    \href{https://developers.google.com/maps/documentation/traffic}, Zugriff am 21.05.2025.

    \bibitem{josm}
    JOSM – Java OpenStreetMap Editor.
    \url{https://josm.openstreetmap.de},
    Zugriff am 21.05.2025.

    \bibitem{mobidata}
    MobiData BW – Mobilitätsdatenplattform Baden-Württemberg.
    \url{https://www.mobidata-bw.de},
    Zugriff am 21.05.2025.

    \bibitem{mobidata_karte}
    MobiData BW: Karte der Dauerzählstellen im Straßenverkehr,
    \url{https://mobidata-bw.de/dataset/karte_strassenverkehrszaehlung}, Zugriff am 21.05.2025.

    \bibitem{mobidata_stunden}
    MobiData BW: Stundenwerte an Dauerzählstellen (Straßenverkehr),
    \url{https://mobidata-bw.de/dataset/stundenwerte_dauerzaehlstellen}, Zugriff am 21.05.2025.

    \bibitem{osm}
    OpenStreetMap.
    \url{https://www.openstreetmap.org},
    Zugriff am 21.05.2025.

    \bibitem{osm-export}
    OpenStreetMap Wiki: Export.
    \url{https://wiki.openstreetmap.org/w/index.php?title=Export&oldid=2822860},
    Zugriff am 21.06.2025.

    \bibitem{sumo-doc}
    SUMO Dokumentation.
    \url{https://sumo.dlr.de},
    Zugriff am 21.05.2025.

    \bibitem{sumo-rl}
    sumo-rl: Reinforcement Learning Environments for Traffic Signal Control in SUMO.
    GitHub Repository.
    \url{https://github.com/LucasAlegre/sumo-rl},
    Zugriff am 21.05.2025.

    \bibitem{sumo-tools}
    SUMO Toolchain: Netzwerk- und Routengeneratoren.
    \url{https://sumo.dlr.de/docs/Tools.html},
    \url{https://sumo.dlr.de/docs/netconvert.html},
    \url{https://sumo.dlr.de/docs/duarouter.html},
    Zugriff am 21.05.2025.

    \bibitem{svzbw}
    Straßenverkehrszentrale Baden-Württemberg.
    \url{https://www.svz-bw.de},
    Zugriff am 23.05.2025.

    \bibitem{tomtom}
    TomTom Traffic API.
    \url{https://developer.tomtom.com/traffic-api},
    Zugriff am 23.05.2025.

    \bibitem{umweltbundesamt-motorisierungsgrad}
    \textit{Umweltbundesamt}
    \url{https://www.umweltbundesamt.de/daten/private-haushalte-konsum/mobilitaet-privater-haushalte#-hoher-motorisierungsgrad}

    \bibitem{umweltbundesamt-emissionen}
    \textit{Umweltbundesamt}
    \url{https://www.umweltbundesamt.de/daten/verkehr/emissionen-des-verkehrs#pkw-fahren-heute-klima-und-umweltvertraglicher}

    \bibitem{umweltbundesamt-klimaschutz}
    \textit{Umweltbundesamt}
    \url{https://www.umweltbundesamt.de/themen/verkehr/klimaschutz-im-verkehr#bepreisung},
    Zugriff am 25.05.2025.

    \bibitem{umweltbundesamt-umweltzonen}
    \textit{Umweltbundesamt}
    \url{https://www.umweltbundesamt.de/themen/luft/luftschadstoffe/feinstaub/umweltzonen-in-deutschland},
    Zugriff am 25.05.2025.

    \bibitem{baden-wuerttemberg}
    \textit{Baden-wuerttemberg}
    \url{https://www.baden-wuerttemberg.de/de/service/presse/pressemitteilung/pid/land-startet-testfeld-mit-ki-gesteuerten-ampeln},  Zugriff am 25.05.2025.

    \bibitem{Bundesanstallt}
    \textit{Bundesanstallt für Straßenwesen}
    \url{https://edocs.tib.eu/files/e01fn19/166939879X.pdf},
    Zugriff am 25.05.2025.

    \bibitem{Kraftfahrt-Bundesamt}
    \textit{Kraftfahrt-Bundesamt}
    \url{https://www.kba.de/DE/Presse/Pressemitteilungen/Nr1Segmente/2025/pm33_2025_nr1_seg_07_25_komplett.html},
    Zugriff am 25.05.2025.

    \bibitem{Europäische-Umweltagentur}
    \textit{Europäische Umweltagentur}
    \url{https://www.kba.de/DE/Presse/Pressemitteilungen/Nr1Segmente/2025/pm33_2025_nr1_seg_07_25_komplett.html},
    Zugriff am 25.05.2025.

    \bibitem{Inrix-Traffic-Scorecard}
    \textit{Inrix Traffic Scorecard}
    \url{https://inrix.com/scorecard/},
    Zugriff am 25.05.2025.

    \bibitem{lubw}
    \textit{Landesanstalt für Umwelt Baden-Württemberg}
    \url{https://www.lubw.baden-wuerttemberg.de/luft/verkehrsemissionen}, Zugriff am 25.05.2025.

    \bibitem{DSGVO}
    \textit{Datenschutz-Grundverordnung}
    \url{https://dsgvo-gesetz.de/}, Zugriff am 25.05.2025.

    \bibitem{KI4LSA}
    \textit{Künstliche Intelligenz für LichtSignalAnlagen}
    \url{https://www.iosb-ina.fraunhofer.de/de/geschaeftsbereiche/maschinelles-lernen/forschungsthemen-und-projekte/projekt-KI4LSA.html}, Zugriff am 25.05.2025.

    \bibitem{Low-traffic-Amsterdam}
    \textit{Low-traffic Amsterdam}
    \url{https://openresearch.amsterdam/en/page/49784/low-traffic-amsterdam}, Zugriff am 25.05.2025.

    \bibitem{Traffic-Flow-Theory}
    \textit{Traffic Flow Theory – A State‑of‑the‑Art Report}
    \url{https://rosap.ntl.bts.gov/view/dot/35775/dot_35775_DS1.pdf}, Zugriff am 25.05.2025.

    \bibitem{KI-Ampel}
    \textit{KI-Ampeln in BW}
    \url{https://www.baden-wuerttemberg.de/de/service/presse/pressemitteilung/pid/land-startet-testfeld-mit-ki-gesteuerten-ampeln}, Zugriff am 25.05.2025.

    \bibitem{dynamischen-Verkehrsumlegung}
    \textit{Verfahren zur dynamischen Verkehrsumlegung}
    \url{https://www.isv.uni-stuttgart.de/vuv/publikationen/downloads/200503_Fr_DynUmlg-SVT.pdf}, Zugriff am 25.05.2025.

    \bibitem{Sumo-Publikation}
    \textit{Sumo-Publikation}
    \url{https://sumo.dlr.de/pdf/sysmea_v5_n34_2012_4.pdf}, Zugriff am 25.05.2025.

    \bibitem{sutton-barto}
    \textit{Sutton, Reinforcement Learning: An Introduction}
    \url{http://incompleteideas.net/book/the-book-2nd.html}

    \bibitem{Sumo-git}
    \textit{Sumo-Git}
    \url{https://github.com/eclipse-sumo/sumo}

    \bibitem{Gymnasium}
    \textit{Gymnasium}
    \url{https://gymnasium.farama.org/index.html}

    \bibitem{Stable-Baselines3}
    \textit{Stable-Baselines3}
    \url{https://stable-baselines3.readthedocs.io/en/master/}

    \bibitem{pytorch}
    \textit{Pytorch}
    \url{https://pytorch.org/}

    \bibitem{TensorFlow}
    \textit{TensorFlow}
    \url{https://www.tensorflow.org/}

    \bibitem{RLlib}
    \textit{RLlib}
    \url{https://docs.ray.io/en/latest/rllib/index.html}

    \bibitem{Sumo-rl-publications}
    \textit{Sumo-rl Publications}
    \url{https://lucasalegre.github.io/sumo-rl/examples/publications/}\bibitem{alegre2021}
    Alegre, L. N., et al. (2021).
    \emph{Quantifying the impact of non-stationarity in reinforcement learning-based traffic signal control}.
    PeerJ Computer Science.
    \url{https://doi.org/10.7717/peerj-cs.589}

    \bibitem{hwang2023}
    Hwang, S., et al. (2023).
    \emph{Information-Theoretic State Space Model for Multi-View Reinforcement Learning}.
    OpenReview.
    \url{https://openreview.net/forum?id=YourPaperID}

    \bibitem{reza2023}
    Reza, M., et al. (2023).
    \emph{A citywide TD-learning based intelligent traffic signal control for autonomous vehicles: Performance evaluation using SUMO}.
    Wiley Online Library.
    \url{https://doi.org/10.1002/YourDOI}

    \bibitem{ghanadbashi2023}
    Ghanadbashi, H., et al. (2023).
    \emph{Handling uncertainty in self-adaptive systems: an ontology-based reinforcement learning model}.
    Springer.
    \url{https://doi.org/10.1007/YourDOI}

    \bibitem{almeida2022}
    Almeida, A., et al. (2022).
    \emph{Multiagent Reinforcement Learning for Traffic Signal Control: a k-Nearest Neighbors Based Approach}.
    CEUR Workshop Proceedings.
    \url{http://ceur-ws.org/Vol-YourVolumeID}

    \bibitem{zheng2022}
    Zheng, Y., et al. (2022).
    \emph{From Local to Global: A Curriculum Learning Approach for Reinforcement Learning-based Traffic Signal Control}.
    IEEE Xplore.
    \url{https://doi.org/10.1109/YourDOI}

    \bibitem{ault2021}
    Ault, A., Sharon, G. (2021).
    \emph{Reinforcement Learning Benchmarks for Traffic Signal Control}.
    OpenReview.
    \url{https://openreview.net/forum?id=YourPaperID}

    \bibitem{agand2021}
    Agand, R., et al. (2021).
    \emph{EcoLight: Reward Shaping in Deep Reinforcement Learning for Ergonomic Traffic Signal Control}.
    arXiv preprint.
    \url{https://arxiv.org/abs/YourArxivID}

    \bibitem{Kooperative-Lichtsignalsteuerung}
    \textit{Kooperative Lichtsignalsteuerung: Integration von Fahrzeugen in die Steuerung vernetzter Verkehrssysteme}
    \url{https://mediatum.ub.tum.de/doc/1382853/123509.pdf}

    \bibitem{SURTRAC}
    \textit{SURTRAC}
    \url{https://www.ri.cmu.edu/pub_files/2013/1/13-0315.pdf}

    \bibitem{Q-Learning}
    \textit{Q-Learning}
    \url{https://www.datacamp.com/tutorial/introduction-q-learning-beginner-tutorial}

    \bibitem{Deep Q-Learning}
    \textit{Deep Q-Learning}
    \url{https://medium.com/@samina.amin/deep-q-learning-dqn-71c109586bae}

    \bibitem{Proximal Policy Optimization}
    \textit{Proximal Policy Optimization}
    \url{https://spinningup.openai.com/en/latest/algorithms/ppo.html}

    \bibitem{PettingZoo}
    \textit{PettingZoo}
    \url{https://pettingzoo.farama.org/}

    \bibitem{Sumo-HBEFA}
    \textit{Sumo-HBEFA}
    \url{https://sumo.dlr.de/docs/Models/Emissions/HBEFA-based.html}

    \bibitem{osm-git}
    \textit{OSM-Git}
    \url{https://github.com/openstreetmap}

    \bibitem{osm-Guide}
    \textit{OSM-Guide}
    \url{https://wiki.openstreetmap.org/wiki/Beginners%27_guide}

    \bibitem{Verkehrszählungen in Baden-Württemberg}
    \textit{Verkehrszählungen in Baden-Württemberg}
    \url{https://www.lubw.baden-wuerttemberg.de/luft/verkehrszaehlungen-in-baden-wuerttemberg#karte}

    \bibitem{Sumo-tls}
    \textit{Sumo - Traffic Light System}
    \url{https://sumo.dlr.de/docs/Simulation/Traffic_Lights.html}

    \bibitem{Sumo-osm}
    \textit{Sumo - OSM import}
    \url{https://sumo.dlr.de/docs/Networks/Import/OpenStreetMap.html}


\end{thebibliography}

\end{document}
